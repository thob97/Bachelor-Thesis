\subsubsection{Vergleich}\myCheckmark %
Ähnlich wie bei der \nameref{subsection:umfrage} wird sich zu dieser Erhebungstechnik auch [gesondert] Gedanken gemacht. Ziel dadurch soll es sein möglichst Wertvolle und aussagekräftige Ergebnisse zu erzielen und dabei möglichst zeiteffizient vorzugehen. 

\myNewSection
%Was es bringen soll
Durch den Vergleich sollen lediglich Inspiration sowie allgemeine und \glqq offensichtliche\grqq{} Anforderungen gesammelt werden. %
%Was nicht: Abgrenzung
Sie soll nicht dazu verleiten Funktionen und Designs zu kopieren oder sich beeinflussen zu lassen. %
	%Wie dagegen angekommen wird.
	Deshalb werden nur wenige Apps zum Vergleich herangezogen und diese auch nur kurzweilig getestet.%
		%Warum: weiterer Punkt: Zeitaufwand
		%[Außerdem] würde das testen weiterer App auch zu viel Aufwand und Zeit kosten.\newline%
\newline%
%Was: Auswahl von Apps
Bei der Auswahl der Apps wurde versuche diejenigen zu wählen, welche möglichst nützliche Informationen liefern können. %
	%Apple & Google
	Dabei wurden einmal der native Apple iOS Kalender\cite{A_calendarApple} und Google Kalender\cite{A_calendarGoogle} zum Vergleich ausgewählt. Denn es wird vermutet, dass diese Unternehmen durch ihren Erfolg, ihrer Größe und dadurch dass sie eigene Richtlinien für Apps festgelegt haben\cite{konventionen_guidelinesApple, konventionen_guidelinesGoogle}, besonders Achtsam bei der Entwicklung dieser Apps waren.\newline%
	%Calendars
	Des Weiteren wurde eine App nach Kundenbewertungen ausgewählt. Denn wohlmöglich wurde solch eine App deshalb so positiv bewertet, weil sie über Funktionen verfügt, welche bei den anderen beiden nicht vorhanden sind. Die Wahl viel dementsprechend auf Calendars\cite{A_calendarReviews}.%