\subsubsection{Vergleich} %
Ähnlich wie bei der \nameref{subsection:umfrage} werden auch für diese Erhebungstechnik weitere Überlegungen angestellt. Das Ziel dabei ist es, möglichst wertvolle und aussagekräftige Ergebnisse zu erzielen und dabei zugleich zeiteffizient vorzugehen..%
%
%
\newline%
\myNewSection%
%Was es bringen soll
Durch den Vergleich sollen lediglich Inspiration sowie allgemeine und offensichtliche Anforderungen gesammelt werden. %
%Was nicht: Abgrenzung
Das Ziel ist es jedoch nicht, Funktionen und Designs von anderen Apps zu kopieren oder sich von ihnen beeinflussen zu lassen. %
	%Wie dagegen angekommen wird.
	Aus diesem Grund werden nur wenige Apps zum Vergleich herangezogen und diese auch nur kurzzeitig getestet.%
		%Warum: weiterer Punkt: Zeitaufwand
		%[Außerdem] würde das testen weiterer App auch zu viel Aufwand und Zeit kosten.\newline%
\newline%
%Was: Auswahl von Apps
Bei der Auswahl der Apps wurde versucht, solche auszuwählen, die möglichst nützliche Informationen liefern können. %
	%Apple & Google
	Dazu wurden der native Apple iOS Kalender\cite{A_calendarApple} und der Google Kalender\cite{A_calendarGoogle} ausgewählt, da angenommen wird, dass diese Unternehmen aufgrund ihres Erfolgs, ihrer Größe und da sie eigenen Richtlinien für Apps haben \cite{konventionen_platforms_ios, konventionen_guidelinesGoogle}, besonders sorgfältig bei der Entwicklung dieser Apps vorgegangen sind.\newline%
	%Calendars
	Darüber hinaus wurde eine App aufgrund von Kundenbewertungen ausgewählt. Bei einer solchen App wäre es nämlich beispielsweise möglich, dass sie aufgrund von Funktionen, die in den anderen beiden Apps nicht vorhanden sind, so positiv bewertet wurde. Aus diesem Grund wurde sich für Calendars\cite{A_calendarReviews} entschieden.%