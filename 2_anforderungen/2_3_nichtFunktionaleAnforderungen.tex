\subsection{Nicht Funktionale Anforderungen}\label{subsection:anforderung:nichtFunktionaleAnforderungen}
%Was: nfA Erläutern + Aufzählung/Bewertung
In diesem Abschnitt werden die nicht funktionalen Anforderungen, das sind all jene Eigenschaften, welche die App besitzen soll, erläutert und in eine Aufzählung der Wichtigkeit nach bewertet.\newline%
%Warum: priorisierung + zeit
Eine solche Priorisierung hilft dabei zu erkennen, welche Anforderungen für diese Arbeit am wichtigsten sind. Das ist hilfreich, da die begrenzte Arbeitszeit es nicht erlaubt, alle Anforderungen gleich intensiv zu berücksichtigen.\newline%
%Was
Dafür wird die MoSCoW-Priorisierung verwendet. Dabei werden die Anforderungen in die Kategorien \glqq Must-have\grqq{} (unbedingt notwendig für das Projekt), \glqq Should-have\grqq{} (wichtig, aber nicht notwendig für den Erfolg der Anwendung), \glqq Could-have\grqq{} (wünschenswert, aber nicht unbedingt notwendig), und \glqq Won't-have\grqq{} (erwähnenswert, werden aber bewusst nicht umgesetzt) eingeteilt. Im Folgenden werden diese Kategorien mit M, S, C und W abgekürzt.%
%
\begin{itemize}
	%-----------------------------Stärken von Pcs und Handys-----------------------------
	\item \textbf{M Stärken von Pcs und Handys}: %
		%Was+Warum+ref
		Die wohl wichtigste Anforderung an die App besteht darin, die Vorteile von PCs und Handys optimal zu nutzen, da dies ein Hauptziel dieser Arbeit ist. Hierbei sollen alle Erkenntnisse berücksichtigt werden, die in \secref{section:pcVsPhone} gesammelt wurden.
		%Da aber bereits viele davon ausgiebig im diesem Abschnitt behandelt wurden, werden nur jene dieser Erkenntnisse, [zu welchen noch was hinzugefügt werden kann], erneut hier erwähnt. %
		%Wie/Auswirkung/wichtigkeit	
%		\myComment{
%		Viele dieser Eigenschaften sind emergent und entstehen erst durch die Zusammenwirkung von den nicht funktionalen und funktionalen Anforderungen. Dementsprechend werden einige dieser Eigenschaften in  erwähnt.
%		}

		%Beispiel-hidden
%		\myComment{Funktionen welche auf einen Endgerät besser Funktionieren als auf den anderen, sollten vielleicht überdacht oder Verlagert werden. So sollte zum Beispiel die Tastatureingabe auf dem Handy wahrscheinlich versucht werden zu überdenken. Entweder könnte man alternative Eingabe nutzen, wie zum Beispiel VoiceToSpeech, oder falls möglich wird die Texteingabe auf dem Pc verlegt. \newline}

	%-----------------------------Benutzbarkeit-----------------------------
	\item \textbf{M Benutzbarkeit}: %
		%Was: Benutzbarkeit
		In dieser Arbeit wird unter \glqq Benutzbarkeit\grqq{} verstanden, dass eine Anwendung intuitiv, einfach und effektiv nutzbar ist.\newline%
		%Warum: PcVsPhone Stärke
		Eine Erkenntnis aus \secref{section:pcVsPhone} ist, dass genau das eine Stärke der Handys ist. Daher sollte diese Eigenschaft auch in der zu entwickelnden App sichtbar werden.\newline%
			%Warum: nützlich + intuitiv
			Die Benutzbarkeit kann dabei helfen, dass die App nicht nur nützliches im Konzept, sonder auch nützlich für den Endnutzer ist sowie das möglichst viele Personen auch ohne große Einarbeitung und Vorwissen die App nutzen können.\newline%
		%Wie: Design
		Um die Benutzbarkeit der App zu verbessern, scheint das Design der App eine wichtige Variable zu sein, da es das einzige ist, mit dem der Nutzer interagiert. 
			%Warum: in PcVsPhone erwähnt
				Diese Erkenntnis wurde bereits in \secref{section:pcVsPhone} festgestellt, zusammen mit der Feststellung, dass Richtlinien für gutes Design wichtig sind. Unter anderem wurde begründet, dass Handy-Nutzer Apps bevorzugen \cite{pcVsphone_mobileAppVsWebTimeSpent}, da diese passende Richtlinien befolgen, während Websites eher für PCs ausgelegt sind und daher weniger auf für Handys passende Konventionen achten. % 
			%Warum: quelle
			Eine weitere Statistik, die auf die Bedeutung einer guten Benutzbarkeit hinweist, ist, dass Benutzer eher dazu neigen, eine Anwendung weiter zu nutzen, wenn sie intuitiv und einfach zu bedienen ist \cite{pcVsphone_peopleWillRevisitMobileIfEasyToUse}.\newline%
	%Wie/Auswirkung: ref
	Dementsprechend wird die Benutzbarkeit als eine wichtige Anforderung bewertet.% und der Sicherstellung davon wird deshalb auch einen eigenen \secref{section:design} gewidmet.
		
		
	%-----------------------------Wartbarkeit-----------------------------
	\item \textbf{S Wartbarkeit, Erweiterbarkeit, Verständlichkeit}: %
		%Warum
		Da, wie zuvor erwähnt, die Bachelorarbeit nur eine begrenzte Bearbeitungszeit zulässt, muss damit gerechnet werden, dass nicht alle Funktionen bis zum Abgabetermin umgesetzt werden können.  %
		%Was
		Daher sollte der Quellcode so gestaltet sein, dass er für zukünftige Entwicklungen und auch für Dritte leicht zugänglich ist. %
		%Wie
		Um dies zu erreichen, muss der Programmcode möglichst wartbar sein, was bedeutet, dass er verständlich und erweiterbar sein sollte.\newline%
		%hidden-agile-arbeitsweise
%		\myComment{
%			Außerdem sollten späte Änderungen in den Anforderungen nicht allzu umständlich umzusetzen sein, da wir in dieser Arbeit und durch unsere Agile-Arbeitsweise genau diese Veränderungen erwarten. \newline
%			Deswegen werden die die drei stark miteinander verbundenen Anforderungen Wartbarkeit, Erweiterbarkeit und Verständlichkeit als wichtig eingeschätzt. \newline
%			}
		%ref
		%Wie versucht wird diese Anforderung zu ermöglichen wird im \secref{section:implementierung} besprochen.
		

	%-----------------------------Qualität \& Korrektheit-----------------------------
	\item \textbf{S Qualität \& Korrektheit}: %
		%Was:
		Mit dieser Anforderung soll sichergestellt werden, dass die App sich genau so verhält, wie zuvor spezifiziert. Anders ausgedrückt soll es zu keinen unerwarteten Situationen wie Fehlern oder Abstürzen kommen. %
		%Warum:
		Falls dies dennoch der Fall ist, würde dies nicht nur die Benutzbarkeit einschränken, sondern auch die Nutzer irritieren und möglicherweise dazu bewegen, die App nicht weiter zu nutzen. %
			%Quellen:
			Laut einer Umfrage würden 88\% der Nutzer die App bei einem Fehler verlassen\cite{nfA_bugsAbandon}. Daten von Google bekräftigen diese Aussage, so handeln laut ihnen 54\% aller 1-Sterne-Bewertungen im Play Store von Fehlern oder Stabilitätsproblemen \cite{nfA_bugsReview}. %
			%Auswirkung: -> wichtig
			Aus diesem Grund wird diese Anforderung als durchaus wichtig eingeschätzt.\newline%
		%Wie
		Um die Qualität sicherzustellen, darf man entweder beim Programmieren keine Fehler machen, was aus Erfahrung keine sinnvolle Annahme ist, oder man testet die Software ausgiebig genug, um nachzuweisen, dass die Software keine Defekte aufweist. 
		%Auswirkung + ref
		Daher wird die zu erstellende Anwendung ausgiebig getestet. %Wie genau das vonstatten geht im \secref{section:implementierung} erwähnt.
	
	%-----------------------------Performance-----------------------------
	\item \textbf{C Leistung}: %
		%Warum: Erkenntnis PcVsPhone
		Eine Erkenntnis aus \secref{section:pcVsPhone} ist, dass Anwendungen für das Handy nicht zu viel Leistung und Ressourcen benötigen sollten. %
		%Was:
		Dementsprechend wird versucht, den Speicher-, Prozessor- und Netzwerkverbrauch zu minimieren. %
		%Warum: 
			%Speicherbedarf-> Löschen
			Ein zu großer Speicherbedarf könnte zum Beispiel dazu führen, dass die App aufgrund begrenzter Speicherkapazität nicht genutzt oder entfernt wird. %
			%CPU Aufwand -> rucklern + batterie -> Benutzbarkeit und Benutzererfahrung
			Eine zu aufwändige App kann auch zu Ruckeln auf der grafischen Oberfläche und schneller entleerender Batterie führen. Beides davon würde zu einer Einschränkung der Benutzbarkeit und der Benutzererfahrung führen. %
			%Was: Ladezeit
			Der als am wichtigsten bewertete Grund für die Beachtung der Performance ist jedoch, dass eine zu netzwerk- oder prozessorintensive App zu langen Ladezeiten führen kann. %
				%Warum: Stärke PcVsPhone
				Diese langen Ladezeiten würden gegen die Stärke des Smartphones für kurzweilige Aufgaben widersprechen. %
				%Quelle:
				Eine Messung von Google bekräftigt diese Annahme. So werden laut ihr 53\% aller Webseiten verlassen, wenn das Laden länger als drei Sekunden dauert\cite{pcVsphone_threeSeconds}.
	
	%-----------------------------Reichweite-----------------------------
	\item \textbf{C Reichweite}: %
		%Was
		Auch wichtig, aber im Vergleich zu den anderen Anforderungen eher zweitrangig, ist die Reichweite. %
			%Warum
			Zwar soll es möglichst vielen Handy-Nutzern ermöglicht werden, die zu erstellende App zu nutzen, jedoch wird es sich bei der App wahrscheinlich eher um eine Nischenanwendung handeln. Denn bei den bereits existierenden CLI-Terminkalendern, auf welche die App aufbauen soll, scheint es sich, anhand der eher wenigen GitHub-Sterne, auch bereits um Nischenanwendungen zu handeln\cite{cli_calcurseGitHub, cli_khal}. % 
			%Auswirkung:
			Daher soll nicht zu viel Aufwand in diese Eigenschaft fließen.\newline%
		%Wie: crossplatform
		Am wichtigsten für die Reichweite ist es, dass die App für Android- und iOS-Systeme verfügbar ist, da dies die beliebtesten Systeme für Handys sind\cite{pcVsphone_mobileOperatingSystem}. %
		%Wie: Alter von Handys
		Aber auch das Alter könnte dafür wichtig sein. Wenn die zu erstellende App eine zu hohe Android- oder Apple-Version voraussetzt, können ältere Handys diese nicht verwenden.%
%
%
%
%hidden-useless nfA
%\myComment{
%	\item \textbf{Sicherheit}: Die Anforderung Sicherheit steht dafür, dass die in der App verwendeten Daten\footnote{Zum Beispiel die Kalendereinträge} nicht von dritten mitgelesen werden können. \newline%
%	Für dieses Projekt gibt es drei verschiedene Standort-Möglichkeiten der zu schützenden Daten und somit auch drei verschiedene Angriffsflächen für einen Dritten die Daten zu stehlen.\newline%
%	Erstens die Situation, dass sich die Daten auf dem Handy befinden. Die Sicherheit hierzu wird jedoch vom Betriebsystem sichergestellt. So bietet Appel zum Beispiel für jede App eine Sandbox, welche verhindert, dass andere Apps auf die zu schützenden Daten zugreifen könnten \cite{nfA_sandbox}.\newline%
%	Die zweite und dritte Situation wären einmal, dass sich die Daten auf dem Backend befinden oder während der Kommunikation von Endgerät zu Backend. Beide dieser Situationen werden meist Framework und Datenbank geschützt, da aktuelle Software oft standardmäßig bereits starke Verschlüsselung benutzen. \newline%
%	Diese Anforderung wird also nicht als niedrigstes eingestuft, weil sie am unwichtigsten betrachten wird, sonder weil wir für dieses Projekt nur wenig Einfluss darauf ausüben können. Es ist lediglich wichtig ein Backend und Framework zu finden, welche Verschlüsselung anbieten.
%}

\end{itemize}