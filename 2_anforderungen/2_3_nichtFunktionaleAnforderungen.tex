\subsection{Nicht Funktionale Anforderungen} \myCheckmark
Dieser Abschnitt handelt von den nicht funktionalen Anforderungen. Das sind all jene Eigenschaften welche die App besitzen soll. Dabei werden die Anforderungen nicht nur erläutert sondern auch durch eine Aufzählung nach der Wichtigkeit bewertet.\newline%
Einerseits hilft diese Priorisierung nämlich dabei zu erkennen welche Anforderungen für diese Arbeit am wichtigsten sind, und andererseits lässt es die begrenzte Arbeitszeit nicht zu alle Anforderungen gleich intensive Beachtung zu schenken.

\myNewSection %todo remove
\myTextTodo{LEITFADEN: 1.Was ist die Eigenschaft, 2.Warum ist es wichtig, 3.Wie wird es umgesetzt?} %%%

\myTextTodo{\textbf{PcVsPhone Leistung -> n.f.A:} Deshalb wird darauf geachtet, dass die App nicht zu Ressourcen-Aufwändig wird, wie bereits in den nicht funktionalen Anforderungen erwähnt. Da es sich bei der App um einen Kalender und nicht um komplizierte 3D-Darstellungen oder Algorithmen handelt, wird aber davon ausgegangen, dass es zu keinen Performance-Problemen kommen sollte.}

\myNewSection
\myTextTodo{\textbf{PcVsPhone Internet -> n.f.A}: Deshalb wollen wir bei der App eine schnelle Ladezeit sicherstellen. Um das zu erreichen wird es wichtig sein beim Abtausch von Daten über das Internet diese möglichst brandbreiten-effizient zu übertragen, sowie die App zur Leistung zur optimieren}

\myNewSection
\myTextTodo{\textbf{PcVsPhone Speicher -> nfA}: Es wird zwar davon ausgegangen, dass die App keine absurde Menge von Daten benötigen wird. Trotzdem wird darauf geachtet, dass die App keine große Menge an Speicher annimmt, um auch Geräte mit kleineren Kapazität zu unterstützen. Besonders sollten nicht zuviele Daten geladen werden, damit es zu keiner langen Ladezeit kommt.\newline
Pc als Speicherablage/was man irgendwann mal braucht. Handy nur für wichtigstes/was man unterwegs/immer baucht. Pc sehr große Anwendungen (200gb) alleine. Auf Handy muss nach speichere eher optimiert werden.}

\begin{itemize}

	\item \textbf{Stärken von Pcs und Handys}: die wohl wichtigste Eigenschaft und auch eine Hauptaufgabe dieser Arbeit ist es, die Vorzüge von Pc's und Handy's in der App zu nutzen. Funktionen welche auf einen Endgerät besser Funktionieren als auf den anderen, sollten vielleicht überdacht oder Verlagert werden. So sollte zum Beispiel die Tastatureingabe auf dem Handy wahrscheinlich versucht werden zu überdenken. Entweder könnte man alternative Eingabe nutzen, wie zum Beispiel VoiceToSpeech, oder falls möglich wird die Texteingabe auf dem Pc verlegt. \newline%
	Diese Eigenschaft hat große Auswirkungen auf die funktionalen Anforderung haben, da jede Funktion mit dieser Eigenschaft überdenken werden muss. \newline%
	Wie sich die beiden Systeme in Ihren Stärken überhaupt unterscheiden, wurde im Kapitel ... behandelt.
		
	\item \textbf{Wartbarkeit, Erweiterbarkeit, Verständlichkeit}: Da die Bachelorarbeit, wie bereits erwähnt, nur eine kurze Bearbeitungszeit zulässt, muss man damit rechnen, dass nicht alle Funktionen bis zur Abgabeschluss umgesetzt werden können. Daher soll der Quellcode möglichst gut für in der Zukunft liegende und Fremde Weiterentwicklung ausgelegt sein. Um das zu erreichen muss der Quellcode möglichst Wartbar sein, was anders ausgedrückt bedeutet, dass der Quellcode verständlich und erweiterbar sein muss.\newline%
		Des Weiteren sollten späte Änderungen in den Anforderungen nicht allzu umständlich umzusetzen sein, da wir in dieser Arbeit und durch unsere Agile-Arbeitsweise genau diese Veränderungen erwarten. \newline%
		Deswegen werden die die drei stark miteinander verbundenen Anforderungen Wartbarkeit, Erweiterbarkeit und Verständlichkeit als wichtig eingeschätzt. \newline%
		Wie versucht wird diese Anforderung zu ermöglichen wird im Kapitel ... besprochen.
		
	\item \textbf{Benutzbarkeit}: Unter Benutzbarkeit verstehen wir, dass die App intuitiv, einfach und effektiv zu nutzen ist. \newline%
	Die App soll nicht nur nützliches im Konzept, sonder auch nützlich für den Endnutzer sein. Daher soll die App möglichst effektiv zu nutzen sein. Wichtige und oft genutzt Funktionen sollten also zum Beispiel leicht zugänglich sein, anstatt diese hinter mehreren Seiten zu verstecken und damit die Nutzung zu erschweren. \newline%
	Des Weiteren sollen möglichst viele Personen auch ohne große Einarbeitung und Vorwissen die App nutzen können. Dafür muss die App intuitiv und einfach sein. Ein Beispiel wie man das umsetzen könnte ist, das alle Funktionen haben eine klare Bedeutung haben und sind genau dort aufzufinden wo man sie auch erwartet. \newline%
	Die wahrscheinlich wichtigste Variabel um eine gute Benutzbarkeit zu ermöglichen ist das Design, denn sie ist das einzige mit welchem der Nutzer interagiert.\newline%
	Im Kaptitel ... wurde bereits festgestellt, dass das Design sowie Richtlinien dazu sehr wichtig scheinen. So wurde unter anderem Begründet, dass Handynutzer Apps präferieren\cite{pcVsphone_mobileAppVsWebTimeSpent}, da diese passende Richtlinien befolgen, während Websites auch für Pc's ausgelegt sind und daher weniger auf für Handys passende Konventionen achten. Außerdem scheint es, dass wenn eine Anwendung intuitiv und leicht zu nutzen ist, Benutzer eher dazu geneigt sind die Anwendung weiter zu nutzen\cite{pcVsphone_peopleWillRevisitMobileIfEasyToUse}.\newline%
	Daher sind wir ziemlich sicher, dass die Benutzbarkeit eine wichtige Anforderung für diese App darstellt. Deshalb wird sich dem Design ein eigenen Abschnitt im Kapitel ... gewidmet.

%todo remove
\myComment{
	Dass das Design sowie die Richtlinien dafür nicht nur wichtig für uns erscheint sondern es dafür auch Andeutungen gibt, haben wir im Kapitel ... festgestellt. So wurde unter anderem Begründet, dass Handynutzer Apps präferieren\cite{}, da diese passende Richtlinien befolgen, während Websites auch für Pc's ausgelegt sind und daher weniger auf für Handys passende Konventionen achten. Außerdem scheint es, dass wenn eine Anwendung intuitiv und leicht zu nutzen ist, Benutzer eher dazu geneigt sind die Anwendung weiter zu nutzen\cite{pcVsphone_peopleWillRevisitMobileIfEasyToUse}.\newline%
	Wegen diesen Gründen betrachten wir die Benuztbarkeit als eine wichtige Eigenschaft. Deshalb gibt es in ... einen eigenen Abschnitt, welcher sich mit dem Design auseinander setzt.
	
	Für die Benutzbarkeit in Apps ist das Design entscheidend, da dies das einzige ist womit der Nutzer interagiert. Dass das Design sowie Richtlinien für jenes wichtig für Apps sind haben wir im Kapitel ... festgestellt. So wurde unter anderem Begründet, dass Handynutzer Apps präferieren\cite{}, da diese passende Richtlinien befolgen, während Websites auch für Pc's ausgelegt sind und daher weniger auf für Handys passende Konventionen achten. \newline%
	Ein weiteres Indiz was die Wichtigkeit dieser Eigenschaft untermauert ist, dass wenn eine Anwendung intuitiv und leicht zu nutzen ist, die Benutzer eher dazu geneigt sind die Anwendung weiter zu nutzen\cite{pcVsphone_peopleWillRevisitMobileIfEasyToUse}.\newline%
	Daher schätzen wir die Benutzbarkeit als eine wichtige Eigenschaft für die App ein. Die App soll immerhin nicht nur nützliches im Konzept sein, sonder auch nützlich für den Endnutzer sein.
	
	Die App soll nicht nur nützliches im Konzept sein, sonder auch nützlich für den Endnutzer sein. Dafür sollte sie also möglichst effektiv und intuitiv zu benutzen sein.\newline%
	Für die Benutzbarkeit in Apps ist dafür das Design entscheidend, da dies das einzige ist womit der Nutzer interagiert. Dass das Design sowie Richtlinien für jenes wichtig für Apps sind haben wir im Kapitel ... festgestellt. So wurde unter anderem Begründet, dass Handynutzer Apps präferieren\cite{}, da diese passende Richtlinien befolgen, während Websites auch für Pc's ausgelegt sind und daher weniger auf für Handys passende Konventionen achten. \newline%
	Ein weiteres Indiz was die Wichtigkeit dieser Eigenschaft untermauert ist, dass wenn eine Anwendung intuitiv und leicht zu nutzen ist, die Benutzer eher dazu geneigt sind die Anwendung weiter zu nutzen\cite{pcVsphone_peopleWillRevisitMobileIfEasyToUse}.\newline%
	Deshalb gibt es in ... einen eigenen Abschnitt, welcher sich mit dem Design und Richtlinien auseinander setzt. 
	
	 Ein Beispiel dafür: wichtig Funktionen sollten leicht zugänglich sein, anstatt diese hinter mehreren Seiten zu verstecken und damit die Nutzung zu erschweren.\newline%
		Um möglichst viele Personen unserer Zielgruppe auch ohne große Einarbeitung und Vorwissen den Zugang zu ermöglichen, soll die App außerdem möglichst intuitiv sein. Also alle Funktionen haben eine klare Bedeutung und sind genau dort aufzufinden wo man sie auch erwartet.\newline% 
		Diese Anforderungen hängen am meisten von der Grafischenoberfläche der App ab, da dass das einzige ist mit was der Nutzer interagiert. Deswegen gibt es in ... einen eigenen Abschnitt, welcher sich mit dem Design auseinander setzt.
}%%%

		
	\item \textbf{Qualität/Korrektheit}: Mit der Anforderung Qualität soll sichergestellt werden, dass die App sich genau so zu verhalten hat wie zuvor Spezifiziert. Es soll also zu keinen unerwarteten Situationen wie Fehler und Abstürzen kommen.\newline%
	Falls doch würden das nicht nur die Benutzbarkeit einschränken, sondern auch die Nutzer irritieren und möglicherweise dazu bewegen die App nicht weiter zu nutzen. So würden laut einer Umfrage 88\% von Nutzern die App bei einem "Bug" verlassen\cite{nfA_bugsAbandon}. Daten von Google bekräftigen diese Aussage. So handeln 54\% aller 1-Sterne-Bewertungen im Play Store von "Bugs" oder "Stability"\cite{nfA_bugsReview}. Deshalb sehen wir die Anforderung Qualität auch als durchaus wichtig an.\newline%
	Um die Qualität sicherzustellen darf man entweder beim Programmieren keine Fehler machen, was keine Sinnvolle Annahme ist, da Agilen-Arbeitsweisen Fehler erwartet werden, oder man testet die Software ausgiebig genug, sodass man nachweisen kann, dass die Software keine Defekte besitzt. Daher werden wir unsere App testen. Wie genau das vonstatten geht im Kapitel ... erwähnt.
	
	\item \textbf{Reichweite}: Auch wichtig aber im Vergleich zu den anderen Anforderungen eher zweitrangig ist die Reichweite. Zwar sollen möglichst vielen Handy-Nutzern ermöglicht werden die App zu benutzen, jedoch handelt es sich bei um eine nischen-Anwendung und daher sollte nicht zuviel Aufwand in diese Eigenschaft fließen.\newline%
	Am wichtigsten für die Reichweite ist es, dass die App von Android und iOS genutzt werden können, da dies die weitverbeitesten Systeme im Handymarkt sind\cite{}. Andererseits kann auch das Alter von Handys kann hierzu betrachtet werden. Wenn unsere App zu viel Ressourcen benötigt oder eine zu hohe Android oder Apple Version beanspruchen, könnte Sie von älteren Handys nicht benutzt werden.
	%todo Verteilung der Handy os-verionen %mabye doch eher unwichtig, da quelle: viele leute kaufen oft neue handys
	
	\item \textbf{Sicherheit}: Die Anforderung Sicherheit steht dafür, dass die in der App verwendeten Daten\footnote{Zum Beispiel die Kalendereinträge} nicht von dritten mitgelesen werden können. \newline%
	Für dieses Projekt gibt es drei verschiedene Standort-Möglichkeiten der zu schützenden Daten und somit auch drei verschiedene Angriffsflächen für einen Dritten die Daten zu stehlen.\newline%
	Erstens die Situation, dass sich die Daten auf dem Handy befinden. Die Sicherheit hierzu wird jedoch vom Betriebsystem sichergestellt. So bietet Appel zum Beispiel für jede App eine Sandbox, welche verhindert, dass andere Apps auf die zu schützenden Daten zugreifen könnten \cite{nfA_sandbox}.\newline%
	Die zweite und dritte Situation wären einmal, dass sich die Daten auf dem Backend befinden oder während der Kommunikation von Endgerät zu Backend. Beide dieser Situationen werden meist Framework und Datenbank geschützt, da aktuelle Software oft standardmäßig bereits starke Verschlüsselung benutzen. \newline%
	Diese Anforderung wird also nicht als niedrigstes eingestuft, weil sie am unwichtigsten betrachten wird, sonder weil wir für dieses Projekt nur wenig Einfluss darauf ausüben können. Es ist lediglich wichtig ein Backend und Framework zu finden, welche Verschlüsselung anbieten.

\end{itemize}