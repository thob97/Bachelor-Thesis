\subsection{Nicht Funktionale Anforderungen}\label{subsection:anforderung:nichtFunktionaleAnforderungen}\myCheckmark
%Was: nfA Erläutern + Aufzählung/Bewertung
In diesem Abschnitt werden die nicht funktionalen Anforderungen, das sind all jene Eigenschaften welche die App besitzen soll, erläutert und durch eine Aufzählung der Wichtigkeit nach bewertet.\newline%
%Warum: priorisierung + zeit
Solch eine Priorisierung hilft zu erkennen welche Anforderungen für diese Arbeit am wichtigsten sind. Dies ist hilfreich, da es die begrenzte Arbeitszeit nicht zu lässt alle Anforderungen gleich intensive Beachtung zu schenken.\newline%
%Was
Dafür wird die MoSCoW Priorisierung verwendet. Dabei werden [Objekte, Items] zu den Kategorien \glqq must-have\grqq{}(essential für das Projekt), \glqq should-have\grqq{}(wichtig aber nicht essential), \glqq could-have\grqq{}(wäre schön zu haben), und \glqq won't-have\grqq{}(auch erwähnenswert aber nicht bedient benötigt) zugeordnet. Im folgenden werden diese Kategorien mit M, S, C und W abgekürzt.

\begin{itemize}
	%-----------------------------Stärken von Pcs und Handys-----------------------------
	\item \textbf{M Stärken von Pcs und Handys}: %
		%Was+Warum+ref
		Die wohl wichtigste Eigenschaft, weil dies eine Hauptaufgabe dieser Arbeit ist, ist die Vorzüge von Pc's und Handy's in der App zu nutzen. Dabei wird versucht alle die in \secref{section:pcVsPhone} gesammelten Erkenntnisse während der Entwicklung zu beachten.
		%Da aber bereits viele davon ausgiebig im diesem Abschnitt behandelt wurden, werden nur jene dieser Erkenntnisse, [zu welchen noch was hinzugefügt werden kann], erneut hier erwähnt. %
		%Wie/Auswirkung/wichtigkeit	
		\myComment{
		Viele dieser Eigenschaften sind emergent und entstehen erst durch die Zusammenwirkung von den nicht funktionalen und funktionalen Anforderungen. Dementsprechend werden einige dieser Eigenschaften in  erwähnt.
		}

		%Beispiel-hidden
		\myComment{Funktionen welche auf einen Endgerät besser Funktionieren als auf den anderen, sollten vielleicht überdacht oder Verlagert werden. So sollte zum Beispiel die Tastatureingabe auf dem Handy wahrscheinlich versucht werden zu überdenken. Entweder könnte man alternative Eingabe nutzen, wie zum Beispiel VoiceToSpeech, oder falls möglich wird die Texteingabe auf dem Pc verlegt. \newline}

	%-----------------------------Benutzbarkeit-----------------------------
	\item \textbf{M Benutzbarkeit}: %
		%Was: Benutzbarkeit
		Unter Benutzbarkeit wird in dieser Arbeit verstanden, dass eine Anwendung intuitiv, einfach und effektiv zu nutzen ist.\newline%
		%Warum: PcVsPhone Stärke
		Eine Erkenntnis aus \secref{section:pcVsPhone} ist, dass genau das eine Stärke der Handys ist. Dementsprechend soll diese Eigenschaft auch in der zu erstellenden App [ersichtlich/erkennbar] werden.\newline%
			%Warum: nützlich + intuitiv
			So kann die Benutzbarkeit dabei helfen, dass die App nicht nur nützliches im Konzept, sonder auch nützlich für den Endnutzer ist sowie das möglichst viele Personen auch ohne große Einarbeitung und Vorwissen die App nutzen können.
		%Wie: Design
		Um die Benutzbarkeit zu beeinflussen scheint eine wichtige Variabel das Design dar App zu sein, denn sie ist das einzige mit welchem der Nutzer interagiert. 
			%Warum: in PcVsPhone erwähnt
			Diese Erkenntnis und das Richtlinien dazu wichtig sind wurde in \secref{section:pcVsPhone} bereits festgestellt. So wurde unter anderem Begründet, dass Handy-Nutzer Apps präferieren\cite{pcVsphone_mobileAppVsWebTimeSpent}, da diese passende Richtlinien befolgen, während Websites auch für Pc's ausgelegt sind und daher weniger auf für Handys passende Konventionen achten. 
			%Warum: quelle
			Eine weitere Statistik welche die Nützlichkeit von einer guten Benutzbarkeit andeutet ist, dass wenn eine Anwendung intuitiv und leicht zu nutzen ist, Benutzer eher dazu geneigt sind die Anwendung weiter zu nutzen\cite{pcVsphone_peopleWillRevisitMobileIfEasyToUse}.\newline%
	%Wie/Auswirkung: ref
	Dementsprechend wird die Benutzbarkeit als eine wichtige Anforderung bewertet.% und der Sicherstellung davon wird deshalb auch einen eigenen \secref{section:design} gewidmet.
		
		
	%-----------------------------Wartbarkeit-----------------------------
	\item \textbf{S Wartbarkeit, Erweiterbarkeit, Verständlichkeit}: %
		%Warum
		Da, wie bereits erwähnt, die Bachelorarbeit nur eine kurze Bearbeitungszeit zulässt, muss damit gerechnet werden, dass nicht alle Funktionen bis zur Abgabeschluss umgesetzt werden können. %
		%Was
		Daher soll der Quellcode möglichst gut für in der Zukunft liegende und Fremde Weiterentwicklung ausgelegt sein. %
		%Wie
		Um das zu erreichen muss der Programmcode möglichst Wartbar sein, was anders ausgedrückt bedeutet, dass dieser verständlich und erweiterbar sein muss.\newline%
		%hidden-agile-arbeitsweise
		\myComment{
			Außerdem sollten späte Änderungen in den Anforderungen nicht allzu umständlich umzusetzen sein, da wir in dieser Arbeit und durch unsere Agile-Arbeitsweise genau diese Veränderungen erwarten. \newline
			Deswegen werden die die drei stark miteinander verbundenen Anforderungen Wartbarkeit, Erweiterbarkeit und Verständlichkeit als wichtig eingeschätzt. \newline
			}
		%ref
		%Wie versucht wird diese Anforderung zu ermöglichen wird im \secref{section:implementierung} besprochen.
		

	%-----------------------------Qualität \& Korrektheit-----------------------------
	\item \textbf{S Qualität \& Korrektheit}: %
		%Was:
		Mit dieser Anforderung soll sichergestellt werden, dass die App sich genau so zu verhalten hat wie zuvor Spezifiziert. Anders beschrieben soll es zu keinen unerwarteten Situationen wie Fehler oder Abstürzen kommen. %
		%Warum:
		Falls doch würden das nicht nur die Benutzbarkeit einschränken, sondern auch die Nutzer irritieren und möglicherweise dazu bewegen die App nicht weiter zu nutzen. %
			%Quellen:
			So würden laut einer Umfrage 88\% von Nutzern die App bei einem Fehler verlassen\cite{nfA_bugsAbandon}. Daten von Google bekräftigen diese Aussage. So handeln 54\% aller 1-Sterne-Bewertungen im Play Store von Fehlern oder Stabilitätsproblemen\cite{nfA_bugsReview}. %
			%Auswirkung: -> wichtig
			Deshalb wird diese Anforderung auch als durchaus wichtig eingeschätzt.\newline%
		%Wie
		Um die Qualität sicherzustellen darf man entweder beim Programmieren keine Fehler machen, was aus Erfahrung keine Sinnvolle Annahme ist, oder man testet die Software ausgiebig genug, sodass man nachweisen kann, dass die Software keine Defekte besitzt. 
		%Auswirkung + ref
		Daher wird die zu erstellende Anwendung ausgiebig getestet. %Wie genau das vonstatten geht im \secref{section:implementierung} erwähnt.
	
	%-----------------------------Performance-----------------------------
	\item \textbf{C Performance}: %
		%Warum: Erkenntnis PcVsPhone
		Eine Erkenntnis aus \secref{section:pcVsPhone} ist, dass Anwendungen für das Handy nicht zu viel Leistung und Ressourcen benötigen sollten. %
		%Was:
		Dementsprechend wird versucht auf den Speicher-, Prozessor- und Netzwerkverbrauch zu achten. %
		%Warum: 
			%Speicherbedarf-> Löschen
			So könnte zum Beispiel ein zu großer Speicherbedarf dazu führen, dass die App aus begrenzten Speicherkapazität nicht genutzt oder entfernt wird. %
			%CPU Aufwand -> rucklern + batterie -> Benutzbarkeit und Benutzererfahrung
			Weiter könnte eine zu Aufwändige App zu ruckeln auf der grafischen Oberfläche und einer schneller entleerenden Batterie führen. Bei beidem wird davon ausgegangen, dass dies die Benutzbarkeit und Benutzererfahrung einschränken könnte. %
			%Was: Ladezeit
			Der aber als am wichtigsten betrachteter Grund, warum es Sinnvoll ist die Performance zu betrachten, ist, dass eine Netzwerk oder Prozessor aufwändige App zu langen Ladezeiten führen kann. %
				%Warum: Stärke PcVsPhone
				Diese langen Ladezeiten würden gegen die Stärke, dass das Handy gut für kurzweilige Aufgaben ist, wirken. %
				%Quelle:
				Eine Messung von Google bekräftigt diese Annahme. So werden laut ihr 53\% aller Webseiten verlassen wann das Laden länger als drei Sekunden dauert\cite{pcVsphone_threeSeconds}.
	
	%-----------------------------Reichweite-----------------------------
	\item \textbf{C Reichweite}: %
		%Was
		Auch wichtig aber im Vergleich zu den anderen Anforderungen eher zweitrangig ist die Reichweite. %
			%Warum
			Zwar sollen möglichst vielen Handy-Nutzern ermöglicht werden die zu erstellende App zu benutzen, jedoch wird es sich der App wahrscheinlich eher um eine Nischenanwendung handeln. Denn bei den bereits existierenden CLI-Terminkalender, auf welche die App aufbauen soll, scheint es sich, durch Betrachtung der GitHub Sterne, auch bereits um Nischenanwendungen zu handeln\cite{cli_calcurseGitHub, cli_khal}. % 
			%Auswirkung:
			Daher soll nicht zu viel Aufwand in diese Eigenschaft fließen.\newline%
		%Wie: crossplatform
		Am wichtigsten für die Reichweite ist es, dass die App für Android und iOS Systemen verfügbar ist, da dies die beliebtesten Systeme für Handys sind\cite{pcVsphone_mobileOperatingSystem}. %
		%Wie: Alter von Handys
		Aber auch das alter könnte dafür wichtig sein. Wenn die zu erstellende App eine zu hohe Android oder Apple Version voraussetzt, können ältere Handys diese nicht verwenden.



%hidden-useless nfA
\myComment{
	\item \textbf{Sicherheit}: Die Anforderung Sicherheit steht dafür, dass die in der App verwendeten Daten\footnote{Zum Beispiel die Kalendereinträge} nicht von dritten mitgelesen werden können. \newline%
	Für dieses Projekt gibt es drei verschiedene Standort-Möglichkeiten der zu schützenden Daten und somit auch drei verschiedene Angriffsflächen für einen Dritten die Daten zu stehlen.\newline%
	Erstens die Situation, dass sich die Daten auf dem Handy befinden. Die Sicherheit hierzu wird jedoch vom Betriebsystem sichergestellt. So bietet Appel zum Beispiel für jede App eine Sandbox, welche verhindert, dass andere Apps auf die zu schützenden Daten zugreifen könnten \cite{nfA_sandbox}.\newline%
	Die zweite und dritte Situation wären einmal, dass sich die Daten auf dem Backend befinden oder während der Kommunikation von Endgerät zu Backend. Beide dieser Situationen werden meist Framework und Datenbank geschützt, da aktuelle Software oft standardmäßig bereits starke Verschlüsselung benutzen. \newline%
	Diese Anforderung wird also nicht als niedrigstes eingestuft, weil sie am unwichtigsten betrachten wird, sonder weil wir für dieses Projekt nur wenig Einfluss darauf ausüben können. Es ist lediglich wichtig ein Backend und Framework zu finden, welche Verschlüsselung anbieten.
}

\end{itemize}