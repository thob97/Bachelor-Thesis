\subsection{Vorgehensweise}
Als Entwickler ist einem oft garnicht bewusst was überhaupt gebaut werden soll. Meistens liegt das an der hohen Komplexität der Domäne. Es ist schwer zu durchschauen und herrauszufinden was die Software in diesem Anwendungsgebiet leisten kann und soll. \newline
Daher ist eine Regel in der Softwaretechnik, dass man sich beim Bau von Software nicht auf die intuitiven Eindrücke verlassen soll, stattdessen müssen die Anforderungen Systematisch erhoben werden <Quelle SWT>. \newline
Genau um diese Systematischeerhebung werden wir uns nun Gedanken machen. Für dieses Problem gibt es eine Menge Erhebungstechniken. Da jede Technik Vor- und Nachteile hat und wir möglichst viele nützliche Informationen sammeln wollen, wird sich nicht für eine Technik entschieden. Stattdessen es werden mehrere betrachtet und benutzt. Folgende werden wir dazu betrachten: Introspektion, Umfragen, User Feedback, Iterative development, Inspiration durch Vergleiche.

\begin{itemize}
	\item \textbf{Introspektion}: Bei den Introspektion versucht man selbst durch 'Nachdenken' Anforderungen zu erheben. Das hat den Vorteil, dass es anders als die anderen Erhebungstechniken keinen großen Einstiegtaufwand hat, man es sowieso ständig und immer betreibt und aus eigenen Überlegungen keine Missverständnisse entstehen können. Jedoch muss man sich dafür mit der Domäne auskennen. Wenn man diese nicht versteht, können einen auch keine Ideen einfallen.
	\item \textbf{User Feedback}: Da es eine schwere Aufgabe als einzelne Person alle Anforderungen und Wünsche vieler Nutzer zu erraten und überblicken, sollte man eine passende Technik nutzen. Ihr kommt das User Feedback ins Spiel. Dabei geben einen Nutzer Feedback über die Software. Dabei werden nicht nur existierende Funktionen bewertet, sondern es können auch neue Wünsche und Funktionen geäußert und entdeckt werden. Ein weiterer Vorteil ist, dass man eine der besten Arten von Feedback bekommt, und zwar direkt von der Zielgruppe. \newline Jedoch werden wir diese Technik trotz der Vorteile nicht genutzt, denn um sie auszuführen benötigte es eine lauffähige Software. Diese nach jeder neuen Iteration neu zu kompilieren und bereitzustellen wäre ein größer Aufwand. Außerdem wird davon ausgegangen, dass sich keine Gruppe gefunden werden kann, welche groß genug ist und das auch bis zum ende der Arbeit bleibt. Stattdessen würde jeder Nutzer eher wahrscheinlich das Interesse nach jeder Iteration mehr verlieren.
	\item \textbf{Umfragen}: Von daher wurde sich stattdessen für die Umfrage entschieden. Dabei werden sich einige Fragen ausgedacht und einmalig an die Zielgruppe gestellt. Das hat die Vorteile, dass es leichter ist freiwillige Nutzer für ein einmalige Frage, statt eines dauerhaften Aufgabe, zu finden. Sowie dass man Fragen in beliebige Richtungen stellen kann, anstatt Feedback zu allen Möglichen Themen zu bekommen. Jedoch hat diese Technik, genau wie die Vorherige, den Nachteil, dass das schriftliche Feedback anhand fehlendes Kontextes leicht missverstanden werden. Außerdem könnten einzelne Nachrichten als viel zu wichtig eingestuft werden, wobei sie nicht Repräsentativ für die eigentliche Zielgruppe ist.
	\item \textbf{Inspiration durch Vergleiche}: Es gibt zwar noch keine App wie wir sie Entwickeln wollen und von daher müssen wir uns für die 'revelotionären-Ideen' andere Erhebungs-Techniken überlegen, jedoch wird es in der App trotzdem einige ähnliche Funktionen und Anforderungen zu konventionellen Kalender-Apps geben. Auch wenn man sich vorerst Vorstellt, dass die Anforderungen und Funktionen einer normalen Kalender-App simpel scheinen, so gilt auch hier die zuvor erwähnte Regel. Es könnte zum Beispiel bereits Eigenheiten und etablierte Standards in Kalender-Apps geben, welche man ohne Vergleiche nicht finden würde. Oder es gibt als \"selbstverständlich\" angesehene Funktionen, welche deshalb von niemanden angesprochen aber trotzdem erwartet werden. Von daher macht es Sinn, sich zumindestens für die allgemeine und typischen Funktionen, welche in unserer sowie in Konventionellen Apps auftauchen würden, Inspiration zu suchen. 
	\item \textbf{Domänenwissen}: Durch das einarbeiten in die Domäne CLI-Kalender könnte einen bewusst werden, was für Funktionen und Anforderungen sie besitzen. Dieses wissen könnte wohlmöglich auch einige neue Ideen und Anforderungen für die App inspirieren. Jedoch haben wir uns gegen das einarbeiten in die Domäne entschieden. Die einarbeitungszeit wird zu hoch verglichen mit den Erwerb an Informationen eingeschätzt, da sich die App doch sehr vom CLI-Programm unterscheiden wird. Zudem besitzen viele der CLI-Kalender Eigenheiten, welche leicht als zu aussagekräftig und Repräsentativ für alle Kalender, eingeschätzt werden könnten.  
	\item \textbf{Iterative Develompent}: Das Iterative Arbeiten kann auch als Erhebungstechink bezeichnet werden. Während jeder Iteration bietet sich die Chance die Anforderungen zu überdenken und sein zuvor neu gelerntes darauf anzuwenden. Da wir eine Agile-Arbeitsweise betreiben passiert das also sowieso nebenbei.
\end{itemize}

\subsubsection{Umfrage}\label{subsection:umfrage}
%Was+ Warum
Um möglichst wertvolle und aussagekräftige Ergebnisse aus der Umfrage zu erzielen, wurden die Durchführungsbedingungen und Fragen sorgfältig überdacht. %
%Was+ Warum
Die Wahl des Standorts und die Formulierung der Fragen wurden dabei als entscheidende Faktoren für den Erfolg der Umfrage identifiziert.%
%
\newline%
\myNewSection
\textbf{Standort}: %
%Was: Auswahlmöglichkeiten
Zur Auswahl stehen die folgenden Möglichkeiten: die Durchführung der Umfrage im Bekanntenkreis, an der Universität oder in Online-Foren.\newline%
%Was+Warum: keine Zielgruppe in 1&2 -> entfallen
Es wird vermutet, dass sich weder im Bekanntenkreis noch an der Universität viele Nutzer der Zielgruppe finden lassen.\newline%
%Auswirkung -> OnlineForums
Dementsprechend fällt die Entscheidung auf die Durchführung der Umfrage in Online-Foren. %
%Was: erneut frage stellen
Jedoch ergibt sich nun die Frage, in welchen Foren die Umfrage veröffentlicht werden soll. %
	%Was/Auswirkung: großes Spektrum an Auswahlmöglichkeiten
	Dadurch ergibt sich ein noch viel größeres Spektrum an Auswahlmöglichkeiten. %
	%Was/Warum: nicht zuviel Zeit verschwenden -> Suchanfragen
	Um nicht zu viel Zeit mit der Suche nach einem maßgeschneiderten Forum zu verschwenden, wurden ungefähr die ersten 20 Ergebnisse einer Google-Suche mit dem Suchbegriff \glqq CLI-Calendar Forum\grqq{} betrachtet. %
	%Aufzählung
	Dabei wurden unter anderem folgende Foren vorgeschlagen: \glqq reddit: r/commandline\grqq{}\cite{forum_rCommandLine} , \glqq Stack Exchange: Unix \& Linux\grqq{}\cite{forum_unixAndLinux}, \glqq archlinux: Forums\grqq{}\cite{forum_archlinux}, \glqq Debian User Forums\grqq{}\cite{forum_debianUserForums}, \glqq Linux Mint Forums\grqq{}\cite{forum_linuxMintForums}, \glqq Puppy Linux Discussion Forum\grqq{}\cite{forum_puppyLinux}. %
	%Warum: andere geringe Reichweite
	Viele der betrachteten Foren beschränken sich auf ein einzelnes Betriebssystem und haben daher vermutlich eine geringere Reichweite. %
		%Was: Wahl auf reddit
		Das einzige Forum, das dabei heraussticht, ist Reddit. Dementsprechend fiel auch die Wahl auf dieses Forum.\newline%
%Was+Warum: kleiner Aufwand -> r/Apps veröffentlichen  
Da der zusätzliche Aufwand für das Veröffentlichen der Umfrage auf einem weiteren Reddit-Forum als gering eingeschätzt wird, wird die gleiche Umfrage auch auf r/androidapps\cite{forum_rAndroidapps} und r/iosapps\cite{forum_rIOSapps} veröffentlicht. 
	%Warum: weiteres feedback
	Das Ziel ist es, so auch Einblicke in die Wünsche und Erwartungen der App-Nutzer zu erhalten.%
%
\newline
\myNewSection
\textbf{Fragen}: %
%Was vordefiniert VS offenes Konstrukt
Für die zu stellenden Fragen wurde überlegt, ob sie vordefiniert oder als offenes Konstrukt gestaltet werden sollen. %
%Was: Vordefiniert
	%Vorteil: gezielte Antworten
	Vordefinierte Fragen haben den Vorteil, dass gezielt Antworten auf bestimmte Fragen erhalten werden können. %
	%Nachteil: zu sehr an Fragen orientieren
	Allerdings besteht hierbei die Möglichkeit, dass sich Nutzer zu sehr an den Fragen orientieren und dadurch wichtige und interessante Ideen nicht zum Vorschein kommen. %
	%Nachteil: fällt schwer Fragen zu überlegen
	Zudem wurde sich zu diesem Zeitpunkt noch keine Anforderungen bezüglich der Anwendung überlegt, was die Erstellung von Fragen erschwert. %
%Auswirkung:
Daher wurde sich letztendlich für offene Freitextfragen und -antworten Konstrukt entschieden.%
%
\myNewSection
Die Umfragen können unter folgenden Links abgerufen werden: %
\newline%
\url{https://www.reddit.com/r/iosapps/comments/10k3d2c/developing_an_app_for_clicalendars_opinion_poll/}
\newline%
\url{https://www.reddit.com/r/androidapps/comments/10k3k7w/developing_an_app_for_clicalendars_opinion_poll/}
\newline%
\url{https://www.reddit.com/r/commandline/comments/10k38bc/developing_an_app_for_clicalendars_opinion_poll/}
\subsubsection{Vergleich}\myCheckmark %
Ähnlich wie bei der \nameref{subsection:umfrage} wird sich zu dieser Erhebungstechnik auch [gesondert] Gedanken gemacht. Ziel dadurch soll es sein möglichst Wertvolle und aussagekräftige Ergebnisse zu erzielen und dabei möglichst zeiteffizient vorzugehen. 

\myNewSection
%Was es bringen soll
Durch den Vergleich sollen lediglich Inspiration sowie allgemeine und \glqq offensichtliche\grqq{} Anforderungen gesammelt werden. %
%Was nicht: Abgrenzung
Sie soll nicht dazu verleiten Funktionen und Designs zu kopieren oder sich beeinflussen zu lassen. %
	%Wie dagegen angekommen wird.
	Deshalb werden nur wenige Apps zum Vergleich herangezogen und diese auch nur kurzweilig getestet.%
		%Warum: weiterer Punkt: Zeitaufwand
		%[Außerdem] würde das testen weiterer App auch zu viel Aufwand und Zeit kosten.\newline%
\newline%
%Was: Auswahl von Apps
Bei der Auswahl der Apps wurde versuche diejenigen zu wählen, welche möglichst nützliche Informationen liefern können. %
	%Apple & Google
	Dabei wurden einmal der native Apple iOS Kalender\cite{A_calendarApple} und Google Kalender\cite{A_calendarGoogle} zum Vergleich ausgewählt. Denn es wird vermutet, dass diese Unternehmen durch ihren Erfolg, ihrer Größe und dadurch dass sie eigene Richtlinien für Apps festgelegt haben\cite{konventionen_guidelinesApple, konventionen_guidelinesGoogle}, besonders Achtsam bei der Entwicklung dieser Apps waren.\newline%
	%Calendars
	Des Weiteren wurde eine App nach Kundenbewertungen ausgewählt. Denn wohlmöglich wurde solch eine App deshalb so positiv bewertet, weil sie über Funktionen verfügt, welche bei den anderen beiden nicht vorhanden sind. Die Wahl viel dementsprechend auf Calendars\cite{A_calendarReviews}.%