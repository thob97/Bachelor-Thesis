\subsection{Vorgehensweise}\label{subsection:anforderung:vorgehensweise}
%Warum: Unbewusst was gebaut werden soll
Als Entwickler ist es oft schwierig zu durchschauen und zu verstehen, welche Funktionalitäten die Software in dem Anwendungsbereich erfüllen sollte.\newline%
%Was: Erhebungstechniken Vergleichen
Deshalb wird in diesem Unterabschnitt genau darüber nachgedacht. Es wird überlegt, wie die Anforderungen am besten ermittelt werden können. Dazu werden verschiedene Erhebungstechniken verglichen. %
%
\begin{itemize}
	\item \textbf{Introspektion}: %
		%Was: def
		Für die Introspektion versucht man, durch Nachdenken und Selbstreflexion Anforderungen zu erheben. %
		%Auswirkung: Benutzung
		Da jede Entscheidung in diesem Bereich gut durchdacht sein sollte, wird diese Erhebungstechnik durchgängig und fast immer angewendet. %
		%Vorteil: Missverständnisse + kein vorbereitungs-aufwand
		Diese Methode hat zwei Vorteile: Zum einen können Missverständnisse vermieden werden, da man auf Basis eigener Überlegungen arbeitet. Zum anderen ist kein großer Vorbereitungsaufwand nötig und man kann sofort loslegen. %
		%Nachteil: Domäne auskennen
		Allerdings erfordert die Introspektion ein Mindestmaß an Verständnis der Domäne, da man sonst keine Ideen entwickeln kann.%
	\item \textbf{User Feedback}: %
		%Warum: Schwer alles zu überblicken
		Als einzelne Person ist es schwierig, alle Anforderungen und Wünsche der Nutzer zu erraten und zu überblicken. Daher sind Erhebungstechniken wie das User Feedback nützlich. %
		%Was + Warum: Userfeedback
		Dabei geben Nutzer Feedback über die Software. Damit werden nicht nur existierende Funktionen bewertet, sondern es können auch neue Wünsche und Funktionen geäußert und entdeckt werden. %
		%Was: nicht nutzen
		Allerdings wird diese Technik nicht genutzt, %
			%Nachteil: lauffähige software
			da dafür eine lauffähige Software erforderlich ist und erwartet wird, dass eine solche erst zum Ende der Bearbeitungszeit verfügbar steht.%
				%Trimmed
				%Denn einerseits benötigte es dafür eine lauffähige Software und diese nach jeder neuen Iteration neu zu kompilieren und bereitzustellen wäre ein größer Aufwand. %
				%%Nachteil: Testgruppe finden
				%Außerdem wird vermutet, dass sich das finden einer Testgruppe, welche über mehrere Iterationen die App testet als schwer herausstellen könnte. Wahrscheinlich würde das Interesse nach jeder Iteration mehr schwinden und so verfallen auch die Nutzer.%
	\item \textbf{Umfragen}: %
	%Was: Umfrage 
	Von daher wurde sich stattdessen für eine Umfrage entschieden. Dabei werden einige Fragen formuliert und einmalig an die Zielgruppe gestellt. %
	%Vorteil: Nutzer Findung leichter
	Das hat den Vorteil, dass es vermutlich einfacher ist, freiwillige Nutzer für eine einmalige Frage als für einen dauerhaften Test zu finden. %
	%Vorteil: Fragen\Antworten können gelenkt werden
	Außerdem können gezielte Fragen gestellt werden, um so Feedback zu gewünschten Themen zu erhalten. % 
	%Nachteil: schriftliches Feedback missverstanden %TODO -> in Fazit
	Jedoch hat diese Technik, genau wie die vorherige, den Nachteil, dass schriftliches Feedback aufgrund fehlender Kontextinformationen leicht missverstanden werden kann. %
	%Nachteil: einzelnes Feedback nicht als zu wichtig ansehen %TODO -> in Fazit
	Es muss auch darauf geachtet werden, dass einzelne Antworten nicht als zu wichtig interpretiert werden. Denn obwohl sie für einen Nutzer wichtig sein können, repräsentieren sie möglicherweise nicht die gesamte Zielgruppe.%
	\item \textbf{Inspiration durch Vergleiche}: %
		%Warum: ähnliche Funktionen
		Es existiert zwar noch keine App, wie jene, welche in dieser Arbeit entwickelt werden soll, dennoch ist anzunehmen, dass es in dieser App ähnliche Funktionen und Anforderungen wie in konventionellen Kalender-Apps geben wird. %
		%Was: simpel -> intuitive Eindrücke?
		Auch wenn die Anforderungen und Funktionen einer normalen Kalender-App zuerst einfach erscheinen mögen, sollte man sich nicht nur auf seine intuitiven Eindrücke verlassen. %
			%Warum:
			Es könnten beispielsweise Eigenheiten und etablierte Standards in Kalender-Apps geben, die man ohne Vergleiche nicht finden würde. Oder es gibt als Funktionen, die als selbstverständlich angesehen werden und deshalb von niemandem angesprochen werden, aber dennoch erwartet werden. %
		%Schlussfolgerung: Sinnvoll
		Daher scheint es sinnvoll, sich zumindest für die allgemeinen und typischen Funktionen Inspiration zu suchen.%
	\item \textbf{Domänenwissen}: %
		%Was: def
		Durch das Einarbeiten in die Domäne der CLI-Terminkalender könnte bewusst werden, welche Funktionen und Anforderungen sie besitzen %
		%Warum: neue Ideen
		Dieses Wissen könnte zu Inspiration für neue Ideen und Anforderungen für die App führen führen. %
		%Was: dagegen Entschieden
		Allerdings wurde sich gegen das Einarbeiten in die Domäne entschieden. %
			%Warum: Einarbeitungszeit
			Einerseits wird die Einarbeitungszeit als zu hoch eingeschätzt, da es viele CLI-Terminkalender gibt und diese meist komplex und unterschiedlich zu bedienen sind und zudem verschiedene Eigenheiten bieten. %
			%Warum: geringer Informationserwerb
			Andererseits wird der Erwerb von Informationen als zu gering eingeschätzt. Es wird nämlich vermutet, dass die zu erstellende App sich aufgrund der Unterschiede zwischen PC und Handy sehr von CLI-Terminkalendern unterscheiden wird. Immerhin ist das Ziel nicht, ein solches Programm zu portieren, sondern die Stärken von PC und Handy zu nutzen.%
	\item \textbf{Iterative Develompent}: %
		%Was: def
		Das iterative Arbeiten kann auch als Erhebungstechnik bezeichnet werden, da sich während jeder Iteration die Chance bietet, die Anforderungen zu überdenken und das zuvor neu Gelernte darauf anzuwenden. %
		%Warum: passiert nebenbei 
		Da für diese Arbeit eine agile Arbeitsweise gewählt wurde, wird auch diese Technik angewendet.%
\end{itemize}

\subsubsection{Umfrage}\label{subsection:umfrage}
%Was+ Warum
Um möglichst wertvolle und aussagekräftige Ergebnisse aus der Umfrage zu erzielen, wurden die Durchführungsbedingungen und Fragen sorgfältig überdacht. %
%Was+ Warum
Die Wahl des Standorts und die Formulierung der Fragen wurden dabei als entscheidende Faktoren für den Erfolg der Umfrage identifiziert.%
%
\newline%
\myNewSection
\textbf{Standort}: %
%Was: Auswahlmöglichkeiten
Zur Auswahl stehen die folgenden Möglichkeiten: die Durchführung der Umfrage im Bekanntenkreis, an der Universität oder in Online-Foren.\newline%
%Was+Warum: keine Zielgruppe in 1&2 -> entfallen
Es wird vermutet, dass sich weder im Bekanntenkreis noch an der Universität viele Nutzer der Zielgruppe finden lassen.\newline%
%Auswirkung -> OnlineForums
Dementsprechend fällt die Entscheidung auf die Durchführung der Umfrage in Online-Foren. %
%Was: erneut frage stellen
Jedoch ergibt sich nun die Frage, in welchen Foren die Umfrage veröffentlicht werden soll. %
	%Was/Auswirkung: großes Spektrum an Auswahlmöglichkeiten
	Dadurch ergibt sich ein noch viel größeres Spektrum an Auswahlmöglichkeiten. %
	%Was/Warum: nicht zuviel Zeit verschwenden -> Suchanfragen
	Um nicht zu viel Zeit mit der Suche nach einem maßgeschneiderten Forum zu verschwenden, wurden ungefähr die ersten 20 Ergebnisse einer Google-Suche mit dem Suchbegriff \glqq CLI-Calendar Forum\grqq{} betrachtet. %
	%Aufzählung
	Dabei wurden unter anderem folgende Foren vorgeschlagen: \glqq reddit: r/commandline\grqq{}\cite{forum_rCommandLine} , \glqq Stack Exchange: Unix \& Linux\grqq{}\cite{forum_unixAndLinux}, \glqq archlinux: Forums\grqq{}\cite{forum_archlinux}, \glqq Debian User Forums\grqq{}\cite{forum_debianUserForums}, \glqq Linux Mint Forums\grqq{}\cite{forum_linuxMintForums}, \glqq Puppy Linux Discussion Forum\grqq{}\cite{forum_puppyLinux}. %
	%Warum: andere geringe Reichweite
	Viele der betrachteten Foren beschränken sich auf ein einzelnes Betriebssystem und haben daher vermutlich eine geringere Reichweite. %
		%Was: Wahl auf reddit
		Das einzige Forum, das dabei heraussticht, ist Reddit. Dementsprechend fiel auch die Wahl auf dieses Forum.\newline%
%Was+Warum: kleiner Aufwand -> r/Apps veröffentlichen  
Da der zusätzliche Aufwand für das Veröffentlichen der Umfrage auf einem weiteren Reddit-Forum als gering eingeschätzt wird, wird die gleiche Umfrage auch auf r/androidapps\cite{forum_rAndroidapps} und r/iosapps\cite{forum_rIOSapps} veröffentlicht. 
	%Warum: weiteres feedback
	Das Ziel ist es, so auch Einblicke in die Wünsche und Erwartungen der App-Nutzer zu erhalten.%
%
\newline
\myNewSection
\textbf{Fragen}: %
%Was vordefiniert VS offenes Konstrukt
Für die zu stellenden Fragen wurde überlegt, ob sie vordefiniert oder als offenes Konstrukt gestaltet werden sollen. %
%Was: Vordefiniert
	%Vorteil: gezielte Antworten
	Vordefinierte Fragen haben den Vorteil, dass gezielt Antworten auf bestimmte Fragen erhalten werden können. %
	%Nachteil: zu sehr an Fragen orientieren
	Allerdings besteht hierbei die Möglichkeit, dass sich Nutzer zu sehr an den Fragen orientieren und dadurch wichtige und interessante Ideen nicht zum Vorschein kommen. %
	%Nachteil: fällt schwer Fragen zu überlegen
	Zudem wurde sich zu diesem Zeitpunkt noch keine Anforderungen bezüglich der Anwendung überlegt, was die Erstellung von Fragen erschwert. %
%Auswirkung:
Daher wurde sich letztendlich für offene Freitextfragen und -antworten Konstrukt entschieden.%
%
\myNewSection
Die Umfragen können unter folgenden Links abgerufen werden: %
\newline%
\url{https://www.reddit.com/r/iosapps/comments/10k3d2c/developing_an_app_for_clicalendars_opinion_poll/}
\newline%
\url{https://www.reddit.com/r/androidapps/comments/10k3k7w/developing_an_app_for_clicalendars_opinion_poll/}
\newline%
\url{https://www.reddit.com/r/commandline/comments/10k38bc/developing_an_app_for_clicalendars_opinion_poll/}
\subsubsection{Vergleich}\myCheckmark %
Ähnlich wie bei der \nameref{subsection:umfrage} wird sich zu dieser Erhebungstechnik auch [gesondert] Gedanken gemacht. Ziel dadurch soll es sein möglichst Wertvolle und aussagekräftige Ergebnisse zu erzielen und dabei möglichst zeiteffizient vorzugehen. 

\myNewSection
%Was es bringen soll
Durch den Vergleich sollen lediglich Inspiration sowie allgemeine und \glqq offensichtliche\grqq{} Anforderungen gesammelt werden. %
%Was nicht: Abgrenzung
Sie soll nicht dazu verleiten Funktionen und Designs zu kopieren oder sich beeinflussen zu lassen. %
	%Wie dagegen angekommen wird.
	Deshalb werden nur wenige Apps zum Vergleich herangezogen und diese auch nur kurzweilig getestet.%
		%Warum: weiterer Punkt: Zeitaufwand
		%[Außerdem] würde das testen weiterer App auch zu viel Aufwand und Zeit kosten.\newline%
\newline%
%Was: Auswahl von Apps
Bei der Auswahl der Apps wurde versuche diejenigen zu wählen, welche möglichst nützliche Informationen liefern können. %
	%Apple & Google
	Dabei wurden einmal der native Apple iOS Kalender\cite{A_calendarApple} und Google Kalender\cite{A_calendarGoogle} zum Vergleich ausgewählt. Denn es wird vermutet, dass diese Unternehmen durch ihren Erfolg, ihrer Größe und dadurch dass sie eigene Richtlinien für Apps festgelegt haben\cite{konventionen_guidelinesApple, konventionen_guidelinesGoogle}, besonders Achtsam bei der Entwicklung dieser Apps waren.\newline%
	%Calendars
	Des Weiteren wurde eine App nach Kundenbewertungen ausgewählt. Denn wohlmöglich wurde solch eine App deshalb so positiv bewertet, weil sie über Funktionen verfügt, welche bei den anderen beiden nicht vorhanden sind. Die Wahl viel dementsprechend auf Calendars\cite{A_calendarReviews}.%