\subsection{Vorgehensweise}\label{subsection:anforderung:vorgehensweise}\myCheckmark
%Warum: Unbewusst was gebaut werden soll
Als Entwickler ist einem oft garnicht bewusst was überhaupt gebaut werden soll, da es schwer zu durchschauen und herauszufinden ist was die Software in dem Anwendungsgebieten  leisten soll.\newline%
%Was: Erhebungstechniken Vergleichen
Deshalb wird sich in diesem Unterabschnitt genau dazu Gedanken gemacht. Es wird überlegt wie die Anforderungen für diese Arbeit am besten erhoben werden können. Dazu werden eine Reihe von Erhebungstechniken verglichen. %
%
\begin{itemize}
	\item \textbf{Introspektion}: %
		%Was: def
		Bei der Introspektion wird versucht selbstständig durch das Nachdenken Anforderungen zu erheben. 
		%Auswirkung: Benutzung
		Da jede Entscheidung diese Arbeit [sowieso] gut überdacht sein sollte, wurde diese Erhebungstechnik durchgängig und fast immer benutzt. %
		%Vorteil: Missverständnisse + kein vorbereitungs-aufwand
		Diese Technik hat zwei Vorteile. Erstens können aus eigenen Überlegungen keine Missverständnisse entstehen. Zweitens benötigt es keine großen Vorbereitungs-Aufwand, [da man sofort loslegen kann]. %
		%Nachteil: Domäne auskennen
		Jedoch muss man sich dafür mit der Domäne auskennen. Wenn man diese nicht versteht, können einen auch keine Ideen einfallen.%
	\item \textbf{User Feedback}: %
		%Warum: Schwer alles zu überblicken
		Als einzelne Person ist es eine schwere Aufgabe alle Anforderungen und Wünsche vieler Nutzer zu erraten und überblicken. Daher sind Erhebungstechniken wie das User Feedback nützlich. %
		%Was + Warum: Userfeedback
		Dabei geben einen Nutzer Feedback über die Software. Damit werden nicht nur existierende Funktionen bewertet, sondern es können auch neue Wünsche und Funktionen geäußert und entdeckt werden. %
		%Was: nicht nutzen
		Jedoch wird diese Technik nicht genutzt. %
			%Nachteil: lauffähige software
			Denn dafür benötigt es eine lauffähige Software und es wird erwartet, dass diese erst zum Ende der Bearbeitungszeit bereit steht.%
				%Trimmed
				%Denn einerseits benötigte es dafür eine lauffähige Software und diese nach jeder neuen Iteration neu zu kompilieren und bereitzustellen wäre ein größer Aufwand. %
				%%Nachteil: Testgruppe finden
				%Außerdem wird vermutet, dass sich das finden einer Testgruppe, welche über mehrere Iterationen die App testet als schwer herausstellen könnte. Wahrscheinlich würde das Interesse nach jeder Iteration mehr schwinden und so verfallen auch die Nutzer.%
	\item \textbf{Umfragen}: %
	%Was: Umfrage 
	Von daher wurde sich stattdessen eine Umfrage entschieden. Dabei werden sich einige Fragen ausgedacht und einmalig an die Zielgruppe gestellt. %
	%Vorteil: Nutzer Findung leichter
	Das hat die Vorteile, dass es vermutlich leichter ist freiwillige Nutzer für ein einmalige Frage, statt eines dauerhaften Testens, zu finden. %
	%Vorteil: Fragen\Antworten können gelenkt werden
	Außerdem können Fragen in beliebige Richtungen stellen kann. So bekommt man Feedback zu gewünschten anstatt zu allen möglichen Themen. % 
	%Nachteil: schriftliches Feedback missverstanden %TODO -> in Fazit
	Jedoch hat diese Technik, genau wie die Vorherige, den Nachteil, dass das schriftliche Feedback anhand fehlendes Kontextes leicht missverstanden werden. %
	%Nachteil: einzelnes Feedback nicht als zu wichtig ansehen %TODO -> in Fazit
	Außerdem muss darauf geachtet werden einzelne Nachrichten nicht als zu wichtig einzustufen. Denn sie könnten zwar für einen Nutzer wichtig sein, aber es muss nicht die eigentliche Zielgruppe repräsentieren.%
	\item \textbf{Inspiration durch Vergleiche}: %
		%Warum: ähnliche Funktionen
		Es existiert zwar noch keine App wie jene welche in dieser Arbeit entwickelt werden soll, jedoch wird angenommen, dass es in dieser App trotzdem einige ähnliche Funktionen und Anforderungen zu konventionellen Kalender-Apps geben wird. %
		%Was: simpel -> intuitive Eindrücke?
		Auch wenn die Anforderungen und Funktionen einer normalen Kalender-App zuerst simpel scheinen, so gilt auch hier, dass man sich nicht auf seine intuitiven Eindrücke verlassen sollte. %
			%Warum:
			Es könnte zum Beispiel bereits Eigenheiten und etablierte Standards in Kalender-Apps geben, welche man ohne Vergleiche nicht finden würde. Oder es gibt als \"selbstverständlich\" angesehene Funktionen, welche deshalb von niemanden angesprochen aber trotzdem erwartet werden. %
		%Schlussfolgerung: Sinnvoll
		Von daher scheint es Sinnvoll sich mindestens für die allgemeine und typischen Funktionen Inspiration zu suchen.%
	\item \textbf{Domänenwissen}: %
		%Was: def
		Durch das einarbeiten in die Domäne CLI-Terminkalender kann bewusst werden was für Funktionen und Anforderungen sie besitzen. %
		%Warum: neue Ideen
		Dieses Wissen könnte zu Inspiration von neue Ideen und Anforderungen für die App führen führen. %
		%Was: dagegen Entschieden
		Jedoch wurde sich gegen das Einarbeiten in die Domäne entschieden. %
			%Warum: Einarbeitungszeit
			Einerseits wird die Einarbeitungszeit als zu hoch eingeschätzt. Denn es existieren viele CLI-Terminkalender und diese lassen sich meistens eher komplex und unterschiedlich bedienen und bieten darüberhinaus noch verschiedene Eigenheiten. %
			%Warum: geringer Informationserwerb
			Andererseits wird der Erwerb an Informationen als zu gering eingeschätzt. So wird nämlich durch die Unterschiede des Pc's und des Handys vermutet, dass die zu erstellende App sich sehr von CLI-Terminkalender unterscheiden wird. Immerhin ist das Ziel auch nicht solch ein Programm zu portieren, sondern die Stärken des Handys und Pc's zu nutzen.%
	\item \textbf{Iterative Develompent}: %
		%Was: def
		Das iterative Arbeiten kann auch als Erhebungstechnik bezeichnet werden. Während jeder Iteration bietet sich die Chance die Anforderungen zu überdenken und sein zuvor neu gelerntes darauf anzuwenden. %
		%Warum: passiert nebenbei 
		Da für diese Arbeit eine Agile-Arbeitsweise betrieben wird, wird diese Technik auch [nebensächlich/währenddessen/dabei] angewendet.%
\end{itemize}

\subsubsection{Umfrage}\label{subsection:umfrage}
%Was+ Warum
Um möglichst wertvolle und aussagekräftige Ergebnisse aus der Umfrage zu erzielen, wurden die Durchführungsbedingungen und Fragen sorgfältig überdacht. %
%Was+ Warum
Die Wahl des Standorts und die Formulierung der Fragen wurden dabei als entscheidende Faktoren für den Erfolg der Umfrage identifiziert.%
%
\newline%
\myNewSection
\textbf{Standort}: %
%Was: Auswahlmöglichkeiten
Zur Auswahl stehen die folgenden Möglichkeiten: die Durchführung der Umfrage im Bekanntenkreis, an der Universität oder in Online-Foren.\newline%
%Was+Warum: keine Zielgruppe in 1&2 -> entfallen
Es wird vermutet, dass sich weder im Bekanntenkreis noch an der Universität viele Nutzer der Zielgruppe finden lassen.\newline%
%Auswirkung -> OnlineForums
Dementsprechend fällt die Entscheidung auf die Durchführung der Umfrage in Online-Foren. %
%Was: erneut frage stellen
Jedoch ergibt sich nun die Frage, in welchen Foren die Umfrage veröffentlicht werden soll. %
	%Was/Auswirkung: großes Spektrum an Auswahlmöglichkeiten
	Dadurch ergibt sich ein noch viel größeres Spektrum an Auswahlmöglichkeiten. %
	%Was/Warum: nicht zuviel Zeit verschwenden -> Suchanfragen
	Um nicht zu viel Zeit mit der Suche nach einem maßgeschneiderten Forum zu verschwenden, wurden ungefähr die ersten 20 Ergebnisse einer Google-Suche mit dem Suchbegriff \glqq CLI-Calendar Forum\grqq{} betrachtet. %
	%Aufzählung
	Dabei wurden unter anderem folgende Foren vorgeschlagen: \glqq reddit: r/commandline\grqq{}\cite{forum_rCommandLine} , \glqq Stack Exchange: Unix \& Linux\grqq{}\cite{forum_unixAndLinux}, \glqq archlinux: Forums\grqq{}\cite{forum_archlinux}, \glqq Debian User Forums\grqq{}\cite{forum_debianUserForums}, \glqq Linux Mint Forums\grqq{}\cite{forum_linuxMintForums}, \glqq Puppy Linux Discussion Forum\grqq{}\cite{forum_puppyLinux}. %
	%Warum: andere geringe Reichweite
	Viele der betrachteten Foren beschränken sich auf ein einzelnes Betriebssystem und haben daher vermutlich eine geringere Reichweite. %
		%Was: Wahl auf reddit
		Das einzige Forum, das dabei heraussticht, ist Reddit. Dementsprechend fiel auch die Wahl auf dieses Forum.\newline%
%Was+Warum: kleiner Aufwand -> r/Apps veröffentlichen  
Da der zusätzliche Aufwand für das Veröffentlichen der Umfrage auf einem weiteren Reddit-Forum als gering eingeschätzt wird, wird die gleiche Umfrage auch auf r/androidapps\cite{forum_rAndroidapps} und r/iosapps\cite{forum_rIOSapps} veröffentlicht. 
	%Warum: weiteres feedback
	Das Ziel ist es, so auch Einblicke in die Wünsche und Erwartungen der App-Nutzer zu erhalten.%
%
\newline
\myNewSection
\textbf{Fragen}: %
%Was vordefiniert VS offenes Konstrukt
Für die zu stellenden Fragen wurde überlegt, ob sie vordefiniert oder als offenes Konstrukt gestaltet werden sollen. %
%Was: Vordefiniert
	%Vorteil: gezielte Antworten
	Vordefinierte Fragen haben den Vorteil, dass gezielt Antworten auf bestimmte Fragen erhalten werden können. %
	%Nachteil: zu sehr an Fragen orientieren
	Allerdings besteht hierbei die Möglichkeit, dass sich Nutzer zu sehr an den Fragen orientieren und dadurch wichtige und interessante Ideen nicht zum Vorschein kommen. %
	%Nachteil: fällt schwer Fragen zu überlegen
	Zudem wurde sich zu diesem Zeitpunkt noch keine Anforderungen bezüglich der Anwendung überlegt, was die Erstellung von Fragen erschwert. %
%Auswirkung:
Daher wurde sich letztendlich für offene Freitextfragen und -antworten Konstrukt entschieden.%
%
\myNewSection
Die Umfragen können unter folgenden Links abgerufen werden: %
\newline%
\url{https://www.reddit.com/r/iosapps/comments/10k3d2c/developing_an_app_for_clicalendars_opinion_poll/}
\newline%
\url{https://www.reddit.com/r/androidapps/comments/10k3k7w/developing_an_app_for_clicalendars_opinion_poll/}
\newline%
\url{https://www.reddit.com/r/commandline/comments/10k38bc/developing_an_app_for_clicalendars_opinion_poll/}
\subsubsection{Vergleich}\myCheckmark %
Ähnlich wie bei der \nameref{subsection:umfrage} wird sich zu dieser Erhebungstechnik auch [gesondert] Gedanken gemacht. Ziel dadurch soll es sein möglichst Wertvolle und aussagekräftige Ergebnisse zu erzielen und dabei möglichst zeiteffizient vorzugehen. 

\myNewSection
%Was es bringen soll
Durch den Vergleich sollen lediglich Inspiration sowie allgemeine und \glqq offensichtliche\grqq{} Anforderungen gesammelt werden. %
%Was nicht: Abgrenzung
Sie soll nicht dazu verleiten Funktionen und Designs zu kopieren oder sich beeinflussen zu lassen. %
	%Wie dagegen angekommen wird.
	Deshalb werden nur wenige Apps zum Vergleich herangezogen und diese auch nur kurzweilig getestet.%
		%Warum: weiterer Punkt: Zeitaufwand
		%[Außerdem] würde das testen weiterer App auch zu viel Aufwand und Zeit kosten.\newline%
\newline%
%Was: Auswahl von Apps
Bei der Auswahl der Apps wurde versuche diejenigen zu wählen, welche möglichst nützliche Informationen liefern können. %
	%Apple & Google
	Dabei wurden einmal der native Apple iOS Kalender\cite{A_calendarApple} und Google Kalender\cite{A_calendarGoogle} zum Vergleich ausgewählt. Denn es wird vermutet, dass diese Unternehmen durch ihren Erfolg, ihrer Größe und dadurch dass sie eigene Richtlinien für Apps festgelegt haben\cite{konventionen_guidelinesApple, konventionen_guidelinesGoogle}, besonders Achtsam bei der Entwicklung dieser Apps waren.\newline%
	%Calendars
	Des Weiteren wurde eine App nach Kundenbewertungen ausgewählt. Denn wohlmöglich wurde solch eine App deshalb so positiv bewertet, weil sie über Funktionen verfügt, welche bei den anderen beiden nicht vorhanden sind. Die Wahl viel dementsprechend auf Calendars\cite{A_calendarReviews}.%