\subsection{Vorgehensweise}\label{subsection:anforderung:vorgehensweise}
%Warum: Unbewusst was gebaut werden soll
Als Entwickler ist es oft schwierig zu durchschauen und zu verstehen, welche Funktionalitäten die Software in dem Anwendungsbereich erfüllen sollte.\newline%
%Was: Erhebungstechniken Vergleichen
Deshalb wird in diesem Unterabschnitt genau darüber nachgedacht. Es wird überlegt, wie die Anforderungen am besten ermittelt werden können. Dazu werden verschiedene Erhebungstechniken verglichen. %
%
\begin{itemize}
	\item \textbf{Introspektion}: %
		%Was: def
		Für die Introspektion versucht man, durch Nachdenken und Selbstreflexion Anforderungen zu erheben. %
		%Auswirkung: Benutzung
		Da jede Entscheidung in diesem Bereich gut durchdacht sein sollte, wird diese Erhebungstechnik durchgängig und fast immer angewendet. %
		%Vorteil: Missverständnisse + kein vorbereitungs-aufwand
		Diese Methode hat zwei Vorteile: Zum einen können Missverständnisse vermieden werden, da man auf Basis eigener Überlegungen arbeitet. Zum anderen ist kein großer Vorbereitungsaufwand nötig und man kann sofort loslegen. %
		%Nachteil: Domäne auskennen
		Allerdings erfordert die Introspektion ein Mindestmaß an Verständnis der Domäne, da man sonst keine Ideen entwickeln kann.%
	\item \textbf{User Feedback}: %
		%Warum: Schwer alles zu überblicken
		Als einzelne Person ist es schwierig, alle Anforderungen und Wünsche der Nutzer zu erraten und zu überblicken. Daher sind Erhebungstechniken wie das User Feedback nützlich. %
		%Was + Warum: Userfeedback
		Dabei geben Nutzer Feedback über die Software. Damit werden nicht nur existierende Funktionen bewertet, sondern es können auch neue Wünsche und Funktionen geäußert und entdeckt werden. %
		%Was: nicht nutzen
		Allerdings wird diese Technik nicht genutzt, %
			%Nachteil: lauffähige software
			da dafür eine lauffähige Software erforderlich ist und erwartet wird, dass eine solche erst zum Ende der Bearbeitungszeit verfügbar steht.%
				%Trimmed
				%Denn einerseits benötigte es dafür eine lauffähige Software und diese nach jeder neuen Iteration neu zu kompilieren und bereitzustellen wäre ein größer Aufwand. %
				%%Nachteil: Testgruppe finden
				%Außerdem wird vermutet, dass sich das finden einer Testgruppe, welche über mehrere Iterationen die App testet als schwer herausstellen könnte. Wahrscheinlich würde das Interesse nach jeder Iteration mehr schwinden und so verfallen auch die Nutzer.%
	\item \textbf{Umfragen}: %
	%Was: Umfrage 
	Von daher wurde sich stattdessen für eine Umfrage entschieden. Dabei werden einige Fragen formuliert und einmalig an die Zielgruppe gestellt. %
	%Vorteil: Nutzer Findung leichter
	Das hat den Vorteil, dass es vermutlich einfacher ist, freiwillige Nutzer für eine einmalige Frage als für einen dauerhaften Test zu finden. %
	%Vorteil: Fragen\Antworten können gelenkt werden
	Außerdem können gezielte Fragen gestellt werden, um so Feedback zu gewünschten Themen zu erhalten. % 
	%Nachteil: schriftliches Feedback missverstanden %TODO -> in Fazit
	Jedoch hat diese Technik, genau wie die vorherige, den Nachteil, dass schriftliches Feedback aufgrund fehlender Kontextinformationen leicht missverstanden werden kann. %
	%Nachteil: einzelnes Feedback nicht als zu wichtig ansehen %TODO -> in Fazit
	Es muss auch darauf geachtet werden, dass einzelne Antworten nicht als zu wichtig interpretiert werden. Denn obwohl sie für einen Nutzer wichtig sein können, repräsentieren sie möglicherweise nicht die gesamte Zielgruppe.%
	\item \textbf{Inspiration durch Vergleiche}: %
		%Warum: ähnliche Funktionen
		Es existiert zwar noch keine App, wie jene, welche in dieser Arbeit entwickelt werden soll, dennoch ist anzunehmen, dass es in dieser App ähnliche Funktionen und Anforderungen wie in konventionellen Kalender-Apps geben wird. %
		%Was: simpel -> intuitive Eindrücke?
		Auch wenn die Anforderungen und Funktionen einer normalen Kalender-App zuerst einfach erscheinen mögen, sollte man sich nicht nur auf seine intuitiven Eindrücke verlassen. %
			%Warum:
			Es könnten beispielsweise Eigenheiten und etablierte Standards in Kalender-Apps geben, die man ohne Vergleiche nicht finden würde. Oder es gibt als Funktionen, die als selbstverständlich angesehen werden und deshalb von niemandem angesprochen werden, aber dennoch erwartet werden. %
		%Schlussfolgerung: Sinnvoll
		Daher scheint es sinnvoll, sich zumindest für die allgemeinen und typischen Funktionen Inspiration zu suchen.%
	\item \textbf{Domänenwissen}: %
		%Was: def
		Durch das Einarbeiten in die Domäne der CLI-Terminkalender könnte bewusst werden, welche Funktionen und Anforderungen sie besitzen %
		%Warum: neue Ideen
		Dieses Wissen könnte zu Inspiration für neue Ideen und Anforderungen für die App führen führen. %
		%Was: dagegen Entschieden
		Allerdings wurde sich gegen das Einarbeiten in die Domäne entschieden. %
			%Warum: Einarbeitungszeit
			Einerseits wird die Einarbeitungszeit als zu hoch eingeschätzt, da es viele CLI-Terminkalender gibt und diese meist komplex und unterschiedlich zu bedienen sind und zudem verschiedene Eigenheiten bieten. %
			%Warum: geringer Informationserwerb
			Andererseits wird der Erwerb von Informationen als zu gering eingeschätzt. Es wird nämlich vermutet, dass die zu erstellende App sich aufgrund der Unterschiede zwischen PC und Handy sehr von CLI-Terminkalendern unterscheiden wird. Immerhin ist das Ziel nicht, ein solches Programm zu portieren, sondern die Stärken von PC und Handy zu nutzen.%
	\item \textbf{Iterative Develompent}: %
		%Was: def
		Das iterative Arbeiten kann auch als Erhebungstechnik bezeichnet werden, da sich während jeder Iteration die Chance bietet, die Anforderungen zu überdenken und das zuvor neu Gelernte darauf anzuwenden. %
		%Warum: passiert nebenbei 
		Da für diese Arbeit eine agile Arbeitsweise gewählt wurde, wird auch diese Technik angewendet.%
\end{itemize}

\subsubsection{Umfrage}
Um möglichst Wertvolle und Aussagekräftige Ergebnisse aus der Umfrage zu erzielen haben wir folgende zwei Themen betrachtet: der Standort der Durchführung und die zu stellenden Fragen.

\myNewSection
\textbf{Standort}: Zur Auswahl stehen uns hier die folgenden Möglichkeiten: Die Umfrage in Bekanntenkreis, der Universität, Online-Forums. \newline
Da es sehr wichtig ist, dass wir eine möglichst große Reichweite auf die Zielgruppe haben, fallen die ersten beiden Auswahlmöglichkeiten weg. \newline
Also bleiben uns nur noch die Online-Forums. Auch hier muss sich jedoch erneut die Frage über den Standort gestellt werden: 'In welchen Foren soll die Umfrage veröffentlicht werden'. Dadurch resultiert ein noch viel größeres Spektrum an Auswahlmöglichkeiten. Um dadurch nicht zuviel Zeit in die Suche eines maßgeschneiderten Forums zu verschwenden, wurden ungefähr die ersten 20-Google-Ergebnisse, der Suche 'CLI-Calendar Forum', betrachtet. Dabei erhielten wir folgende Forum Vorschläge: reddit: r/commandline\cite{rCommandLine} , Stack Exchange: Unix \& Linux\cite{unixAndLinux}, archlinux: Forums\cite{archlinux}, Debian User Forums\cite{debianUserForums}, Linux Mint Forums\cite{linuxMintForums}, Puppy Linux Discussion Forum\cite{puppyLinux}. Da sich viele dieser Foren auf ein einzelnes Betriebssystem beschränken schätzen wir die Reichweite als eher gering ein. Das einzige Forum dabei was hinaussticht und allgemeiner angesiegelt ist, ist reddit. \newline
Um die Zielgruppe nicht nur bei CLI Nutzern zu belassen, sondern auch App-liebhaber anzusprechen, wird die gleiche Umfrage auch auf r/androidapps\cite{rAndroidapps} und r/iosapps\cite{rIOSapps} veröffentlicht.

\myNewSection
\textbf{Fragen}: Bei den zu stellenden Fragen wurde überlegt ob sie vordefiniert werden oder ein Offenes Konstrukt genutzt wird. Vordefinierte Fragen haben zwar den Vorteil, dass man gezielte Antworten bekommt, jedoch besteht hier die Möglichkeit, dass sich die Nutzer zu sehr an die Fragen orientieren und dadurch beeinflusst werden und wichtige und interessante Ideen nicht zum Vorschein kommen. Außerdem wurden sich in diesem Punkt der Arbeit noch keine Anforderungen überlegt, was die Erstellung von Fragen erschwierigt und zeitlich kostspielig macht. Daher haben wir uns für Frei-Text Fragen/Antworten entschieden.

\myNewSection
Die Umfragen wurden auf folgenden Seiten veröffentlicht: \myTodo
\subsubsection{Vergleich} %
Ähnlich wie bei der \nameref{subsection:umfrage} werden auch für diese Erhebungstechnik weitere Überlegungen angestellt. Das Ziel dabei ist es, möglichst wertvolle und aussagekräftige Ergebnisse zu erzielen und dabei zugleich zeiteffizient vorzugehen..%
%
%
\newline%
\myNewSection%
%Was es bringen soll
Durch den Vergleich sollen lediglich Inspiration sowie allgemeine und offensichtliche Anforderungen gesammelt werden. %
%Was nicht: Abgrenzung
Das Ziel ist es jedoch nicht, Funktionen und Designs von anderen Apps zu kopieren oder sich von ihnen beeinflussen zu lassen. %
	%Wie dagegen angekommen wird.
	Aus diesem Grund werden nur wenige Apps zum Vergleich herangezogen und diese auch nur kurzzeitig getestet.%
		%Warum: weiterer Punkt: Zeitaufwand
		%[Außerdem] würde das testen weiterer App auch zu viel Aufwand und Zeit kosten.\newline%
\newline%
%Was: Auswahl von Apps
Bei der Auswahl der Apps wurde versucht, solche auszuwählen, die möglichst nützliche Informationen liefern können. %
	%Apple & Google
	Dazu wurden der native Apple iOS Kalender\cite{A_calendarApple} und der Google Kalender\cite{A_calendarGoogle} ausgewählt, da angenommen wird, dass diese Unternehmen aufgrund ihres Erfolgs, ihrer Größe und da sie eigenen Richtlinien für Apps haben \cite{konventionen_guidelinesApple, konventionen_guidelinesGoogle}, besonders sorgfältig bei der Entwicklung dieser Apps vorgegangen sind.\newline%
	%Calendars
	Darüber hinaus wurde eine App aufgrund von Kundenbewertungen ausgewählt. Bei einer solchen App wäre es nämlich beispielsweise möglich, dass sie aufgrund von Funktionen, die in den anderen beiden Apps nicht vorhanden sind, so positiv bewertet wurde. Aus diesem Grund wurde sich für Calendars\cite{A_calendarReviews} entschieden.%