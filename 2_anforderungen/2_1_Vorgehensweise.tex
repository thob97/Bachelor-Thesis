\subsection{Vorgehensweise}\label{subsection:anforderung:vorgehensweise}\myCheckmark
%Warum: Unbewusst was gebaut werden soll
Als Entwickler ist einem oft garnicht bewusst was überhaupt gebaut werden soll, da es schwer zu durchschauen und herauszufinden ist was die Software in dem Anwendungsgebieten  leisten soll.\newline%
%Was: Erhebungstechniken Vergleichen
Deshalb wird sich in diesem Unterabschnitt genau dazu Gedanken gemacht. Es wird überlegt wie die Anforderungen für diese Arbeit am besten erhoben werden können. Dazu werden eine Reihe von Erhebungstechniken verglichen. %
%
\begin{itemize}
	\item \textbf{Introspektion}: %
		%Was: def
		Bei der Introspektion wird versucht selbstständig durch das Nachdenken Anforderungen zu erheben. 
		%Auswirkung: Benutzung
		Da jede Entscheidung diese Arbeit [sowieso] gut überdacht sein sollte, wurde diese Erhebungstechnik durchgängig und fast immer benutzt. %
		%Vorteil: Missverständnisse + kein vorbereitungs-aufwand
		Diese Technik hat zwei Vorteile. Erstens können aus eigenen Überlegungen keine Missverständnisse entstehen. Zweitens benötigt es keine großen Vorbereitungs-Aufwand, [da man sofort loslegen kann]. %
		%Nachteil: Domäne auskennen
		Jedoch muss man sich dafür mit der Domäne auskennen. Wenn man diese nicht versteht, können einen auch keine Ideen einfallen.%
	\item \textbf{User Feedback}: %
		%Warum: Schwer alles zu überblicken
		Als einzelne Person ist es eine schwere Aufgabe alle Anforderungen und Wünsche vieler Nutzer zu erraten und überblicken. Daher sind Erhebungstechniken wie das User Feedback nützlich. %
		%Was + Warum: Userfeedback
		Dabei geben einen Nutzer Feedback über die Software. Damit werden nicht nur existierende Funktionen bewertet, sondern es können auch neue Wünsche und Funktionen geäußert und entdeckt werden. %
		%Was: nicht nutzen
		Jedoch wird diese Technik nicht genutzt. %
			%Nachteil: lauffähige software
			Denn dafür benötigt es eine lauffähige Software und es wird erwartet, dass diese erst zum Ende der Bearbeitungszeit bereit steht.%
				%Trimmed
				%Denn einerseits benötigte es dafür eine lauffähige Software und diese nach jeder neuen Iteration neu zu kompilieren und bereitzustellen wäre ein größer Aufwand. %
				%%Nachteil: Testgruppe finden
				%Außerdem wird vermutet, dass sich das finden einer Testgruppe, welche über mehrere Iterationen die App testet als schwer herausstellen könnte. Wahrscheinlich würde das Interesse nach jeder Iteration mehr schwinden und so verfallen auch die Nutzer.%
	\item \textbf{Umfragen}: %
	%Was: Umfrage 
	Von daher wurde sich stattdessen eine Umfrage entschieden. Dabei werden sich einige Fragen ausgedacht und einmalig an die Zielgruppe gestellt. %
	%Vorteil: Nutzer Findung leichter
	Das hat die Vorteile, dass es vermutlich leichter ist freiwillige Nutzer für ein einmalige Frage, statt eines dauerhaften Testens, zu finden. %
	%Vorteil: Fragen\Antworten können gelenkt werden
	Außerdem können Fragen in beliebige Richtungen stellen kann. So bekommt man Feedback zu gewünschten anstatt zu allen möglichen Themen. % 
	%Nachteil: schriftliches Feedback missverstanden %TODO -> in Fazit
	Jedoch hat diese Technik, genau wie die Vorherige, den Nachteil, dass das schriftliche Feedback anhand fehlendes Kontextes leicht missverstanden werden. %
	%Nachteil: einzelnes Feedback nicht als zu wichtig ansehen %TODO -> in Fazit
	Außerdem muss darauf geachtet werden einzelne Nachrichten nicht als zu wichtig einzustufen. Denn sie könnten zwar für einen Nutzer wichtig sein, aber es muss nicht die eigentliche Zielgruppe repräsentieren.%
	\item \textbf{Inspiration durch Vergleiche}: %
		%Warum: ähnliche Funktionen
		Es existiert zwar noch keine App wie jene welche in dieser Arbeit entwickelt werden soll, jedoch wird angenommen, dass es in dieser App trotzdem einige ähnliche Funktionen und Anforderungen zu konventionellen Kalender-Apps geben wird. %
		%Was: simpel -> intuitive Eindrücke?
		Auch wenn die Anforderungen und Funktionen einer normalen Kalender-App zuerst simpel scheinen, so gilt auch hier, dass man sich nicht auf seine intuitiven Eindrücke verlassen sollte. %
			%Warum:
			Es könnte zum Beispiel bereits Eigenheiten und etablierte Standards in Kalender-Apps geben, welche man ohne Vergleiche nicht finden würde. Oder es gibt als \"selbstverständlich\" angesehene Funktionen, welche deshalb von niemanden angesprochen aber trotzdem erwartet werden. %
		%Schlussfolgerung: Sinnvoll
		Von daher scheint es Sinnvoll sich mindestens für die allgemeine und typischen Funktionen Inspiration zu suchen.%
	\item \textbf{Domänenwissen}: %
		%Was: def
		Durch das einarbeiten in die Domäne CLI-Terminkalender kann bewusst werden was für Funktionen und Anforderungen sie besitzen. %
		%Warum: neue Ideen
		Dieses Wissen könnte zu Inspiration von neue Ideen und Anforderungen für die App führen führen. %
		%Was: dagegen Entschieden
		Jedoch wurde sich gegen das Einarbeiten in die Domäne entschieden. %
			%Warum: Einarbeitungszeit
			Einerseits wird die Einarbeitungszeit als zu hoch eingeschätzt. Denn es existieren viele CLI-Terminkalender und diese lassen sich meistens eher komplex und unterschiedlich bedienen und bieten darüberhinaus noch verschiedene Eigenheiten. %
			%Warum: geringer Informationserwerb
			Andererseits wird der Erwerb an Informationen als zu gering eingeschätzt. So wird nämlich durch die Unterschiede des Pc's und des Handys vermutet, dass die zu erstellende App sich sehr von CLI-Terminkalender unterscheiden wird. Immerhin ist das Ziel auch nicht solch ein Programm zu portieren, sondern die Stärken des Handys und Pc's zu nutzen.%
	\item \textbf{Iterative Develompent}: %
		%Was: def
		Das iterative Arbeiten kann auch als Erhebungstechnik bezeichnet werden. Während jeder Iteration bietet sich die Chance die Anforderungen zu überdenken und sein zuvor neu gelerntes darauf anzuwenden. %
		%Warum: passiert nebenbei 
		Da für diese Arbeit eine Agile-Arbeitsweise betrieben wird, wird diese Technik auch [nebensächlich/währenddessen/dabei] angewendet.%
\end{itemize}

\subsubsection{Umfrage}
Um möglichst Wertvolle und Aussagekräftige Ergebnisse aus der Umfrage zu erzielen haben wir folgende zwei Themen betrachtet: der Standort der Durchführung und die zu stellenden Fragen.

\myNewSection
\textbf{Standort}: Zur Auswahl stehen uns hier die folgenden Möglichkeiten: Die Umfrage in Bekanntenkreis, der Universität, Online-Forums. \newline
Da es sehr wichtig ist, dass wir eine möglichst große Reichweite auf die Zielgruppe haben, fallen die ersten beiden Auswahlmöglichkeiten weg. \newline
Also bleiben uns nur noch die Online-Forums. Auch hier muss sich jedoch erneut die Frage über den Standort gestellt werden: 'In welchen Foren soll die Umfrage veröffentlicht werden'. Dadurch resultiert ein noch viel größeres Spektrum an Auswahlmöglichkeiten. Um dadurch nicht zuviel Zeit in die Suche eines maßgeschneiderten Forums zu verschwenden, wurden ungefähr die ersten 20-Google-Ergebnisse, der Suche 'CLI-Calendar Forum', betrachtet. Dabei erhielten wir folgende Forum Vorschläge: reddit: r/commandline\cite{rCommandLine} , Stack Exchange: Unix \& Linux\cite{unixAndLinux}, archlinux: Forums\cite{archlinux}, Debian User Forums\cite{debianUserForums}, Linux Mint Forums\cite{linuxMintForums}, Puppy Linux Discussion Forum\cite{puppyLinux}. Da sich viele dieser Foren auf ein einzelnes Betriebssystem beschränken schätzen wir die Reichweite als eher gering ein. Das einzige Forum dabei was hinaussticht und allgemeiner angesiegelt ist, ist reddit. \newline
Um die Zielgruppe nicht nur bei CLI Nutzern zu belassen, sondern auch App-liebhaber anzusprechen, wird die gleiche Umfrage auch auf r/androidapps\cite{rAndroidapps} und r/iosapps\cite{rIOSapps} veröffentlicht.

\myNewSection
\textbf{Fragen}: Bei den zu stellenden Fragen wurde überlegt ob sie vordefiniert werden oder ein Offenes Konstrukt genutzt wird. Vordefinierte Fragen haben zwar den Vorteil, dass man gezielte Antworten bekommt, jedoch besteht hier die Möglichkeit, dass sich die Nutzer zu sehr an die Fragen orientieren und dadurch beeinflusst werden und wichtige und interessante Ideen nicht zum Vorschein kommen. Außerdem wurden sich in diesem Punkt der Arbeit noch keine Anforderungen überlegt, was die Erstellung von Fragen erschwierigt und zeitlich kostspielig macht. Daher haben wir uns für Frei-Text Fragen/Antworten entschieden.

\myNewSection
Die Umfragen wurden auf folgenden Seiten veröffentlicht: \myTodo
\subsubsection{Vergleich} %
Ähnlich wie bei der \nameref{subsection:umfrage} werden auch für diese Erhebungstechnik weitere Überlegungen angestellt. Das Ziel dabei ist es, möglichst wertvolle und aussagekräftige Ergebnisse zu erzielen und dabei zugleich zeiteffizient vorzugehen..%
%
%
\newline%
\myNewSection%
%Was es bringen soll
Durch den Vergleich sollen lediglich Inspiration sowie allgemeine und offensichtliche Anforderungen gesammelt werden. %
%Was nicht: Abgrenzung
Das Ziel ist es jedoch nicht, Funktionen und Designs von anderen Apps zu kopieren oder sich von ihnen beeinflussen zu lassen. %
	%Wie dagegen angekommen wird.
	Aus diesem Grund werden nur wenige Apps zum Vergleich herangezogen und diese auch nur kurzzeitig getestet.%
		%Warum: weiterer Punkt: Zeitaufwand
		%[Außerdem] würde das testen weiterer App auch zu viel Aufwand und Zeit kosten.\newline%
\newline%
%Was: Auswahl von Apps
Bei der Auswahl der Apps wurde versucht, solche auszuwählen, die möglichst nützliche Informationen liefern können. %
	%Apple & Google
	Dazu wurden der native Apple iOS Kalender\cite{A_calendarApple} und der Google Kalender\cite{A_calendarGoogle} ausgewählt, da angenommen wird, dass diese Unternehmen aufgrund ihres Erfolgs, ihrer Größe und da sie eigenen Richtlinien für Apps haben \cite{konventionen_guidelinesApple, konventionen_guidelinesGoogle}, besonders sorgfältig bei der Entwicklung dieser Apps vorgegangen sind.\newline%
	%Calendars
	Darüber hinaus wurde eine App aufgrund von Kundenbewertungen ausgewählt. Bei einer solchen App wäre es nämlich beispielsweise möglich, dass sie aufgrund von Funktionen, die in den anderen beiden Apps nicht vorhanden sind, so positiv bewertet wurde. Aus diesem Grund wurde sich für Calendars\cite{A_calendarReviews} entschieden.%