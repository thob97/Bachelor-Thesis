\subsection{Vorgehensweise}
Als Entwickler ist einem oft garnicht bewusst was überhaupt gebaut werden soll. Meistens liegt das an der hohen Komplexität der Domäne. Es ist schwer zu durchschauen und herrauszufinden was die Software in diesem Anwendungsgebiet leisten kann und soll. \newline
Daher ist eine Regel in der Softwaretechnik, dass man sich beim Bau von Software nicht auf die intuitiven Eindrücke verlassen soll, stattdessen müssen die Anforderungen Systematisch erhoben werden <Quelle SWT>. \newline
Genau um diese Systematischeerhebung werden wir uns nun Gedanken machen. Für dieses Problem gibt es eine Menge Erhebungstechniken. Da jede Technik Vor- und Nachteile hat und wir möglichst viele nützliche Informationen sammeln wollen, wird sich nicht für eine Technik entschieden. Stattdessen es werden mehrere betrachtet und benutzt. Folgende werden wir dazu betrachten: Introspektion, Umfragen, User Feedback, Iterative development, Inspiration durch Vergleiche.

\begin{itemize}
	\item \textbf{Introspektion}: Bei den Introspektion versucht man selbst durch 'Nachdenken' Anforderungen zu erheben. Das hat den Vorteil, dass es anders als die anderen Erhebungstechniken keinen großen Einstiegtaufwand hat, man es sowieso ständig und immer betreibt und aus eigenen Überlegungen keine Missverständnisse entstehen können. Jedoch muss man sich dafür mit der Domäne auskennen. Wenn man diese nicht versteht, können einen auch keine Ideen einfallen.
	\item \textbf{User Feedback}: Da es eine schwere Aufgabe als einzelne Person alle Anforderungen und Wünsche vieler Nutzer zu erraten und überblicken, sollte man eine passende Technik nutzen. Ihr kommt das User Feedback ins Spiel. Dabei geben einen Nutzer Feedback über die Software. Dabei werden nicht nur existierende Funktionen bewertet, sondern es können auch neue Wünsche und Funktionen geäußert und entdeckt werden. Ein weiterer Vorteil ist, dass man eine der besten Arten von Feedback bekommt, und zwar direkt von der Zielgruppe. \newline Jedoch werden wir diese Technik trotz der Vorteile nicht genutzt, denn um sie auszuführen benötigte es eine lauffähige Software. Diese nach jeder neuen Iteration neu zu kompilieren und bereitzustellen wäre ein größer Aufwand. Außerdem wird davon ausgegangen, dass sich keine Gruppe gefunden werden kann, welche groß genug ist und das auch bis zum ende der Arbeit bleibt. Stattdessen würde jeder Nutzer eher wahrscheinlich das Interesse nach jeder Iteration mehr verlieren.
	\item \textbf{Umfragen}: Von daher wurde sich stattdessen für die Umfrage entschieden. Dabei werden sich einige Fragen ausgedacht und einmalig an die Zielgruppe gestellt. Das hat die Vorteile, dass es leichter ist freiwillige Nutzer für ein einmalige Frage, statt eines dauerhaften Aufgabe, zu finden. Sowie dass man Fragen in beliebige Richtungen stellen kann, anstatt Feedback zu allen Möglichen Themen zu bekommen. Jedoch hat diese Technik, genau wie die Vorherige, den Nachteil, dass das schriftliche Feedback anhand fehlendes Kontextes leicht missverstanden werden. Außerdem könnten einzelne Nachrichten als viel zu wichtig eingestuft werden, wobei sie nicht Repräsentativ für die eigentliche Zielgruppe ist.
	\item \textbf{Inspiration durch Vergleiche}: Es gibt zwar noch keine App wie wir sie Entwickeln wollen und von daher müssen wir uns für die 'revelotionären-Ideen' andere Erhebungs-Techniken überlegen, jedoch wird es in der App trotzdem einige ähnliche Funktionen und Anforderungen zu konventionellen Kalender-Apps geben. Auch wenn man sich vorerst Vorstellt, dass die Anforderungen und Funktionen einer normalen Kalender-App simpel scheinen, so gilt auch hier die zuvor erwähnte Regel. Es könnte zum Beispiel bereits Eigenheiten und etablierte Standards in Kalender-Apps geben, welche man ohne Vergleiche nicht finden würde. Oder es gibt als \"selbstverständlich\" angesehene Funktionen, welche deshalb von niemanden angesprochen aber trotzdem erwartet werden. Von daher macht es Sinn, sich zumindestens für die allgemeine und typischen Funktionen, welche in unserer sowie in Konventionellen Apps auftauchen würden, Inspiration zu suchen. 
	\item \textbf{Domänenwissen}: Durch das einarbeiten in die Domäne CLI-Kalender könnte einen bewusst werden, was für Funktionen und Anforderungen sie besitzen. Dieses wissen könnte wohlmöglich auch einige neue Ideen und Anforderungen für die App inspirieren. Jedoch haben wir uns gegen das einarbeiten in die Domäne entschieden. Die einarbeitungszeit wird zu hoch verglichen mit den Erwerb an Informationen eingeschätzt, da sich die App doch sehr vom CLI-Programm unterscheiden wird. Zudem besitzen viele der CLI-Kalender Eigenheiten, welche leicht als zu aussagekräftig und Repräsentativ für alle Kalender, eingeschätzt werden könnten.  
	\item \textbf{Iterative Develompent}: Das Iterative Arbeiten kann auch als Erhebungstechink bezeichnet werden. Während jeder Iteration bietet sich die Chance die Anforderungen zu überdenken und sein zuvor neu gelerntes darauf anzuwenden. Da wir eine Agile-Arbeitsweise betreiben passiert das also sowieso nebenbei.
\end{itemize}

\subsubsection{Umfrage}
Um möglichst Wertvolle und Aussagekräftige Ergebnisse aus der Umfrage zu erzielen haben wir folgende zwei Themen betrachtet: der Standort der Durchführung und die zu stellenden Fragen.

\myNewSection
\textbf{Standort}: Zur Auswahl stehen uns hier die folgenden Möglichkeiten: Die Umfrage in Bekanntenkreis, der Universität, Online-Forums. \newline
Da es sehr wichtig ist, dass wir eine möglichst große Reichweite auf die Zielgruppe haben, fallen die ersten beiden Auswahlmöglichkeiten weg. \newline
Also bleiben uns nur noch die Online-Forums. Auch hier muss sich jedoch erneut die Frage über den Standort gestellt werden: 'In welchen Foren soll die Umfrage veröffentlicht werden'. Dadurch resultiert ein noch viel größeres Spektrum an Auswahlmöglichkeiten. Um dadurch nicht zuviel Zeit in die Suche eines maßgeschneiderten Forums zu verschwenden, wurden ungefähr die ersten 20-Google-Ergebnisse, der Suche 'CLI-Calendar Forum', betrachtet. Dabei erhielten wir folgende Forum Vorschläge: reddit: r/commandline\cite{rCommandLine} , Stack Exchange: Unix \& Linux\cite{unixAndLinux}, archlinux: Forums\cite{archlinux}, Debian User Forums\cite{debianUserForums}, Linux Mint Forums\cite{linuxMintForums}, Puppy Linux Discussion Forum\cite{puppyLinux}. Da sich viele dieser Foren auf ein einzelnes Betriebssystem beschränken schätzen wir die Reichweite als eher gering ein. Das einzige Forum dabei was hinaussticht und allgemeiner angesiegelt ist, ist reddit. \newline
Um die Zielgruppe nicht nur bei CLI Nutzern zu belassen, sondern auch App-liebhaber anzusprechen, wird die gleiche Umfrage auch auf r/androidapps\cite{rAndroidapps} und r/iosapps\cite{rIOSapps} veröffentlicht.

\myNewSection
\textbf{Fragen}: Bei den zu stellenden Fragen wurde überlegt ob sie vordefiniert werden oder ein Offenes Konstrukt genutzt wird. Vordefinierte Fragen haben zwar den Vorteil, dass man gezielte Antworten bekommt, jedoch besteht hier die Möglichkeit, dass sich die Nutzer zu sehr an die Fragen orientieren und dadurch beeinflusst werden und wichtige und interessante Ideen nicht zum Vorschein kommen. Außerdem wurden sich in diesem Punkt der Arbeit noch keine Anforderungen überlegt, was die Erstellung von Fragen erschwierigt und zeitlich kostspielig macht. Daher haben wir uns für Frei-Text Fragen/Antworten entschieden.

\myNewSection
Die Umfragen wurden auf folgenden Seiten veröffentlicht: \myTodo
\subsubsection{Vergleich} %
Ähnlich wie bei der \nameref{subsection:umfrage} werden auch für diese Erhebungstechnik weitere Überlegungen angestellt. Das Ziel dabei ist es, möglichst wertvolle und aussagekräftige Ergebnisse zu erzielen und dabei zugleich zeiteffizient vorzugehen..%
%
%
\newline%
\myNewSection%
%Was es bringen soll
Durch den Vergleich sollen lediglich Inspiration sowie allgemeine und offensichtliche Anforderungen gesammelt werden. %
%Was nicht: Abgrenzung
Das Ziel ist es jedoch nicht, Funktionen und Designs von anderen Apps zu kopieren oder sich von ihnen beeinflussen zu lassen. %
	%Wie dagegen angekommen wird.
	Aus diesem Grund werden nur wenige Apps zum Vergleich herangezogen und diese auch nur kurzzeitig getestet.%
		%Warum: weiterer Punkt: Zeitaufwand
		%[Außerdem] würde das testen weiterer App auch zu viel Aufwand und Zeit kosten.\newline%
\newline%
%Was: Auswahl von Apps
Bei der Auswahl der Apps wurde versucht, solche auszuwählen, die möglichst nützliche Informationen liefern können. %
	%Apple & Google
	Dazu wurden der native Apple iOS Kalender\cite{A_calendarApple} und der Google Kalender\cite{A_calendarGoogle} ausgewählt, da angenommen wird, dass diese Unternehmen aufgrund ihres Erfolgs, ihrer Größe und da sie eigenen Richtlinien für Apps haben \cite{konventionen_guidelinesApple, konventionen_guidelinesGoogle}, besonders sorgfältig bei der Entwicklung dieser Apps vorgegangen sind.\newline%
	%Calendars
	Darüber hinaus wurde eine App aufgrund von Kundenbewertungen ausgewählt. Bei einer solchen App wäre es nämlich beispielsweise möglich, dass sie aufgrund von Funktionen, die in den anderen beiden Apps nicht vorhanden sind, so positiv bewertet wurde. Aus diesem Grund wurde sich für Calendars\cite{A_calendarReviews} entschieden.%