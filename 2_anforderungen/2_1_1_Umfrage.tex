\subsubsection{Umfrage}\label{subsection:umfrage}\myCheckmark
%Was+ Warum
Um möglichst Wertvolle und Aussagekräftige Ergebnisse aus der Umfrage zu erzielen wurde sich nochmal getrennt zu dieser Gedanken gemacht. %
%Was+ Warum
Dabei wurde der Standort der Durchführung und die zu stellenden Fragen betrachtet, da sie für die Umfrage die entscheidendsten Faktoren zu seien scheinen.%

\myNewSection
\textbf{Standort}: %
%Was: Auswahlmöglichkeiten
Zur Auswahl stehen die folgenden Möglichkeiten: Die Umfrage in Bekanntenkreis, der Universität, Online-Forums.\newline%
%Was+Warum: keine Zielgruppe in 1&2 -> entfallen
Es wird vermutet, dass sich weder im Bekanntenkreis noch der Universität Nutzer der Zielgruppe befinden. Dementsprechend entfallen diese Auswahlmöglichkeiten.\newline%
%Auswirkung -> OnlineForums
Nach dem Ausschlussverfahren fällt die Entscheidung auf die Online-Forums. %
%Was: erneut frage stellen
Jedoch muss sich hier erneut die Frage über den Standort gestellt werden: \glqq In welchen Foren soll die Umfrage veröffentlicht werden\grqq{}. %
	%Was/Auswirkung: großes Spektrum an Auswahlmöglichkeiten
	Dadurch resultiert ein noch viel größeres Spektrum an Auswahlmöglichkeiten. %
	%Was/Warum: nicht zuviel Zeit verschwenden -> Suchanfragen
	Um nicht zu viel Zeit in die Suche eines maßgeschneiderten Forums zu verschwenden, wurden ungefähr die ersten 20 Ergebnisse, der Suchanfrage \glqq CLI-Calendar Forum\grqq{}, betrachtet. %
	%Aufzählung
	Dabei wurden unteranderem die folgende Forums vorgeschlagen: \glqq reddit: r/commandline\grqq{}\cite{forum_rCommandLine} , \glqq Stack Exchange: Unix \& Linux\grqq{}\cite{forum_unixAndLinux}, \glqq archlinux: Forums\grqq{}\cite{forum_archlinux}, \glqq Debian User Forums\grqq{}\cite{forum_debianUserForums}, \glqq Linux Mint Forums\grqq{}\cite{forum_linuxMintForums}, \glqq Puppy Linux Discussion Forum\grqq{}\cite{forum_puppyLinux}. %
	%Warum: andere geringe Reichweite
	Da sich viele dieser Foren auf ein einzelnes Betriebssystem beschränken wird die Reichweite als eher gering eingeschätzt. %
		%Was: Wahl auf reddit
		Das einzige Forum was dabei hervorsticht, da es allgemeiner angesiedelt ist, ist reddit. Dementsprechend fiel auch die Wahl darauf.\newline%
%Was+Warum: kleiner Aufwand -> r/Apps veröffentlichen  
Da der Mehrfachaufwand für das Veröffentlichen der Umfrage auf einem zusätzlichem Reddit Forum als klein eingeschätzt wird, wird die gleiche Umfrage auch auf r/androidapps\cite{forum_rAndroidapps} und r/iosapps\cite{forum_rIOSapps} veröffentlicht. 
	%Warum: weiteres feedback
	Das Ziel davon ist es [so auch] Einblicke zu dem was App-Nutzer sich wünschen und erwarten zu erhalten.%

\myNewSection
\textbf{Fragen}: %
%Was vordefiniert VS offenes Konstrukt
Für die zu stellenden Fragen wurde sich überlegt, ob sie vordefiniert werden oder ein offenes Konstrukt genutzt wird. %
%Was: Vordefiniert
	%Vorteil: gezielte Antworten
	Vordefinierte Fragen haben zwar den Vorteil, dass man Antworten gezielt aus die gestellten Fragen bekommt. %
	%Nachteil: zu sehr an Fragen orientieren
	Jedoch besteht hier die Möglichkeit, dass sich die Nutzer zu sehr an die Fragen orientieren und dadurch beeinflusst werden und wichtige und interessante Ideen nicht zum Vorschein kommen. %
	%Nachteil: fällt schwer Fragen zu überlegen
	Außerdem wurden sich zu diesem Punkt in der Arbeit noch keine Anforderungen überlegt, was die Erstellung von Fragen erschwert. %
%Auswirkung:
Daher wurde sich für ein offenes Konstrukt beziehungsweise Frei-Text Fragen und Antworten entschieden.%

\myNewSection
Die Umfragen sind unter folgenden Links abrufbar: \myTodo%