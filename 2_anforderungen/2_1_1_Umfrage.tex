\subsubsection{Umfrage}\label{subsection:umfrage}
%Was+ Warum
Um möglichst wertvolle und aussagekräftige Ergebnisse aus der Umfrage zu erzielen, wurden die Durchführungsbedingungen und Fragen sorgfältig überdacht. %
%Was+ Warum
Die Wahl des Standorts und die Formulierung der Fragen wurden dabei als entscheidende Faktoren für den Erfolg der Umfrage identifiziert.%
%
\newline%
\myNewSection
\textbf{Standort}: %
%Was: Auswahlmöglichkeiten
Zur Auswahl stehen die folgenden Möglichkeiten: die Durchführung der Umfrage im Bekanntenkreis, an der Universität oder in Online-Foren.\newline%
%Was+Warum: keine Zielgruppe in 1&2 -> entfallen
Es wird vermutet, dass sich weder im Bekanntenkreis noch an der Universität viele Nutzer der Zielgruppe finden lassen.\newline%
%Auswirkung -> OnlineForums
Dementsprechend fällt die Entscheidung auf die Durchführung der Umfrage in Online-Foren. %
%Was: erneut frage stellen
Jedoch ergibt sich nun die Frage, in welchen Foren die Umfrage veröffentlicht werden soll. %
	%Was/Auswirkung: großes Spektrum an Auswahlmöglichkeiten
	Dadurch ergibt sich ein noch viel größeres Spektrum an Auswahlmöglichkeiten. %
	%Was/Warum: nicht zuviel Zeit verschwenden -> Suchanfragen
	Um nicht zu viel Zeit mit der Suche nach einem maßgeschneiderten Forum zu verschwenden, wurden ungefähr die ersten 20 Ergebnisse einer Google-Suche mit dem Suchbegriff \glqq CLI-Calendar Forum\grqq{} betrachtet. %
	%Aufzählung
	Dabei wurden unter anderem folgende Foren vorgeschlagen: \glqq reddit: r/commandline\grqq{}\cite{forum_rCommandLine} , \glqq Stack Exchange: Unix \& Linux\grqq{}\cite{forum_unixAndLinux}, \glqq archlinux: Forums\grqq{}\cite{forum_archlinux}, \glqq Debian User Forums\grqq{}\cite{forum_debianUserForums}, \glqq Linux Mint Forums\grqq{}\cite{forum_linuxMintForums}, \glqq Puppy Linux Discussion Forum\grqq{}\cite{forum_puppyLinux}. %
	%Warum: andere geringe Reichweite
	Viele der betrachteten Foren beschränken sich auf ein einzelnes Betriebssystem und haben daher vermutlich eine geringere Reichweite. %
		%Was: Wahl auf reddit
		Das einzige Forum, das dabei heraussticht, ist Reddit. Dementsprechend fiel auch die Wahl auf dieses Forum.\newline%
%Was+Warum: kleiner Aufwand -> r/Apps veröffentlichen  
Da der zusätzliche Aufwand für das Veröffentlichen der Umfrage auf einem weiteren Reddit-Forum als gering eingeschätzt wird, wird die gleiche Umfrage auch auf r/androidapps\cite{forum_rAndroidapps} und r/iosapps\cite{forum_rIOSapps} veröffentlicht. 
	%Warum: weiteres feedback
	Das Ziel ist es, so auch Einblicke in die Wünsche und Erwartungen der App-Nutzer zu erhalten.%
%
\newline
\myNewSection
\textbf{Fragen}: %
%Was vordefiniert VS offenes Konstrukt
Für die zu stellenden Fragen wurde überlegt, ob sie vordefiniert oder als offenes Konstrukt gestaltet werden sollen. %
%Was: Vordefiniert
	%Vorteil: gezielte Antworten
	Vordefinierte Fragen haben den Vorteil, dass gezielt Antworten auf bestimmte Fragen erhalten werden können. %
	%Nachteil: zu sehr an Fragen orientieren
	Allerdings besteht hierbei die Möglichkeit, dass sich Nutzer zu sehr an den Fragen orientieren und dadurch wichtige und interessante Ideen nicht zum Vorschein kommen. %
	%Nachteil: fällt schwer Fragen zu überlegen
	Zudem wurde sich zu diesem Zeitpunkt noch keine Anforderungen bezüglich der Anwendung überlegt, was die Erstellung von Fragen erschwert. %
%Auswirkung:
Daher wurde sich letztendlich für offene Freitextfragen und -antworten Konstrukt entschieden.%
%
\myNewSection
Die Umfragen können unter folgenden Links abgerufen werden: %
\newline%
\url{https://www.reddit.com/r/iosapps/comments/10k3d2c/developing_an_app_for_clicalendars_opinion_poll/}
\newline%
\url{https://www.reddit.com/r/androidapps/comments/10k3k7w/developing_an_app_for_clicalendars_opinion_poll/}
\newline%
\url{https://www.reddit.com/r/commandline/comments/10k38bc/developing_an_app_for_clicalendars_opinion_poll/}