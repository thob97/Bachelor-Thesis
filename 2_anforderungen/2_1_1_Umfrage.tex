\subsubsection{Umfrage}
Um möglichst Wertvolle und Aussagekräftige Ergebnisse aus der Umfrage zu erzielen haben wir folgende zwei Themen betrachtet: der Standort der Durchführung und die zu stellenden Fragen.

\myNewSection
\textbf{Standort}: Zur Auswahl stehen uns hier die folgenden Möglichkeiten: Die Umfrage in Bekanntenkreis, der Universität, Online-Forums. \newline
Da es sehr wichtig ist, dass wir eine möglichst große Reichweite auf die Zielgruppe haben, fallen die ersten beiden Auswahlmöglichkeiten weg. \newline
Also bleiben uns nur noch die Online-Forums. Auch hier muss sich jedoch erneut die Frage über den Standort gestellt werden: 'In welchen Foren soll die Umfrage veröffentlicht werden'. Dadurch resultiert ein noch viel größeres Spektrum an Auswahlmöglichkeiten. Um dadurch nicht zuviel Zeit in die Suche eines maßgeschneiderten Forums zu verschwenden, wurden ungefähr die ersten 20-Google-Ergebnisse, der Suche 'CLI-Calendar Forum', betrachtet. Dabei erhielten wir folgende Forum Vorschläge: reddit: r/commandline\cite{rCommandLine} , Stack Exchange: Unix \& Linux\cite{unixAndLinux}, archlinux: Forums\cite{archlinux}, Debian User Forums\cite{debianUserForums}, Linux Mint Forums\cite{linuxMintForums}, Puppy Linux Discussion Forum\cite{puppyLinux}. Da sich viele dieser Foren auf ein einzelnes Betriebssystem beschränken schätzen wir die Reichweite als eher gering ein. Das einzige Forum dabei was hinaussticht und allgemeiner angesiegelt ist, ist reddit. \newline
Um die Zielgruppe nicht nur bei CLI Nutzern zu belassen, sondern auch App-liebhaber anzusprechen, wird die gleiche Umfrage auch auf r/androidapps\cite{rAndroidapps} und r/iosapps\cite{rIOSapps} veröffentlicht.

\myNewSection
\textbf{Fragen}: Bei den zu stellenden Fragen wurde überlegt ob sie vordefiniert werden oder ein Offenes Konstrukt genutzt wird. Vordefinierte Fragen haben zwar den Vorteil, dass man gezielte Antworten bekommt, jedoch besteht hier die Möglichkeit, dass sich die Nutzer zu sehr an die Fragen orientieren und dadurch beeinflusst werden und wichtige und interessante Ideen nicht zum Vorschein kommen. Außerdem wurden sich in diesem Punkt der Arbeit noch keine Anforderungen überlegt, was die Erstellung von Fragen erschwierigt und zeitlich kostspielig macht. Daher haben wir uns für Frei-Text Fragen/Antworten entschieden.

\myNewSection
Die Umfragen wurden auf folgenden Seiten veröffentlicht: \myTodo