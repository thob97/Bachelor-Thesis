\myComment{\subsection*{Stichpunkte2}} 

\myComment{

	\myNewSection
	\textbf{Prezi:}
	\begin{enumerate}
		\item Hier geht es Allgemein um die Frage “was die App überhaupt können soll”. Und sobald ich einige wichtige Anforderungen gesammelt habe, würde ich diese nach Wichtigkeit bewerten, damit ,falls es am Ende Zeitlich knapp wird, die wichtigsten Funktionen bereits Implementiert und konzipiert sind.
		\item Beim Punkt “Vorgehensweise”… geht es darum, dass ich mir Gedanken darüber mache, wie ich die Anforderungen überhaupt erhebe. Zum Beispiel über Typische Features, eigene Ideen, kleine Umfragen, Inspiration durch andere Apps oder über das einlesen in die CLI-Kalender Domäne.
		\item ...
	\end{enumerate}
	
	\myNewSection
	\textbf{Funktionale Anforderungen}
	\begin{enumerate}
		\item Einmal wäre da die \textbf{Verbindung mit Github}. Damit ist gemeint, dass Github als Backend genutzt wird um die CLI-Kalender mit dem App-Kalender zu synchonisieren. Ich könnte mir gut vorstellen, dass ich es als Backendserver wähle, weil Github beliebt, kostenlos und zugänglich ist. Außerdem bietet es auch interessante und nützliche features, welche auch in der App nützlich sein könnten. Wie zum Beispiel Versionskontrolle, branches und die Möglichkeit repos zu teilen)
		\item Die zweite funktionale Anforderung ist: “\textbf{Adapter für etablierte CLI-Kalender}”. Damit ist gemeint, dass die App die Sprache der bereits existierenden CLI-Kalender verstehen soll. Also dass eine Synchronisation zwischen App und Pc möglich sein soll. Das sehe ich einfach mal als Grundanforderung, für die App an.
		\item ...
	\end{enumerate}
	
	\myNewSection
	\textbf{Nicht Funktionale Anforderungen}
	\begin{enumerate}
		\item Dabei wäre einmal \textbf{Wartbarkeit und Erweiterbarkeit} wichtig. Denn Änderungen sollen leicht durchzuführen sein und das Projekt soll auch möglichst Verständlich sein. Ziel davon ist es damit fremde Weiterentwicklung zu ermöglicht und vereinfachen.
		\item Bei der zweiten n.f.Anforderung habe ich die \textbf{Bedienbarkeit} ausgewählt. Denn ich könnte mir gut Vorstellen, dass es wichtig ist, dass die App für Benutzer leicht zu bedienen ist und dass sie die App möglichst effektiv nutzen können  [-> um es möglichst Zugänglich und nützlich zu machen]
	\end{enumerate}
	
}