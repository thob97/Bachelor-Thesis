%%% hidden subsection for a better structure in latex editor: "texifier"
\myComment{\subsection*{Übersicht}}
%Einleitung
	%Was + Warum:
	Da wir nun wissen, welche Aufgaben auf dem PC und welche auf dem Handy gut funktionieren, kann als nächstes die Frage behandelt werden \glqq was die zu erstellende App überhaupt können und leisten soll\grqq{}. %und wie diese Eigenschaften auf die App übertragen werden können. %
	%Warum:
	%Da das [Resultat,Erkenntnisse,Antwort] dieser Frage die darauf folgenden Abschnitte stark beeinflusst, wurde sich damit so früh wie möglich befasst.%
%Übersicht
\newline
\textbf{Überblick:}
	%Vorgehensweise
	Zunächst wird untersucht, auf welche Arten und mit welchen Techniken versucht wird, diese Frage zu beantworten. %
	%fA +nfA
	Anschließend werden die gewünschten Eigenschaften und Funktionen der App aufgeführt, begründet und bewertet.%
%Ergebnisse
\newline
\textbf{Ergebnisse:} %
%Was: 1: Vorgehensweise
Für die Erhebung der Anforderungen wurden die Techniken der Introspektion, der Umfrage und des Vergleichs gewählt, da sie den besten Ausgleich zwischen Informationsgehalt und Zeitaufwand für diese Arbeit bieten. %
%Was 2: n.f.A
Bei den Anforderungen wurde stets versucht, Entscheidungen entsprechend der Stärken von Handys und PCs zu treffen. Dementsprechend wurden die nicht funktionalen Anforderungen \glqq Stärken von PCs und Handys\grqq{} als am wichtigsten bewertet. Die Stärke des Handys \glqq Benutzbarkeit\grqq{} wurde dabei separat betrachtet und als zweitwichtigste nicht funktionale Anforderung bewertet. Weitere gewünschte Eigenschaften sind \glqq Wartbarkeit\grqq{}, \glqq Qualität\grqq{} und eine hohe \glqq Reichweite\grqq{}.%
%Was 3: f.A.
Zuletzt wurden die Funktionen einer \glqq Verbindung zum Backend\grqq{}, einer \glqq grafischen Darstellung für den Kalender\grqq{} sowie eines \glqq Übersetzers für CLI-Terminkalender\grqq{} als unbedingt erforderlich eingeschätzt. Sie bilden also das Grundgerüst der App. Aber auch Funktionen wie das \glqq Erstellen, Bearbeiten und Löschen von Erinnerungen\grqq{}, \glqq Benachrichtigungen\grqq{} und \glqq Konfigurationen auf dem PC\grqq{} werden für diese Anwendung als wichtig erachtet.
%
%
%Maybe AbschlussPrezi
%\myComment{
%(---Außerdem sind viele Vorzüge des Handys erst durch optimierung entstanden, zum Beispiel die intuitive Benutzung durch Gesten und Design, daher muss sich überlegt werden wie diese umzusetzen ist (n.f.A.)---)
%
%}
