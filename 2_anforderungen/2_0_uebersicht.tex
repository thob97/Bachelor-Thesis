%%% hidden subsection for a better structure in latex editor: "texifier"
\myComment{\subsection*{Übersicht}}\myCheckmark
%Einleitung
	%Was + Warum:
	Da wir nun wissen, welche Aufgaben auf dem Pc und welche auf dem Handy gut funktionieren, kann als nächstes die Frage behandelt werden \glqq was die zu erstellende App überhaupt können und leisten soll\grqq{} und wie diese Eigenschaften auf die App übertragen werden können. %
	%Warum:
	%Da das [Resultat,Erkenntnisse,Antwort] dieser Frage die darauf folgenden Abschnitte stark beeinflusst, wurde sich damit so früh wie möglich befasst.%
%Übersicht
\newline
\textbf{Übersicht:}
	%Vorgehensweise
	Dabei wird im \secref{subsection:anforderung:vorgehensweise} behandelt auf welche Arten und mit welchen Techniken versucht wird dies Frage zu beantworten. %
	%fA +nfA
	In \secref{subsection:anforderung:nichtFunktionaleAnforderungen} und \secref{subsection:anforderung:funktionaleAnforderungen} werden die gewünschten Eigenschaften sowie einige der Funktionen der App aufgezählt, begründet und bewertet.%
%Ergebnisse
\newline
\textbf{Ergebnisse:} %
%Was: 1: Vorgehensweise
Für die Erhebung wurden sich die Techniken Introspektion, Umfrage und Vergleich entschieden, denn diese schienen für die Arbeit den besten Ausgleich zwischen Informationen und Zeitkosten zu liefern. %
%Was 2: n.f.A
Bei den Anforderungen wurde stets versucht Entscheidung entsprechend der Stärken des Handys und Pc's zu treffen. Dementsprechend wurde die nicht funktionalen Anforderungen \glqq Stärken von Pcs und Handys\grqq{} als am wichtigsten bewertet. Eine Stärke des Handys ist die \glqq Benutzbarkeit\grqq{}, weswegen sie auch nochmal getrennt betrachtet und als zweit wichtigste nicht funktionale Anforderung bewertet wurde. %
%Was 3: f.A.
Zuletzt wurden die Funktionen einer \glqq Verbindung zum Backend\grqq{}, einer \glqq grafische Darstellung für den Kalender\grqq{} sowie einen \glqq Übersetzer für CLI-Terminkalender\grqq{} als [unbedingt] erforderlich eingeschätzt. Aus ihnen besteht also das Grundgerüst der App. Aber auch Funktionen wie das \glqq Erstellen, Bearbeiten und Löschen von Einträgen \grqq{}, \glqq Benachrichtigungen \grqq{} und \glqq Konfigurationen auf dem Pc \grqq{} werden für diese Anwendung als wichtig eingeschätzt.


%Maybe AbschlussPrezi
\myComment{
(---Außerdem sind viele Vorzüge des Handys erst durch optimierung entstanden, zum Beispiel die intuitive Benutzung durch Gesten und Design, daher muss sich überlegt werden wie diese umzusetzen ist (n.f.A.)---)

}
