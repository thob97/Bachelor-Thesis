\subsection{Funktionale Anforderungen}\label{subsection:anforderung:funktionaleAnforderungen}
%Was
Dieser Abschnitt beschäftigt sich mit den funktionalen Anforderungen, also dem, was das System können soll. Wie bereits im \secref{subsection:anforderung:nichtFunktionaleAnforderungen} , wird auch hier aus denselben Gründen die MoSCoW-Priorisierung verwendet.





%Wiederholung
%\myComment{
%
%	\textbf{Vorab/Kurze wiederholung:} In der \nameref{section:zielsetzung} wurde das Ziel mit \glqq das Erstellen einer App welche die Stärken des Pc's und Handys nutzt\grqq beschrieben. Um auch die Vorzüge des Pc's benutzen zu können, wird keine Standalone App entwickelt. Stattdessen baut die zu erstellende App auf die bereits existierenden CLI-Terminkalender auf und kommuniziert mit dem Pc. \myTextTodo{-> beliebtheit+bekanntheit+es muss nur eine App für das Handy entworfen werden, da für den Pc schon alles verfügbar/existiert - damit erste beide fA Sinnvoller/weniger überfüllt}
%
%}


\begin{itemize}
%Must haves
	\item \textbf{M Verbindung mit Backend:} %
		%Was
		Damit das Handy und der Pc zusammen genutzt werden können, müssen sich diese miteinander verbinden können. Dazu soll ein Backend verwendet werden. Die Wahl fiel auf GitHub, da es für diese Arbeit mehrere Vorteile bietet. %
		%Warum: 
			%Funktionen
			Es bietet es einerseits viele interessante Funktionen, die in die App integriert werden könnten, wie beispielsweise die Versionsverwaltung, Zugriffskontrolle und die Möglichkeit von Entwicklungszweigen. %
			%Beliebt
			Darüber hinaus wird vermutet, dass CLI-Terminkalender-Nutzer auch gerne Git nutzen, da es sich dabei um eine beliebte CLI-Anwendung handelt\footnote{Beispielsweise ist Git bereits standardmäßig in Ubuntu vorinstalliert.\cite{nfA_ubuntuManifestGIT}}. Dementsprechend müssten viele Nutzer keine neue Anwendung erlernen, um die App zu nutzen.%
		%
	\item \textbf{M [+ C] Übersetzer für CLI-Terminkalender:} %
		%Was
		Damit Daten verarbeitet und ausgetauscht werden können, ist neben der Verbindung der Geräte auch eine Kommunikation erforderlich. %
			%Parser vs neue CLI Anwendung
			Zur Ermöglichung dieser Kommunikation bieten sich zwei Optionen an: die Erstellung eines neuen CLI-Terminkalenders, der mit der App kommunizieren kann, oder die Erstellung eines Parsers bzw. Übersetzers für bereits existierende CLI-Terminkalender.
				%Entscheidung+Warum:
				Es wurde sich für die zweite Option entschieden, da sich dadurch einerseits auf die App konzentriert werden kann und andererseits mehrere verschiedene CLI-Terminkalender mit der App funktionieren können. Dadurch würde sich die Zielgruppe und der Nutzen der App erhöhen.\newline%
		%Was + Warum: fokus auf einen -> zeit
		Zunächst wird sich jedoch nur auf die Übersetzung für einen CLI-Terminkalender konzentriert, da einer ausreicht, um die restlichen Funktionen der App zu testen und vorzuführen.%
		%Was+Warum: Später weitere -> reichweite
		Weitere CLI-Terminkalender können immer noch zu einem späteren Zeitpunkt hinzugefügt werden, um so die Zielgruppe zu erweitern.%
		%
	\item \textbf{M [+ C] Kalender Darstellung:} %
		%Was+Warum: simpel+kurz+unterwegs -> ansehen von Terminen 
		Eine Aufgabe, die gut zum Handy passt, da es eine kurzweilige und einfache Aufgabe ist, die man durchaus unterwegs lösen möchte, ist das Ansehen von anstehenden Terminen. Dafür benötigt es eine Darstellung für den Kalender. %
		%Was+Warum: Design
		Dabei wurde sich von bereits existierenden und konventionellen Darstellungen inspiriert, um eine möglichst einfache und intuitive Benutzung zu ermöglichen. Die verglichenen Apps zeigten dabei oft mehrere verschiedene Darstellungen für den Kalender an. %
		%Was+Warum: Monatsansicht -> Zeit + benötigt
		Um Zeit zu sparen und da nicht alle Darstellungen zum Benutzen des Kalenders notwendig sind, wurden vorerst nur zwei Darstellungen fokussiert. Es wurde sich für die Monats- und Tagesansicht entschieden, da sie zusammen einen guten Ausgleich zwischen Details und Übersicht bieten.\newline%
		%Was+Warum:  Weitere später -> Details+Übersicht
		Um verschiedene Darstellungen an Details und Übersicht zu ermöglichen können nachwirkend noch weitere Ansichten wie eine Listen-, Wochen- oder Jahresansicht hinzugefügt werden.%

%should haves
	\item \textbf{S Einträge erstellen, bearbeiten, löschen:} %
		%Was+Warum: simpel+kurz+unterwegs -> Einträge hinzufügen, bearbeiten, löschen
		Eine weitere Aufgabe, die gut zum Handy passt, da es eine kurzweilige und einfache Aufgabe ist, die man durchaus unterwegs lösen möchte, ist das Erstellen, Bearbeiten und Löschen von Terminen. %
		%Was+Warum: nicht 1zu1 -> da Syntax mit vielen Sonderzeichen
		Da die Erstellung von CLI-Terminkalendereinträgen eine eigene Syntax mit vielen Sonderzeichen erfordert, sollte diese Funktion jedoch nicht genau wie auf dem PC über ein Terminal und Texteingabe gelöst werden. %
		%Was: Todoliste
		Stattdessen sollen neue Einträge auf dem Handy in einer separaten Erinnerungsliste erstellt werden. Für den PC sollen die Erinnerungen mit Hilfe von GitHub sichtbar gemacht werden. %
			%Nachteil: Doppelarbeit
			Zwar entsteht hierbei der Nachteil, dass eine gewisse Doppelarbeit entsteht, da die Einträge auf dem Handy lediglich als Erinnerung dienen und zu einem späteren Zeitpunkt auf dem PC vervollständigt werden müssen. %
			%Vorteil: Sinnvolle Aufteilung
			Jedoch hat diese Methode den Vorteil, dass die Aufgabe passend den Stärken der Geräte aufgeteilt wird. Das Handy wird für einfache und kleine Einträge genutzt, während auf dem PC die Einträge vervollständigt werden können, um ausführliche Beschreibungen mit komplexen Zeichen und Syntax zu erstellen.%
			%Warum: Einfachheit -> weniger komplexe Gedanken
			%Außerdem ist ein weiterer Vorteil die Einfachheit dieser Lösung. So muss sich dadurch zum Beispiel keine [komplexen] Gedanken über wie mit Push-Konflikten umgegangen wird [gemacht werden].%
			%
	\item \textbf{S Einschränkungen:}	%
		%Einleitung
		Im \secref{section:pcVsPhone} wurden wiederholt Einschränkungen erwähnt, die dem Handy bewusst auferlegt wurden, um so positiven Eigenschaften wie Einfachheit zu betonen %
		%Was+Warum: Begrenzen -> zu kurzweilige und einfache Aufgaben lenken 
		Eine ähnliche Idee wird auch hier verfolgt. Einige Funktionen sollen begrenzt werden, um die Nutzung der App auf die Stärken des Handys zu beschränken, beispielsweise auf kurzweilige und einfache Aufgaben. %
		%Beispiele:
		So könnten zum Beispiel eine maximale Textlänge für neue Einträge auf dem Handy festgelegt werden, um zu vermeiden, dass zu viel auf dem Handy geschrieben wird. Außerdem könnte die maximale Anzahl der Erinnerungseinträge begrenzt werden, um so eine unübersichtlich lange Liste auf dem Handy zu vermeiden. %
		%
	\item \textbf{S Benachrichtigungen:} %
		%Was
		Eine Aufgabe, die sich besser für das Handy eignet als für den PC, sind Erinnerungsbenachrichtigungen für Termine. 
			%Warum:Handy
			Das Handy kann den Nutzer fast immer benachrichtigen, da wie in \secref{section:pcVsPhone} erwähnt wurde, dass davon ausgegangen wird, dass das Handy fast immer beim Nutzer angeschaltet ist. Im Gegensatz dazu ist der PC stationär und wird in der Regel ausgeschaltet, wenn er nicht benutzt wird. Dadurch hat er weniger Möglichkeiten, den Nutzer zu benachrichtigen.\newline%
		%Auswirkungen
		Daher wird in der App die Möglichkeit für Benachrichtigungen von Terminen angeboten. % 
		Allerdings wurde entschieden, diese Funktion nicht für die zuvor erwähnten Erinnerungseinträge anzubieten, da dies dazu führen könnte, dass Erinnerungen als Termine genutzt werden. Dies würde dem zuvor festgelegten Design widersprechen, das vorsieht, dass das Handy für kurze Aufgaben wie Erinnerungen genutzt wird, während auf dem PC komplexere Aufgaben erledigt werden.%
		%
		%
	\item \textbf{S Konfiguration auf dem Pc:} %
		%Was
		Für Optionen, die von möglicherweise Nutzern ändern wollen, wird eine Konfigurationsmöglichkeit bereitgestellt. %
			%Beispiel
			Dazu gehören beispielsweise die Standardzeit für Erinnerungen oder Einstellungen zu den zuvor erwähnten Einschränkungen. %
		%Was+ Warum
		Da in \secref{section:pcVsPhone} festgestellt wurde, dass sich PCs eher als Handys zum Konfigurieren eignen, soll diese Funktion über den PC ermöglicht werden.%
	%
%could haves
	\item \textbf{C Suchfunktion:} %
		%Was
		Um unterwegs schnell gezielte Informationen zu Terminen abrufen zu können, soll eine Suchfunktion implementiert werden. %
		%Warum
		Da die Informationen auch über die grafische Oberfläche oder den PC abgerufen werden können, wird diese Aufgabe jedoch als zweitrangig betrachtet.%
	%
	\item \textbf{C Weitere Kalender Abonnieren \& Teilen:} %
		%Was + Warum:
		Die Funktionen, mehrere Terminkalender als einen darzustellen sowie eigene Terminkalender zu teilen und variable Zugriffsrechte zu bestimmen, sind in allen drei getesteten Apps wiederkehrende Funktionen und scheinen dementsprechend Standardfunktionen in Kalendern zu sein. %Falls die CLI-Terminkalender diese Funktion nicht bereits unterstützen, kann diese mithilfe dieser Anwendung ermöglicht werden.%
		
	\item \textbf{C Offline Funktionen:} % 
		%Was
		Da die App lediglich zum Herunterladen und Hochladen von Dateien eine Internetverbindung benötigt, ist eine permanente Verbindung nicht zwingend erforderlich. %
		%Warum
		Da jedoch fast alle Smartphones heutzutage über eine Internetverbindung verfügen, wird dieser Funktion vorerst weniger Aufmerksamkeit geschenkt.%

%wont haves
	\item \textbf{W Anleitung:} %
		%Was
		Um die Nutzung der App zu vereinfachen, könnte es beim ersten Start eine Einführung mittels Anleitung geben. %
		%Warum:
		Noch besser wäre jedoch, wenn die App so intuitiv gestaltet ist, dass keine Anleitung benötigt wird.%


	\item \textbf{W Commit-History:} %
		%Was
		Eine möglicherweise interessante Funktion, die durch GitHub ermöglicht wird, ist die Darstellung der Commit-History als Graph in der App. Dadurch können die letzten Veränderungen und Versionen betrachtet werden. %
		%Warum
		Da unter anderem für die Darstellung lange Listen und viele Details erforderlich sind, wird diese Funktion jedoch nicht als besonders geeignet für die Nutzung auf dem Handy angesehen. %
		%Auswirkung:
		Daher wird diese Funktion vorerst nicht implementiert.%



\end{itemize}