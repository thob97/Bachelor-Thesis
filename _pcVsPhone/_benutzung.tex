\subsection{Benutzung}\myCheckmark
%Was: Einstiegsaufwand
	%Pc
	Um Pc's zu benutzen, bedarf es eines gewissen Einstiegsaufwands. %
		%Warum: Stationär
		So muss sich einerseits, da Desktops stationär sind, zuerst zum Standort des Pc's begeben werden. %
		%Warum: Startuptime
		Darüber hinaus muss der Pc vor jeder Nutzung eingeschaltet werden. Denn es wird davon ausgegangen, dass Pc's normalerweise ausgeschaltet sind, wenn sie derzeit nicht in Benutzung sind. Das Anschalten dauert in der Regel ein paar Sekunden. Laut dem Benchmark \glqq Startup Timer\grqq{} beträgt die schnellste aufgezeichnete Startzeit bis der Pc nutzbar ist elf Sekunden\cite{pcVsphone_boottime}.\newline%
	%Handys
	Bei Handys existiert dieser Aufwand hingegen nicht. %
		%Warum: Mobil + Relevanz -> Immer bei einem
		So wird davon ausgegangen, dass die meisten Nutzer ihr Handy immer bei sich tragen. Das liegt einerseits an der zuvor erwähnten Mobilität, aber andererseits auch in der \nameref{subsection:motivation} erwähnten Relevanz und Beliebtheit von Handys. %
		%Warum:
		Weiterhin wird davon ausgegangen, dass Handys normalerweise nicht ausgeschalten werden, um unter anderem für wichtige Anrufe oder Nachrichten erreichbar zu sein. Das Handy ist also immer an und benötigt dementsprechend keine Startzeit.\newline%		
%Was: Multitasking%-------------------------Reword-------------------------------------
Ein weiterer Aspekt bei der Verwendung ist, wie Anwendungen auf den jeweiligen Geräten genutzt werden können. %
	%Handy: 
	Dabei können Handys immer nur eine Anwendung gleichzeitig darstellen. Das könnte eine Limitation des Betriebssystems oder der Leistung sein. Vermutlich wird das Multitasking aufgrund des kleinen Displays nicht unterstützt.\newline%
	%Pc:
	Währenddessen unterstützt der Pc genau diese Funktion. Er kann mehrere Anwendungen gleichzeitig ausführen und darstellen.%
	
	
\myComment{

\myTextTodo{\textbf{One task at a time}: Although many mobile operating systems now offer a split-screen mode, the small screen size limits its usefulness. The fact is, in most cases, users on mobile devices must focus on one window at a time. This limitation means that it’s difficult to combine multiple sources of information and carry out complex tasks. These mobile constraints are no problem if the task is simple, unimportant, or open-ended. However, when the task is goal-based and has high stakes, these constraints are reason enough to save the task for another device}

}