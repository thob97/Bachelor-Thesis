\subsection{Benutzung}

\myNewSection
\textbf{Benutzung} Der wohl wichtigste Unterschied zwischen Pc und Handy und damit auch die wichtigste Erkenntnisse ist, dass Benutzer die Geräte für verschiedene Aufgaben nutzen. ... (Warum) ... . Alle zuvor genannten Unterschiede spielen wahrscheinlich in diesen Sachverhalt mit ein.%newline%
So wird für das Arbeiten an wichtigen Aufgaben anscheinend lieber der Pc genutzt\cite{pcVsphone_easyUseVsImportantTask}. Wir sehen diese Annahme als Sinnvoll an, denn die meisten vorherigen Vergleiche deuten darauf hin. So kann der Pc zum Beispiel mit seinem großen Bildschirm mehrere Informationen gleichzeitig darstellen, die Eingabegeräte lassen Präziser eingaben zu, die Leistung lässt einen aufwändige Anwendungen benutzen und es können viele Konfigurationen und Optionen getroffen werden. \newline%

Anzeichen, dass dies nicht nur eine Annahme ist, sondern auch in der Praxis Benutzer den Pc häufiger für wichtige Aufgaben, wie der Arbeit, nutzen, sieht man zum Beispiel anhand Suchanfragen. \newline

So sind Suchanfragen am Pc eher auf Arbeit ausgelegt\cite{}. Auch Emails welche über die Arbeit handeln werden öfter auf dem Pc geöffnet\cite{}.

Ein weiteres Indiz dafür ist, dass die Arbeit am Pc Zeitintensiver scheint. So machen Pc's 67\% der Zeitnutzung auf Websites aus und es gibt mehr Pageviews.\newline

Währeddessen scheinen Handys eher für Simple und Kurzweilige Aufgaben genutzt zu werden.

So werden von Handys 68\% der Webseitaufrufe getätigt aber sie die Besuchdauer mit 33\% ist kurzweiliger\cite{}. Das hat wahrscheinlich damit zu tun, dass auf Handys auch nach anderen meist simpleren und kurzweiligeren Themen gesucht werden. Am meisten wird auf Ihnen wohl nach "Maps, food, news" gesucht.  









Wir nehmen an, dass Aufgaben als wichtig gelten, wenn sie anspruchsvoll, aufwändig, zeitintensiv sind oder viele Informationen benötigen. Beispiele für wichtige Aufgaben wären Erledigungen für die Arbeit, während Freizeitaktivitäten das Gegenteil darstellen.\newline%
Mit dieser Annahme macht die Aussage, dass wichtige Aufgaben generelle lieber auf dem Pc als auf dem Handy erledigt werden, Sinn. So lässt einen der große Bildschirm mehrere Informationen gleichzeitig darstellen, die Eingabegeräte sind Präziser, 



- Die Benutzung erschließt sich aus allen zuvor genannten Unterschieden.

- Pc für wichtigere Aufgaben -> z.B. weil Software nur auf dem Pc verfügbar -> weil Bildschirm/Übersicht, Eingabe, Leistung, Anpassung/Optionen
- Handy leichtere Aufgaben -> mehr Benutzerfreundlich/auf Userexperience ausgelegt -> weil Design, intuitiver Input, allInOne, reibungslose Software und Hardware Lösung
- Kurze vs lange Sitzungen

\myTextTodo{Durch die stärkere Performance von Desktops ist es ihnen auch erlaubt aufwändigere Software zu betreiben. Besonders auch durch die Eingabegeräte mit Maus und Tastatur ist es möglich bei Software präzise eingaben und Tastenkombinationen vorrauszusetzen. Dadurch bieten die Softwares oft viele Optionen  
Viele Optionen -> Handy eher simpel. Bei Handy viele Softwares nach bestimmten Konventionen -> bei Desktop vielfältiger}

\myTextTodo{An Desktops verbringen die Menschen tendenziell längere aufeinanderfolgende Sitzungen, was dazu führt, dass sie mehr Zeit auf ein und derselben Webseite verbringen. Im Gegensatz dazu verbringen die Menschen auf ihren Handys kürzere Sitzungen, greifen aber im Laufe des Tages dutzende Male zum Handy. Sie spielen vielleicht ein Spiel, scrollen in sozialen Netzwerken oder stellen eine Frage bei Google und legen dann ihr mobiles Gerät wieder weg. Aktivitäten wie das Warten auf einen Kaffee, Pausen bei der Arbeit oder die Suche nach Restaurants in der Nähe führen dazu, dass sie viel Zeit am Handy verbringen, aber nur in kürzeren Schüben} \newline