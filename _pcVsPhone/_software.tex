\subsection{Software}\myCheckmark
%Was: Pc Software -> Konfigurierbar
Software für den Pc sind oft konfigurierbar. %
	%Beispiel: Installation
	So fängt zum Beispiel die Auswahl verschiedener Optionen bereits bei der Installation der Software an. Unter anderem werden Fragen zum Speicherort, der automatischen Aktualisierung, der Desktop Verknüpfung und dem automatischen Starten gestellt.\newline%
%Was: Handy Software -> Benutzerfreundilchkeit
Handys hingegen scheinen weniger auf Konfigurierbarkeit, sondern eher auf Benutzerfreundlichkeit ausgelegt zu sein. %
	%Warum: Begründung + Beispiel(Installation)
	So benötigt die Installation von neuen Apps oft nur einen Knopfdruck. Rudimentäre Fragen wie der Speicherstandort oder die Erstellung einer Verknüpfung werden hingegen vom Betriebssystem übernommen. Das Abarbeiten einer Liste von Optionen wird also auf dem Handy übersprungen und dadurch wird angenommen, dass sich der Übergang von der Installation bis zum Nutzen der App flüssiger für den Nutzer anfühlt.\newline%
%Was: Pc Software Allgemein -> Pc besser Konfigurierbar 
Dieses Verhalten gilt auch über die Installation hinaus. Generell scheinen Anwendungen mit vielen Optionen besser auf dem Pc zu funktionieren. %
	%Begründung: Aus Erfahrung besser auf Pc
	So werden Aufgaben wie das Editieren von Videos, Entwicklungsumgebungen oder das Erstellen von Steuererklärungen werden erfahrungsgemäß zumeist auf dem Computer bearbeitet. %
		%Begründung: Zuvor Erwähnten Unterschieden -> viele Optionen schwer zu überschauen + klicken
		Womöglich liegt das an den zuvor erwähnten Unterschieden und Limitierungen des Handys. Eines davon wäre zum Beispiel die Displaygröße. Wenn eine Anwendung viele Optionen bietet, lassen sich diese bei einem kleinen Display mit ungenauerer Eingabe schwieriger auswählen und überschauen.\newline%
%Was: Pc Software Allgemein -> Weniger Optionen %---------TODO re check this part----------
Aus Erfahrung spiegelt sich dieser Sachverhalt auch in Apps wieder. So werden für Handys meist nur die relevanten Optionen dargestellt. Das würde bedeuten, dass die restlichen eher nebensächlichen Optionen bereits vom Entwickler oder Betriebssystem getroffen worden.\newline% 
	%Was: Entwickler machen sich mehr Gedanken: Optionen
	Daher wird vermutet, dass sich Entwickler bei der Erstellung von Apps mehr Gedanken über passende Entscheidungen machen, da ihnen bewusst ist, dass die Nutzer die Optionen später nicht selbst anpassen können.\newline%
	%Warum: Vorteil: Userexperience
	Das hätte im besten Fall den Vorteil, dass sich Handynutzer weniger bis keine Gedanken um die Auswahl von Optionen machen müssen und sich dementsprechend die App intuitiver anfühlt. Schließlich wurde bereits die passende Auswahl vom Entwickler getroffen.%
%
%
%
%
%
\myComment{

%%%->Benutzung%%%

	%Zusammenfassung
	Während der Pc also erneut Optionsfreiheit anbietet scheint das Handy wieder Benutzerfreundlicher zu sein. So lässt die Pc Software einen sehr viele Optionen zum selbst konfigurieren, während Apps sofort einsatzbereit sind. 
	
	%Was
	%Warum
	%Zusammenfassung
	%Auswirkung
	- Handy Benutzerfreundlicher\\
	- Handy Besser für einfache Aufgaben, da man dort meistens eh nicht viel konfigurieren möchte
	- Generell scheinen Komplexe Aufgaben mit vielen Optionen besser auf dem Pc zu funktionieren.
	
	\myNewSection
	\myTextTodo{
	-> Konfiguration VS Einfachheit -> Optimierung
	}

}
