\myComment{\subsection*{Software}}  

\myNewSection
\textbf{Software} Ein Ähnliches Verhalten lässt sich auch bei der Software feststellen. So bietet die Software für Pc's oft viele Optionen an. Angefangen vom Speicherstandort bis zum Design und Tastaturkommandos. Oft fängt die Auswahl von Optionen schon bei der Installation der Software an. So werden beim installieren oft Fragen zu dem Speicherort, automatische Aktualisierung, Desktopverknüpfung und Automatischemstarten gestellt.\newline%
Bei Handys wird sich hingegen wieder nach Benutzerfreundlichkeit im Tausch gegen die Optionsfreiheit orientiert. So benötigt die Installation von neuen Apps oft nur einen Klick/Button. Rudimentäre Fragen wie Speicherstandort werden hingegen vom Betriebssystem übernommen. Das abarbeiten einer Liste von Optionen wird auf dem Handy also übersprungen. Dadurch fühlt sich der Übergang um von der Installation bis zum Nutzen der App um einiges flüssiger an. \newline%
Dieses Verhalten hört nicht schon bei der Installation auf. Generell scheinen Komplexe Aufgaben mit vielen Optionen besser auf dem Pc zu funktionieren. So werden für Anwendungen zum Editieren von Videos, Entwicklungsumgebungen oder zum erstellen von Steuererklärungen meistens zum Pc anstatt zum Handy gegriffen. Wohlmöglich liegt das an den zuvor erwähnten Unterschieden. Wenn eine Anwendung viele Optionen bietet, lassen sich diese bei einem kleinen Display mit ungenauer Eingabe schlicht schlecht auswählen.\newline%
Daher werden bei Apps auch oft weniger Optionen angeboten. Stattdessen machen sich Entwickler zuvor Gedanken um die besten Einstellungen für den Nutzer zu treffen. Für Optionen welche trotzdem dem Nutzer überlassen werden sollen, sollten nur die nötigsten angeboten werden. Sonst leidet schnell der Überblick darüber. \newline% 
Während der Pc also erneut Optionsfreiheit anbietet scheint das Handy wieder Benutzerfreundlicher zu sein. So werden für das Handy meist nur relevante Optionen angezeigt, während die restliche Optionen vom Betriebssystem oder dem Entwickler bereits getroffen wurden.

\myComment{(oft wird erst nach einer Auswahl von Optionen gefragt wenn diese gerade auch wirklich benötigt wird zb Kamerazugriff)}