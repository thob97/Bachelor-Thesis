\subsection{Software}
%Was: Pc Software -> Konfigurierbar
Die Software für den PC ist oft konfigurierbar. %
	%Beispiel: Installation
	Bereits bei der Installation können verschiedene Optionen ausgewählt werden, wie beispielsweise der Speicherort, die automatische Aktualisierung, die Desktop-Verknüpfung und das automatische Starten.\newline%
%Was: Handy Software -> Benutzerfreundilchkeit
In puncto Konfigurierbarkeit scheinen Handys eher auf Benutzerfreundlichkeit ausgelegt zu sein. %
	%Warum: Begründung + Beispiel(Installation)
	Bei der Installation neuer Apps genügt oft ein Knopfdruck. Fragen wie der Speicherort oder die Erstellung einer Verknüpfung werden vom Betriebssystem übernommen. Das Abarbeiten von Optionen entfällt hierbei. Dementsprechend wird angenommen, dass sich der Übergang von der Installation bis zum Nutzen der App für den Nutzer flüssiger gestaltet.\newline%
%Was: Pc Software Allgemein -> Pc besser Konfigurierbar 
Dieses Verhalten scheint sich auch über die Installation hinaus fortzusetzen. Beispielsweise scheinen Anwendungen mit vielen Optionen generell besser auf dem PC zu funktionieren. %
	%Begründung: Aus Erfahrung besser auf Pc
	So werden Aufgaben wie das Editieren von Videos, die Verwendung von Entwicklungsumgebungen oder das Erstellen von Steuererklärungen werden erfahrungsgemäß üblicherweise am Computer ausgeführt. %
		%Begründung: Zuvor Erwähnten Unterschieden -> viele Optionen schwer zu überschauen + klicken
		Möglicherweise liegt das an den zuvor erwähnten Unterschieden und Limitierungen von Handys, wie beispielsweise der Displaygröße. Bei einem kleinen Display und ungenauer Eingabe könnten viele Optionen dazu führen, dass sie schwerer auswählbar und überschaubar werden.\newline%
%Was: Pc Software Allgemein -> Weniger Optionen %---------TODO re check this part----------
Aus Erfahrung spiegelt sich dieser Sachverhalt auch in Apps wieder. So werden für Handys meist nur die relevanten Optionen dargestellt. Das würde bedeuten, dass die restlichen eher nebensächlichen Optionen bereits vom Entwickler oder Betriebssystem getroffen worden.\newline% 
	%Was: Entwickler machen sich mehr Gedanken: Optionen
	Daher wird vermutet, dass sich Entwickler bei der Erstellung von Apps mehr Gedanken über passende Entscheidungen machen, da ihnen bewusst ist, dass die Nutzer die Optionen später nicht selbst anpassen können.\newline%
	%Warum: Vorteil: Userexperience
	Im besten Fall führt das dazu, dass Handynutzer sich weniger oder gar keine Gedanken über die Auswahl von Optionen machen müssen und sich die App dementsprechend intuitiver anfühlt.%
%
%
%
%
%
%\myComment{
%
%%%%->Benutzung%%%
%
%	%Zusammenfassung
%	Während der Pc also erneut Optionsfreiheit anbietet scheint das Handy wieder Benutzerfreundlicher zu sein. So lässt die Pc Software einen sehr viele Optionen zum selbst konfigurieren, während Apps sofort einsatzbereit sind. 
%	
%	%Was
%	%Warum
%	%Zusammenfassung
%	%Auswirkung
%	- Handy Benutzerfreundlicher\\
%	- Handy Besser für einfache Aufgaben, da man dort meistens eh nicht viel konfigurieren möchte
%	- Generell scheinen Komplexe Aufgaben mit vielen Optionen besser auf dem Pc zu funktionieren.
%	
%	\myNewSection
%	\myTextTodo{
%	-> Konfiguration VS Einfachheit -> Optimierung
%	}
%
%}
