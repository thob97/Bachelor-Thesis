\subsection{Software}\myCheckmark
%Was: Pc Software -> Konfigurierbar
Software für den Pc sind oft sehr Konfigurierbar. %
	%Beispiel: Installation
	So fängt die Auswahl verschiedener Optionen bereits bei der Installation der Software an. Unteranderem werden Fragen zum Speicherort, der automatische Aktualisierung, der Desktop Verknüpfung und dem automatischem Starten gestellt.\newline%
%Was: Handy Software -> Benutzerfreundilchkeit
Handys scheinen hingegen weniger auf Konfigurierbarkeit sondern eher auf Benutzerfreundlichkeit ausgelegt zu sein. %
	%Warum: Begründung + Beispiel(Installation)
	So benötigt die Installation von neuen Apps oft nur einen Knopfdruck. Rudimentäre Fragen wie der Speicherstandort oder die Erstellung einer Verknüpfung werden hingegen vom Betriebssystem übernommen. Das Abarbeiten einer Liste von Optionen wird also auf dem Handy übersprungen und dadurch fühlt sich der Übergang um von der Installation bis zum Nutzen der App [flüssiger] an.\newline%
%Was: Pc Software Allgemein -> Pc besser Konfigurierbar 
Dieses Verhalten gilt auch über die Installation hinaus. Generell scheinen Anwendungen mit vielen Optionen besser auf dem Pc zu funktionieren. %
	%Begründung: Aus Erfahrung besser auf Pc
	So werden zum Beispiel für solche Aufgaben wie das Editieren von Videos, Entwicklungsumgebungen oder zum erstellen von Steuererklärungen, aus Erfahrung, meistens zum Pc gegriffen. %
		%Begründung: Zuvor Erwähnten Unterschieden -> viele Optionen schwer zu überschauen + klicken
		Wohlmöglich liegt das an den zuvor erwähnten Unterschieden und Limitierungen des Handys. Eins davon wäre zum Beispiel die Display Größe. Wenn eine Anwendung viele Optionen bietet, lassen sich diese bei einem kleinen Display mit ungenauer Eingabe [schlicht] schlecht auswählen und überschauen.\newline%
%Was: Pc Software Allgemein -> Weniger Optionen %---------TODO re check this part----------
Aus Erfahrung spiegelt sich dieser [Sachverhalt] auch durchaus in Apps wieder. Meist werden für Handys nur die relevanten Optionen dargestellt. Die restliche eher nebensächlichen Optionen müssen also bereits vom Entwickler oder Betriebssystem getroffen worden sein.\newline% 
	%Was: Entwickler machen sich mehr Gedanken: Optionen
	Es wird vermutet, dass sich die Entwickler mehr Gedanken für Entscheidung bei der Erstellung der App machen, da sie wissen, dass die Nutzer diese Optionen später nicht selbst ändern können.\newline%
	%Warum: Vorteil: Userexperience
	Das hat im besten Fall den Vorteil, dass man sich als Handy Nutzer weniger [oder sogar/bis] keine Gedanken um die Auswahl von Optionen machen muss. Denn es wurde für einem bereits eine passende Auswahl getroffen.%








\myComment{

%%%->Benutzung%%%

	%Zusammenfassung
	Während der Pc also erneut Optionsfreiheit anbietet scheint das Handy wieder Benutzerfreundlicher zu sein. So lässt die Pc Software einen sehr viele Optionen zum selbst konfigurieren, während Apps sofort einsatzbereit sind. 
	
	%Was
	%Warum
	%Zusammenfassung
	%Auswirkung
	- Handy Benutzerfreundlicher\\
	- Handy Besser für einfache Aufgaben, da man dort meistens eh nicht viel konfigurieren möchte
	- Generell scheinen Komplexe Aufgaben mit vielen Optionen besser auf dem Pc zu funktionieren.
	
	\myNewSection
	\myTextTodo{
	-> Konfiguration VS Einfachheit -> Optimierung
	}

}
