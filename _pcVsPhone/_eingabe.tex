\subsection{Eingabe} 
%Was: Maus
PCs werden in der Regel mithilfe einer Maus bedient. Mit ihr können Objekte auf der Oberfläche präzise ausgewählt werden. Außerdem verfügt die Maus über verschiedene Tasten, mit denen eine Interaktion mit den Objekten auf verschiedene Arten möglich ist.\newline%
%Was: Handy: Zeigen+Gesten
Auf dem Handy wird hingegen der Touchscreen als Mausersatz benutzt. Durch Berühren des Bildschirms lassen sich Mausfunktionen simulieren. Anstatt beispielsweise das gewünschte Objekt mit der Maus auszuwählen, wird es mit dem Finger berührt. Weitere Funktionen wie Zoomen, Scrollen und Rechtsklick können durch Gesten ausgeführt werden.\newline%
	%Warum: Intuitive
	Dadurch wird die Bedienung auf dem Handy intuitiver, insbesondere das Auswählen von Objekten fühlt sich natürlicher an.\newline%
		%Geste
		Erfahrungsgemäß gilt das gleiche für die Gesten. Scheinbar fühlen sie sich intuitiv an und haben sich bereits als Standard in den Handymarkt eingegliedert, dass aus eigener Erfahrung viele Menschen lieber das Handy nutzen statt den Pc um Bilder anzuschauen. Die Möglichkeiten mithilfe von Gesten zu wischen und vergrößern fühlt sich wohl angenehmer an, als über Verwendung von Maus und Tastatur die Bilder zu betrachten.\newline%
	%Warum: Unpräzise -> Design
	Dafür ist die Eingabe über den Touchscreen im Vergleich zur Maus ungenauer. Da der Finger deutlich größer ist als der Mauszeiger, lassen sich kleine Objekte nicht so konsistent auswählen. Um diesem Problem entgegenzuwirken, ist ein passendes Design erforderlich. Beispielsweise wird für Apps eine Mindestgröße für Buttons vorgeschrieben: \glqq On a touchscreen, buttons need a hit target of at least 44x44 points to accommodate a fingertip\grqq{}\cite{konventionen_buttonSize}.%
\myNewSection%
%Was: Keyboard unterschiede
Zur Eingabe von Text auf einem PC wird üblicherweise eine physische Tastatur verwendet, während auf Handys eine Softwaretastatur zum Einsatz kommt. Aufgrund der unterschiedlichen Größe und Bedienung bieten beide Varianten verschiedene Vor- und Nachteile.\newline%
	%Warum Pc: schneller und präziser + Tasten Anzahl und Tastenkombinationen
	Eine Hardware-Tastatur ermöglicht durch ihre Größe und das Tippgefühl schnelleres und präziseres Schreiben im Vergleich zur Touchscreen-Tastatur auf Handys. Außerdem kann sie im Vergleich zur mobilen Variante ungefähr das Dreifache an Tasten darstellen.\footnote{Full-size Tastaturen können bis zu 108 Tasten haben\cite{pcVsphone_pcKeyboardSize}. Die Standard deutsche Texttastatur auf iOS 16.1.2 hat hingegen rund 33 Tasten. Selbst nachgezählt und kontrolliert auf iPhone SE1, SE2 und 13 mini.}. Des Weiteren unterstützt Hardwaretastaturen Tastenkombinationen. % 
		%Auswirkung: TODO ~maybe~ remove
		Dadurch können Programme, die lange Eingaben, viele verschiedene Symbole oder Tastenkombinationen erfordern, in der Regel auf einem PC besser bedient werden. Ein Beispiel dafür sind Programmiereditoren.\newline
	%Warum Handy: Mobiler und intuitiver
	Die Handy-Tastatur hat den Vorteil der Mobilität und das Tippen fühlt sich oft intuitiver an, da sich die Tastatur je nach Anwendung variabel darstellen lässt. %
		%Beispiel:
		Ein Beispiel dafür ist die Taschenrechner-Anwendung, bei der nur die benötigten Zahlen und Symbole auf der Tastatur angezeigt werden und dementsprechend keine Buchstaben.%


%\myComment{
%
%	%%%->Benutzung%%%
%	\myNewSection
%	%Zusammenfassung
%	Zusammenfassend hat das Handy also auch in diesem Vergleich an Leistung, in diesem Fall die präzise fehlerfreie und schnelle Eingabe, verloren und dafür an Mobilität gewonnen. Dafür fühlt es sich aber in vielen Anwendungen intuitiver und damit leichter zu benutzen an.
%	
%	\myNewSection
%	\myTextTodo{
%	-> Leistung(schnellere präzisere fehlerfreie eingabe) vs Portabilität\\
%	-> um den entgegenzuwirken Optimierung(Design-Eingabe auf Anwendung Anpassen + Gesten + Touch) -> Intuitive
%	}
%	
%	\myNewSection	
%	%Auswirkung
%	- Da die Handy-Tastatur also Fehleranfälliger und schwerer zu benutzen ist, nehmen wir an, dass Handy Nutzer weniger gern lange texte auf dem Handy schreiben. \newline%
%	- Andersherum benutzen Handy Nutzer für kurze und einfachere Aufgaben lieber das Handy. Denn bei diesen Aufgaben benötigt es nicht viele verschiedene Tasten, Präzise Eingabe, schnelle Eingabe (-> bedienung wird nicht erschwert) und ist daher mit der Handy Eingabe (Touch, Geste + Anwendungstatsatur) (-> bedienung wird verbessert) intuitiver\newline%
%	- Dafür muss die App aber dementsprechend auch optimiert werden. (leichte Anwendung, Touch + Gesten Unterstützung + richtige Tastatur)
%	- Schwierige und Komplexe Aufgaben funktionieren dafür besser auf dem Pc, da lange Eingabe + verschiedene Symbole + Tastenkombinationen in der Regel für diese benötigt werden. 
%
%}