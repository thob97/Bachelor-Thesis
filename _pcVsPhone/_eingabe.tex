\subsection{Eingabe} \myCheckmark
%Was: Maus
Pc's werden meistens mithilfe einer Maus bedient. Durch sie lassen sich Objekte auf der Oberfläche präzise auswählen. Des Weiteren besitzt die Maus verschiedene Tasten und kann so auf verschiedenen Weisen mit den Objekten interagieren.\newline%
%Was: Handy: Zeigen+Gesten
Auf dem Handy wird hingegen der Touchscreen als Mausersatz benutzt. Damit lassen sich Funktionen der Maus simulieren. So wird zum Beispiel, anstatt die Maus auf das gewünschte Objekt zu bewegen, das Objekt mit dem Finger berührt. Weitere Funktionen wie das Vergrößern, Blättern und der Rechtsklick lassen sich durch Gesten betätigen.\newline%
	%Warum: Intuitive
	Dadurch wird die Bedienung auf dem Handy intuitiver. %
		%Zeigen
		 Insbesondere das Auswählen von Objekten auf dem Handy fühlt sich natürlicher an.\newline%[da der Zeigefinger, wie der Name bereits verrät, für diese Aufgabe wie gemacht ist. Gewissermaßen wird der Zeigefinger bereits zum Zeigen beziehungsweise zum Auswählen von Objekten in der realen Welt benutzt.]\newline%
		%Geste
		Erfahrungsgemäß gilt das gleiche für die Gesten. Scheinbar fühlen sie sich intuitiv an und haben sich bereits als Standard in den Handymarkt eingegliedert, dass aus eigener Erfahrung viele Menschen lieber das Handy nutzen statt den Pc um eigene Bilder anzuschauen. Die Möglichkeiten mithilfe von Gesten zu wischen und vergrößern fühlt sich wohl besser an, als mit Maus und Tastatur Bilder zu öffnen.\newline%
	%Warum: Unpräzise -> Design
	Dafür ist die Eingabe über den Touchscreen ungenauer verglichen mit der Eingabe über die Maus. Der Finger ist um einiges größer als der Mauszeiger und so lassen sich kleine Objekte nicht konsistent auswählen. Um dem entgegenzuwirken fordert es ein passendes Design. So wird für Apps zum Beispiel eine Mindestgröße für Buttons vorgeschrieben \glqq On a touchscreen, buttons need a hit target of at least 44x44 points to accommodate a fingertip\grqq{}\cite{konventionen_buttonSize}.%
\myNewSection%
%Was: Keyboard unterschiede
Zur Texteingaben auf einem Pc wird meistens eine physische Tastatur verwendet, während Handys für diese Aufgabe eine Softwaretastatur nutzen. Durch die unterschiedliche Größe und der unterschiedlichen Bedienung bieten die beiden Tastaturvarianten verschiedene Vorteile.\newline%
	%Warum Pc: schneller und präziser + Tasten Anzahl und Tastenkombinationen
	So lässt es sich auf Hardwaretastaturen durch ihre Größe und dem Tippgefühl, schneller und präziser schreiben. Außerdem kann sie im Vergleich zu Handy-Tastaturen ungefähr das dreifache an Tasten darstellen\footnote{Full-size Tastaturen können bis zu 108 Tasten haben\cite{pcVsphone_pcKeyboardSize}. Die Standard deutsche Texttastatur auf iOS 16.1.2 hat 33 Tasten. Selbst nachgezählt und kontrolliert auf iPhone SE1, SE2 und 13 mini.}. Des Weiteren unterstützt Hardwaretastaturen Tastenkombinationen. % 
		%Auswirkung: TODO ~maybe~ remove
		Dadurch lassen sich Programme, welche lange Eingaben, viele verschiedene Symbole oder Tastenkombinationen benötigen, in der Regel auch besser auf dem Pc bedienen. Ein Beispiel dafür sind Programmiereditoren.\newline
	%Warum Handy: Mobiler und intuitiver
	Dafür ist die Handy-Tastatur mobiler und das Tippen fühlt sich oft intuitiver an, da sich die Tastatur je nach Anwendung variabel darstellen lässt. %
		%Beispiel:
		So bildet die Handy-Tastatur zum Beispiel für Taschenrechner-Anwendungen nur die benötigten Zahlen und Symbole ab und zeigt dementsprechend keine Buchstaben.%


\myComment{

	%%%->Benutzung%%%
	\myNewSection
	%Zusammenfassung
	Zusammenfassend hat das Handy also auch in diesem Vergleich an Leistung, in diesem Fall die präzise fehlerfreie und schnelle Eingabe, verloren und dafür an Mobilität gewonnen. Dafür fühlt es sich aber in vielen Anwendungen intuitiver und damit leichter zu benutzen an.
	
	\myNewSection
	\myTextTodo{
	-> Leistung(schnellere präzisere fehlerfreie eingabe) vs Portabilität\\
	-> um den entgegenzuwirken Optimierung(Design-Eingabe auf Anwendung Anpassen + Gesten + Touch) -> Intuitive
	}
	
	\myNewSection	
	%Auswirkung
	- Da die Handy-Tastatur also Fehleranfälliger und schwerer zu benutzen ist, nehmen wir an, dass Handy Nutzer weniger gern lange texte auf dem Handy schreiben. \newline%
	- Andersherum benutzen Handy Nutzer für kurze und einfachere Aufgaben lieber das Handy. Denn bei diesen Aufgaben benötigt es nicht viele verschiedene Tasten, Präzise Eingabe, schnelle Eingabe (-> bedienung wird nicht erschwert) und ist daher mit der Handy Eingabe (Touch, Geste + Anwendungstatsatur) (-> bedienung wird verbessert) intuitiver\newline%
	- Dafür muss die App aber dementsprechend auch optimiert werden. (leichte Anwendung, Touch + Gesten Unterstützung + richtige Tastatur)
	- Schwierige und Komplexe Aufgaben funktionieren dafür besser auf dem Pc, da lange Eingabe + verschiedene Symbole + Tastenkombinationen in der Regel für diese benötigt werden. 

}