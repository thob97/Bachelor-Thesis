\subsection{Eingabe}

\myNewSection
\textbf{User Input}: Die meisten Pc's werden mithilfe physischer Tastaturen bedient. Handys besitzen hingegen nur eine Software-Tastatur welches im Vergleich zu Hardware-Tastaturen in der Vielfalt der Tasten, Größe und Tippgefühl sehr eingeschränkt sind. So sind Sonderzeichen oft nur durch mehrere extra Schritte zu benutzen und die Funktionstasten fehlen oft ganz. Des Weiteren verleitet einen die kleine Größe und das fehlende Feedback einer echten Tastatur dazu langsamer zu schreiben und öfters Fehleingaben zu machen.\newline%
Da die Handy-Tastatur also Fehleranfälliger und schwerer zu benutzen ist, nehmen wir an, dass Handy Nutzer weniger gern lange texte auf dem Handy schreiben. \newline%
Neben der Tastatur wird auch eine Computermaus benutzt um den Pc zu steuern. Sie bietet im eine sehr präzise Eingabe für die Oberfläche. Auf dem Handy wird hingegen der Touch-Input als Maus-Ersatz benutzt. Dieser kann sich wohlmöglich intuitiver anfühlen, jedoch ist er um einiges unpräziser. Um dem Entgegenzuwirken fordert es ein passendes Design. So wird für Apps zum Beispiel eine Mindestgröße für Buttons vorgeschrieben "On a touchscreen, buttons need a hit target of at least 44x44 points to accommodate a fingertip"\cite{konventionen_buttonSize}. \newline%
Des Weiteren besitzen Computer-Mäuse mehrere Tasten für unterschiedliche Funktionen. Die wahrscheinlich meist benutzten sind der Links-click fürs Bestätigen, rechts-click für Optionen und scroll-wheel zum Blättern. Um diese etablierten Funktionen auch auf dem Handy zu ermöglichen werden Gesten benutzt. So gibt es die Wischgeste zum Blättern und long-press-Geste welcher oft den Rechtsklick ersetzt. \newline%
Jedoch geht es bei den Gesten nicht nur darum das Handy auf die Funktionen des Pc's anzupassen, sondern darüber hinaus gibt es auch noch weitere Handy eigene Gesten für Funktionen. Außerdem fühlt sich die Nutzung von Gesten oft Naturelle und intuitiver an. So benutzen aus meiner Erfahrung viele Menschen ihr Handy lieber zum Durchstöbern ihrer Bilder als den Pc, da die Möglichkeiten des Swipens und Herranzomens sich besser anfühlen, als mit Maus und Tastatur Bilder zu öffnen. \newline%
Während das Handy in der vorherigen Vergleichen immer Leistung gegen Mobilität getauscht hat, sieht es etwas anders aus. Zwar hat das Handy hier erneut an Mobilität gewonnen, da weder eine Hardware-Tatsatur noch eine Maus vonnöten ist und auch hat es in dem Sinne an Leistung verloren, da präzisen Eingaben erschwert wurden, dafür fühlen sich aber simple Aktionen(Gesten) oft um einiges besser auf dem Handy an. 

\myTextTodo{Typing on mobile devices is still a pain. Even alternative input options like swipe keyboards and voice-to-text are often inaccurate and slow users down. Users fear making mistakes and anticipate the interaction cost of data entry on their mobile devices. Particularly if the activity is personally important (like an important email to a client), users might choose a larger device to avoid mistakes.}