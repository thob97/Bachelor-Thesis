\subsection{Auswertung}\myTodo
%Einleitung
Nun folgt eine Aufzählung der gesammelten Erkenntnisse und Schlussfolgerungen dieses Abschnittes. % 
Alles was zuvor in den vorherigen Unterabschnitten und Vergleichen begründet wurde, wird hier nicht erneut [begründet/aufgesagt/erwähnt].%todo mabye warum?

% erfordert/bedarf/voraussetzt/benötigt/brauchen
% denn/darüberhinaus/außerdem/des weiteren/zusätzlich/daneben
\myNewSection
Pc's funktionieren für jene Aufgaben gut welche:
\begin{enumerate}%
	\item viel Leistung benötigt.\newline%
	Denn durch den dauerhaften Zugang zu Strom kann performante Hardware benutze werden. Das Ethernet bietet eine schnelle und stabile Internetverbindung. Und durch die Größe des Pc's kann große Hardware, wie zum Beispiel Festplatten mit viel Kapazität, verbaut werden.%
	\item schnelle, präzise oder vielfältige Eingaben erfordern.\newline%
	Denn die Maus und die Tastatur lassen sich durch ihre Größe und dem physischen Feedback schneller und präziser bedienen. Die vielfältige Eingaben wird durch die hohe Anzahl an Tasten und die Möglichkeit für Tastenkombinationen ermöglicht.%
	\item viele Informationen gleichzeitig darstellen oder brauchen.\newline%
	Denn mithilfe des großem Displays können viele Details dargestellt werden. Darüberhinaus kann durch das Multitasking weitere Informationen von anderen Anwendungen [besorgt/dargestellt/herangeschafft] werden.%
	\item viele Optionen und Konfigurationen anbieten oder benötigen.\newline%
	Denn die Hardware und das Betriebssystem kann bei Desktops beliebig ausgetauscht und konfiguriert werden. Außerdem ist ein großer Bildschirm hilfreich wenn viele Optionen dargestellt werden sollen.% 
	\item (lange dauern/viel Zeit benötigt).\newline% 
	Denn einerseits benötigen Aufgaben auf dem Pc generell etwas mehr Startzeit, da der [Einstiegs/anfangs] Aufwand größer ist. Andererseits helfen die schnellen und präzisen Eingaben dabei lange Aufgaben schneller bewältigen zu können.%   
\end{enumerate}%
%
\myNewSection
Handys funktionieren für jene Aufgaben gut welche:
\begin{enumerate}
	\item Ressourcen schonend sind oder nicht viel Leistung benötigen.\newline%
	Denn das Handy besitzt durch die Batterie nur begrenzt Strom, daher sind die Komponenten eher auf Effizienz ausgelegt. Das WLAN oder die mobilen Dates sind oft langsamer und instabiler im vergleich zum Ethernet. Und durch die kleine Größe sind auch nur kleine Festplatten mit begrenzter Kapazität möglich.%
	\item keine schnelle, präzise oder vielfältige Eingabe erfordern.\newline%
	Denn das kleine Display kann nur wenig Tasten gleichzeitig darstellen. Der Finger ist größer und damit auch zum Auswählen unpräziser als ein Mauszeiger. Und dementsprechend ist die Eingabe auch langsamer, da es sonst zu fehleranfällig würde.%
	\item nur wenig Informationen darstellen oder brauchen.\newline%
	Denn das Display vom Handy ist sehr klein und wirkliches Multitasking wird meistens auch nicht unterstützt.%
	\item ohne viele Optionen und Konfiguration auskommen.\newline%
	Denn das Betriebssystem und die Hardware sind [nicht modular/abgeschlossen/nicht änderbar]. Außerdem übernimmt das Betriebssystem viele eigentlich optionale Entscheidungen, wie zum Beispiel das Installationsverzeichnis von Anwendungen. Des Weiteren lassen sich durch das kleine Display auch nicht viele Optionen gleichzeitig darstellen.%
	\item kurzweilig sind.\newline%kurze Aufgaben welche man schnell lösen will
	Denn einerseits ist das Handy immer bei einem und es benötigt keinen [Einstiegs/anfangs] Aufwand. Und andererseits fällt das Abarbeiten von langen Aufgaben auf dem Handy schwerer, da die Eingabe langsamer und unpräziser ist.%
	\item (man Unterwegs lösen möchte.)\newline%
	Denn dadurch dass das Handy mobil ist, immer bei einem ist und immer über Internet verfügt ist genau das möglich.%
	\item (einfach und intuitiv zu lösen seien sollen).\newline%
	Denn erstens wird die Darstellung der App mithilfe von Richtlinien benutzerfreundlicher. Zweitens ist die Bedienung des Handys durch unteranderem die Gesten intuitiver. Und zuletzt muss sich der Nutzer keine Gedanken um die Konfiguration machen, das Betriebssystem, die Software und Hardware wurden bereits passend für das Handy konfiguriert.%
\end{enumerate}

\myNewSection%
%Was: Schlussfolgerung -> Handys leichte Aufgaben vs Pc's komplexe Aufgaben
(Schlussfolgernd/Zusammengefasst) und angesichts der Punkte der Aufzählung scheinen Handys für jene Aufgaben gut zu funktionieren welche als einfach, simpel oder leicht zu beschreiben sind. Während die Aufgaben des Pc's eher als aufwändig, komplex oder wichtig zu beschreiben sind.\newline%
%Begründung/Quellen
	%Warum: Studie -> Pc wichtig + Handy simpel
	Die Erkenntnisse lassen sich auch durch eine Studie bekräftigen. So scheinen die Befragten, Pc's lieber für wichtige Aufgaben zu nutzen. Währenddessen werden Handys als leichter zu benutzen bewertet, was zu den einfachen und simplen Aufgaben des Handys passen würde\cite{pcVsphone_easyUseVsImportantTask}.\newline%
	%allgemeine Nutzerverhalten
	Aber auch das allgemeine Nutzerverhalten im Internet deutet darauf hin.
		%Warum: Googel Suche -> Handy simpel + Pc Komplex
		So sind einerseits die Suchanfragen auf Google je nach Gerät anders. Auf dem Pc werden zum Beispiel eher aufwändige und komplexe Kategorien wie Computer, Elektronik, Arbeit, Ausbildung und Wissenschaft angefragt. Während auf dem Handy oft eher nach simplen und kurzen Aufgaben wie nach Essen, Nachrichten und Sport gesucht wird\cite{pcVsphone_onWebsites_DevicesDistrubition_TimeSpent_Bouncrate_PageViews_Categories}.\newline%
			%Warum: Website Visits
			Das diese Aufgaben auch wirklich einfacher sind, kann anhand der Aufrufe und Dauer von Webseiten beobachten werden. So kommen zum Beispiel 68\% alles Webseiten-Aufrufe von Handys, aber sie machen nur 33\% der Zeit die auf Webseiten verbracht werden aus\cite{pcVsphone_onWebsites_DevicesDistrubition_TimeSpent_Bouncrate_PageViews_Categories}.\newline% 
		%Auswirkung:
		Das Handy wird beim Surfen also eher für kurzweilige und schnelle Aufgaben benutzt, während sich auf dem Pc mehr Zeit gelassen wird.\newline%
		%Warum: Emails
		Ein ähnliches Verhalten lässt sich auch bei Emails feststellen. Laut einem Survey von Adobe werden für Emails mit Arbeitsthemen lieber der Pc genutzt während für private Emails stattdessen zum Handy gegriffen wird. Da Arbeit oft mit aufwändigen, komplexen und wichtigen Aufgaben verbunden wird, deutet auch diese Aussage mit der Erkenntnis überein\cite{pcVsphone_personalEmailsVsWorkEmails}.