\subsection{Auswertung}\myCheckmark
%Einleitung
Nun folgt eine Aufzählung der gesammelten Erkenntnisse und Schlussfolgerungen dieses Abschnittes. % 
%Alles was zuvor in den vorherigen Unterabschnitten begründet wurde, wird hier nicht erneut [gleich detailreich/nur oberflächlich] [begründet/aufgesagt/erwähnt].%todo mabye warum?

% erfordert/bedarf/voraussetzt/benötigt/brauchen
% denn/darüberhinaus/außerdem/des weiteren/zusätzlich/daneben
\myNewSection
Pc's scheinen für jene Aufgaben gut zu funktionieren welche:
\begin{enumerate}%
	\item viel Leistung benötigt.\newline%
	Denn durch den dauerhaften Zugang zu Strom kann performante Hardware benutzt werden. Das Ethernet bietet eine schnelle und stabile Internetverbindung. Und durch die Größe des Pc's kann große Hardware, wie zum Beispiel Festplatten mit viel Kapazität, verbaut werden.%
	%
	\item schnelle, präzise oder vielfältige Eingaben erfordern.\newline%
	Die Maus und die Tastatur lassen sich durch ihre Größe und dem physischen Feedback schneller und präziser bedienen. Die vielfältigen Eingaben werden durch die hohe Anzahl an Tasten und die Möglichkeit für Tastenkombinationen ermöglicht.%
	%
	\item viele Informationen gleichzeitig darstellen oder benötigt.\newline%
	Denn mithilfe des großen Displays können viele Details dargestellt werden. Darüber hinaus können durch das Multitasking weitere Informationen von anderen Anwendungen dargestellt werden.%
	%
	\item viele Optionen und Konfigurationen anbieten oder benötigt.\newline%
	Die Hardware und das Betriebssystem kann bei Desktops beliebig ausgetauscht und konfiguriert werden. Außerdem ist ein großer Bildschirm hilfreich wenn viele Optionen dargestellt werden sollen.% 
	%
	\item langwierig sind oder viel Zeit benötigt.\newline% 
	Einerseits benötigen Aufgaben auf dem Pc generell mehr Startzeit, da der Einstiegsaufwand größer ist. Andererseits helfen die schnellen und präzisen Eingaben dabei lange Aufgaben schneller bewältigen zu können.%   
	%
\end{enumerate}%
%
\myNewSection
Handys scheinen hingegen für jene Aufgaben gut zu funktionieren welche:
\begin{enumerate}
	\item Ressourcenschonend sind und nicht viel Leistung benötigen.\newline%
	Denn das Handy besitzt durch die Batterie nur begrenzt Strom und die Komponenten sind daher eher auf Effizienz ausgelegt. Das WLAN oder die mobilen Daten sind oft langsamer und instabiler im Vergleich zum Ethernet. Und durch die kleine Größe sind auch nur kleine Festplatten mit begrenzter Kapazität möglich.%
	%
	\item keine schnelle, präzise und vielfältige Eingabe erfordern.\newline%
	Das kleine Display kann nur wenig Tasten gleichzeitig darstellen. Der Finger ist größer und damit zum Auswählen unpräziser als ein Mauszeiger. Dementsprechend ist die Eingabe auch langsamer, da sie sonst zu fehleranfällig werden würde.%
	%
	\item nur wenig Informationen darstellen oder benötigen.\newline%
	Das Display vom Handy ist relativ klein und Multitasking wird meistens auch nicht unterstützt.%
	%
	\item ohne viele Optionen und Konfiguration auskommen.\newline%
	Das Betriebssystem und die Hardware sind nicht änderbar. Außerdem übernimmt das Betriebssystem viele eigentlich optionale Entscheidungen, wie zum Beispiel das Installationsverzeichnis von Anwendungen. Des Weiteren lassen sich durch das relativ kleine Display nicht viele Optionen gleichzeitig darstellen.%
	%
	\item kurzweilig sind oder wenig Zeit benötigen.\newline%kurze Aufgaben welche man schnell lösen will
	Das Handy steht stets zur Disposition des Nutzers und es erfordert nur einen geringen Einstiegsaufwand. Des weiteren fällt das Abarbeiten von langen Aufgaben auf dem Handy schwerer, da die Eingabe langsamer und unpräziser ist.%
	%
	\item lohnenswert sind unterwegs zu lösen\newline%
	Da das Handy mobil ist, sich meistens beim Eigentümer befindet und grundsätzlich über Internet verfügt, ist es möglich Aufgaben unterwegs zu bearbeiten.%
	%
	\item einfach und intuitiv seien sollen.\myTodo\newline%
	Denn erstens wird die Darstellung der App mithilfe von Richtlinien benutzerfreundlicher. Zweitens ist die Bedienung des Handys durch unteranderem die Gesten intuitiver. Und zuletzt muss sich der Nutzer keine Gedanken um die Konfiguration machen, das Betriebssystem, die Software und Hardware wurden bereits passend für das Handy konfiguriert.%
	%
\end{enumerate}

\myNewSection
%Was: Quellen
Diese gesammelten Erkenntnisse lassen sich auch durch Statistiken und Studien deuten. %
%Warum: Studie -> Pc wichtig + Handy simpel
So nutzen zum Beispiel Befragte laut einer Studie, Pc's lieber für wichtige Aufgaben. Währenddessen werden Handys als leichter zu benutzen bewertet, was zu den einfachen und simplen Aufgaben des Handys passen würde\cite{pcVsphone_easyUseVsImportantTask}.\newline%
%allgemeine Nutzerverhalten
Aber auch das allgemeine Nutzerverhalten im Internet deutet darauf hin.
	%Warum: Googel Suche -> Handy simpel + Pc Komplex
	So sind einerseits die Suchanfragen auf Google je nach Gerät anders. Auf dem Pc werden zum Beispiel eher aufwändige und komplexe Kategorien wie Computer, Elektronik, Arbeit, Ausbildung und Wissenschaft angefragt. Während auf dem Handy oft eher nach simplen und kurzen Themen wie Essen, Nachrichten und Sport gesucht wird\cite{pcVsphone_onWebsites_DevicesDistrubition_TimeSpent_Bouncrate_PageViews_Categories}.\newline%
		%Warum: Website Visits
		Das diese Aufgaben auch wirklich einfacher und kurzweiliges sind und Handys dafür benutzt werden, kann anhand der Aufrufe und Dauer von Webseiten beobachten werden. So kommen zum Beispiel 68\% alles Webseiten-Aufrufe von Handys, aber sie machen nur 33\% der Zeit die auf Webseiten verbracht werden aus\cite{pcVsphone_onWebsites_DevicesDistrubition_TimeSpent_Bouncrate_PageViews_Categories}. Das bedeutet also, das Handys für sehr viele kleine Suchanfragen genutzt werden. Pc's werden [bei diesem Thema] hingegen für wenigere Lange Aufgaben genutzt.\newline% 
		%Warum: Handy threeSeconds
		Ein weiteres Indiz dafür, dass die kürze und dementsprechend die Einfachheit von Aufgaben für Handys wichtig sind, ist eine Messung von Google. So verlassen rund 53\% von Handy Nutzern eine Webseite, wenn sie länger als drei Sekunden braucht zu laden\cite{pcVsphone_threeSeconds}.\newline%
		%Warum: Emails
		Ein ähnliches Verhalten lässt sich auch bei Emails feststellen. Laut einem Survey von Adobe werden für Emails mit Arbeitsthemen lieber der Pc genutzt während für private Emails stattdessen zum Handy gegriffen wird\cite{pcVsphone_personalEmailsVsWorkEmails}. Da angenommen wird, dass Arbeit eher mit langwierige Aufgaben welche viele Informationen benötigen verbunden werden, während private Aufgaben eher kurzweilig und einfach sind, wie zum Beispiel das aussuchen von Essen oder dem Schreiben einer Nachricht, stimmt auch dieses [Indiz] mit den Erkenntnissen überein.

%%%%%%%%%%%%%Backup
\myComment{

	\myNewSection%
	\myTextTodo{Maybe nicht zusammenfassen zu Einfache/Schwere sondern einfach nur belegen? + threeSecondLoad erwähnen?}\newline%
	%Was: Schlussfolgerung -> Handys leichte Aufgaben vs Pc's komplexe Aufgaben
	(Schlussfolgernd/Zusammengefasst) und angesichts der Punkte der Aufzählung scheinen Handys für jene Aufgaben gut zu funktionieren welche als einfach, simpel oder leicht zu beschreiben sind. Während die Aufgaben des Pc's eher als aufwändig, komplex oder wichtig zu beschreiben sind.\newline%
	%Begründung/Quellen
		%Warum: Studie -> Pc wichtig + Handy simpel
		Die Erkenntnisse lassen sich auch durch eine Studie bekräftigen. So scheinen die Befragten, Pc's lieber für wichtige Aufgaben zu nutzen. Währenddessen werden Handys als leichter zu benutzen bewertet, was zu den einfachen und simplen Aufgaben des Handys passen würde\cite{pcVsphone_easyUseVsImportantTask}.\newline%
		%allgemeine Nutzerverhalten
		Aber auch das allgemeine Nutzerverhalten im Internet deutet darauf hin.
			%Warum: Googel Suche -> Handy simpel + Pc Komplex
			So sind einerseits die Suchanfragen auf Google je nach Gerät anders. Auf dem Pc werden zum Beispiel eher aufwändige und komplexe Kategorien wie Computer, Elektronik, Arbeit, Ausbildung und Wissenschaft angefragt. Während auf dem Handy oft eher nach simplen und kurzen Aufgaben wie nach Essen, Nachrichten und Sport gesucht wird\cite{pcVsphone_onWebsites_DevicesDistrubition_TimeSpent_Bouncrate_PageViews_Categories}.\newline%
				%Warum: Website Visits
				Das diese Aufgaben auch wirklich einfacher und kurzweiliges sind, kann anhand der Aufrufe und Dauer von Webseiten beobachten werden. So kommen zum Beispiel 68\% alles Webseiten-Aufrufe von Handys, aber sie machen nur 33\% der Zeit die auf Webseiten verbracht werden aus\cite{pcVsphone_onWebsites_DevicesDistrubition_TimeSpent_Bouncrate_PageViews_Categories}.\newline% 
			%Auswirkung:
			Das Handy wird beim Surfen also eher für kurzweilige und schnelle Aufgaben benutzt, während sich auf dem Pc mehr Zeit gelassen wird.\newline%
			%Warum: Emails
			Ein ähnliches Verhalten lässt sich auch bei Emails feststellen. Laut einem Survey von Adobe werden für Emails mit Arbeitsthemen lieber der Pc genutzt während für private Emails stattdessen zum Handy gegriffen wird. Da Arbeit oft mit aufwändigen, komplexen und wichtigen Aufgaben verbunden wird, deutet auch diese Aussage mit der Erkenntnis überein\cite{pcVsphone_personalEmailsVsWorkEmails}.

}