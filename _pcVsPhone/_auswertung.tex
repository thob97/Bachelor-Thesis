\subsection{Auswertung}
%Einleitung
Im Folgenden werden die Erkenntnisse und Schlussfolgerungen dieses Abschnittes aufgelistet. % 
%Alles was zuvor in den vorherigen Unterabschnitten begründet wurde, wird hier nicht erneut [gleich detailreich/nur oberflächlich] [begründet/aufgesagt/erwähnt].%todo mabye warum?
%
% erfordert/bedarf/voraussetzt/benötigt/brauchen
% denn/darüberhinaus/außerdem/des weiteren/zusätzlich/daneben
\newline%
\myNewSection%
Pc's scheinen für jene Aufgaben gut zu funktionieren welche: %
\begin{enumerate}%
	\item viel Leistung benötigen.\newline%
	Durch den ständigen Zugang zur Stromversorgung können leistungsstarke Hardwarekomponenten genutzt werden. Zudem ermöglicht eine kabelgebundene Ethernet-Verbindung eine schnelle und stabile Internetverbindung. Aufgrund der Größe des PCs können auch große Hardwarekomponenten, wie etwa Festplatten mit einer hohen Kapazität, eingebaut werden.%
	%
	\item schnelle, präzise oder vielfältige Eingaben erfordern.\newline%
	Die Maus und die Tastatur ermöglichen aufgrund ihrer Größe und des haptischen Feedbacks eine schnellere und präzisere Bedienung. Die vielfältigen Eingaben werden durch die hohe Anzahl an Tasten und die Möglichkeit für Tastenkombinationen ermöglicht.%
	%
	\item viele Informationen gleichzeitig darstellen oder benötigen.\newline%
	Dank des großen Displays und der Möglichkeit, mehrere Displays zu verwenden, können viele Details gleichzeitig dargestellt werden. Zudem erlaubt Multitasking die gleichzeitige Anzeige von Informationen aus verschiedenen Anwendungen.%
	%
	\item viele Optionen und Konfigurationen anbieten oder benötigen.\newline%
	Die Hardware, das Betriebssystem und weitere Software können bei Desktops nach Belieben ausgetauscht und konfiguriert werden. Zudem ist ein großer Bildschirm von Vorteil, wenn viele Optionen dargestellt werden müssen.% 
	%
	\item langwierig sind oder viel Zeit benötigen.\newline% 
	Einerseits benötigen Aufgaben auf dem Pc generell mehr Startzeit, da der Einstiegsaufwand größer ist. Andererseits helfen die schnellen und präzisen Eingaben dabei lange Aufgaben schneller bewältigen zu können.%   
	%
\end{enumerate}%
%
\myNewSection
Handys scheinen hingegen für jene Aufgaben gut zu funktionieren welche: %
\begin{enumerate}
	\item Ressourcenschonend sind und nicht viel Leistung benötigen.\newline%
	Denn das Handy verfügt aufgrund der begrenzten Batteriekapazität über begrenzten Strom, weshalb die Komponenten auf Effizienz ausgelegt sind. Zusätzlich sind WLAN- oder mobile Datenverbindungen oft langsamer und instabiler im Vergleich zu einer kabelgebundenen Ethernet-Verbindung. Darüber hinaus ist aufgrund der begrenzten Größe des Geräts nur eine kleine Festplatte mit eingeschränkter Kapazität möglich.%
	%
	\item keine schnelle, präzise und vielfältige Eingabe erfordern.\newline%
	Das relativ kleine Display kann nur wenige Tasten gleichzeitig darstellen. Der Finger ist größer als ein Mauszeiger und daher unpräziser beim Auswählen. Folglich ist die Eingabe oft langsamer und fehleranfälliger.%
	%
	\item nur wenig Informationen darstellen oder benötigen.\newline%
	Das Display vom Handy ist relativ klein und Multitasking wird meistens nicht unterstützt.%
	%
	\item ohne viele Optionen und Konfigurationen auskommen.\newline%
	Das Betriebssystem und die Hardware sind nicht anpassbar. Zusätzlich übernimmt das Betriebssystem viele eigentlich optionale Entscheidungen, wie beispielsweise das Installationsverzeichnis von Anwendungen. Außerdem können aufgrund des relativ kleinen Displays nicht viele Optionen gleichzeitig angezeigt werden.%
	%
	\item kurzweilig sind oder wenig Zeit benötigen.\newline%kurze Aufgaben welche man schnell lösen will
	Das Handy steht dem Nutzer jederzeit zur Verfügung und erfordert nur einen geringen Einstiegsaufwand. Des weiteren fällt das Abarbeiten langer Aufgaben auf dem Handy aufgrund der langsameren und unpräziseren Eingabe schwieriger.%
	%
	\item lohnenswert sind unterwegs zu lösen.\newline%
	Da das Handy mobil ist und sich üblicherweise beim Besitzer befindet sowie in der Regel über eine Internetverbindung verfügt, können Aufgaben unterwegs bearbeitet werden.%
	%
	\item einfach und intuitiv seien sollen.\newline%
	Denn zum einen wird die Darstellung der App mithilfe von Richtlinien benutzerfreundlicher. Zum anderen ist die Bedienung des Handys durch unteranderem die Gesten intuitiver. Und schließlich muss sich der Nutzer keine Gedanken über die Konfiguration machen, da das Betriebssystem, die Software und die Hardware bereits passend für das Handy voreingestellt sind.%
	%
\end{enumerate}

\myNewSection
%Was: Quellen
Diese gesammelten Erkenntnisse lassen sich auch durch Statistiken und Studien deuten. %
%Warum: Studie -> Pc wichtig + Handy simpel
Eine Studie zeigt zum Beispiel, dass Befragte Computer lieber für wichtige Aufgaben nutzen. Handys hingegen werden als einfacher zu bedienen bewertet, was zu der Erkenntnis der einfachen und intuitiven Aufgaben des Handys passt \cite{pcVsphone_easyUseVsImportantTask}.\newline%
%allgemeine Nutzerverhalten
Aber auch das allgemeine Nutzerverhalten im Internet deutet darauf hin.
	%Warum: Googel Suche -> Handy simpel + Pc Komplex
	Auf Google werden beispielsweise je nach Gerät unterschiedliche Suchanfragen gestellt. Auf dem PC werden eher aufwändige und komplexe Kategorien wie Computer, Elektronik, Arbeit, Ausbildung und Wissenschaft angefragt, während auf dem Handy oft eher einfache und kurze Themen wie Essen, Nachrichten und Sport gesucht werden \cite{pcVsphone_onWebsites_DevicesDistrubition_TimeSpent_Bouncrate_PageViews_Categories}.\newline%
		%Warum: Website Visits
		Dass diese Aufgaben wirklich einfacher und schneller sind und Handys dafür bevorzugt werden, lässt sich anhand der Aufrufe und Dauer von Webseitbesuchen beobachten. Obwohl 68\% aller Webseitenaufrufe von Handys stammen, machen sie nur 33\% der auf Webseiten verbrachten Zeit aus \cite{pcVsphone_onWebsites_DevicesDistrubition_TimeSpent_Bouncrate_PageViews_Categories}. Das bedeutet, dass Handys hauptsächlich für viele kurze Suchanfragen genutzt werden. PCs werden hingegen eher für weniger, aber dafür längere Aufgaben genutzt.\newline% 
		%Warum: Handy threeSeconds
		Ein weiteres Indiz dafür, dass für Handys die Kürze und Einfachheit von Aufgaben wichtig sind, ist eine Messung von Google. Demnach verlassen etwa 53\% der Handy-Nutzer eine Webseite, wenn sie länger als drei Sekunden benötigt, um zu laden \cite{pcVsphone_threeSeconds}.\newline%
		%Warum: Emails
		Ein ähnliches Verhalten lässt sich auch bei E-Mails feststellen. Laut einer Umfrage von Adobe werden für berufliche E-Mails eher PCs genutzt, während für private E-Mails stattdessen zum Handy gegriffen wird\cite{pcVsphone_personalEmailsVsWorkEmails}. Da angenommen wird, dass Arbeit eher mit langwierigen Aufgaben verbunden ist, die viele Informationen erfordern, während private Aufgaben eher kurzweilig und einfach sind, wie zum Beispiel das Aussuchen von Essen oder das Schreiben einer Nachricht, stimmt auch dieses Indiz mit den Erkenntnissen überein.%
%
%
%
%%%%%%%%%%%%%%Backup
%\myComment{
%
%	\myNewSection%
%	\myTextTodo{Maybe nicht zusammenfassen zu Einfache/Schwere sondern einfach nur belegen? + threeSecondLoad erwähnen?}\newline%
%	%Was: Schlussfolgerung -> Handys leichte Aufgaben vs Pc's komplexe Aufgaben
%	(Schlussfolgernd/Zusammengefasst) und angesichts der Punkte der Aufzählung scheinen Handys für jene Aufgaben gut zu funktionieren welche als einfach, simpel oder leicht zu beschreiben sind. Während die Aufgaben des Pc's eher als aufwändig, komplex oder wichtig zu beschreiben sind.\newline%
%	%Begründung/Quellen
%		%Warum: Studie -> Pc wichtig + Handy simpel
%		Die Erkenntnisse lassen sich auch durch eine Studie bekräftigen. So scheinen die Befragten, Pc's lieber für wichtige Aufgaben zu nutzen. Währenddessen werden Handys als leichter zu benutzen bewertet, was zu den einfachen und simplen Aufgaben des Handys passen würde\cite{pcVsphone_easyUseVsImportantTask}.\newline%
%		%allgemeine Nutzerverhalten
%		Aber auch das allgemeine Nutzerverhalten im Internet deutet darauf hin.
%			%Warum: Googel Suche -> Handy simpel + Pc Komplex
%			So sind einerseits die Suchanfragen auf Google je nach Gerät anders. Auf dem Pc werden zum Beispiel eher aufwändige und komplexe Kategorien wie Computer, Elektronik, Arbeit, Ausbildung und Wissenschaft angefragt. Während auf dem Handy oft eher nach simplen und kurzen Aufgaben wie nach Essen, Nachrichten und Sport gesucht wird\cite{pcVsphone_onWebsites_DevicesDistrubition_TimeSpent_Bouncrate_PageViews_Categories}.\newline%
%				%Warum: Website Visits
%				Das diese Aufgaben auch wirklich einfacher und kurzweiliges sind, kann anhand der Aufrufe und Dauer von Webseiten beobachten werden. So kommen zum Beispiel 68\% alles Webseiten-Aufrufe von Handys, aber sie machen nur 33\% der Zeit die auf Webseiten verbracht werden aus\cite{pcVsphone_onWebsites_DevicesDistrubition_TimeSpent_Bouncrate_PageViews_Categories}.\newline% 
%			%Auswirkung:
%			Das Handy wird beim Surfen also eher für kurzweilige und schnelle Aufgaben benutzt, während sich auf dem Pc mehr Zeit gelassen wird.\newline%
%			%Warum: Emails
%			Ein ähnliches Verhalten lässt sich auch bei Emails feststellen. Laut einem Survey von Adobe werden für Emails mit Arbeitsthemen lieber der Pc genutzt während für private Emails stattdessen zum Handy gegriffen wird. Da Arbeit oft mit aufwändigen, komplexen und wichtigen Aufgaben verbunden wird, deutet auch diese Aussage mit der Erkenntnis überein\cite{pcVsphone_personalEmailsVsWorkEmails}.
%
%}