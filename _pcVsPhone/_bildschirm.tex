\subsection{Bildschirm}

\myNewSection
\textbf{Bildschirm}: Die Bildschirme von Pc's sind nicht nur um einiges größer als die des Handys, sondern es können auch mehrere gleichzeitig benutzt werden \cite{pcVsphone_screenResolutionStats}\cite{pcVsphone_screenResolutionToSize}\footnote{Abschätzung der Monitorgröße durch die Monitorauflösung: die meisten Monitore sind 14 bis 23 Zoll groß}.\newline%
Dadurch können Pc's sehr viele Informationen gleichzeitig darstellen. Auf den kleineren Bildschirm des Smartphones lässt sich hingegen nur ein kleinerer Teil davon abbilden. Wenn man als Beispiel doch versuchen würde, die Darstellung des Pc's auf ein Handy abzubilden, müsst die Anzeige um ein vielfaches verkleinert werden, und dadurch würde die Lesbarkeit enorm leiden. Vielleicht erfreuen sich deshalb größere Handy-Bildschirme immer mehr Beliebtheit \cite{pcVsphone_screenSizePhone}.\newline% 
Des Weiteren Unterscheiden sich Handys und Pc's auch in Ihrer Ausrichtung des Bildschirms. Während Pc-Bildschirme meist landscape-orientation operieren, werden Handys meist in portrait-orientation benutzt.\newline%
Durch die verschiedene Ausrichtung und Bildschirm Größe sollte einen also Bewusst werden, dass unterschiedliche Darstellung verschieden gut auf den beiden Geräten funktionieren.
So wird die Darstellung einer großen detaillierten Tabelle mit vielen Columns auf einen Pc wahrscheinlich leichter Fallen. Dafür lassen sich auf Handys lange vertikale Listen mit  einfacheren Details besser überfliegen (als auf dem Pc).\newline%
Deshalb muss sich für die Darstellung auf dem Handy besonders Gedanken zum Design gemacht werden. Wenn zu viel oder zu wenig Informationen Dargestellt sind wirkt sich das schnell auf die Leserlichkeit und Übersichtlichkeit und damit auch auf die Qualität der App aus. Für Pc-Anwendungen gilt zwar das gleiche, aber in der Regel ist es durch den größeren Bildschirm schwerer eine unübersichtliche Anwendung zu erstellen. Daher gibt für Apps bereits einige Guidelines\cite{konventionen_guidelinesApple}\cite{konventionen_guidelinesGoogle}.\newline%
Das diese Richtlinien viel Beachtung geschenkt wird und sie allmählich zu Konventionen werden anstatt nur irgendwelche Wunschvorstellung zu sein, sieht man daran, dass die beiden größten Handyverkäufern Apple und Google diese Richtlinien pflegen und anwenden\cite{pcVsphone_vendors}\footnote{Samsung gilt als Partner von Google und wird daher als Ersatz in diesem Vergleich genommen\myTodo \myTextTodo{wirklich Partner?}}. Die Verbreitung/Anwendung dieser Richtlinien hat den gewünschten Vorteil, dass sich verschiedene Apps ähnlich bedienen lassen und anfühlen. Dadurch muss der Nutzer nicht für jede Anwendung eine neue Bedienung lernen. Das ist wahrscheinlich ein großer Grund dafür, warum Handys als leichter zu bedienen bewertet werden\cite{pcVsphone_easyUseVsImportantTask}. Außerdem gibt das auch ein Indiz warum auf dem Handy Apps sehr viel beliebter sind als Websites\cite{pcVsphone_mobileAppVsWebTimeSpent}, denn für Webseiten muss keine der Appeigenen Richtlinien befolgt werden.\newline%

Das Handy gewinnt also auch hier dank seines kleineren Bildschirms an Mobilität. Dafür verliert es aber an der Anzahl an der gleichzeitig darstellbaren Informationen. Außerdem muss sich nun detaillierte Gedanken über die Darstellung der App gemacht werden, da ein schlechtes Design auf einem kleinen Bildschirm weniger verzeihend ist. Wenn einem dies jedoch gelingt, kann die App im Vergleich zu Pc Anwendungen um einiges intuitiver und leichter werden.

 
\myComment{Da diese Konventionen bereits sehr etabliert sind, immerhin werden sie von Android und Apple empfohlen, die beiden größten Handy Hersteller \cite{}, werden sie auch sehr häufig in Apps eingesetzt + werden in eigenen Apps verwendet. -> nutzer gewöhnt sich an konventionen -> neue nutzer finden apps simpler (da alle das gleiche navigaions muster) verwenden -> auf pc's gibt es keine solche konventionen/regeln -> schlechtes design lässt sich auf dem pc allgemein eher verzeihen, da großer bildschirm + präzisere eingabe -> apps simpler / pc anwendungen oft komplizierter quotate? hier benuzten oder in "hier, usage, nfA, einleitung konventionen"?}
		
\myNewSection
\myTextTodo{n.f.A: Also sollte einen klar werden, dass man sich für die grafische-Oberfäche der App andere Gedanken machen muss als wie für eine Webseite auf einem Pc. Das wir dies als sehr wichtig betrachten haben wir bereits in ... betont (Benutzbarkeit).}
\newline
\myTextTodo{Less content in each screenful means less context and a higher cognitive load for users.}
\newline
\myTextTodo{Kleinerer Bildschirm: Designer müssen sich noch mehr Gedanken machen wie man die Software gut bedienen kann, da..}