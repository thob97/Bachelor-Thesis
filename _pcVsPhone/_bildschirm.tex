\subsection{Bildschirm}\myCheckmark
%Warum: Bildschirmgröße
Pc's bieten einem die Möglichkeit mehrere Monitore gleichzeitig zu nutzen und die Bildschirme sind in der Regel auch um einiges größer\cite{pcVsphone_screenResolutionStats}\cite{pcVsphone_screenResolutionToSize}\footnote{Die Monitorgröße wurde durch die Monitorauflösung abgeschätzt: die meisten Monitore sind 14 bis 23 Zoll groß}.\newline%
	%Was: Anzahl an darstellbaren Informationen
	Dadurch kann der Pc sehr viele Informationen gleichzeitig darstellen.\newline%
%Warum: Hoch- vs Querformat
Außerdem Unterscheiden sich Handys und Pc's auch in ihrer Ausrichtung des Bildschirms. Während Pc-Bildschirme meist im Landschaftsmodus operieren, werden Handys meist im Porträtmodus benutzt.\newline%
	%Was: Verschiedene Darstellungen/Anwendungen
	Durch die verschiedene Ausrichtung und Bildschirmgröße folgt die Erkenntnis, dass unterschiedliche Darstellung [und damit auch Anwendungen] verschieden gut auf den beiden Geräten funktionieren. %
		%Beispiel
		So wird die Darstellung einer großen detaillierten Tabelle auf einen Pc wahrscheinlich leichter Fallen. Dafür lassen sich auf Handys lange vertikale Listen mit einfacheren Details besser überfliegen.\newline%
%Auswirkung: Design
Dementsprechend muss sich für das Handy Gedanken zum \nameref{section:design} gemacht werden. %
	%Begründung :Richtlinien + Google/Apple
	Diese Erkenntnis scheint auch bereits etabliert zu sein. So haben Apple und Google für die beiden beliebtesten Plattformen iOS und Android\cite{pcVsphone_mobileOperatingSystem} ihre eigenen Konventionen veröffentlicht\cite{konventionen_guidelinesApple, konventionen_guidelinesGoogle}. %
	%Warum: intuitiver + einfacher
	Durch die Anwendung von diesen Richtlinien kann die App intuitiver und einfacher werden. %
		%Begründung: Regeln abgestimmt für die Eigenschaften des Handys
		Das liegt einerseits daran, dass diese Richtlinien, [sorgfältig ausgewählte] Regeln abgestimmt auf die Eigenschaften des Handys, nennen. So ist zum Beispiel eine Regel, dass Buttons eine mindestens 44x44 Punkte groß sein müssen, um die Größe des Fingers zu berücksichtigen\cite{konventionen_buttonSize}.\newline%
		%Begründung: Reichweite + Einfluss -> Verbreitung -> Konvention -> ähnliches Design+Zurechtfinden
		Andererseits spielt wahrscheinlich auch die Verbreitung dieser Richtlinien eine Rolle. So wirkt sich die große Reichweite und der Einfluss von Apple und Google gewiss auch auf die Bekanntheit und Verbreitung ihrer Richtlinien aus. %
		%Konvention
		Daher wird vermutet, dass immer mehr Apps diese Richtlinien folgen und die Regeln dadurch allmählich zu Konventionen werden. %
		%ähnliches Design -> intuitiver
		Dadurch würden sich verschiedene Apps ähnlich bedienen lassen und Nutzer müssten nicht für jede Anwendung eine neue Bedienung lernen. Dementsprechend findet sich Nutzer in neuen Apps schneller zurecht.\newline%
		%Quelle: Richtlinien sind nützlich :TODO: quelle wirklich hier verwenden? oder in auswertung oder design oder nfA?
		Ein Indiz dafür, dass Richtlinien und dessen Verbreitung Anwendungen einfacher, intuitiver oder wenigstens beliebter machen ist, dass auf dem Handy Apps sehr viel beliebter als Webseiten sind\cite{pcVsphone_mobileAppVsWebTimeSpent}. Denn für Webseiten muss keine der Regeln befolgt werden. [Für Apps gibt es einige, welche zwingend für die Veröffentlichung der App benötigt werden.]














\myComment{

		%Old Backup: TODB Remove
		%Die Richtlinien sowie dessen Verbreitung sind wahrscheinlich ein [großer] Grund dafür, warum Handys als leichter zu bedienen bewertet werden\cite{pcVsphone_easyUseVsImportantTask}. Außerdem gibt das auch ein Indiz warum auf dem Handy Apps sehr viel beliebter sind als Websites\cite{pcVsphone_mobileAppVsWebTimeSpent}, denn für Webseiten muss keine der Appeigenen Richtlinien befolgt werden. Bei Apps gibt es einige Regeln, welche zwingend für die Veröffentlichung der App benötigt werden.\newline%


	%%%->Benutzung%%%
	
	%Zusammenfassung
	Die Bildschirme von Handy und Pc unterscheiden sich. So ist der Bildschirm vom Handy kleiner und dadurch mobiler, während der Pc Monitor durch seine Größe viele Informationen gleichzeitig darstellen kann. Daher kam es zu der Erkenntnis, dass unterschiedliche Anwendungen und Darstellungen verschieden gut auf den beiden Geräten funktionieren.
	Um der Informationsarmut des Handys entgegenzuwirken muss muss sich nun detaillierte Gedanken über die Darstellung der App gemacht werden. Wenn einem dies gelingt, kann die App im Vergleich zu Pc Anwendungen sogar um einiges intuitiver und leichter werden.	
		
	
	
	
			
	\myNewSection
	%Auswirkung
	- Verschiedene Darstellungen und Anwendungen können unterschiedlich gut abgebildet werden.\\
	-> Das Handy braucht um gut zu funktionieren andere Designentscheidungen als Pc's\\
	-> Schwere Aufgaben (viele Details) funktionieren besser auf dem Pc\\
	-> Nutzer benutzen Handys gerne, da sie durch das Design einfach und intuitive zu nutzen sind\\
	-> Einfache Aufgaben funktionieren gut auf dem Handy, da genau diese Aufgaben oft nicht viele Informationen Darstellen müssen + Handy durch andere Punkte bereits intuitive und einfach, was diesen Punkt noch mehr verstärkt. 
	- 
	
	%Was
	%Warum
	%Zusammenfassung
	%Auswirkung
	
	\myNewSection
	\myTextTodo{
	-> Leistung(Darstellbare Informationen) vs Portabilität\\
	- um dem entgegenzuwirken Optimierung(Design) -> intuitive + einfach\\ 
	}

}




\myComment{Da diese Konventionen bereits sehr etabliert sind, immerhin werden sie von Android und Apple empfohlen, die beiden größten Handy Hersteller \cite{}, werden sie auch sehr häufig in Apps eingesetzt + werden in eigenen Apps verwendet. -> nutzer gewöhnt sich an konventionen -> neue nutzer finden apps simpler (da alle das gleiche navigaions muster) verwenden -> auf pc's gibt es keine solche konventionen/regeln -> schlechtes design lässt sich auf dem pc allgemein eher verzeihen, da großer bildschirm + präzisere eingabe -> apps simpler / pc anwendungen oft komplizierter quotate? hier benuzten oder in "hier, usage, nfA, einleitung konventionen"?}