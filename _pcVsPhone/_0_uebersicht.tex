%%% hidden subsection for a better structure in latex editor: "texifier"
\myComment{\subsection*{Übersicht}}\myCheckmark%
%
%\textbf{Einleitung}
	%Warum: unterschiede -> ziel
	Um das in \secref{section:zielsetzung} erwähnte Ziel umzusetzen, muss zuerst herausgearbeitet werden, was überhaupt die Stärken und Schwächen von Pc's und Handys sind. %
	%Was + Was nicht
	Daher wird in diesem Abschnitt versucht Erkenntnisse über diese zu gewinnen, indem wesentliche Unterschiede zwischen Pc's und Handys verglichen werden. %
	Da es sich bei Pc’s jedoch um eine Kategorie von Systemen handelt\footnote{zum Beispiel Laptops, Desktops oder mini-Pc's}, wird sich während der Vergleiche auf eine davon beschränkt, da dieser Abschnitt sonst zu umfangreich sein würde. Dabei wurde der Desktop gewählt, weil bei diesem vermutet wird, dass die Unterschiede zwischen ihm und dem Handy am stärksten sind. \newline%
	%Trimmed
	%Jedoch handelt dieser Abschnitt nicht davon möglichst alle Unterschiede aufzuzählen, sondern eher davon [überwiegenden, starken, wesentlichen, den ausschlaggebende] Unterschiede zu betrachten und so für diese Arbeit wichtige Erkenntnisse, [also die Stärken und Schwächen des Handys und Pc's], zu [gewinnen/identifizieren]. [Denn genau diese Erkenntnisse sollen uns dabei helfen die Stärken und Schwächen des Handys und Pc's zu identifizieren.]\newline%
\textbf{Übersicht:} %
%1 Teil: Unterschiede:
	In diesem Abschnitt werden die folgenden Bereiche der Geräte verglichen: die Leistung, die Internetverbindung, den Speicher, die Eingabemöglichkeiten, die Bildschirme, die Mobilität, wie die Geräte benutzt werden, die Hardware, das Betriebssystem und die Software. %
	%2 Teil: Erkenntnisse:
	Zum Schluss werden die Erkenntnisse zusammengefasst und daraus Schlussfolgerungen gezogen.\newline%
\textbf{Ergebnisse:} %
	%Design
	Beim Vergleich der Bildschirme kam es zu der Erkenntnis, dass für Handys das Design und dessen Richtlinien wichtig sind. Diese können Auswirkungen auf die Intuitivität und die Einfachheit der Anwendung haben.\newline%
	%Pc's
	Zudem kam es durch die verglichenen Unterschiede zu der Erkenntnis, dass Pc's gut für Aufgaben funktionieren welche\myTodo %
		%Aufzählung
		viel Leistung benötigen, schnelle, präzise oder vielfältige Eingaben erfordern, viele Informationen gleichzeitig darstellen oder benötigen, viele Optionen und Konfigurationen anbieten oder benötigen, sowie langwierig sind oder viel Zeit benötigen. %
	%Handys
	Handys funktionieren hingegen für jene Aufgaben gut welche\myTodo %
		%Aufzählung
		ressourcenschonend sind oder nicht viel Leistung benötigen, keine schnelle, präzise oder vielfältige Eingabe erfordern, nur wenig Informationen darstellen oder benötigen, ohne viele Optionen und Konfigurationen auskommen, kurzweilig sind oder wenig Zeit benötigen, lohnenswert sind unterwegs zu lösen, sowie einfach und intuitiv seien sollen.