%%% hidden subsection for a better structure in latex editor: "texifier"
\myComment{\subsection*{Übersicht}}%
%
%\textbf{Einleitung}
	%Warum: unterschiede -> ziel
	Um das in \secref{section:zielsetzung} genannte Ziel zu erreichen, muss zunächst festgestellt werden, was die Stärken und Schwächen von PCs und Handys sind. %
	%Was + Was nicht
	In diesem Abschnitt werden daher wesentliche Unterschiede zwischen PCs und Handys verglichen, um so zu versuchen Erkenntnisse über ihre Stärken zu gewinnen. %
	Da es sich bei PCs jedoch um eine Kategorie von Systemen handelt\footnote{Beispielsweise gehören Laptops, Desktops oder Mini-PCs zu dieser Kategorie.}, wird sich auf einen davon beschränkt, da dieser Abschnitt sonst zu umfangreich würde. Dabei wurde der Desktop gewählt, da vermutet wird, dass die Unterschiede zwischen ihm und dem Handy am stärksten ausgeprägt sind. \newline%
	%Trimmed
	%Jedoch handelt dieser Abschnitt nicht davon möglichst alle Unterschiede aufzuzählen, sondern eher davon [überwiegenden, starken, wesentlichen, den ausschlaggebende] Unterschiede zu betrachten und so für diese Arbeit wichtige Erkenntnisse, [also die Stärken und Schwächen des Handys und Pc's], zu [gewinnen/identifizieren]. [Denn genau diese Erkenntnisse sollen uns dabei helfen die Stärken und Schwächen des Handys und Pc's zu identifizieren.]\newline%
\textbf{Überblick:} %
%1 Teil: Unterschiede:
	In diesem Abschnitt werden folgende Aspekte der Geräte verglichen: Leistung, Internetverbindung, Speicher, Eingabemöglichkeiten, Bildschirme, Mobilität, Nutzung, Hardware, Betriebssystem und Software, da diese als wesentliche Unterschiede betrachtet werden. %
	%2 Teil: Erkenntnisse:
	Abschließend werden die Erkenntnisse zusammengefasst und Schlussfolgerungen daraus gezogen.\newline%
\textbf{Ergebnisse:} %
	%Design
	Beim Vergleich der Bildschirme wurde erkannt, dass für Handys das Design und dessen Richtlinien wichtig sind. Diese können sich auf die Intuitivität und Benutzerfreundlichkeit der Anwendung auswirken.\newline%
	%Pc's
	Zudem kam es durch den Vergleich der Unterschiede zu der Erkenntnis, dass PCs gut für Aufgaben funktionieren %
		%Aufzählung
		die viel Leistung benötigen, schnelle, präzise oder vielfältige Eingaben erfordern, viele Informationen gleichzeitig darstellen oder benötigen, viele Optionen und Konfigurationen anbieten oder benötigen sowie langwierig sind oder viel Zeit benötigen.\newline
	%Handys
	Handys hingegen funktionieren gut für Aufgaben %
		%Aufzählung
		die ressourcenschonend sind und nicht viel Leistung benötigen, keine schnelle, präzise und vielfältige Eingabe erfordern, nur wenige Informationen darstellen oder benötigen, ohne viele Optionen und Konfigurationen auskommen, kurzweilig sind oder wenig Zeit benötigen, lohnenswert sind, unterwegs zu lösen, sowie einfach und intuitiv sein sollen.%