\subsection{Hardware}\myCheckmark
%Was: Modulär
Die meisten Desktops sind hinsichtlich ihrer Hardware modular. %
	%Warum: Konfigurierbarkeit/Anpassungsfreiheit 
	So wird dem Benutzer die Option gegeben viele Hardwarekomponenten nach belieben auszutauschen. %
	%Warum: Auswirkung: Größe
	Dementsprechend muss das Gehäuse des Pc's auf diese Anpassungen aber auch ausgelegt sein, denn es gibt Hardwarekomponenten in einer Varietät von Größen und Formen. Die Modularität kommt also mit dem Nachteil, dass das Gehäuse groß genug sein muss um auch die verschiedenen Hardware-Optionen zu ermöglichen.\newline%
%Was: Handy allInOne 	
Das Handy unterstützt dementsprechend diese Konfigurierbarkeit nicht. Dafür bietet es sich als all-in-one-Gerät an. %
	%Warum: Ease of use
	Während man sich beim Pc auch Gedanken um Monitor und Eingabegeräte machen muss, fällt die Wahl bei einem Handy leichter aus. Bei ihm wird alles zusammen in einem System angeboten. Unter anderem besitzen sie sogar, anders als die meisten Pc's, eine Kamera, GPS, Gyroskop und die Option für biometrisches login Verfahren.\newline%
	
	
	
\myComment{

	%%%->Benutzung%%%
	
	%Zusammenfassung
	Hier erkennen wir also einen Abtausch zwischen Anpassungsfähigkeit und Einfachheit. Während die meisten Pc's sehr viele Optionen zur Konfiguration geben wurde beim Handy für einen bereits alle Entscheidungen getroffen.%
	
	%Was
	%Warum
	%Zusammenfassung
	%Auswirkung
	- Handys werden als leichter angesehen (all in one - ease of use) -> gut für Simple Aufgaben VS Pc eher für Leute die sich auskennen oder konfigurieren wollen
	- Dafür sind Pc's sehr anpassungsfähig. -> Gut für Arbeit / wichtige und Komplexe Aufgaben. Da manche Aufgaben zum Beispiel sehr spezifische Komponenten benötigen. Zum Beispiel einen schnellen Prozessor, oder eine schnelle Grafikkarte, oder viel Speicher, eine soundkarte, sehr schnelle netzwerkgeschwindigkeit, oder viele Monitore usw...
	
	\myNewSection
	\myTextTodo{
	-> Leistuns(Modulär) vs Portabilität\\
	-> Aber auch Modularität vs Abgeschlossenheit -> Easy of use \\
	- 
	}


}