\subsection{Hardware}

\myNewSection
\textbf{Hardware}: Desktops sind sehr anpassungsfähig. Den Benutzer wird die Option gegeben viele Hardwarekomponenten nach belieben auszutauschen. Da es Pc-Komponenten in einer Varietät von Größen und Formen gibt\footnote{So gibt es zum Beispiel sehr kleine Festplatten bis hin zu großen}, müssen die Gehäuse dafür auch dementsprechend Groß sein. \newline%
Die Modularität hat also den Nachteil, dass das Gehäuse groß genug sein muss, um auch die verschiedenen Hardware-Optionen zu ermöglichen. Das Handy unterstützen diese Optionsfreiheit also nicht. \newline%
Dafür bietet sich das Handy als attraktives allInOne-Device an. Während man sich bei der Anschaffung eines Pc's auch Gedanken um Monitor und Eingabegeräte machen muss, fällt die Wahl bei einem Handy leichter aus. Hier wird alles zusammen angeboten, unteranderem besitzt es sogar, anders als die meisten Pc's, eine Kamera, GPS, GyroSensor und die option für biometric login.\newline%
Hier erkennen wir also einen Abtausch zwischen Anpassungsfähigkeit und Einfachheit. Während die meisten Pc's sehr viele Optionen zur Konfiguration geben wurde beim Handy für einen bereits alle Entscheidungen getroffen.