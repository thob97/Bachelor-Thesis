\subsection{Hardware}
%Was: Modulär
Die meisten Desktop-Computer sind hinsichtlich ihrer Hardware modular aufgebaut. %
	%Warum: Konfigurierbarkeit/Anpassungsfreiheit 
	Dadurch haben Benutzer die Möglichkeit, viele Hardwarekomponenten nach Belieben auszutauschen. %
	%Warum: Auswirkung: Größe
	Allerdings muss das Gehäuse des PCs auch auf diese Anpassungen ausgelegt sein, da es Hardwarekomponenten in verschiedenen Größen und Formen gibt. Die Modularität hat also den Nachteil, dass das Gehäuse groß genug sein muss, um verschiedene Hardwareoptionen zu ermöglichen.\newline%
%Was: Handy allInOne 	
Das Handy unterstützt dementsprechend diese Konfigurierbarkeit nicht. Dafür bietet es sich als all-in-one-Gerät an. %
	%Warum: Ease of use
	Dadurch entfällt im Gegensatz zum PC die Notwendigkeit, sich Gedanken über die Auswahl eines Monitors oder Eingabegeräts zu machen, da alles bereits in einem System integriert ist. Zudem bieten Handys Funktionen, die bei den meisten PCs nicht vorhanden sind, wie beispielsweise eine Kamera, GPS, Gyroskop und die Option für biometrisches Login-Verfahren.\newline%
%
%
% 
%\myComment{
%
%	%%%->Benutzung%%%
%	
%	%Zusammenfassung
%	Hier erkennen wir also einen Abtausch zwischen Anpassungsfähigkeit und Einfachheit. Während die meisten Pc's sehr viele Optionen zur Konfiguration geben wurde beim Handy für einen bereits alle Entscheidungen getroffen.%
%	
%	%Was
%	%Warum
%	%Zusammenfassung
%	%Auswirkung
%	- Handys werden als leichter angesehen (all in one - ease of use) -> gut für Simple Aufgaben VS Pc eher für Leute die sich auskennen oder konfigurieren wollen
%	- Dafür sind Pc's sehr anpassungsfähig. -> Gut für Arbeit / wichtige und Komplexe Aufgaben. Da manche Aufgaben zum Beispiel sehr spezifische Komponenten benötigen. Zum Beispiel einen schnellen Prozessor, oder eine schnelle Grafikkarte, oder viel Speicher, eine soundkarte, sehr schnelle netzwerkgeschwindigkeit, oder viele Monitore usw...
%	
%	\myNewSection
%	\myTextTodo{
%	-> Leistuns(Modulär) vs Portabilität\\
%	-> Aber auch Modularität vs Abgeschlossenheit -> Easy of use \\
%	- 
%	}
%
%
%}