\subsection{Internet}\label{PcVsPhone:Internet}
%Was: Pc
Was PCs betrifft, so sind sie oft mit schnellen und stabilen Ethernet-Anschlüssen verbunden, die jedoch stationär sind. %
%Was: Handy
Im Gegensatz dazu werden Handys in der Regel über WLAN oder mobile Daten genutzt. Diese Alternativen sind wesentlich mobiler, da sie keine Kabelverbindung erfordern. Stattdessen wird die Verbindung über die Luft hergestellt, was es ermöglicht, dass das Handy fast überall eine Internetverbindung herstellen kann. Allerdings ist zu beachten, dass diese Verbindungen oft langsamer und fehleranfälliger sind als kabelgebundene Verbindungen, da die Luft im Vergleich zum Kabel ein schwächeres Übertragungsmedium darstellt.%
%
%\myComment{
%
%	%%%->Benutzung%%%
%	%Warum: internet wichtig für Handy ----------- Leistung + Internet -> Quelle
%	Wobei besonders diese Schnelligkeit und Stabilität auf dem Handy wichtig zu seien scheint, vielleicht genau weil sie auf dem Handys oft fehlt. So verlassen die hälfte aller Handy Nutzer laut Google die Website wenn sie länger als drei Sekunden braucht zu laden\cite{pcVsphone_threeSeconds}.\newline%
%	
%	%Zusammenfassung
%	Also wird auch hier, wie im Vergleich der \nameref{PcVsPhone:Leistung}, Performance, in diesem Fall in Form von Schnelligkeit und Stabilität, gegen Mobilität getauscht. Schnelligkeit und Stabilität scheint den Nuztern wichtig zu sein\newline%
%	
%	\myNewSection
%	\myTextTodo{
%	-> Leistung(Schnelle und stabile Verbindung) vs Portabilität\\
%	- Schnelle und stabile Verbindung vs Portable Verbindung -> Optimierung wichtig (internet stabilität + schnelligkeit)
%	}
%	
%	\myNewSection
%	%Auswirkungen
%	- Das führt zum Beispiel dazu, dass für Anwendungen welche von eine schnelle und stabile Internetverbindung profitieren, eher der Pc präferiert wird. So zum Beispiel bei online Spielen oder Videoübertragungen.
%	- Das Handy muss für den Netzwerkverkehr optimiert werden, sonst stellt dies ein flaschenhalz her, um nicht länger als 3 sekunden zu laden
%	-> kleine + kurze Aufgaben funktionieren gut auf dem handy. Da dafür weder eine stabile(kurze) noch schnelle(kleine) internetverbindung nötig ist + die portabilität ein prositiver faktor, da für solche Aufgaben nicht der Aufwand des Pc's anschaltens betrieben werden muss
%
%}