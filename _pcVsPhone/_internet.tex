\subsection{Internet}\label{PcVsPhone:Internet}\myCheckmark
%Was: Pc
Pc's sind oft mit schnellen und stabilen aber dafür auch stationären Ethernet Anschlüssen verbunden. %
%Was: Handy
Handys hingegen werden meistens mit dem WLAN oder den mobilen Daten genutzt. Diese Alternativen sind um einiges mobiler, da für sie kein Kabel benötigt wird. Stattdessen funktioniert die Verbindung über die Luft, dadurch kann das Handy an fast jedem Standort eine Internetverbindung bekommen. Jedoch kommt das mit dem Nachteil, dass diese Verbindungen oft langsamer und fehleranfälliger sind, da die Luft verglichen zum Kabel ein schwächeres Übertragungsmedium ist.%

\myComment{

	%%%->Benutzung%%%
	%Warum: internet wichtig für Handy ----------- Leistung + Internet -> Quelle
	Wobei besonders diese Schnelligkeit und Stabilität auf dem Handy wichtig zu seien scheint, vielleicht genau weil sie auf dem Handys oft fehlt. So verlassen die hälfte aller Handy Nutzer laut Google die Website wenn sie länger als drei Sekunden braucht zu laden\cite{pcVsphone_threeSeconds}.\newline%
	
	%Zusammenfassung
	Also wird auch hier, wie im Vergleich der \nameref{PcVsPhone:Leistung}, Performance, in diesem Fall in Form von Schnelligkeit und Stabilität, gegen Mobilität getauscht. Schnelligkeit und Stabilität scheint den Nuztern wichtig zu sein\newline%
	
	\myNewSection
	\myTextTodo{
	-> Leistung(Schnelle und stabile Verbindung) vs Portabilität\\
	- Schnelle und stabile Verbindung vs Portable Verbindung -> Optimierung wichtig (internet stabilität + schnelligkeit)
	}
	
	\myNewSection
	%Auswirkungen
	- Das führt zum Beispiel dazu, dass für Anwendungen welche von eine schnelle und stabile Internetverbindung profitieren, eher der Pc präferiert wird. So zum Beispiel bei online Spielen oder Videoübertragungen.
	- Das Handy muss für den Netzwerkverkehr optimiert werden, sonst stellt dies ein flaschenhalz her, um nicht länger als 3 sekunden zu laden
	-> kleine + kurze Aufgaben funktionieren gut auf dem handy. Da dafür weder eine stabile(kurze) noch schnelle(kleine) internetverbindung nötig ist + die portabilität ein prositiver faktor, da für solche Aufgaben nicht der Aufwand des Pc's anschaltens betrieben werden muss

}