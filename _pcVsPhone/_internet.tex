\subsection{Internet}\label{PcVsPhone:Internet}\myCheckmark
%Was: Pc+Handy
Während Pc's oft mit schnellen und stabilen, aber dafür auch stationären, Ethernet-Anschlüssen verbunden sind, werden Handys oft mit dem WLAN, der mobileren aber dafür häufig langsameren Alternative, genutzt. 
%Zusammenfassung
Also wird auch hier, wie im Vergleich der \nameref{PcVsPhone:Leistung}, Performance, in diesem Fall in Form von Schnelligkeit und Stabilität, gegen Mobilität getauscht. \newline%
%Warum: internet wichtig für Handy
Jedoch scheint besonders diese Schnelligkeit und Stabilität auf dem Handy wichtig zu sein, vielleicht genau weil sie auf dem Handys oft fehlt. So verlassen die hälfte aller Handy Nutzer laut Google die Website wenn sie länger als drei Sekunden braucht zu laden\cite{pcVsphone_threeSeconds}.\newline%
%Was+Warum: Handy -> Mobile Daten
Außerdem besitzen Handys Zugang zu \dq Mobile-Daten\dq. Dabei ähnelt sich die Funktion mit der des WLAN's, mit dem Unterschied, dass Sie die Stärken und Schwächen des WLAN's noch weiter in die extreme ziehen. So erweitern sich die mobile Nutzung vom Haus hin zu fast jedem Standort. Jedoch gelingt dies erneut nur durch den Austausch von Leistung. So ist das Datennetz für die \dq Mobilen-Daten\dq häufig langsamer und instabiler als die des Heiminternets. Das liegt wahrscheinlich zuteil an dem bereits fehleranfälligeren Übertragungsmedium der Mobilen-Daten. So wird wenn es um Stabilität geht die Verbindung über ein Kabel über die Verbindung über die Luft bevorzugt.\newline%
%Auswirkungen
Das führt zum Beispiel dazu, dass für Anwendungen welche von eine schnelle und stabile Internetverbindung profitieren, eher der Pc präferiert wird. So zum Beispiel bei online Spielen oder Videoübertragungen.




%old - todo remove
\myComment{
		Handys bieten trotzdem auch hier einige Vorzüge. Das WLAN macht das Handy um einiges mobiler, dass es kein Kabel benötigt. Während Laptops auch WLAN besitzen, sind viele Desktops nicht damit ausgestattet. Außerdem besitzen die meisten Handys einen dauerhaften Internetzugang durch 'Mobile Daten'. Diese sind zwar oft langsamer und nur begrenzt verfügbar, jedoch ermöglicht das einen Internetzugang von fast jedem Standort.
		}