\subsection{Leistung}\label{PcVsPhone:Leistung}
%Was: Pc
In Bezug auf Leistung gelten PCs im Allgemeinen als schneller und stärker im Vergleich zu Smartphones. 
	%Warum: 
	Das liegt an verschiedenen Faktoren. Zum einen haben die meisten PCs einen konstanten Zugang zu Strom, während Smartphones häufig nur über Batterien verfügen. Darüber hinaus können PCs aufgrund ihres größeren Volumens leistungsstärkere Hardware-Komponenten verbauen und größere Kühlsysteme nutzen, was die Verwendung dieser stärkeren Hardware überhaupt erst ermöglicht.\newline%
%Was + Warum: Handy
Im Gegensatz dazu ist die Hardware von Smartphones in der Regel auf Portabilität und Energieeffizienz ausgelegt, um eine möglichst lange Akkulaufzeit zu ermöglichen.\newline%
%
%
%
%
%
%
%\myComment{
%
%	%%%->Benutzung%%%
%	%Zusammenfassung: Leistung VS Portabilität
%	Während Pc's also auf möglichst performante Leistung ausgelegt sind, wird stattdessen beim Handy eher auf die Portabilität geachtet.\newline%
%	
%	\myNewSection
%	\myTextTodo{
%	-> Leistung vs Portabilität\\
%	- Aufwändige Hardware vs Optimierung(Leistung + Batterie) 
%	}
%	
%	
%	\myNewSection
%	%Auswirkungen
%	-Eine mögliche Auswirkung davon ist, dass aufwändige Anwendungen öfters nur auf dem genutzt(da schneller) oder gar nur auf dem Pc unterstützt werden. Solch eine aufwändige Anwendung wäre zum Beispiel das exportieren von Videodateien. 
%	-Währenddessen wird/muss auf dem Handy wahrscheinlich eher auf die Optimierung geachtet, um wenig Strom zu verbrauchen und so eine längere Betriebszeit zu ermöglichen. Und um handys schneller zu gestalten (3 sekunden regel)
%
%}