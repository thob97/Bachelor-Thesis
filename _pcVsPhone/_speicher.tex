\subsection{Speicher}
%Warum: Speicher
Was den Speicher betrifft, so müssen die Speichermodule in Handys im Vergleich zu Festplatten für PCs kleiner sein, um in den Formfaktor des Geräts zu passen. Dem entsprechend bieten Handy-Speichermodule in der Regel auch eine geringere Kapazität als die größeren Festplatten von PCs. %   
	%Quellen
	Laut einer Studie von Counterpoint besitzen Handys im Durchschnitt rund 118 Gigabyte Speicherkapazität\cite{pcVsphone_storageSmartphone}. Im Gegensatz dazu zeigt ein Report von Seagate, dass sie im vierten Quartal 2022 im Durchschnitt PC-Festplatten mit einer Größe von rund 8 Terabyte verkauft haben\cite{pcVsphone_storageSeagate}\footnote{Die hohe Festplattengröße kommt vermutlich durch Server zustande. Der Speicher von Heimcomputern fällt dementsprechend wahrscheinlich etwas kleiner aus.}.% 
%
%\myComment{
%
%	%%%->Benutzung%%%
%	%Zusammenfassung
%	Also wird, wie bereits im Vergleich der \nameref{PcVsPhone:Leistung} und dem \nameref{PcVsPhone:Internet}, auch beim Speicher die Leistung zugunsten der Mobilität eingeschränkt.
%		
%	\myNewSection
%	\myTextTodo{
%	-> Leistung(Kapazität) vs Portabilität\\
%	- große erweiterbare Festplatten + Datengrab vs kleine Festplatten + Optimierung und nur wichtigstes\\ 
%	}
%	
%	\myNewSection	
%	%Auswirkung
%	-Dadurch dass der Speicher auf Handys also relativ klein ist, wird vermutet, dass Handy-Nutzer eher auf ihren Speicher achten. Während der Pc gerne als Speicherablage genutzt wird, wird auf dem Handy eher wichtige Daten und welche man häufiger braucht gespeichert.
%	 -Optimierung für App nötig. Wenn Speicher knapp wird wahrscheinlich erst eine App mit Großem Speicherbedarf gelöscht/nicht benutzt
%
%}