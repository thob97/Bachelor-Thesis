\subsection{Speicher}\myCheckmark
%Zusammenfassung
Wie bereits im Vergleich der \nameref{PcVsPhone:Leistung} und dem \nameref{PcVsPhone:Internet} wird auch beim Speicher die Leistung zugunsten der Mobilität eingeschränkt. 
%Warum: Speicher
Dadurch dass der Speicher in Handys möglichst klein sein soll, bieten sie dementsprechend meistens auch weniger Kapazität. 
	%Quellen
	So besitzen Handys im Durchschnitt, laut einer Studie von CointerPpoint, rund 118 Gigabyte Speicherkapazität\cite{pcVsphone_storageSmartphone}. Im Gegensatz zeigt ein Report von Seagate, dass sie im vierten Quartal 2022 im Durchschnitt Festplatten mit einer Größe von rund 8 Terabyte verkauften\cite{pcVsphone_storageSeagate}. 
	%Was: Erkenntnis -> Handy Speicher < Pc Speicher
	Zwar sind Server wahrscheinlich der größte Einfluss dafür wie es zu diesen hohen Speichergrößen kommt, da diese oft besonders viel Kapazität benötigen, jedoch bietet diese Erkenntnis trotzdem ein starkes Anzeichen dafür, dass Pc's generell mehr Speicher besitzen als Handys.\newline%
%Auswirkung
Dadurch dass der Speicher auf Handys also relativ klein ist, wird vermutet, dass Handy-Nutzer eher auf ihren Speicher achten. Während der Pc gerne als Speicherablage genutzt wird, wird auf dem Handy eher wichtige Daten und welche man häufiger braucht gespeichert.
   


