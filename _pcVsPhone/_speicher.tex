\subsection{Speicher}\myCheckmark
%Warum: Speicher
Die Speichermodule in Handys müssen im vergleich zu Festplatte für den Pc um einiges kleiner sein, um in den Formfaktor ihres Gerätes zu passen. Dementsprechend bieten die Handy Speichermodule meistens auch weniger Kapazität als die größeren Pc-Festplatten. %   
	%Quellen
	So besitzen Handys im Durchschnitt, laut einer Studie von CointerPpoint, rund 118 Gigabyte Speicherkapazität\cite{pcVsphone_storageSmartphone}. Im Gegensatz zeigt ein Report von Seagate, dass sie im vierten Quartal 2022 im Durchschnitt Pc Festplatten mit einer Größe von rund 8 Terabyte verkauft haben\cite{pcVsphone_storageSeagate}\footnote{Die hohe Festplatten Größe kommt vermutlich durch Server zustande. Der Speicher von Heimcomputern fällt dementsprechend wahrscheinlich etwas kleiner aus.}. 

\myComment{

	%%%->Benutzung%%%
	%Zusammenfassung
	Also wird, wie bereits im Vergleich der \nameref{PcVsPhone:Leistung} und dem \nameref{PcVsPhone:Internet}, auch beim Speicher die Leistung zugunsten der Mobilität eingeschränkt.
		
	\myNewSection
	\myTextTodo{
	-> Leistung(Kapazität) vs Portabilität\\
	- große erweiterbare Festplatten + Datengrab vs kleine Festplatten + Optimierung und nur wichtigstes\\ 
	}
	
	\myNewSection	
	%Auswirkung
	-Dadurch dass der Speicher auf Handys also relativ klein ist, wird vermutet, dass Handy-Nutzer eher auf ihren Speicher achten. Während der Pc gerne als Speicherablage genutzt wird, wird auf dem Handy eher wichtige Daten und welche man häufiger braucht gespeichert.
	 -Optimierung für App nötig. Wenn Speicher knapp wird wahrscheinlich erst eine App mit Großem Speicherbedarf gelöscht/nicht benutzt

}