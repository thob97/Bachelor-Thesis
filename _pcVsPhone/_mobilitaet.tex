\subsection{Mobilität}
%Was: Mobilität
Eines der Hauptziele des Handys scheint die Mobilität zu sein. %
	%Warum
	So scheinen alle zuvor erwähnten Unterschiede eine Auswirkung der von Handys gewünschten Eigenschaft: Mobilität. %
		%Beispiel:
		Wenn das Handy nicht mobil sein müsste, könnte es beispielsweise mehr Leistung, einen größeren Bildschirm und eine stabilere Internetverbindung besitzen.\newline% Dadurch wären aber wahrscheinlich auch einige der dabei entstandenen Vorzüge, wie zum Beispiel die der Design-Richtlinien und der Berührungseingabe, verfallen [da genau diese Anpassung dann nicht nötig gewesen wären]. \newline%
%Handy vs Laptop
Im Vergleich zu Desktop-PCs sind Handys aufgrund ihrer kompakten Größe mobiler. Obwohl Laptops auch mobiler als Desktops sind, können sie in Bezug auf Mobilität nicht mit Handys konkurrieren, da Handys in die meisten Hosentaschen passen, während für Laptops oft eine größere Tasche oder ein Rucksack benötigt wird.%
%
%
%
%\myComment{
%
%	%%%->Benutzung%%%
%	
%	%Auswirkung
%	- Handy ist immer bei einem\\
%	-> wird unter anderem für aufgaben benutzt, welche on-the-go auftreten (maps, foot, kleine aufgaben)\\
%	-> wird oft genutzt, da immer bei einem + kleiner aufwand + gut für kurze aufgaben
%	- Zwar ist dies keine Stärke welche wir direkt in der App einbauen und nutzen können. Jedoch vielleicht die größte Stärke des Handys und eine Deutung dafür, dass Smartphone apps beleibt sind und sich diese Arbeit lohnt.
%
%
%}