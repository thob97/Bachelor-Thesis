\subsection{Mobilität}\myCheckmark
%Was: Mobilität
Eines der Hauptziele des Handys ist es mobil zu sein. %
	%Warum
	So sind alle zuvor erwähnten Unterschiede eine Auswirkung der von Handys gewünschten Eigenschaft Mobilität. %
		%Beispiel:
		Wenn es nicht mobil sein müsste könnte das Handy mehr Leistung, ein größerer Bildschirm, eine stabilere Internetverbindung und so weiter, besitzen. Dadurch wären aber wahrscheinlich auch einige der dabei entstandenen Vorzüge, wie zum Beispiel die der Design-Richtlinien und der Berührungseingabe, verfallen [da genau diese Anpassung dann nicht nötig gewesen wären]. \newline%
%Handy vs Laptop
Durch die kleine Größe der Smartphones genießen es also eine hohe Mobilität. Zwar sind Laptops auch schon um einiges mobiler als Desktops, jedoch kommen sie in dem Aspekt nicht an Handys heran. Handys passen in die meisten Hosentaschen, während für Laptops oft eine größere Tasche oder einen Rucksack benötigt wird.%






\myComment{

	%%%->Benutzung%%%
	
	%Auswirkung
	- Handy ist immer bei einem\\
	-> wird unter anderem für aufgaben benutzt, welche on-the-go auftreten (maps, foot, kleine aufgaben)\\
	-> wird oft genutzt, da immer bei einem + kleiner aufwand + gut für kurze aufgaben
	- Zwar ist dies keine Stärke welche wir direkt in der App einbauen und nutzen können. Jedoch vielleicht die größte Stärke des Handys und eine Deutung dafür, dass Smartphone apps beleibt sind und sich diese Arbeit lohnt.


}