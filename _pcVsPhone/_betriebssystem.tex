\subsection{Betriebssystem}

\myNewSection
\textbf{Betriebssysteme} Ähnlich sieht es auch bei Betriebssystemen aus. Bei Desktops gibt es eine große Vielfalt an zu wählenden Betriebssystemen. Dem Nutzer ist die Option überlassen sich für eins zu entscheiden. Selbst mit alter Hardware lassen sich oft aktuelle Betriebssysteme installieren. Das schlimmste was dabei passieren kann, ist, dass sich der Pc langsam verhält oder sich nicht starten lässt.\newline%
Handys wird ein festes Betriebssystem vorgesetzt. Das Betriebssystem lässt sich nicht ohne weiters ändern und wird meist nur drei bis fünf Jahre unterstützt \cite{pcVsphone_deviceSupportGoogle}\cite{pcVsphone_deviceSupportApple}. Danach gilt die Hardware als zu alt und es lassen sich keine neuere Versionen des Betriebsystems installieren. Zwar kann man das Handy dann noch weiter benutzen, jedoch leidet darunter die Sicherheit und Performance. Das hat also den Nachteil, dass man Theoretisch sein Handy alle drei bis fünf Jahre wechseln muss. Es bietet aber auch einen Vorteil. Dadurch, dass die Herstelle des Betriebsystems genau wissen welche Hardware zurzeit unterstützt wird, ist es auch genau auf diese Komponenten optimiert. Dadurch sollte das Handy in seiner Lebensdauer mit wenig Fehlern und Rucklern funktionieren.\newline%
Während der Pc einen also die Freiheiten für Optionen und Wahlmöglichkeiten lässt, ist das Handy erneut auf Einfachheit und Benutzerfreundlichkeit ausgelegt. (Immerhin muss man sich keine Gedanken um ein passendes Betriebssystem machen und die Leistung wird durch Optimierung garantiert)
	
