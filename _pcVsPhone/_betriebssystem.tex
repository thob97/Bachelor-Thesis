\subsection{Betriebssystem}
%Was: Pc: Vielfalt + Entscheidungsfreiheit
Bei Desktops gibt es eine Vielfalt an Betriebssystemen zur Auswahl und der Nutzer kann sich für eins oder mehrere davon entscheiden. %
	%Was: lange Unterstützung
	Selbst mit älterer Hardware können oft noch aktuelle Betriebssysteme installiert werden. So unterstützt beispielsweise der fast zehn Jahre alte Prozessor Intel Pentium J1750 das weltweit meistgenutzte und immer noch aktuelle Betriebssystem Windows 10 \cite{pcVsphone_intelWindowsSupport, pcVsphone_destkopOperatingSystem, pcVsphone_windowsVersions}.\newline%
%Was: Handy Betriebssysteme
Für Handys wird im Gegensatz zu Desktops ein festes Betriebssystem vorgegeben, das sich nicht ohne Weiteres ändern lässt. %
	%Was: keine Entscheidungsfreiheit + kurze Unterstützung
	Zudem werden in der Regel Handy-Betriebssysteme nur für einen Zeitraum von drei bis fünf Jahren unterstützt \cite{pcVsphone_deviceSupportGoogle, pcVsphone_deviceSupportApple}. %
		%Nachteil: Sicherheit und Leitsung
		Obwohl das Handy nach dieser Zeitspanne noch funktioniert, leiden Sicherheit und Leistung des Geräts ohne weitere Software-Updates. Das hat den Nachteil, dass das Handy alle drei bis fünf Jahre ersetzt werden sollte.\newline%
		%Vorteil: Optimierung
		Allerdings könnte die kürzere Unterstützungszeit der Handy-Betriebssysteme auch Vorteile bieten. Die Entwickler können sich auf eine kleinere Anzahl von Handys konzentrieren und die Software besser auf diese optimieren. Daher wird angenommen, dass Handys während des unterstützten Zeitraums beispielsweise mit weniger Defekten und Rucklern auskommen und eine flüssigere grafische Oberfläche bieten.\newline%
%
%
%
%\myComment{
%
%%%%->Benutzung%%%
%
%	%Zusammenfassung
%	Während der Pc einen also die Freiheiten für Optionen und Wahlmöglichkeiten lässt, ist das Handy auf Einfachheit und Benutzerfreundlichkeit ausgelegt. (Immerhin muss man sich keine Gedanken um ein passendes Betriebssystem machen und die Leistung wird durch Optimierung garantiert). -> Handy Betriebssystem ist auf genau auf die Hardware ausgelegt.
%		
%	%Auswirkung
%	-> Handys wirken auf Nutzer einfacher. Man muss sich weniger Gedanken um OS machen,  'it just works'. -> passend für simple Aufgaben, diese sollten auch einfach zu lösen sein
%	-> Pcs sind mehr konfigurierbar und erweiterbar -> anpassbarkeit ist gut für komplexe und schwierige aufgaben, denn diese brauchen wohlmöglich komplexe und spezifische umgebungen
%	
%	\myNewSection
%	\myTextTodo{
%	-> Konfiguration VS Ease of use\\
%	- 
%	}
%
%}