\subsection{Betriebssystem}\myCheckmark
%Was: Pc: Vielfalt + Entscheidungsfreiheit
Bei Desktops gibt es eine große Vielfalt an zu wählenden Betriebssystemen und dem Nutzer ist die Option überlassen sich für eins oder mehrere davon zu entscheiden. %
	%Was: lange Unterstützung
	Selbst mit alter Hardware lassen sich oft aktuelle Betriebssysteme installieren. So unterstützt der knapp zehn Jahre alter Prozessor Intel Pentium J1750 das weltweit meist genutzte Betriebssystem Windows 10\cite{pcVsphone_intelWindowsSupport, pcVsphone_destkopOperatingSystem, pcVsphone_windowsVersions}\newline%
%Was: Handy Betriebssysteme
Für Handys wird hingegen ein festes Betriebssystem vorgesetzt. %
	%Was: keine Entscheidungsfreiheit + kurze Unterstützung
	Das Betriebssystem lässt sich nicht ohne weiters ändern und wird meist nur drei bis fünf Jahre unterstützt \cite{pcVsphone_deviceSupportGoogle}\cite{pcVsphone_deviceSupportApple}. %
		%Nachteil: Sicherheit und Leitsung
		Zwar kann das Handy nach dieser Zeitspanne noch weiter betrieben werden, jedoch leidet ohne weitere Softwareunterstützung die Sicherheit und Leistung darunter. Das hat also den Nachteil, dass alle drei bis fünf Jahre das Handy gewechselt werden sollte.\newline%
		%Vorteil: Optimierung
		Die kürzere Unterstützungszeit der Handy Betriebssysteme bietet aber auch einen Vorteil. Dadurch können sich die Betriebssystem Entwickler auf eine kleinere Anzahl von Handys konzentrieren und können die Software dementsprechend gut auf diese Optimieren. Dadurch sollten Handys in ihrer Lebensdauer mit weniger Defekte und [Ruckler,guten performance, flüssigen userexperinec] funktionieren.\newline%








\myComment{

%%%->Benutzung%%%

	%Zusammenfassung
	Während der Pc einen also die Freiheiten für Optionen und Wahlmöglichkeiten lässt, ist das Handy auf Einfachheit und Benutzerfreundlichkeit ausgelegt. (Immerhin muss man sich keine Gedanken um ein passendes Betriebssystem machen und die Leistung wird durch Optimierung garantiert). -> Handy Betriebssystem ist auf genau auf die Hardware ausgelegt.
		
	%Auswirkung
	-> Handys wirken auf Nutzer einfacher. Man muss sich weniger Gedanken um OS machen,  'it just works'. -> passend für simple Aufgaben, diese sollten auch einfach zu lösen sein
	-> Pcs sind mehr konfigurierbar und erweiterbar -> anpassbarkeit ist gut für komplexe und schwierige aufgaben, denn diese brauchen wohlmöglich komplexe und spezifische umgebungen
	
	\myNewSection
	\myTextTodo{
	-> Konfiguration VS Ease of use\\
	- 
	}

}