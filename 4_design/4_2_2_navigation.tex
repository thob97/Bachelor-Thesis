\subsubsection{Appleisten}
%Was
Bei den Appleisten wurde generell versucht, die Funktionen der Seite und der Tasten so klar wie möglich darzustellen. %
	%Wie: deutlichkeit
	Dazu wurden für die Seiten und Tasten möglichst kurze und deutliche Wörter und Icons verwendet. %
	%Konsistenz:
	Außerdem wurde versucht, die Leisten möglichst konsistent mit bereits bestehenden Konventionen zu entwerfen. So befindet sich der Zurückknopf, wie man es erwarten würden, immer am oberen linken Rand. Zudem wurde vorerst ein \glqq Tab-Design\grqq{} gewählt \ref{} \myTodo, da dieses auch in vielen Apps vorkommt.\newline%
	%Design änderung
	Jedoch wurde später beschlossen, von diesem ursprünglichen Design abzuweichen. Die Funktion, eine Erinnerung hinzuzufügen, wird als eine der häufig genutzten Hauptfunktionen der Anwendung angesehen. Allerdings ist sie in diesem Design relativ schwer zu erreichen (Plus-Icon oben rechts). Es wäre besser, diese Aktion unten oder in der Mitte anzubieten, da dieser Bereich für die meisten Nutzer besser erreichbar ist\cite{konventionen_platforms_ios}. Zusätzlich dazu ist der Tab zu der Einstellungsseite relativ groß und immer sichtbar, obwohl davon ausgegangen wird, dass dieser Seite nur einmal oder sehr selten benötigt wird.\newline%
	%Auswirkung
	Dementsprechend wurde das Design abgeändert\ref{}\myTodo. Die Hauptaktion eine Erinnerung zu erstellen ist nun verständlicher erklärt und besser erreichbar. Die Aktion zur Einstellungsseite ist nun kleiner und wurde aus dem meistgenutzten Bereich (unten und mittig) entfernt. Auch die Navigationstaste zur Kalenderseite wurde entfernt, da diese als Hauptseite gilt und dementsprechend immer über den Zurückknopf erreichbar ist.
	%\myTextTodo{anders als das Tab Design verändert sich unser design je nach seite, dies kann das design zwar etwas komplizierter machen da nicht statisch/konsistent, aber dafür kann für jede seite genau die benötigte }
	
%\pic{old} \pic{new}
%Konvention: Use the standard back button
%Text: Use the title area to describe the current screen if it provides useful context
%Keep tabs visible even when their content is unavailable
%Use concrete nouns or verbs as tab titles