%%% hidden subsection for a better structure in latex editor: "texifier"
\myComment{\subsection*{Übersicht}} 

%Einleitung
Wie im \secref{section:anforderungen} beschrieben wurde, ist eine gute Benutzbarkeit ein wichtiger Aspekt für die zu entwickelnde Anwendung. Dabei wurde Insbesondere das Design und die Einhaltung von Designrichtlinien im \secref{section:pcVsPhone} als eine entscheidende Rolle für eine gute Benutzbarkeit hervorgehoben.\newline%
%Übersicht
\textbf{Übersicht:} %
	%Richtlinien
	Im ersten Unterabschnitt werden daher diese Designrichtlinien genauer betrachtet. Hierbei wird zunächst eine Richtlinie ausgewählt und ihr Inhalt genauer erläutert, um die für die Arbeit relevanten Regeln zu identifizieren.  %für eine Entschieden -> welcher Inhalt davon wirklich nützlich/verwendbar -> Erkenntnisse/Schlussfolgerung daraus anstatt Blind alles befolgen um so die Entscheidungen zu verstehen und diese auf nicht behandelte themen anzuwenden
	%Design
	Anschließend wird das konkrete Design der App vorgestellt und einige der getroffenen Entscheidungen begründet.\newline%
%Ergebnisse
\textbf{Ergebnisse:}\myTodo
%AusWahl
Es wurde sich zwischen den Richtlinien Apple und Google für Apples entschieden.
%Ergebnisse
Viele der Regeln in den Designrichtlinien stimmen mit den Stärken des Handys, beziehungsweise den Ergebnissen aus \ref{section:pcVsPhone}, überein. Beispielsweise wird empfohlen, dass die App und ihr Design folgende Merkmale aufweisen sollten: \glqq sofort einsatzbereit sein\grqq{},\glqq Aufgaben vereinfachen",\glqq Informationen reduzieren\grqq{},\glqq Informationen indirekt vermitteln\grqq{},\glqq konsistent innerhalb der App und Plattform sein\grqq{} und \glqq intuitiv sein\grqq{}.
%Design
Abschließend wurden die Richtlinien und ihre Regeln sowie die Merkmale auf die verschiedenen Komponenten der App angewendet und erläutert, einschließlich der Kalenderseite, der Erinnerungsseite, der Einstellungsseite, der Appleisten und der Systemtastatur.%
%
%
%
%
%Trimmed
%\myComment{
%
%\myTextTodo{
%%Design -> starke Auswirkung auf Qualität
%\textbf{Abschnitte der Arbeit}\\
%Im \secref{section:design} wird sich Gedanken über das äußere sowie innere Design der App gemacht. Anders gesagt also Überlegungen zu der grafischen Oberfläche sowie der Architektur. Dieser Abschnitt wird behandelt weil, beide dieser Punkte starke Auswirkungen auf Qualität der App ausüben können. Dabei würde die grafischen Oberfläche besonders die Qualität für den Endnutzer beeinflussen, da dies das einzige ist mit dem Benutzer interagiert. Gleicherweise würde die Architektur die Qualität für den Entwickler entscheiden, da der Quelltext seine Schnittstelle darstellt.\newline
%}
%
%}