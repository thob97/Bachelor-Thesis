\subsubsection{Erinnerungenseite}
%Liste
Die Erinnerungsseite zeichnet sich hauptsächlich durch ihre Liste von Erinnerungen aus. Diese Liste nutzt Animationen, um einen fließenden Übergang beim Erstellen und Abschließen von Erinnerungen zu gewährleisten. Ohne diese Animationen könnten abrupte Änderungen in der Darstellung auftreten und möglicherweise zu Verwirrung führen.\newline%
%Einträge
Die Erinnerungseinträge bestehen aus einem Titel und enthalten optional eine Beschreibung sowie Dateien. %
	%Warum
	Denn es wird angenommen, dass für einfache Erinnerungen wenige Wörter oder ein kurzer Satz ausreichen. Für komplexere Erinnerungen kann stattdessen die Beschreibung genutzt werden. In Fällen, in denen sich die Erinnerung hingegen leichter über eine Datei erklären lässt, soll auch dafür eine Möglichkeit bestehen.\newline%
%Kompakt
Das dazu erstellte Design wurde versucht so kompakt wie möglich zu gestalten, um so auf dem relativ kleinen Bildschirm eines Handys alle notwendigen Informationen angemessen darzustellen. %
	%Optional
	Des Weiteren wurde das Design so gestaltet, dass es ansprechend aussieht, unabhängig davon, ob die optionalen Eingaben vorhanden sind oder nicht.\newline%
%Was: bestätigungsknopf
Das Abschließen einer Erinnerung wird durch einen Knopf neben der entsprechenden Erinnerung ermöglicht. %
	%Warum
	Da der Knopf immer sichtbar ist, lässt er sich deshalb mit minimalem Aufwand betätigen, ohne dass weitere Aktionen und Zwischenschritte erforderlich sind. %
	%Links knopf
	Der Knopf wurde bewusst links von der Erinnerung platziert, da an dieser Position davon ausgegangen wird, dass Nutzer\footnote{gilt nur für Nutzer, die das Handy mit der rechten Hand nutzen} ihn seltener versehentlich betätigen.\newline%
%Erstellung
Außerdem wird, falls bei der Erstellung einer neuen Erinnerung kein Titel gewählt wurde, dieser automatisch erstellt, um dem Nutzer Zeit und Aufwand zu ersparen.%
%
%pic{leere seite} pic{mit 4 Einträgen, nur titel, titel und beschreibung, titel datein, titel beschreibung dateien}
%
%