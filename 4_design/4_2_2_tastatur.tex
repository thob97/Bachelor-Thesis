\subsubsection{Tastatur}%
%Was
Wie zuvor erwähnt, ist geplant, alternative Eingabemöglichkeiten für die Erstellung von Erinnerungen anzubieten. Es wurde entschieden, Fotos, Videos, Sprachnachrichten oder andere auf dem Handy bereits vorhandene Dateien als Optionen anzubieten, da angenommen wird, dass diese Dateitypen nützlich für die Erinnerungen sein könnten und einfach über das Handy zugänglich sind.\newline%
%Wo Platzieren
Anstatt diese Alternativen statisch auf einer Seite anzuzeigen, sollten sie nur dann angezeigt werden, wenn sie benötigt werden. Dadurch werden die dargestellten Informationen reduziert und eine bessere Übersichtlichkeit gewährleistet. Gleichzeitig wird die Bedeutung der Aktionen indirekt vermittelt, da dadurch gezeigt wird, wann und wofür die Aktionen genutzt werden können. Der dafür passende Platz ist die Systemtastatur.\newline%
%Kosistenz
Um konsistent mit dem Betriebssystem zu bleiben, soll die Standardtastatur beibehalten werden. Anstatt eine eigene Tastatur zu entwerfen, werden die Alternativen stattdessen in die bereits existierende Standardtastatur integriert.\newline%
%Icons
Die Aktionen werden wiederum als Icons dargestellt, um mit möglichst wenig Platz ihre Bedeutung zu vermitteln. 
\newline%
Für eine Beispieldarstellung siehe Abbildung \ref{pic:tastatur}.
