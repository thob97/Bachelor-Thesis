\subsubsection{Auswahl}


\myNewSection
%Auswahl von Konvention
	%Was: Mehrere Konventionen
	Es stehen mehrere verschiedene Richtlinien für Apps zur Verfügung, z.B. eine für iOS von Apple und eine für Android von Google\cite{konventionen_platforms_ios, konventionen_guidelinesGoogle}. %
	%Was+Warum: Entscheiden
	Es wird jedoch eine einzige Richtlinie gewählt, da unterschiedliche Richtlinien unterschiedliche Regeln enthalten und sich gegebenenfalls sogar widersprechen können. %
		%Beispiel
		Google bietet beispielsweise eigene Seiten und Regeln für \glqq Floating Action Buttons\grqq{}\cite{konventionen_floatingActionButton}, während diese in den Apple Richtlinien nie erwähnt werden und daher wahrscheinlich auch nicht erwünscht sind. %
	%Was+Warum: Auswahl Apple -> eigene Präferenz(leichter+intuitiver)
	Da aufgrund persönlicher Präferenz die Apps von Apple als noch intuitiver und leichter zu bedienen empfunden werden und dies für die Anforderungen der App von Vorteil ist, wurde sich für die Richtlinien von Apple entschieden.%
	% Aus eigener Präferenz wurde aufgrund der Tatsache, dass die Apps dieser Plattform intuitiver und leichter zu bedienen sind und dies den Anforderungen der App entspricht, entschieden, die Richtlinien von Apple zu verwenden.
	
\myNewSection
%Richtlinien: Inhalt + Was wurde betrachtet + Was nicht
	%Was: Woraus besteht die Richtlinie
	Die Richtlinien von Apple bestehen aus fünf Abschnitten, die unter anderem die Gestaltung einzelner Komponenten wie Textfelder und Knöpfe sowie allgemeine Regeln und Muster für die Erstellung von Apps behandeln. Insgesamt umfasst die Richtlinie etwa 148 Regelungen. %
	%Was: davon interessant
	Davon schienen anfangs etwas mehr als die Hälfte als nützlich für die Arbeit. Bei genauerer Betrachtung stellten sich 34 dieser Einträge als tatsächlich passend heraus, da sie Funktionen und Anforderungen behandeln, die in die Arbeit einfließen sollen. %
	%Was: davon uninteressant
	Die meisten restlichen Einträge behandeln Funktionen, die nicht in die Anwendung integriert werden sollen, wie zum Beispiel die Bedienung per Hardwaretastatur oder NFC-Funktionalität\cite{konventionen_keyboard, konventionen_nfc}. Einige andere Einträge, die eigentlich nützliche Funktionen für diese Arbeit beschreiben, wie zum Beispiel ein Nachtmodus oder Siri-Unterstützung, wurden aufgrund des begrenzten Zeitrahmens vorerst übersprungen\cite{konventionen_darkmode,konventionen_siri}.%