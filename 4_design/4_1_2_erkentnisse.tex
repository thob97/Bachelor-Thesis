\subsubsection{Auswertung \& Erkenntnisse}\label{subsection:design:erkenntnisse}%
%Was
In diesem Abschnitt werden Erkenntnisse und Muster beschrieben, die während der Ausarbeitung der Richtlinien ersichtlich wurden.\newline%
%Warum
Diese Erkenntnisse haben einen allgemeineren Charakter als die Richtlinien und können dementsprechend auf die gesamte App und auch auf Bereiche angewendet werden, für die keine Regeln vorhanden sind.\newline%
Dabei ist zu beachten, dass einige der Erkenntnisse eng miteinander verknüpft sind und deshalb einen fließenden Übergang zwischen ihnen herrscht.%
%
\newline
\myNewSection
Das Design und die App sollten gemäß den Richtlinien folgende Merkmale aufweisen:
\begin{enumerate}%[noitemsep,topsep=0pt,parsep=0pt,partopsep=0pt]
	%%% kurzweilig sind oderwenig Zeit benötigen.
	\item Sofort einsatzbereit sein.%
	\begin{itemize}%[noitemsep,topsep=0pt,parsep=0pt,partopsep=0pt]
		\item[] Denn es wird unter anderem empfohlen:% 
		\item den Inhalt der App möglichst früh zu zeigen\cite{konventionen_patterns_loading}%
		\item die Notwendigkeit für eine Anmeldung möglichst lang zu Verzögern\cite{konventionen_patterns_managingAccounts}%
		\item sowie initiale und unterbrechende Informationsanfragen, wie zum Beispiel nach Bewertungen oder Zugriffsrechten, nur wenn wirklich nötig anzufordern\cite{konventionen_patterns_launching}.%
	\end{itemize}


	%%% einfach und intuitiv seien sollen
	\item Aufgaben vereinfachen.%
	\begin{itemize}%[noitemsep,topsep=0pt,parsep=0pt,partopsep=0pt]
		\item[] Denn es wird unter anderem empfohlen:% 
		\item Texte so klar wie möglich zu verfassen\cite{konventionen_foundations_writing}
		\item wo möglich, Informationen automatisch vom System zu entnehmen, anstatt den Nutzer danach zu fragen\cite{konventionen_patterns_enteringData, konventionen_platforms_ios}
		\item sowie Alternativen zur Texteingabe, wie zum Beispiel \glqq drag \& drop\grqq{} oder eine Liste von Optionen, anzubieten\cite{konventionen_patterns_enteringData}.
		%\item choices instead of text entries
	\end{itemize}

	%%% nur wenig Informationen darstellen oder benötigen
	\item Informationen reduzieren.%
	\begin{itemize}%[noitemsep,topsep=0pt,parsep=0pt,partopsep=0pt]
		\item[] Denn es wird unter anderem empfohlen:% 
		\item Texte so kurz wie möglich zu verfassen\cite{konventionen_foundations_writing}
		\item die Anzahl an Steuerelementen zu begrenzen\cite{konventionen_platforms_ios}
		\item simplifizierte Designs für Icons zu verwenden\cite{konventionen_foundations_icons}
		\item sowie wichtigen Informationen genügend Platz zu geben, indem zum Beispiel eher unwichtige Details ausgelassen werden\cite{konventionen_foundations_layout}.
	\end{itemize}	

	%%% einfach und intuitiv seien sollen
	\item Informationen indirekt vermitteln.%
	\begin{itemize}%[noitemsep,topsep=0pt,parsep=0pt,partopsep=0pt]
		\item[] Denn es wird unter anderem empfohlen:% 
		\item Farbe zu nutzen, um Bedeutung zu vermitteln (beispielsweise indem grün für bestätigende Aktionen genutzt wird) \cite{konventionen_foundations_color}.
		\item Icons zu nutzen, da diese bei richtiger Verwendung ein Konzept sofort verständlich machen können\cite{konventionen_foundations_icons} 
		\item Textgröße, Schriftart und Farbe zu nutzen, um Wichtigkeit zu vermitteln\cite{konventionen_foundations_typography}.
		\item Platzierung von Objekten zu nutzen, um Wichtigkeit zu vermitteln\cite{konventionen_foundations_layout}
		\item sowie zur Situation passende Sprache zu verwenden\cite{konventionen_foundations_writing}.
	\end{itemize}
	
	%%% mit der Annahme, dass es sich lohnt bestehende Richtlinien zu folgen
	\item Konsistent innerhalb der App und Plattform sein.% 
	\begin{itemize}%[noitemsep,topsep=0pt,parsep=0pt,partopsep=0pt]
		\item[] Denn es wird unter anderem empfohlen:% 
		\item die Plattformfarben zu nutzen und bei einmal definierten Farben konsistent zu bleiben\cite{konventionen_foundations_color}
		\item die Plattformdefinierten Textstile zu nutzen\cite{konventionen_foundations_typography}
		\item die Plattformdefinierten Gesten zu nutzen\cite{konventionen_foundations_accessibility} 
		\item sowie auf allen Seiten ein konsistentes Icondesign zu verwenden\cite{konventionen_foundations_icons}.
	\end{itemize}
	
	
	%%% einfach und intuitiv seien sollen
	\item Intuitiv sein.%
	\begin{itemize}
		\item Denn alle vorherigen Punkte haben dies als Ziel.
	\end{itemize}
	

\end{enumerate}
%
%
%was: Erkentnisse überschneiden mit PcVsPhone Stärken
Interessant hierbei ist, dass sich diese Erkenntnisse mit den Ergebnissen aus \secref{section:pcVsPhone} überschneiden. %
	%Beispiel:
	So passt beispielsweise die Erkenntnis \glqq Sofort einsatzbereit sein\grqq{} zur Stärke des Handys, gut für kurzweilige Aufgaben zu sein. Die Erkenntnisse \glqq Aufgaben vereinfachen\grqq{}, \glqq Informationen indirekt vermitteln\grqq{} und \glqq Intuitiv sein\grqq{} passen hingegen zum Ergebnis, dass einfache und intuitive Aufgaben gut zum Handy passen.  \glqq Informationen reduzieren\grqq{} passt hingegen zum Ergebnis, dass auf Handys Aufgaben besser funktionieren, die wenig Informationen darstellen oder benötigen.\newline%
	%Auf in Einzelnen Regeln
	Weiter gibt es in den Richtlinien auch einzelne Regeln, bei denen es zu einer ähnlichen Überschneidung kommt. 
		%Wenig Konfig + Eingabe	
		So rät Apple beispielsweise sowohl von Texteingaben als auch von einer hohen Anzahl an Einstellungen ab\cite{konventionen_patterns_enteringData,konventionen_settings}. Dies überschneidet sich mit den Ergebnissen, dass Handys keine schnelle, präzise und vielfältige Eingabe erfordern sollten und ohne viele Optionen und Konfigurationen auskommen sollten.