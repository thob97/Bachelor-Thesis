\subsubsection{Erkentnisse/Auswertung}\myTodo\label{subsection:design:erkenntnisse}%
%Was
In diesem Abschnitt werden die Erkenntnisse und Muster genannt, welche während der Einarbeitung in die Apple Richtlinien erkenntlich wurden.\newline%
%Warum
Denn diese Erkenntnisse wären allgemeiner als die Richtlinien und könnten dementsprechend auf die ganze App und für Steller für welche keine Regeln existieren angewendet werden.\newline%
\myTextTodo{+ als begründung für die konkreten design entscheidungen im nächsten abschintt. viele der Genannten Erkenntnisse sind stark verschränkt miteinander.}

Das Design und die App sollten [laut] den Richtlinien:

\begin{enumerate}%[noitemsep,topsep=0pt,parsep=0pt,partopsep=0pt]
	%%% kurzweilig sind oderwenig Zeit benötigen.
	\item sofort Einsatzbereit sein\newline%
		Denn es wird unter anderem empfohlen:% 
	\begin{itemize}%[noitemsep,topsep=0pt,parsep=0pt,partopsep=0pt]
		\item den Inhalt der App möglichst früh zu zeigen\cite{konventionen_patterns_loading},%
		\item die Anmeldung möglichst lange zu Verzögern\cite{konventionen_patterns_managingAccounts},%
		\item sowie initiale und unterbrechende Informationsanfragen, wie zum Beispiel nach Bewertungen- oder Zugangsrechten, nur wenn wirklich nötig [anzufordern/aufzufordern]\cite{konventionen_patterns_launching}%
	\end{itemize}


	%%% einfach und intuitiv seien sollen
	\item dabei helfen Aufgaben zu vereinfachen\newline%
		Denn es wird unter anderem empfohlen:% 
	\begin{itemize}%[noitemsep,topsep=0pt,parsep=0pt,partopsep=0pt]
		\item Texte so klar wie möglich zu verfassen\cite{konventionen_foundations_writing},
		\item wo möglich Informationen automatisch vom System zu entnehmen, anstatt den Nutzer nach diesen zu Fragen\cite{konventionen_patterns_enteringData, konventionen_platforms_ios},
		\item sowie alternativen zu Texteingaben, wie zum Beispiel \glqq drag \& drop\grqq{} oder eine Liste von Optionen,  anzubieten\cite{konventionen_patterns_enteringData}
		%\item choices instead of text entries
	\end{itemize}

	%%% nur wenig Informationen darstellen oder benötigen
	\item Informationen reduzieren\newline%
		Denn es wird unter anderem empfohlen:% 
	\begin{itemize}%[noitemsep,topsep=0pt,parsep=0pt,partopsep=0pt]
		\item Texte so kurz wie möglich zu verfassen\cite{konventionen_foundations_writing},
		\item die Anzahl an Steuerelemente zu limitieren\cite{konventionen_platforms_ios},
		\item simplifizierte Designs für Icons zu nutzen\cite{konventionen_foundations_icons}
		\item sowie wichtigen Informationen genügend Platz zu geben, indem zum Beispiel eher unwichtige Details ausgelassen werden\cite{konventionen_foundations_layout}
	\end{itemize}	

	%%% einfach und intuitiv seien sollen
	\item stattdessen [Informationen beziehungsweise Bedeutung] indirekt vermitteln\newline%
		Denn es wird unter anderem empfohlen:% 
	\begin{itemize}%[noitemsep,topsep=0pt,parsep=0pt,partopsep=0pt]
		\item Farbe zu nutzen, sodass Nutzer die zu [versuchen] mitteilende 	Information besser verstehen, aber auch, dass sich nicht gänzlich allein dafür auf Farbe verlassen werden soll\cite{konventionen_foundations_color}
		\item Icons zu nutzen, da diese bei richtiger Verwendung, Nutzer sofort ein Konzept  verstehen lassen\cite{konventionen_foundations_icons} 
		\item Textgröße, Font und Farbe zu nutzen um damit Wichtigkeit zu vermitteln\cite{konventionen_foundations_typography}
		\item Platzierung von Objekten zu nutzen um Wichtigkeit zu vermitteln,
		\item sowie zur Situation passende Sprache zu verwenden\cite{konventionen_foundations_writing}
	\end{itemize}
	
	%%% mit der Annahme, dass es sich lohnt bestehende Richtlinien zu folgen
	\item konsistent innerhalb der App und Platform sein\newline% 
		Denn es wird unter anderem empfohlen:% 
	\begin{itemize}%[noitemsep,topsep=0pt,parsep=0pt,partopsep=0pt]
		\item die Platform Farben zu nutzen und konsistent bei einmal definierten Farben zu bleiben\cite{konventionen_foundations_color}
		\item die Platformdefinierten Textstyles zu nutzen\cite{konventionen_foundations_typography}
		\item die Platformdefinierten Gesten zu nutzen\cite{konventionen_foundations_accessibility} 
		\item sowie über alle Seiten eine konsistentes Icondesign zu verwenden\cite{konventionen_foundations_icons}
	\end{itemize}
	
	
	%%% einfach und intuitiv seien sollen
	\item intuitiv sein\newline%
		Denn es wird unter anderem empfohlen:% 
	\begin{itemize}%[noitemsep,topsep=0pt,parsep=0pt,partopsep=0pt]
		\item alle zuvor erwähnten Regeln + 2 Quellen
		\item ...
		\item ...
		\item ...
	\end{itemize}

\end{enumerate}


%was
[Interessant/Nennenswert] hierbei ist es, dass sich [einige/alle/...] diese Erkenntnisse mit den Vermutungen aus \secref{section:pcVsPhone} [übereinstimmen/ähneln/überschneiden]. %
	%
	so \^ ...
	aber auch settings, eingabe vermutungen
	%Wenig Konfig + Eingabe
	So ratet Apple unteranderem einerseits von Texteingaben und einer hohen Anzahl an Einstellungen ab\cite{konventionen_enteringDate,konventionen_settings}. 