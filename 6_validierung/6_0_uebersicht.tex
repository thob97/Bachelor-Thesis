%%% hidden subsection for a better structure in latex editor: "texifier"
\myComment{\subsection*{Übersicht}} 
%
%Einleitung
In diesem Abschnitt wird versucht festzustellen, ob die erstellte Anwendung auch das erfüllt was sich zuvor vorgenommen wurde.\newline%
%
%Übersicht
\textbf{Überblick:} %
Dazu wird eine Anforderungsverifizierung mit den funktionalen und nicht funktionalen Anforderungen aus dem \secref{section:anforderungen} durchgeführt.\newline%
%
%Zusammenfassung
\textbf{Erkenntnisse:}\newline%
	%f.A.
	Im Hinblick auf die funktionalen Anforderungen konnten alle \glqq Must-haves\grqq{} und \glqq Should-haves\grqq{} erfolgreich umgesetzt werden. Lediglich der Parser und die Funktion, Dateien zu Erinnerungen hinzuzufügen, blieben unvollständig, da andere Funktionen aufgrund von Zeitmangel priorisiert wurden. %
	%n.f.A
	Die nicht funktionalen Eigenschaften \glqq Wartbarkeit\grqq{} und \glqq Reichweite\grqq{} konnten aufgrund derselben Einschränkungen nicht vollständig implementiert werden. Jedoch konnten die übrigen nicht funktionalen Anforderungen überwiegend bis vollständig erfüllt werden.%
%
%
%
%\myComment{
%	
%	%Einleitung: Warum
%	\textbf{Abschnitte der Arbeit}\\
%	Um zuletzt festzustellen ob wir auch wirklich das erschaffen haben, was wir uns zuvor als Ziel setzten, wird die Software mithilfe verschiedener Methoden im \secref{section:evaluation} Validiert.\newline%
%
%}