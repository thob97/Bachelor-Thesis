% !TeX encoding = UTF-8

\documentclass[serif,article,noparskip,de]{config-agse-thesis}

%Todo: Maybe Titel anpassen:
%Entwicklung einer Kalender-App für Geeks: Ein Versuch die Lücke zwischen Terminal und Smartphone zu überbrücken
\newcommand{\thesisTitle}{Entwicklung einer Kalender-App für Geeks: Ein Versuch wie man die Lücke zwischen Terminal und Smartphone überbrücken könnte}
% -> You may use \par (but not \\) to format the title. If you do so, you'll

\newcommand{\studentName}{Thore Brehmer}
%===============================================================================

\hypersetup{pdftitle={\thesisTitle}}
\hypersetup{pdfauthor={\studentName}}

\addbibresource{quellen.bib}

%%% multiline comment
\newcommand{\myComment}[1]{}
\newcommand{\myNewSection}[0]{\ \\}

%TODO Remove later after refactoring demonstration texts
\usepackage{lipsum}

%TODO remove after removing all stichpunkte
%Für farbige Checkmarks
\usepackage{tikz}
\newcommand{\myCheckmark}{}%
\DeclareRobustCommand{\myCheckmark}{%
  \tikz\fill[scale=0.7, color=red]
  (0,.35) -- (.25,0) -- (1,.7) -- (.25,.15) -- cycle;%
}

%für farbige rote circles
\newcommand{\myTodo}{}%
\DeclareRobustCommand{\myTodo}{
	\tikz\draw[red,fill=red] (0,0) circle (.5ex);
}

%für roten Text
\newcommand{\myTextTodo}[1]{\textcolor{red}{#1}}%

%für nameref
\usepackage{hyperref}
\newcommand{\secref}[1]{\autoref{#1}. \nameref{#1}}

\begin{document}

\coverpage[
    student/id=5216879,
    student/mail=thob97@zedat.fu-berlin.de,
    thesis/type=Bachelorarbeit,            % optional, default: Bachelorarbeit
    thesis/group={}, % TODO
                                           % optional, default: AGSE
    thesis/advisor={Prof. Dr.-Ing. Volker Roth},           % optional TODO
    thesis/examiner={Prof. Dr.-Ing. Volker Roth},  % TODO
    %thesis/examiner/2={}, % optional TODO
    thesis/date=\today,                    % optional, default: \today
   %title/size=\LARGE,      % set this value to overwrite automatic font size
   %abstract/separate       % toggle this to move the abstract to its own page
]
{ % Your abstract/Zusammenfassung here:
    %%% hidden subsection for a better structure in latex editor: "texifier"
\myComment{\section*{Zusammenfassung}} 
%
%\input{stichpunkte0.tex} %todo: remove after this sections completion
%
%\myNewSection
%\textbf{WisschenschaftlichesSchreiben:}
%\\ Zusammenfassung des Inhalts
%\\ - wird am häufigsten gelesen
%\\ - max länge vorgegeben (150-300 Wörter)
%
%
%
%
%
%
\textit{Ziele:} In dieser Arbeit wurde versucht, eine passende App für CLI-Terminkalender zu erstellen. \glqq Passend\grqq{} bedeutet in diesem Sinne, dass die Stärken von PC und Handy berücksichtigt werden und somit in die zu erstellende Anwendung einfließen.
%
%
%
%
%
\newline%
\newline%
\textit{Methoden:} 
	Dazu wurde zunächst versucht herauszufinden, was die Stärken von Handys und PCs überhaupt sind, indem die wesentlichen Unterschiede zwischen ihnen verglichen wurden.
	%Anforderungen
	Anschließend wurden mithilfe von Erhebungstechniken die Anforderungen an die zu erstellende App ermittelt und festgehalten.
	%Technologische Überlegungen
	Auf Basis dieser Anforderungen wurden Gedanken zur Auswahl der passenden Technologien gemacht. 
	%Design
	Danach wurde das Design der Anwendung erstellt und überdacht, da dies ein wichtiger Punkt für die Anforderungen sowie eine mögliche Stärken von Handys ist.
	%Implementierung
	Da nun die Anforderungen und das Design der Anwendung bekannt sind sowie die Technologie, mit der sie erstellt werden soll, wurde sich anschließend mit der Implementierung befassen.
	%Evaluation
	Zum Schluss wurde die Anwendung mit einer Anforderungsverifizierung evaluiert, um festzustellen, ob das Erstellte auch das erfüllt, was sich zuvor vorgenommen wurde.
%
%
%
% 
%
\newline%
\newline%
\textit{Ergebnisse}:
%Was
Eine wichtige Erkenntnis, die während der Erarbeitung der Stärken gewonnen wurde, ist, dass für die Intuitivität und Benutzerfreundlichkeit einer Handyanwendung das Design und dessen Richtlinien ein entscheidender Faktor sind. 
		%Handy
		Weiterhin kam es zu dem Ergebnis, dass Handys für jene Aufgaben gut geeignet sind, die kurzweilig sind oder wenig Zeit benötigen, die unterwegs gelöst werden sollen und einfach sowie intuitiv sein sollen. 
		\newline%
		%Pc
		Während PCs gut für jene Aufgaben sind, die viel Leistung benötigen, schnelle, präzise oder vielfältige Eingaben erfordern, viele Informationen gleichzeitig darstellen oder benötigen sowie viele Optionen und Konfigurationen anbieten oder benötigen.
	%Anwendung Resultat
	Laut der Evaluation entstand während der Arbeit eine Anwendung, die in der Lage ist, genau diese Stärken umzusetzen.
%
%
%
% 
%
\newline%
\newline%
\textit{Schlussfolgerungen:} 
Es wird angenommen, dass das Ziel dadurch erfüllt werden konnte. Weiter wird sogar vermutet, dass es sich bei dieser App um eine nützliche Anwendung handeln könnte.
%
%
%
%
%
\newline%
\newline%
\textit{Ausblick}:
	%Generell
			%Generell (rest auf app bezogen)
		Falls sich die Anwendung als nützlich beweist, könnte dies bedeuten, dass es für Anwendungen durchaus lohnenswert sein kann, diese unter Berücksichtigung der Stärken sowohl von Handys als auch von PCs zu entwickeln. Möglicherweise könnte dies zeigen, dass Anwendungen nicht unbedingt als eigenständige Einheiten genutzt werden müssen, sondern voneinander abhängig sein und dennoch oder gerade deshalb nützlich sein können.
	\newline%
	%App
		%evaluation
		Um jedoch die Nützlichkeit abschließend bewerten zu können, benötigt es weiterer Evaluation. %Insbesondere ein Nutzertest könnte hierbei nützliche Erkenntnisse liefern.
		Auch konnte die Anwendung noch nicht vollständig implementiert werden. So fehlen noch einige wenige Funktionen, Design Anpassungen und ausprägungen von nicht funktionalen Anforderungen.
}

\include{eidesstattlicheErklaerung}

\cleardoublepage

\tableofcontents

\cleardoublepage

\pagestyle{fancy}

% Actual content starts here

% !TeX encoding = UTF-8
\section{Einleitung}

\myComment{\subsection*{Stichpunkte1}} 

\myComment{

	\myNewSection
	\textbf{Website:}
	\\Was ist das Problem? Warum ist es ein Problem? Wie bettet es sich in andere Arbeiten ein? Was ist nicht das Problem? Was wird nicht gelöst mit dieser Arbeit? \myTodo

	\myNewSection 
	\textbf{My Notes (Also Website)}: \myTodo
	\\ <Hauptteil> Gewählter Lösungsansatz, Alternativen, Abwägungen
	\\<Hauptteil> Beschreibung besonderer Schwierigkeiten und wie sie gelöst, umgangen oder vermieden wurden (oder warum nicht)
	\\<Hauptteil> Dokumentation der Durchführung und der entstandenen Artefakte
	\\ Alle Behauptungen müssen belegt werden, sei es mit einer Literaturquelle, einem sorgfältigen Argument oder mit eigenen empirischen Daten.

	\myNewSection
	\textbf{Purpose}: Dient dem Autor zur Orientierung, aber findet sich normalerweise später in der Einleitung des Dokumentes wieder.
	\begin{itemize}
		\item 1. Eine Beschreibung des größeren Zusammenhangs, in dem das Dokument angesiedelt ist.
		\item 2. Die Beschreibung des konkreten Problems, das im Dokument behandelt wird. \myCheckmark
		\item 3. Die Charakterisierung der Ziele, die das Dokument erreichen soll (z.B. der Information, die es liefern soll). \myCheckmark
		\item 4. Eine Begründung, warum und für wen das Dokument wichtig ist. \myCheckmark
		\item 5. Die Charakterisierung des Vorwissens der angepeilten Leserschaft. \myCheckmark
		\item 6. Eine Auflistung \underline{relevanter Randbedingungen: Zeitbeschränkungen}, Umfangsbeschränkungen, technische Randbedingungen (Medien etc.), äußere Vorgaben (Standards) für Stil, Organisation oder Format. \myCheckmark
		\item Einführung: \underline{Was ist das Problem? Warum ist es ein Problem?} Wie bettet es sich in andere Arbeiten ein? Was ist nicht das Problem? \underline{Was wird nicht gelöst mit dieser Arbeit?} \myCheckmark
	\end{itemize}



	\myNewSection
	\textbf{Prezi:}
	\begin{enumerate}
		\item Zuerst erzähle ich also etwas über die Motivation bzw. der Aufgabenstellung der Arbeit. Also warum mir das Thema relevant erscheint [und was es leisten kann]. \myCheckmark
		\item Bei der Vorgehensweise erkläre ich mit welcher Grundlegenden Herangehensweise ich versuche das Produkt zu erstellen welches die genannten Erwartungen und Ziele aus der Motivation erfüllt. \myCheckmark
		\item Wie gerade erwähnt wird ein agile Arbeitsstil befolgt -> die \underline{Dargestellten Schritte} lassen sich also nicht wie eigentlich abgebildet voneinander trennen und nacheinander lösen, sondern es herrscht ein fließender Übergang während der Bachelorarbeit.
		\\ \underline{Als Beispiel dazu}… Während der Implementation kommt es bestimmt zu Anforderungensveränderungen. Einfach weil man währenddessen neue Ideen hat oder weil einem klar wird, dass die Anforderungen derzeit nicht umsetzbar oder zu Aufwändig sind.
		\\ Jedoch macht es auch Sinn die \underline{Schritte getrennt zu betrachten}. Undzwar jetzt für den Vortrag sowie später in meiner Ausarbeitung. Damit lässt sich nämlich die Komplexität senken. bzw: es macht die Arbeit etwas Verständlicher \myCheckmark
		\item Obwohl ich mir hierzu also schon ein Paar Gedanken gemacht habe. Würde ich diesen Abschnitt, trotzdem während der Arbeit, nochmal detailierter bearbeiten wollen. Einfach weil ich glaube, dass es nützlich und interessant sein könnet. Aber trotzdem versuche ich auch während der Arbeit ständig den Prozess weiter zu verbessern und zu Überdenken. \myCheckmark
	\end{enumerate}
	
	
	
	\textbf{Prof:} Generell bei jeder Übersicht: (Was in diesem Kapitel gemacht wird) + Entscheidungen +-> Ergebnisse dieses Kapitels. \myCheckmark
	
	
	
	\myNewSection
	\textbf{Motivation:} \myCheckmark
	\begin{enumerate}
		\item Für die Motivation ist es vorerst wichtig zu Wissen was \textbf{CLI-Terminkalender} eigentlich sind. Einfach erklärt sind das Terminkalender ohne Grafische Oberfläche. Die Interaktion mit dem Kalender erfolgt stattdessen über das Terminal.
		\item Ein \textbf{Vorteil} von solchen Kalendern ist zum Beispiel: dass man sich nicht an der Vorgegebenen Grafischen Oberfläche anpassen muss. Bei normalen Kalendern mit Grafischer Oberfläche hätte man bei unerwünschte Funktionen oder Design Entscheidungen, nämlich einfach Pech gehabt. Bei den CLI-Kalendern hat man hingegen sehr viele Freiheiten zur Konfiguration und kann sich dadurch selbst sein gewünschtes Umfeld einstellen. 
		\item \textbf{Deswegen} gibt es auch genügend Benutzer welche sich auf den Terminal und damit auch auf CLI-Kalendern wohler fühlen. 
		\item \textbf{Beispiele} für solche Programme wären “when, remind, khal, calcurse, calendar”.
		\item Nun zum \textbf{Problem bzw. der Motivation}: Es gibt derzeitig keine App welche sich zufriedenstellend mit CLI-Kalendern Verbinden lässt. Denn am Wünschenswertesten wäre eine Lösung, welche die Vorzüge des PCs und die Vorzüge der Handys ausschöpft.
		\item Ich meinte ja grade, dass versucht werden sollte die Stärken der jeweiligen Systeme möglichst effektiv zu nutzen, damit sich die beiden Welten Verbinden lassen. Eine Stärke des PC’s ist es zum \textbf{Beispiel}, dass sie eine Tastatur besitzt und sich dadurch schnell und bequem tippen lässt. Ein Handy besitzt zwar auch eine Software Tastatur, diese ist aber viel kleiner und Sondersymbole sind hinter mehreren Layers versteckt. Daher sollte einen die Erkenntnis kommen, dass sich auf Handys weniger gut tippen lässt. Die aller einfachste Lösung dieses Themas und der Arbeit, einen CLI-Terminkalender also einfach 1 zu 1 auf ein Handy zu Portieren scheint also nicht als sinnvoll.
		\item Ein \textbf{Beispiel} wie solch ein Problem gut gelöst wurde, ist der Ipod Nano. Es wurden nämlich nur die benötigten und für das Gerät passende Funktionen implementiert. Das wären zum Beispiel das Abspielen von Musik. Die Restlichen Funktionen konnten nur mithilfe des Pc’s erledigt werde. Zum Beispiel das hinzufügen oder Löschen von Musik.
		\item Mit der Erkenntniss dass Funktionen die gut auf dem PC funktionieren nicht gut auf dem Handy funktionieren müssen 
		\item Bsp. wie man es nicht macht: ssh, Terminal App -> tippen klobig weil benötigte Sonderzeichen nicht leicht erreichbar und Tastatur sehr klein
		\item Und \textbf{noch ein Grund} warum es solch eine App benötigt, wäre, dass es sicherlich nochmehr Leute gibt, welche sich auch auf dem Terminal wohlfühlen, aber trotzdem gerne die Vorzüge einen Handys nutzen würden (Bsp. Mobilität)
		\item \textbf{Das Ziel} ist es also, sich zu Überlegen, wie die beiden Welten miteinander verbunden werden können, und dabei die passende App zu erstellen.
	\end{enumerate}
	
	
	
	\myNewSection
	\textbf{Allgemeine Vorgehensweise} \myCheckmark
	\begin{enumerate}
		\item \textbf{Generell} gilt… also für alle Folgenden Schritte dieses Vortrags, dass während der Bachelorarbeit immer wieder Entscheidungen getroffen werden. Und jede dieser Entscheidungen sollte sorgfältig gewählt und gut Begründet sein.
		\begin{enumerate}
			\item Grund dafür ist erstens, dass es einerseits Interessant für den Leser und andererseits wichtig für die Evaluation ist: den Gedankengang hinter den getroffenen Entscheidungen nachvollziehen zu können.
			\item Und zweitens, weil diese Fragen einen selbst dabei helfen… eine bessere oder sogar die beste Entscheidung, aus vielen Möglichkeiten, zu finden.
		\end{enumerate}
		
		\item Für den \textbf{Prozess} habe ich mir eine Agile-Arbeitsweise vorgenommen. Und zwar unteranderem wegen den Folgenden drei, eng miteinander verbundenen, Prinzipien:
		\begin{enumerate}
			\item Das erste ist dabei die Arbeit in \textbf{Iterationen}. Denn die Schritte zur bau einer Software lassen sich nicht nacheinander und in einem Rutsch abarbeiten. Daher werde ich mich stattdessen in mehreren kleinen Schritten zum Ziel herantasten. Vorteile davon sind, dass man steht's ein Evaluierbares Produkt hat… und dies ist natürlich sehr nützlich für die Abgabe. Außerdem hilft Iteration auch bei Anforderungsunsicherheiten. Denn ich gehe davon aus, dass sich getroffene Entscheidungen während der Arbeit / bzw. der Implementierung noch ändern könnten, obwohl man zuvor dachte, dass diese Entscheidung die richtige Wahl wäre.
			\item Das zweite Prinzip ist der Kurzer \textbf{Planungshorizont}: Anfangs würde ich nämlich einmal Oberflächlich einige Allgemeine Ziele raussuchen. Danach gilt immer nur für eine oder zwei Iteration im Vorraus im Detail zu Planen. Das hat den Vorteil das Änderungen weniger verheerend sind, da die Planung leicht geändert werden kann und nicht zu viel Zeit Anfangs darin gesteckt wurde.
			\item ODER: Ein kurzer \textbf{Planungshorizont}: würde dabei heflen, dass Änderungen weniger verheerend sind. Undzwar weil nicht zu viel Zeit in die Planung gesteckt wird.
			\item Das letzte Prinzip sind die \textbf{Retrospektiven}. Dabei würde ich mich wöchentlich, also nach je einer Iteration, Fragen was gut und was schlecht lief. Diese Reflexion kann nämlich dabei helfen den Prozess zu Verbessern und Ursache von Mängeln zu beseitigen (konstruktive Qualitätssicherung) (ständige wird sich Schrittweise verbessert)
		\end{enumerate}
	\end{enumerate}
	
	
}%%% %todo: remove after this sections completion

%%% hidden subsection for a better structure in latex editor: "texifier"
\myComment{\subsection*{Übersicht}}\myCheckmark

%Einleitung \ Überblick
\textbf{Überblick:}
Dieser Abschnitt handelt von der konkreten Beschreibung des Problems sowie der Charakterisierung des Ziels. Dabei soll unter anderem auch der Nutzen der Arbeit bewusst werden.\newline
Außerdem wird die Struktur dieses Dokumentes erläutert, sowie das vorausgesetzte Vorwissen der angepeilten Leserschaft genannt. Dadurch soll verdeutlicht werden ob und wie die Ausarbeitung gelesen werden kann.\newline
%Result
\textbf{Ergebnisse:}
Für diese Ausarbeitung werden allgemeine Informatik Kenntnisse vorausgesetzt. Dafür enthält jeder Abschnitt dieser Arbeit, um das Lesen zu vereinfachen und das Querlesen zu ermöglichen, einen Überblick und eine Zusammenfassung.\newline
Während der Beschreibung des Problems kam es zu mehreren Erkenntnissen. Einerseits erfreuen sich Handys Relevanz und Beliebtheit und dementsprechend lohnt es sich Anwendungen auch als App anzubieten. Weiterhin besitzen Handys und Pc's verschiedene Stärken und Schwächen und CLI-Terminkalender sind diesbezüglich eine Interessante Anwendung.
\subsection{Vorraussetzungen}

Im Rahmen dieser Ausarbeitung werden wiederholt begriffe aus dem Jargon Informatik genutzt. Auf jene Begriffe welche mit einem Informatikstudium als Informatischesallgemeinwissen/Allgemeinwissen bezeichnet werden, wird nicht näher eingegangen. Das heißt also im Umkehrschluss, dass ein gewisses informatisches Vorwissen vom Leser vorausgesetzt/erwartet wird%
\footnote{Beispiele Begriffe zur Orientierung:
	\begin{itemize}[noitemsep,topsep=0pt,parsep=0pt,partopsep=0pt]
		\item wird Vorausgesetzt: Framework, Package, Byte
		\item wird nicht Vorausgesetzt: \myTodo
	\end{itemize}
	\nointerlineskip %removes space between itemize and next footnote
}. \newline
Das hat den Grund, dass die Zielgruppe dieses Dokumentes genau diese Vorwissen bereits besitzt. Darüberhinaus wird die Leserlichkeit stark verbessert, wenn der Lesefluss nicht ständig durch Definitionen unterbrochen wird. \newline
Um ein möglichst großes Spektrum der Zielgruppe zu erreichen, werden wichtige und nicht Allgemeine Begriffe erläutert und es werden genügend Beispiele gegeben. Für die Arbeit eher unwichtige Begriffe, sowie nebensächliche Beispiele werden aber meistens nur im [Anhang] oder der Fußnote erwähnt. Andererseits würde dies den leseflus zu sehr beeinträchtigen.
\subsection{Struktur}
Im folgenden wird kurz die Struktur dieses Dokumentes erläutert. Dadurch soll bewusst werden, auf welche Arten die Ausarbeitung gelesen werden kann.

\myNewSection Die Abschnitte 1-7 wurden so gewählt, wie man sich den Ablauf eines 'naiven' Software Projektes vorstellet \footnote{Für eine Beispieldarstellung siehe Abbildung [\ref{fig:wasserfallmodell}]}. 
Zwar ließen sich die Abschnitte während der Bachelorarbeit so gut wie nie getrennt voneinander betrachten und lösen, sondern es herrschte steht's ein fließender Übergang. Trotzdem wurden versucht, diese Abschnitte und auch alle Unterpunkte möglichst gut getrennt voneinander zu betrachten und (modularisieren). 
Durch die Aufteilung in verschiedene Kapitel wirk die Bachelorarbeit sehr viel überschaubarer und auch schaffbarer. (Ganz im Sinne "die Lösung wird in vielen kleinen Schritten erreicht") (Nicht umsonst ist eins der wichtigsten Ziele in der Softwaretechnik Projekte in verdauliche Happen aufzuteilen (keep? + Quelle)). Die Trennung der Abschnitte, bzw. anders gesagt die Modularizierung der Abschnitte ermöglicht das 'Querlesen'. Damit ist gemeint, dass Textabschnitte auch ohne das Lesen des kompletten Dokumentes verstanden werden können. 

\myNewSection
Damit ein gewisser Überblick über alle Kapitel geschaffen wird, enthält jeder Abschnitt am Anfang eines Kapitels eine kurze Einleitung über die Unterpunkte und über was sie handeln.
Außerdem haben alle Abschnitte eine kurze Zusammenfassung. Darin werden wichtige Entscheidungen und Ergebnisse des Kapitels genannt. Das soll einen die Möglichkeit geben ganze Abschnitte oder sogar Kapitel überfliegen zu können und trotzdem die wichtigsten Aspekte und Resultate der Arbeit zu verstehen. Anders gesagt hilft diese Strukturentscheidung erneut beim Querlesen
\subsection{Motivation}\label{subsection:motivation}
%
%Relevanz von Handys steigt
Als Handys erstmals auf den Markt kamen, waren sie im weitesten Sinne in erster Linie als eine mobile Alternative zu den herkömmlichen Telefonsystemen gedacht \cite{einleitung_handy_erfindung} . Heutzutage scheinen Handys aber einen viel wichtigeren Teil unseres Lebens eingenommen zu haben. Anscheinend fühlt sich bereits die Mehrzahl an Leuten unwohl das Haus ohne ihr Handy zu verlassen\cite{pcVsphone_feelingUneasyWhenLeavingPhoneHome}. Des Weiteren werden Handys mittlerweile täglich 40 Minuten mehr als PCs benutzt\cite{pcVsphone_phoneScreenTime,pcVsphone_totalScreenTime,pcVsphone_totalScreenTime2}\footnote{Handy: 3.44 Stunden, PC:2.59 Stunden, zusammen: 6.43 Stunden.}. Das zeigt einen starken Wandel, wenn man bedenkt, dass Computer im Jahr 2011 noch rund 94\% des Marktanteils ausmachten\cite{pcVsphone_smartphoneWebTrafficHigherThanPc}. Diese Veränderung scheinen auch Entwickler und Unternehmen zu bemerken. So kündigte zum Beispiel Google bereits in 2017 den Umstieg auf Mobile-Indexing\footnote{Suchergebnisse werden besser bewertet, wenn Seiten eine mobile Ansicht anbieten.}an\cite{pcVsphone_mobileFirstIndexing} und Unternehmen wie Facebook und Pinterest beziehen den der Großteil ihrer Benutzerschaft aus Applikationen\cite{pcVsphone_socialMediaFacebookMobileUsage,pcVsphone_socialMediaPinterestMobileUsage}.\newline%
Generell scheinen Handys also immer beliebter und wichtiger zu werden. Das spiegelt sich auch in den Nutzerzahlen wieder. Während laut einer Studie 96\% der Befragten ein Handy besitzen ist die Anzahl bei PCs von 73\% auf 59\% innerhalb der letzten vier Jahre\footnote{2018 bis 2022} gesunken\cite{pcVsphone_deviceOwnership}. Auch der Marktanteil des Handys zeigt ein Indiz darauf, denn dieser steht mit 53\% knapp über den des Computers\cite{pcVsphone_smartphoneWebTrafficHigherThanPc}.%
%
%Folgerungen: Apps machen Sinn + Es Handys und Pc's haben verschiedene Stärken 
\newline
\myNewSection%
Basierend auf der Erkenntnis, dass das Handy einen wichtigen Stellenwert eingenommen hat, lassen sich für diese Arbeit zwei interessante Schlussfolgerungen ziehen.\newline
Zum einen, dass es lohnt sich lohnt, Anwendungen nicht nur für den PC, sondern auch für Handys anzubieten.\newline%
Zum anderen, dass das Handy gewisse Vorteile und Stärken im Vergleich zu PCs bieten muss, was dazu führt, dass es an Beliebtheit und Nutzung gewinnt. Dies scheint jedoch beidseitig zu gelten. So gibt es Anwendungen, die trotz der steigenden Handy-Popularität besser auf dem Computer funktionieren. Beispielsweise möchten wahrscheinlich nur wenige Menschen eine wissenschaftliche Arbeit oder Steuererklärung auf einem Handy schreiben.%
%
%Begründung für CLI-Kalendar Anwendung: auf Stärken des Pc's ausgelegt + existiert keine App
\newline
\myNewSection
Eine weitere Gruppe von Anwendungen, die sich besser auf dem Computer als auf dem Handy nutzen lassen, sind die CLI-Terminkalender\footnote{Beispiel CLI-Terminkalender: remind\cite{cli_remind}, khal\cite{cli_khal}, calcurse\cite{cli_calcurse}}. Das Kürzel \glqq CLI\grqq{} steht hierbei für \glqq Command Line Interface\grqq{}, was gleichbedeutend mit einer Kommandozeile oder einem Terminal ist. Bei diesen Terminkalendern gibt es keine grafische Benutzeroberfläche, stattdessen wird über das Terminal interagiert. Die Anwendung erfordert es, unter anderem lange Texte, Befehle und Sonderzeichen zu schreiben. Durch die Tastatur ist dies auf dem Computer einfach auszuführen. Im Gegensatz dazu würde die begrenzte Softwaretastatur des Handys die Bedienung dieser Anwendung erschweren. Wahrscheinlich ist dies auch der Grund, warum es bisher keine erfolgreichen CLI-Terminkalender-Anwendungen für das Handy gibt.%
%
%%%---Kommentare---%%%
%
%%Warum CLI-Kalender gut zu Pc's passen: Keyboard + config Möglichkeit
%\myComment{Diese Eigenheit bringt einige Vorteile mit sich. Zum Beispiel, dass man sich nicht an der vorgegebenen Grafischenoberfläche anpassen muss. Bei "normalen" Kalendern mit Grafischeroberfläche kann man gegen unerwünschte Funktionen oder Design Entscheidungen nichts machen. Bei den CLI-Kalendern hat man hingegen sehr viele Freiheiten zur Konfiguration und kann sich dadurch selbst sein gewünschtes Umfeld einstellen. Besonders die Commandos, die einzige Interaktion mit dem Kalender, lassen sich oft nach belieben anpassen.}
%
%%Alt: keine zufriedenstellende Lösung
%\myComment{Jedoch gibt es keine Zufriedenstellende Lösung für die Verbindung dieser Kalender mit dem Handy. Denn am Wünschenswertesten wäre eine Lösung, welche die Vorzüge des PCs und des Handys betrachten. }
\subsection{Zielsetzung}

Ziel ist es also eine Kalender-App zu erstellen welche die Stärken des Pcs und des Handys ausschöpfen. Oder anders formuliert ist das Ziel die Entwicklung einer Kalender-App für Geeks, mit dem Versuch die Lücke zwischen Terminal und Smartphone zu überbrücken. Ziel ist es also offensichtlich nicht einen CLI-Kalender auf das Handy zu portieren, sondern eher der Versuch eine passend Lösung für dieses Problem zu finden. \newline
Natürlich soll nicht nur irgendeine App erstellt werden sondern eine möglichst sinnvolle (funktionale Anforderungen) und hochwertige (nichtfunktionale Anforderungen). Wie Versucht wird die beiden Forderungen sicherzustellen kann im Kapitel \myTodo Anforderungen betrachtet werden.

\myNewSection
Sehr wichtig nochmal zu betonen ist, dass es sich lediglich um einen Versuch handelt. Die in der Arbeit ermittelten Informationen und Ergebnisse sollen weder als Lösung noch als Beispiel dienen. 
Da die Bearbeitungszeit während einer Bachelorarbeit auch äußerst beschränkt ist, soll das Ziel auch nicht sein ein fertiges Produkt zu liefern, sondern eher sich schrittweise dem fertigen Produkt zu nähern.



%\subsection{Vorgehensweise}

%%% hidden subsection for a better structure in latex editor: "texifier"
\myComment{\subsubsection*{Übersicht}} 
\myTodo


\myComment{

	\textbf{Überblick:} In diesem Abschnitt geht es um die allgemeine Vorgehensweise, welche wir während der Bachelorarbeit befolgen. Zuerst wird dazu erklärt welche Abschnitte diese Arbeit hat und warum es genau diese gibt.
	
	
	Im ersten Teil werden die Abschnitte dieser Arbeit erläutert und begründet. 
	Im Abschnitt ... wird eine generell befolgte Arbeitsweise in der Ausarbeitung erläutert.
	Zuletzt folgt die Auswahl und Begründung zum gewählten und befolgten Prozess.\newline%
	\textbf{Zusammenfassung:} Verglichen wurde das Wasserfall-Modell mit einer Agilen-Arbeitsweise, im Bezug auf diese Arbeit. Wir kamen zum Schluss, dass ein Agiler-Prozess mehr zu dieser Arbeit passt, da es die Prinzipien: Iteration, Planungshorizont und Retrospektiven unterstützt.

}
%%Allgemein Arbeitsweise: Herangehensweise an die Arbeit
\subsubsection{Allgemein Arbeitsweise} \myTodo
Generell gilt, also für alle Folgenden Schritte dieses Vortrags, dass während der Bachelorarbeit immer wieder Entscheidungen getroffen werden. Und jede dieser Entscheidungen sollte sorgfältig gewählt und gut Begründet sein. \newline
Grund dafür ist erstens, dass es einerseits Interessant für den Leser und andererseits wichtig für die Evaluation ist, den Gedankengang hinter den getroffenen Entscheidungen nachvollziehen zu können. \newline
Und zweitens, weil diese Fragen einen selbst dabei helfe eine bessere oder sogar die beste Entscheidung, aus vielen Möglichkeiten, zu finden.
\subsubsection{Abschnitte der Arbeit}
Um das in ... erwähnte Ziel umzusetzen werden eine Vielzahl von Ingenieurmäßigen-Techniken angewandt. \newline
In diesem Kapitel ... wurde geklärt was die Software überhaupt leisten könnte und was für einen Nutzen das bringt. \newline
So behandelt das Kapitel ... die Auseinandersetzung mit der Frage, was solch eine Software überhaupt alles leisten soll. Diese Aufgabe sollte möglichst früh bedacht werden, da sich besonders hier Schwierigkeiten und Ziele für die nächsten Teilschritte schnell auffinden lassen. Dadurch fällt das gestalten der nächsten Schritte also sehr viel leichter. \newline
Im darauf folgende Kapitel ... wird sich Gedanken über die Auswahl von Technologien gemacht. Dabei wird es gezielt nach der Anforderungserhebung behandelt, weil sich bereits dieser Schritt Auswirkungen auf Funktionale und nicht Funktionale Anforderungen haben kann. Man stelle sich vor wir wählen zuerst ein Framework welches bekannt für seine langsame Ausführung ist und erst danach erheben wir die nicht Funktionale Anforderung "schnelle Performance", dann entsteht dadurch ein Konflikt und wir müssten die Wahl über das Framework neu überdenken. \newline
Das nächste Kapitel ... ist sehr wichtig. Es werden sich Gedanken über das äußere sowie innere Design der App gemacht. Also Überlegungen des Grafischenoberfläche sowie der Architektur. Beide dieser Punkte können enorme Auswirkungen auf Qualität der App ausüben. Dabei würde die Grafischenoberfläche besonders die Qualität für den Endnutzer beinflussen und das 'innere Design' die Qualität für den Entwickler.\newline
 An dieser Stelle sollten die wichtigsten Erkenntnisse erhoben worden sein. Nun gehen wir über zum Kapitel ... . Hier wird Überdacht wie die Software aufgebaut sein soll. Also auch hier geht es unter anderem wieder um das 'innere Design'. Ein Großteil der Arbeit in diesem Kapitel wird aber nicht reflektiert/wiedergegeben, undzwar die eigentliche Implementierung. Jede einzelne Implementierungs-Entscheidung zu erwähnen und Begründen würde nicht nur den Rahmen sprengen, sonder auch sehr ermüdent sein. Daher werden nur die wichtisten und allgemeinsten(für viele Punkte zutreffenden) Entscheidungen und Schwierigkeiten erwähnt \newline
 Um zuletzt festzustellen ob wir auch wirklich das erschaffen haben, was wir uns zuvor als Ziel setzten, wird die Software mithilfe verschiedener Methoden Validiert. <ref>
\subsubsection{Prozess}
Eine wichtige Annahme ist, dass man in der Regel genau dann hohe Qualität erhalt, wenn man geeignete/sinnvolle Arbeitsweisen verwendet und diese sorgfältig/gezielt durchführt. Zum Beispiel was für Themen betrachtet werden sollen, wie Umfangreich man diese betrachtet und wann man zum nächsten Thema übergeht.
Da dies ein sehr großer und somit auch wichtiger Teil für die Arbeit ist, sollte man diese Gesamt-Vorgehensweise nicht neu erfinden, sondern sich stattdessen auf vorhandene Erfahrungen halten.
Das erste Modell was wir dazu betrachten ist das Wasserfall-Modell. Dies hätte den Vorteil, dass man Schritte zum Bau der Software, genau wie in dieser Arbeit, voneinander trennen und nacheinander lösen kann. Dadurch würde die Zeitplanung auch einfacher ausfallen, was sich sehr gut bei einer vorgegebenen Zeitangabe, wie bei dieser Arbeit, macht.
Jedoch wurde sich doch dagegen entschieden. Die Annahme "alle Aufgaben lassen sich getrennt voneinander lösen", ist bei dieser Arbeit nicht Sinnvoll. Es ist sehr Wahrscheinlichkeit, dass sich während der Implementierung und dem Design neue Erkenntnisse zu den Anforderungen ergeben. Man stelle sich zum Beispiel vor, dass sich erst bei der eigentlichen Implementierung bewusst wird, dass eine Funktion, aus Schwierigkeit und Zeitgründen, nicht umzusetzen ist. Es wird also stattdessen davon ausgegangen, dass sich die Schritte in diesem Softwareprojekt nicht nacheinander und getrennt voneinander lösen lassen, sondern das zwischen Ihnen ein fließender Übergang herrscht.
Auch die Zeitplanung lässt sich nicht, wie im Wasserfall-Modell vorgesehen, einhalten. Dafür benötigt es ein wohldefiniertes Resultat. In dieser Arbeit herrscht aber große Unsicherheit. Anforderungen, Design sowie die Art der Implementierung müssen alle zuerst überlegt/erhoben werden.

\myNewSection
Stattdessen wurde sich dann doch für eine Agile-Arbeitsweise entschieden. Und zwar unteranderem wegen den Folgenden, eng miteinander verbundenen, Prinzipien:
	\begin{enumerate}
		\item Das erste ist dabei die Arbeit in \textbf{Iterationen}. Es wird sich also in vielen kleinen Schritten zum Ziel herangetastet. Einerseits hat man dadurch steht's ein Evaluierbares Produkt, was sehr nützlich für die Evaluierung ist. Außerdem die Arbeit in kleinen überschaubaren Schritten auch bei der zuvor erwähnten Anforderungsunsicherheiten dieser Arbeit. Jedoch gibt es auch Nachteile bei der Arbeit in Iterationen. So könnte diese bei schlechter Planung oder sehr großer Unsicherheit zu Doppelarbeit führen. Zum Beispiel weil das zuletzt erst erstellte und für richtig gehaltene Produkt nun doch nicht benötigt wird.
		\item Das zweite Prinzip ist der Kurzer \textbf{Planungshorizont}. Auch dies würde dabei helfen, dass Änderungen weniger verheerend sind, weil sich nicht strikt an einen Plan gehalten werden muss.
		\item Das letzte Prinzip sind die \textbf{Korrekturen}. Dabei geht das Prinzip sinnvoll davon aus, das man während eines Projektes Fehler machen wird. Deswegen wird der Prozess so gestaltet, dass Auftretende Fehler gut behoben werden können. Offensichtlich erfüllen die beiden zuvor genannten Prinzipien bereits diese "Regel".
		\item Prozessmodelle passen verschieden gut zu Projekten. So könnte das Wasserfall-Modell gut zu einem bereits sehr definierten Projekt passen und Agile zu einer Revolutionären Idee. Auch das einhalten der Prinzipien und Regeln kann misslingen. Von daher ist es wichtig das Prinzip \textbf{ständige reflektion} zu erwähnen. Dabei werden zum Beispiel wöchentlich, also nach je einer Iteration, Überlegungen über den Prozess gemacht. Diese Reflexion helfen einen dabei, schrittweise, da wo es hapert oder wo es sehr gut läuft, den Prozess zu Verbessern und Ursache von Mängeln zu beseitigen.
	\end{enumerate}











\section{Handy und PC Unterschiede}\label{section:pcVsPhone}

\myNewSection
%%% hidden subsection for a better structure in latex editor: "texifier"
\myComment{\subsection*{Übersicht}}\myTodo
\textbf{Übersicht:} %
%Einleitung
Um das in \secref{section:zielsetzung} erwähnte Ziel umzusetzen muss sich zuerst die Frage gestellt werden, was überhaupt die Unterschiede zwischen Pc und Handy sind. Dabei geht es nicht darum möglichst alle Unterschiede aufzuzählen, sondern eher darum [überwiegenden, starken, wesentlichen, den ausschlaggebende] Unterschiede zu betrachten und so für diese Arbeit wichtige Erkenntnisse, [also die Stärken und Schwächen des Handys und Pc's], zu [gewinnen/identifizieren]. [Denn genau diese Erkenntnisse sollen uns dabei helfen die Stärken und Schwächen des Handys und Pc's zu identifizieren.]\newline%
%1 Teil: Unterschiede:
Dafür werden im ...Abschnitt... die Unterschiede betrachtet.\newline%
%2 Teil: Erkenntnisse:
Und im ...Abschnitt.. werden die Erkenntnisse zusammengefasst und daraus Schlussfolgerungen gezogen.\newline%
\textbf{Ergebnisse:}



\subsection{Leistung}\label{PcVsPhone:Leistung}
%Was: Pc
In Bezug auf Leistung gelten PCs im Allgemeinen als schneller und stärker im Vergleich zu Smartphones. 
	%Warum: 
	Das liegt an verschiedenen Faktoren. Zum einen haben die meisten PCs einen konstanten Zugang zu Strom, während Smartphones häufig nur über Batterien verfügen. Darüber hinaus können PCs aufgrund ihres größeren Volumens leistungsstärkere Hardware-Komponenten verbauen und größere Kühlsysteme nutzen, was die Verwendung dieser stärkeren Hardware überhaupt erst ermöglicht.\newline%
%Was + Warum: Handy
Im Gegensatz dazu ist die Hardware von Smartphones in der Regel auf Portabilität und Energieeffizienz ausgelegt, um eine möglichst lange Akkulaufzeit zu ermöglichen.\newline%
%
%
%
%
%
%
%\myComment{
%
%	%%%->Benutzung%%%
%	%Zusammenfassung: Leistung VS Portabilität
%	Während Pc's also auf möglichst performante Leistung ausgelegt sind, wird stattdessen beim Handy eher auf die Portabilität geachtet.\newline%
%	
%	\myNewSection
%	\myTextTodo{
%	-> Leistung vs Portabilität\\
%	- Aufwändige Hardware vs Optimierung(Leistung + Batterie) 
%	}
%	
%	
%	\myNewSection
%	%Auswirkungen
%	-Eine mögliche Auswirkung davon ist, dass aufwändige Anwendungen öfters nur auf dem genutzt(da schneller) oder gar nur auf dem Pc unterstützt werden. Solch eine aufwändige Anwendung wäre zum Beispiel das exportieren von Videodateien. 
%	-Währenddessen wird/muss auf dem Handy wahrscheinlich eher auf die Optimierung geachtet, um wenig Strom zu verbrauchen und so eine längere Betriebszeit zu ermöglichen. Und um handys schneller zu gestalten (3 sekunden regel)
%
%}
\subsection{Internet}\label{PcVsPhone:Internet}
%Was: Pc
Was PCs betrifft, so sind sie oft mit schnellen und stabilen Ethernet-Anschlüssen verbunden, die jedoch stationär sind. %
%Was: Handy
Im Gegensatz dazu werden Handys in der Regel über WLAN oder mobile Daten genutzt. Diese Alternativen sind wesentlich mobiler, da sie keine Kabelverbindung erfordern. Stattdessen wird die Verbindung über die Luft hergestellt, was es ermöglicht, dass das Handy fast überall eine Internetverbindung herstellen kann. Allerdings ist zu beachten, dass diese Verbindungen oft langsamer und fehleranfälliger sind als kabelgebundene Verbindungen, da die Luft im Vergleich zum Kabel ein schwächeres Übertragungsmedium darstellt.%
%
%\myComment{
%
%	%%%->Benutzung%%%
%	%Warum: internet wichtig für Handy ----------- Leistung + Internet -> Quelle
%	Wobei besonders diese Schnelligkeit und Stabilität auf dem Handy wichtig zu seien scheint, vielleicht genau weil sie auf dem Handys oft fehlt. So verlassen die hälfte aller Handy Nutzer laut Google die Website wenn sie länger als drei Sekunden braucht zu laden\cite{pcVsphone_threeSeconds}.\newline%
%	
%	%Zusammenfassung
%	Also wird auch hier, wie im Vergleich der \nameref{PcVsPhone:Leistung}, Performance, in diesem Fall in Form von Schnelligkeit und Stabilität, gegen Mobilität getauscht. Schnelligkeit und Stabilität scheint den Nuztern wichtig zu sein\newline%
%	
%	\myNewSection
%	\myTextTodo{
%	-> Leistung(Schnelle und stabile Verbindung) vs Portabilität\\
%	- Schnelle und stabile Verbindung vs Portable Verbindung -> Optimierung wichtig (internet stabilität + schnelligkeit)
%	}
%	
%	\myNewSection
%	%Auswirkungen
%	- Das führt zum Beispiel dazu, dass für Anwendungen welche von eine schnelle und stabile Internetverbindung profitieren, eher der Pc präferiert wird. So zum Beispiel bei online Spielen oder Videoübertragungen.
%	- Das Handy muss für den Netzwerkverkehr optimiert werden, sonst stellt dies ein flaschenhalz her, um nicht länger als 3 sekunden zu laden
%	-> kleine + kurze Aufgaben funktionieren gut auf dem handy. Da dafür weder eine stabile(kurze) noch schnelle(kleine) internetverbindung nötig ist + die portabilität ein prositiver faktor, da für solche Aufgaben nicht der Aufwand des Pc's anschaltens betrieben werden muss
%
%}
\subsection{Speicher}\myCheckmark
%Zusammenfassung
Wie bereits im Vergleich der \nameref{PcVsPhone:Leistung} und dem \nameref{PcVsPhone:Internet} wird auch beim Speicher die Leistung zugunsten der Mobilität eingeschränkt. 
%Warum: Speicher
Dadurch dass der Speicher in Handys möglichst klein sein soll, bieten sie dementsprechend meistens auch weniger Kapazität. 
	%Quellen
	So besitzen Handys im Durchschnitt, laut einer Studie von CointerPpoint, rund 118 Gigabyte Speicherkapazität\cite{pcVsphone_storageSmartphone}. Im Gegensatz zeigt ein Report von Seagate, dass sie im vierten Quartal 2022 im Durchschnitt Festplatten mit einer Größe von rund 8 Terabyte verkauften\cite{pcVsphone_storageSeagate}. 
	%Was: Erkenntnis -> Handy Speicher < Pc Speicher
	Zwar sind Server wahrscheinlich der größte Einfluss dafür wie es zu diesen hohen Speichergrößen kommt, da diese oft besonders viel Kapazität benötigen, jedoch bietet diese Erkenntnis trotzdem ein starkes Anzeichen dafür, dass Pc's generell mehr Speicher besitzen als Handys.\newline%
%Auswirkung
Dadurch dass der Speicher auf Handys also relativ klein ist, wird vermutet, dass Handy-Nutzer eher auf ihren Speicher achten. Während der Pc gerne als Speicherablage genutzt wird, wird auf dem Handy eher wichtige Daten und welche man häufiger braucht gespeichert.
   



\subsection{Eingabe}

\myNewSection
\textbf{User Input}: Die meisten Pc's werden mithilfe physischer Tastaturen bedient. Handys besitzen hingegen nur eine Software-Tastatur welches im Vergleich zu Hardware-Tastaturen in der Vielfalt der Tasten, Größe und Tippgefühl sehr eingeschränkt sind. So sind Sonderzeichen oft nur durch mehrere extra Schritte zu benutzen und die Funktionstasten fehlen oft ganz. Des Weiteren verleitet einen die kleine Größe und das fehlende Feedback einer echten Tastatur dazu langsamer zu schreiben und öfters Fehleingaben zu machen.\newline%
Da die Handy-Tastatur also Fehleranfälliger und schwerer zu benutzen ist, nehmen wir an, dass Handy Nutzer weniger gern lange texte auf dem Handy schreiben. \newline%
Neben der Tastatur wird auch eine Computermaus benutzt um den Pc zu steuern. Sie bietet im eine sehr präzise Eingabe für die Oberfläche. Auf dem Handy wird hingegen der Touch-Input als Maus-Ersatz benutzt. Dieser kann sich wohlmöglich intuitiver anfühlen, jedoch ist er um einiges unpräziser. Um dem Entgegenzuwirken fordert es ein passendes Design. So wird für Apps zum Beispiel eine Mindestgröße für Buttons vorgeschrieben "On a touchscreen, buttons need a hit target of at least 44x44 points to accommodate a fingertip"\cite{konventionen_buttonSize}. \newline%
Des Weiteren besitzen Computer-Mäuse mehrere Tasten für unterschiedliche Funktionen. Die wahrscheinlich meist benutzten sind der Links-click fürs Bestätigen, rechts-click für Optionen und scroll-wheel zum Blättern. Um diese etablierten Funktionen auch auf dem Handy zu ermöglichen werden Gesten benutzt. So gibt es die Wischgeste zum Blättern und long-press-Geste welcher oft den Rechtsklick ersetzt. \newline%
Jedoch geht es bei den Gesten nicht nur darum das Handy auf die Funktionen des Pc's anzupassen, sondern darüber hinaus gibt es auch noch weitere Handy eigene Gesten für Funktionen. Außerdem fühlt sich die Nutzung von Gesten oft Naturelle und intuitiver an. So benutzen aus meiner Erfahrung viele Menschen ihr Handy lieber zum Durchstöbern ihrer Bilder als den Pc, da die Möglichkeiten des Swipens und Herranzomens sich besser anfühlen, als mit Maus und Tastatur Bilder zu öffnen. \newline%
Während das Handy in der vorherigen Vergleichen immer Leistung gegen Mobilität getauscht hat, sieht es etwas anders aus. Zwar hat das Handy hier erneut an Mobilität gewonnen, da weder eine Hardware-Tatsatur noch eine Maus vonnöten ist und auch hat es in dem Sinne an Leistung verloren, da präzisen Eingaben erschwert wurden, dafür fühlen sich aber simple Aktionen(Gesten) oft um einiges besser auf dem Handy an. 

\myTextTodo{Typing on mobile devices is still a pain. Even alternative input options like swipe keyboards and voice-to-text are often inaccurate and slow users down. Users fear making mistakes and anticipate the interaction cost of data entry on their mobile devices. Particularly if the activity is personally important (like an important email to a client), users might choose a larger device to avoid mistakes.}
\subsection{Bildschirm}
%Warum: Bildschirmgröße
Pcs bieten die Möglichkeit, mehrere Monitore gleichzeitig zu nutzen und ihre Bildschirme sind in der Regel größer als die von Handys \cite{pcVsphone_screenResolutionStats}\cite{pcVsphone_screenResolutionToSize}\footnote{Die Monitorgröße wurde anhand der Monitorauflösung abgeschätzt; die meisten Monitore sind zwischen 14 und 23 Zoll groß.}.\newline%
	%Was: Anzahl an darstellbaren Informationen
	Dadurch kann der PC viele Informationen gleichzeitig darstellen.\newline%
%Warum: Hoch- vs Querformat
Handys und PCs unterscheiden sich auch in der Ausrichtung ihrer Bildschirme. Während PC-Bildschirme meist im Landschaftsmodus betrieben werden, werden Handys meist im Porträtmodus verwendet.\newline%
	%Was: Verschiedene Darstellungen/Anwendungen
	Dies bedeutet, dass Anwendungen unterschiedlich gut auf den beiden Geräten funktionieren, da sie sich aufgrund der verschiedenen Ausrichtung und Bildschirmgröße unterschiedlich darstellen. %
		%Beispiel
		Zum Beispiel ist die Darstellung einer großen, detaillierten Tabelle auf einem PC wahrscheinlich einfacher, während lange vertikale Listen mit einfachen Details sich besser auf Handys überfliegen lassen.\newline%
%Auswirkung: Design
Dementsprechend wird angenommen, dass bei der Entwicklung von Anwendungen für Mobilgeräte besonderes Augenmerk auf das \nameref{section:design} gelegt werden sollte. %	%Begründung :Richtlinien + Google/Apple
	Diese Annahme wird weiter dadurch bestärkt, dass Apple und Google für die beiden beliebtesten Plattformen iOS und Android\cite{pcVsphone_mobileOperatingSystem} ihre eigenen Designrichtlinien veröffentlicht haben\cite{konventionen_guidelinesApple, konventionen_guidelinesGoogle}.\newline%
	%Warum: intuitiver + einfacher
	Interessant ist, dass durch die Anwendung dieser Richtlinien, die App intuitiver und einfacher werden kann. %
		%Begründung: Regeln abgestimmt für die Eigenschaften des Handys
		Dies liegt einerseits daran, dass diese Richtlinien Regeln enthalten, die speziell auf die Eigenschaften von Mobilgeräten abgestimmt sind. Eine Regel besagt beispielsweise, dass Schaltflächen eine Mindestgröße von 44x44 Pixel haben müssen, um die Größe der Finger zu berücksichtigen.\cite{konventionen_buttonSize}.\newline%
		%Begründung: Reichweite + Einfluss -> Verbreitung -> Konvention -> ähnliches Design+Zurechtfinden
		Andererseits spielt wahrscheinlich auch die Verbreitung dieser Richtlinien eine Rolle. So wirkt sich die Reichweite und der Einfluss von Apple und Google wahrscheinlich auch auf die Bekanntheit und Verbreitung ihrer Richtlinien aus. %
		%Konvention
		Durch eine hohe Bekanntheit und Verbreitung würden immer mehr Apps diesen Richtlinien folgen. %
		%ähnliches Design -> intuitiver
		Dadurch würden sich im Umkehrschluss verschiedene Apps ähnlich bedienen lassen und Nutzer müssten nicht für jede Anwendung eine neue Bedienung erlernen. Dementsprechend finden sich Nutzer in neuen Apps schneller zurecht.\newline%
		%Quelle: Richtlinien sind nützlich :TODO: quelle wirklich hier verwenden? oder in auswertung oder design oder nfA?
		Ein Indiz für die Annahme, dass Richtlinien und deren Verbreitung Anwendungen einfacher und intuitiver machen, wäre die größere Popularität von Handy-Apps im Vergleich zu Webseiten\cite{pcVsphone_mobileAppVsWebTimeSpent}\footnote{Laut der Studie werden rund 88\% der Nutzung von Handys für Apps und 12\% für den Browser verwendet.}. Denn während für Apps zumindest einige der Regeln aus den Richtlinien, wie zum Beispiel die Knopfgröße, zwingend angewendet werden müssen\cite{konventionen_buttonSize}, sind für Webseiten diese Regeln nicht Vorschrift.%
%
%
%
%\myComment{
%
%		%Old Backup: TODB Remove
%		%Die Richtlinien sowie dessen Verbreitung sind wahrscheinlich ein [großer] Grund dafür, warum Handys als leichter zu bedienen bewertet werden\cite{pcVsphone_easyUseVsImportantTask}. Außerdem gibt das auch ein Indiz warum auf dem Handy Apps sehr viel beliebter sind als Websites\cite{pcVsphone_mobileAppVsWebTimeSpent}, denn für Webseiten muss keine der Appeigenen Richtlinien befolgt werden. Bei Apps gibt es einige Regeln, welche zwingend für die Veröffentlichung der App benötigt werden.\newline%
%
%
%	%%%->Benutzung%%%
%	
%	%Zusammenfassung
%	Die Bildschirme von Handy und Pc unterscheiden sich. So ist der Bildschirm vom Handy kleiner und dadurch mobiler, während der Pc Monitor durch seine Größe viele Informationen gleichzeitig darstellen kann. Daher kam es zu der Erkenntnis, dass unterschiedliche Anwendungen und Darstellungen verschieden gut auf den beiden Geräten funktionieren.
%	Um der Informationsarmut des Handys entgegenzuwirken muss muss sich nun detaillierte Gedanken über die Darstellung der App gemacht werden. Wenn einem dies gelingt, kann die App im Vergleich zu Pc Anwendungen sogar um einiges intuitiver und leichter werden.	
%		
%	
%	
%	
%			
%	\myNewSection
%	%Auswirkung
%	- Verschiedene Darstellungen und Anwendungen können unterschiedlich gut abgebildet werden.\\
%	-> Das Handy braucht um gut zu funktionieren andere Designentscheidungen als Pc's\\
%	-> Schwere Aufgaben (viele Details) funktionieren besser auf dem Pc\\
%	-> Nutzer benutzen Handys gerne, da sie durch das Design einfach und intuitive zu nutzen sind\\
%	-> Einfache Aufgaben funktionieren gut auf dem Handy, da genau diese Aufgaben oft nicht viele Informationen Darstellen müssen + Handy durch andere Punkte bereits intuitive und einfach, was diesen Punkt noch mehr verstärkt. 
%	- 
%	
%	%Was
%	%Warum
%	%Zusammenfassung
%	%Auswirkung
%	
%	\myNewSection
%	\myTextTodo{
%	-> Leistung(Darstellbare Informationen) vs Portabilität\\
%	- um dem entgegenzuwirken Optimierung(Design) -> intuitive + einfach\\ 
%	}
%
%}
%
%
%
%
%\myComment{Da diese Konventionen bereits sehr etabliert sind, immerhin werden sie von Android und Apple empfohlen, die beiden größten Handy Hersteller \cite{}, werden sie auch sehr häufig in Apps eingesetzt + werden in eigenen Apps verwendet. -> nutzer gewöhnt sich an konventionen -> neue nutzer finden apps simpler (da alle das gleiche navigaions muster) verwenden -> auf pc's gibt es keine solche konventionen/regeln -> schlechtes design lässt sich auf dem pc allgemein eher verzeihen, da großer bildschirm + präzisere eingabe -> apps simpler / pc anwendungen oft komplizierter quotate? hier benuzten oder in "hier, usage, nfA, einleitung konventionen"?}
\subsection{Mobilität}

\myNewSection
\textbf{Portability}: Einer und wenn nicht sogar der wichtigste Unterschied zwischen Pc's und Handys ist die Mobilität. Alle zuvor erwähnten Unterschiede sind also eigentlich nur eine Auswirkung dieser von Handys gewünschten Eigenschaft. Wenn es nicht mobil sein müsste, wäre mehr Leistung, ein größerer Bildschirm, eine stabilere Internetverbindung usw möglich. Jedoch wären wahrscheinlich auch die entstandenen Vorzüge durch die Design-Richtlinien und der Eingabegesten verfallen. \newline%
Durch die kleine Größe der Smartphones genießen es also eine hohe Mobilität. Zwar sind Laptops auch schon um einiges mobiler als Stand-Pc's, jedoch kommen sie in dem Aspekt nicht an Handys heran. Handys passen in die meisten Hosentaschen, während man für Laptops oft eine größere Tasche oder einen Rucksack benötigt. \newline%

\myTextTodo{Überlegen ob hier oder in Benutzung: durch mobil -> tasks on the go -> ...
Zwar ist dies keine Stärke welche wir direkt in der App einbauen und nutzen können. Jedoch vielleicht die größte Stärke des Handys und eine Deutung dafür, dass Smartphone apps beleibt sind und sich diese Arbeit lohnt.}
\subsection{Benutzung}\myCheckmark
%Was: Einstiegsaufwand
	%Pc
	Um Pc's zu benutzen, bedarf es eines gewissen Einstiegsaufwands. %
		%Warum: Stationär
		So muss sich einerseits, da Desktops stationär sind, zuerst zum Standort des Pc's begeben werden. %
		%Warum: Startuptime
		Darüber hinaus muss der Pc vor jeder Nutzung eingeschaltet werden. Denn es wird davon ausgegangen, dass Pc's normalerweise ausgeschaltet sind, wenn sie derzeit nicht in Benutzung sind. Das Anschalten dauert in der Regel ein paar Sekunden. Laut dem Benchmark \glqq Startup Timer\grqq{} beträgt die schnellste aufgezeichnete Startzeit bis der Pc nutzbar ist elf Sekunden\cite{pcVsphone_boottime}.\newline%
	%Handys
	Bei Handys existiert dieser Aufwand hingegen nicht. %
		%Warum: Mobil + Relevanz -> Immer bei einem
		So wird davon ausgegangen, dass die meisten Nutzer ihr Handy immer bei sich tragen. Das liegt einerseits an der zuvor erwähnten Mobilität, aber andererseits auch in der \nameref{subsection:motivation} erwähnten Relevanz und Beliebtheit von Handys. %
		%Warum:
		Weiterhin wird davon ausgegangen, dass Handys normalerweise nicht ausgeschalten werden, um unter anderem für wichtige Anrufe oder Nachrichten erreichbar zu sein. Das Handy ist also immer an und benötigt dementsprechend keine Startzeit.\newline%		
%Was: Multitasking%-------------------------Reword-------------------------------------
Ein weiterer Aspekt bei der Verwendung ist, wie Anwendungen auf den jeweiligen Geräten genutzt werden können. %
	%Handy: 
	Dabei können Handys immer nur eine Anwendung gleichzeitig darstellen. Das könnte eine Limitation des Betriebssystems oder der Leistung sein. Vermutlich wird das Multitasking aufgrund des kleinen Displays nicht unterstützt.\newline%
	%Pc:
	Währenddessen unterstützt der Pc genau diese Funktion. Er kann mehrere Anwendungen gleichzeitig ausführen und darstellen.%
	
	
\myComment{

\myTextTodo{\textbf{One task at a time}: Although many mobile operating systems now offer a split-screen mode, the small screen size limits its usefulness. The fact is, in most cases, users on mobile devices must focus on one window at a time. This limitation means that it’s difficult to combine multiple sources of information and carry out complex tasks. These mobile constraints are no problem if the task is simple, unimportant, or open-ended. However, when the task is goal-based and has high stakes, these constraints are reason enough to save the task for another device}

}
\subsection{Hardware}\myCheckmark
%Was: Modulär
Die meisten Desktops sind hinsichtlich ihrer Hardware modular. %
	%Warum: Konfigurierbarkeit/Anpassungsfreiheit 
	So wird dem Benutzer die Option gegeben viele Hardwarekomponenten nach belieben auszutauschen. %
	%Warum: Auswirkung: Größe
	Dementsprechend muss das Gehäuse des Pc's auf diese Anpassungen aber auch ausgelegt sein, denn es gibt Hardwarekomponenten in einer Varietät von Größen und Formen. Die Modularität kommt also mit dem Nachteil, dass das Gehäuse groß genug sein muss um auch die verschiedenen Hardware-Optionen zu ermöglichen.\newline%
%Was: Handy allInOne 	
Das Handy unterstützt dementsprechend diese Konfigurierbarkeit nicht. Dafür bietet es sich als all-in-one-Gerät an. %
	%Warum: Ease of use
	Während man sich beim Pc auch Gedanken um Monitor und Eingabegeräte machen muss, fällt die Wahl bei einem Handy leichter aus. Bei ihm wird alles zusammen in einem System angeboten. Unter anderem besitzen sie sogar, anders als die meisten Pc's, eine Kamera, GPS, Gyroskop und die Option für biometrisches login Verfahren.\newline%
	
	
	
\myComment{

	%%%->Benutzung%%%
	
	%Zusammenfassung
	Hier erkennen wir also einen Abtausch zwischen Anpassungsfähigkeit und Einfachheit. Während die meisten Pc's sehr viele Optionen zur Konfiguration geben wurde beim Handy für einen bereits alle Entscheidungen getroffen.%
	
	%Was
	%Warum
	%Zusammenfassung
	%Auswirkung
	- Handys werden als leichter angesehen (all in one - ease of use) -> gut für Simple Aufgaben VS Pc eher für Leute die sich auskennen oder konfigurieren wollen
	- Dafür sind Pc's sehr anpassungsfähig. -> Gut für Arbeit / wichtige und Komplexe Aufgaben. Da manche Aufgaben zum Beispiel sehr spezifische Komponenten benötigen. Zum Beispiel einen schnellen Prozessor, oder eine schnelle Grafikkarte, oder viel Speicher, eine soundkarte, sehr schnelle netzwerkgeschwindigkeit, oder viele Monitore usw...
	
	\myNewSection
	\myTextTodo{
	-> Leistuns(Modulär) vs Portabilität\\
	-> Aber auch Modularität vs Abgeschlossenheit -> Easy of use \\
	- 
	}


}
\subsection{Betriebssystem}\myCheckmark
%Was: Pc: Vielfalt + Entscheidungsfreiheit
Bei Desktops gibt es eine große Vielfalt an zu wählenden Betriebssystemen und dem Nutzer ist die Option überlassen sich für eins oder mehrere davon zu entscheiden. %
	%Was: lange Unterstützung
	Selbst mit alter Hardware lassen sich oft aktuelle Betriebssysteme installieren. So unterstützt der knapp zehn Jahre alter Prozessor Intel Pentium J1750 das weltweit meist genutzte Betriebssystem Windows 10\cite{pcVsphone_intelWindowsSupport, pcVsphone_destkopOperatingSystem, pcVsphone_windowsVersions}\newline%
%Was: Handy Betriebssysteme
Für Handys wird hingegen ein festes Betriebssystem vorgesetzt. %
	%Was: keine Entscheidungsfreiheit + kurze Unterstützung
	Das Betriebssystem lässt sich nicht ohne weiters ändern und wird meist nur drei bis fünf Jahre unterstützt \cite{pcVsphone_deviceSupportGoogle}\cite{pcVsphone_deviceSupportApple}. %
		%Nachteil: Sicherheit und Leitsung
		Zwar kann das Handy nach dieser Zeitspanne noch weiter betrieben werden, jedoch leidet ohne weitere Softwareunterstützung die Sicherheit und Leistung darunter. Das hat also den Nachteil, dass alle drei bis fünf Jahre das Handy gewechselt werden sollte.\newline%
		%Vorteil: Optimierung
		Die kürzere Unterstützungszeit der Handy Betriebssysteme bietet aber auch einen Vorteil. Dadurch können sich die Betriebssystem Entwickler auf eine kleinere Anzahl von Handys konzentrieren und können die Software dementsprechend gut auf diese Optimieren. Dadurch sollten Handys in ihrer Lebensdauer mit weniger Defekte und [Ruckler,guten performance, flüssigen userexperinec] funktionieren.\newline%








\myComment{

%%%->Benutzung%%%

	%Zusammenfassung
	Während der Pc einen also die Freiheiten für Optionen und Wahlmöglichkeiten lässt, ist das Handy auf Einfachheit und Benutzerfreundlichkeit ausgelegt. (Immerhin muss man sich keine Gedanken um ein passendes Betriebssystem machen und die Leistung wird durch Optimierung garantiert). -> Handy Betriebssystem ist auf genau auf die Hardware ausgelegt.
		
	%Auswirkung
	-> Handys wirken auf Nutzer einfacher. Man muss sich weniger Gedanken um OS machen,  'it just works'. -> passend für simple Aufgaben, diese sollten auch einfach zu lösen sein
	-> Pcs sind mehr konfigurierbar und erweiterbar -> anpassbarkeit ist gut für komplexe und schwierige aufgaben, denn diese brauchen wohlmöglich komplexe und spezifische umgebungen
	
	\myNewSection
	\myTextTodo{
	-> Konfiguration VS Ease of use\\
	- 
	}

}
\subsection{Software}
%Was: Pc Software -> Konfigurierbar
Die Software für den PC ist oft konfigurierbar. %
	%Beispiel: Installation
	Bereits bei der Installation können verschiedene Optionen ausgewählt werden, wie beispielsweise der Speicherort, die automatische Aktualisierung, die Desktop-Verknüpfung und das automatische Starten.\newline%
%Was: Handy Software -> Benutzerfreundilchkeit
In puncto Konfigurierbarkeit scheinen Handys eher auf Benutzerfreundlichkeit ausgelegt zu sein. %
	%Warum: Begründung + Beispiel(Installation)
	Bei der Installation neuer Apps genügt oft ein Knopfdruck. Fragen wie der Speicherort oder die Erstellung einer Verknüpfung werden vom Betriebssystem übernommen. Das Abarbeiten von Optionen entfällt hierbei. Dementsprechend wird angenommen, dass sich der Übergang von der Installation bis zum Nutzen der App für den Nutzer flüssiger gestaltet.\newline%
%Was: Pc Software Allgemein -> Pc besser Konfigurierbar 
Dieses Verhalten scheint sich auch über die Installation hinaus fortzusetzen. Beispielsweise scheinen Anwendungen mit vielen Optionen generell besser auf dem PC zu funktionieren. %
	%Begründung: Aus Erfahrung besser auf Pc
	So werden Aufgaben wie das Editieren von Videos, die Verwendung von Entwicklungsumgebungen oder das Erstellen von Steuererklärungen werden erfahrungsgemäß üblicherweise am Computer ausgeführt. %
		%Begründung: Zuvor Erwähnten Unterschieden -> viele Optionen schwer zu überschauen + klicken
		Möglicherweise liegt das an den zuvor erwähnten Unterschieden und Limitierungen von Handys, wie beispielsweise der Displaygröße. Bei einem kleinen Display und ungenauer Eingabe könnten viele Optionen dazu führen, dass sie schwerer auswählbar und überschaubar werden.\newline%
%Was: Pc Software Allgemein -> Weniger Optionen %---------TODO re check this part----------
Aus Erfahrung spiegelt sich dieser Sachverhalt auch in Apps wieder. So werden für Handys meist nur die relevanten Optionen dargestellt. Das würde bedeuten, dass die restlichen eher nebensächlichen Optionen bereits vom Entwickler oder Betriebssystem getroffen worden.\newline% 
	%Was: Entwickler machen sich mehr Gedanken: Optionen
	Daher wird vermutet, dass sich Entwickler bei der Erstellung von Apps mehr Gedanken über passende Entscheidungen machen, da ihnen bewusst ist, dass die Nutzer die Optionen später nicht selbst anpassen können.\newline%
	%Warum: Vorteil: Userexperience
	Im besten Fall führt das dazu, dass Handynutzer sich weniger oder gar keine Gedanken über die Auswahl von Optionen machen müssen und sich die App dementsprechend intuitiver anfühlt.%
%
%
%
%
%
%\myComment{
%
%%%%->Benutzung%%%
%
%	%Zusammenfassung
%	Während der Pc also erneut Optionsfreiheit anbietet scheint das Handy wieder Benutzerfreundlicher zu sein. So lässt die Pc Software einen sehr viele Optionen zum selbst konfigurieren, während Apps sofort einsatzbereit sind. 
%	
%	%Was
%	%Warum
%	%Zusammenfassung
%	%Auswirkung
%	- Handy Benutzerfreundlicher\\
%	- Handy Besser für einfache Aufgaben, da man dort meistens eh nicht viel konfigurieren möchte
%	- Generell scheinen Komplexe Aufgaben mit vielen Optionen besser auf dem Pc zu funktionieren.
%	
%	\myNewSection
%	\myTextTodo{
%	-> Konfiguration VS Einfachheit -> Optimierung
%	}
%
%}

\subsection{Auswertung}\myTodo
%Einleitung
Nun folgt eine Aufzählung der gesammelten Erkenntnisse und Schlussfolgerungen dieses Abschnittes. % 
Alles was zuvor in den vorherigen Unterabschnitten und Vergleichen begründet wurde, wird hier nicht erneut [begründet/aufgesagt/erwähnt].%todo mabye warum?

% erfordert/bedarf/voraussetzt/benötigt/brauchen
% denn/darüberhinaus/außerdem/des weiteren/zusätzlich/daneben
\myNewSection
Pc's funktionieren für jene Aufgaben gut welche:
\begin{enumerate}%
	\item viel Leistung benötigt.\newline%
	Denn durch den dauerhaften Zugang zu Strom kann performante Hardware benutze werden. Das Ethernet bietet eine schnelle und stabile Internetverbindung. Und durch die Größe des Pc's kann große Hardware, wie zum Beispiel Festplatten mit viel Kapazität, verbaut werden.%
	\item schnelle, präzise oder vielfältige Eingaben erfordern.\newline%
	Denn die Maus und die Tastatur lassen sich durch ihre Größe und dem physischen Feedback schneller und präziser bedienen. Die vielfältige Eingaben wird durch die hohe Anzahl an Tasten und die Möglichkeit für Tastenkombinationen ermöglicht.%
	\item viele Informationen gleichzeitig darstellen oder brauchen.\newline%
	Denn mithilfe des großem Displays können viele Details dargestellt werden. Darüberhinaus kann durch das Multitasking weitere Informationen von anderen Anwendungen [besorgt/dargestellt/herangeschafft] werden.%
	\item viele Optionen und Konfigurationen anbieten oder benötigen.\newline%
	Denn die Hardware und das Betriebssystem kann bei Desktops beliebig ausgetauscht und konfiguriert werden. Außerdem ist ein großer Bildschirm hilfreich wenn viele Optionen dargestellt werden sollen.% 
	\item (lange dauern/viel Zeit benötigt).\newline% 
	Denn einerseits benötigen Aufgaben auf dem Pc generell etwas mehr Startzeit, da der [Einstiegs/anfangs] Aufwand größer ist. Andererseits helfen die schnellen und präzisen Eingaben dabei lange Aufgaben schneller bewältigen zu können.%   
\end{enumerate}%
%
\myNewSection
Handys funktionieren für jene Aufgaben gut welche:
\begin{enumerate}
	\item Ressourcen schonend sind oder nicht viel Leistung benötigen.\newline%
	Denn das Handy besitzt durch die Batterie nur begrenzt Strom, daher sind die Komponenten eher auf Effizienz ausgelegt. Das WLAN oder die mobilen Dates sind oft langsamer und instabiler im vergleich zum Ethernet. Und durch die kleine Größe sind auch nur kleine Festplatten mit begrenzter Kapazität möglich.%
	\item keine schnelle, präzise oder vielfältige Eingabe erfordern.\newline%
	Denn das kleine Display kann nur wenig Tasten gleichzeitig darstellen. Der Finger ist größer und damit auch zum Auswählen unpräziser als ein Mauszeiger. Und dementsprechend ist die Eingabe auch langsamer, da es sonst zu fehleranfällig würde.%
	\item nur wenig Informationen darstellen oder brauchen.\newline%
	Denn das Display vom Handy ist sehr klein und wirkliches Multitasking wird meistens auch nicht unterstützt.%
	\item ohne viele Optionen und Konfiguration auskommen.\newline%
	Denn das Betriebssystem und die Hardware sind [nicht modular/abgeschlossen/nicht änderbar]. Außerdem übernimmt das Betriebssystem viele eigentlich optionale Entscheidungen, wie zum Beispiel das Installationsverzeichnis von Anwendungen. Des Weiteren lassen sich durch das kleine Display auch nicht viele Optionen gleichzeitig darstellen.%
	\item kurzweilig sind.\newline%kurze Aufgaben welche man schnell lösen will
	Denn einerseits ist das Handy immer bei einem und es benötigt keinen [Einstiegs/anfangs] Aufwand. Und andererseits fällt das Abarbeiten von langen Aufgaben auf dem Handy schwerer, da die Eingabe langsamer und unpräziser ist.%
	\item (man Unterwegs lösen möchte.)\newline%
	Denn dadurch dass das Handy mobil ist, immer bei einem ist und immer über Internet verfügt ist genau das möglich.%
	\item (einfach und intuitiv zu lösen seien sollen).\newline%
	Denn erstens wird die Darstellung der App mithilfe von Richtlinien benutzerfreundlicher. Zweitens ist die Bedienung des Handys durch unteranderem die Gesten intuitiver. Und zuletzt muss sich der Nutzer keine Gedanken um die Konfiguration machen, das Betriebssystem, die Software und Hardware wurden bereits passend für das Handy konfiguriert.%
\end{enumerate}

\myNewSection%
%Was: Schlussfolgerung -> Handys leichte Aufgaben vs Pc's komplexe Aufgaben
(Schlussfolgernd/Zusammengefasst) und angesichts der Punkte der Aufzählung scheinen Handys für jene Aufgaben gut zu funktionieren welche als einfach, simpel oder leicht zu beschreiben sind. Während die Aufgaben des Pc's eher als aufwändig, komplex oder wichtig zu beschreiben sind.\newline%
%Begründung/Quellen
	%Warum: Studie -> Pc wichtig + Handy simpel
	Die Erkenntnisse lassen sich auch durch eine Studie bekräftigen. So scheinen die Befragten, Pc's lieber für wichtige Aufgaben zu nutzen. Währenddessen werden Handys als leichter zu benutzen bewertet, was zu den einfachen und simplen Aufgaben des Handys passen würde\cite{pcVsphone_easyUseVsImportantTask}.\newline%
	%allgemeine Nutzerverhalten
	Aber auch das allgemeine Nutzerverhalten im Internet deutet darauf hin.
		%Warum: Googel Suche -> Handy simpel + Pc Komplex
		So sind einerseits die Suchanfragen auf Google je nach Gerät anders. Auf dem Pc werden zum Beispiel eher aufwändige und komplexe Kategorien wie Computer, Elektronik, Arbeit, Ausbildung und Wissenschaft angefragt. Während auf dem Handy oft eher nach simplen und kurzen Aufgaben wie nach Essen, Nachrichten und Sport gesucht wird\cite{pcVsphone_onWebsites_DevicesDistrubition_TimeSpent_Bouncrate_PageViews_Categories}.\newline%
			%Warum: Website Visits
			Das diese Aufgaben auch wirklich einfacher sind, kann anhand der Aufrufe und Dauer von Webseiten beobachten werden. So kommen zum Beispiel 68\% alles Webseiten-Aufrufe von Handys, aber sie machen nur 33\% der Zeit die auf Webseiten verbracht werden aus\cite{pcVsphone_onWebsites_DevicesDistrubition_TimeSpent_Bouncrate_PageViews_Categories}.\newline% 
		%Auswirkung:
		Das Handy wird beim Surfen also eher für kurzweilige und schnelle Aufgaben benutzt, während sich auf dem Pc mehr Zeit gelassen wird.\newline%
		%Warum: Emails
		Ein ähnliches Verhalten lässt sich auch bei Emails feststellen. Laut einem Survey von Adobe werden für Emails mit Arbeitsthemen lieber der Pc genutzt während für private Emails stattdessen zum Handy gegriffen wird. Da Arbeit oft mit aufwändigen, komplexen und wichtigen Aufgaben verbunden wird, deutet auch diese Aussage mit der Erkenntnis überein\cite{pcVsphone_personalEmailsVsWorkEmails}.
% !TeX encoding = UTF-8
\section{Anforderungen}

%%% hidden subsection for a better structure in latex editor: "texifier"
\myComment{\subsection*{Übersicht}}\myCheckmark
%Einleitung
	%Was + Warum:
	Da wir nun wissen, welche Aufgaben auf dem Pc und welche auf dem Handy gut funktionieren, kann als nächstes die Frage behandelt werden \glqq was die zu erstellende App überhaupt können und leisten soll\grqq{} und wie diese Eigenschaften auf die App übertragen werden können. %
	%Warum:
	%Da das [Resultat,Erkenntnisse,Antwort] dieser Frage die darauf folgenden Abschnitte stark beeinflusst, wurde sich damit so früh wie möglich befasst.%
%Übersicht
\newline
\textbf{Übersicht:}
	%Vorgehensweise
	Dabei wird im \secref{subsection:anforderung:vorgehensweise} behandelt auf welche Arten und mit welchen Techniken versucht wird dies Frage zu beantworten. %
	%fA +nfA
	In \secref{subsection:anforderung:nichtFunktionaleAnforderungen} und \secref{subsection:anforderung:funktionaleAnforderungen} werden die gewünschten Eigenschaften sowie einige der Funktionen der App aufgezählt, begründet und bewertet.%
%Ergebnisse
\newline
\textbf{Ergebnisse:} %
%Was: 1: Vorgehensweise
Für die Erhebung wurden sich die Techniken Introspektion, Umfrage und Vergleich entschieden, denn diese schienen für die Arbeit den besten Ausgleich zwischen Informationen und Zeitkosten zu liefern. %
%Was 2: n.f.A
Bei den Anforderungen wurde stets versucht Entscheidung entsprechend der Stärken des Handys und Pc's zu treffen. Dementsprechend wurde die nicht funktionalen Anforderungen \glqq Stärken von Pcs und Handys\grqq{} als am wichtigsten bewertet. Eine Stärke des Handys ist die \glqq Benutzbarkeit\grqq{}, weswegen sie auch nochmal getrennt betrachtet und als zweit wichtigste nicht funktionale Anforderung bewertet wurde. %
%Was 3: f.A.
Zuletzt wurden die Funktionen einer \glqq Verbindung zum Backend\grqq{}, einer \glqq grafische Darstellung für den Kalender\grqq{} sowie einen \glqq Übersetzer für CLI-Terminkalender\grqq{} als [unbedingt] erforderlich eingeschätzt. Aus ihnen besteht also das Grundgerüst der App. Aber auch Funktionen wie das \glqq Erstellen, Bearbeiten und Löschen von Einträgen \grqq{}, \glqq Benachrichtigungen \grqq{} und \glqq Konfigurationen auf dem Pc \grqq{} werden für diese Anwendung als wichtig eingeschätzt.


%Maybe AbschlussPrezi
\myComment{
(---Außerdem sind viele Vorzüge des Handys erst durch optimierung entstanden, zum Beispiel die intuitive Benutzung durch Gesten und Design, daher muss sich überlegt werden wie diese umzusetzen ist (n.f.A.)---)

}

\subsection{Vorgehensweise}\label{subsection:anforderung:vorgehensweise}\myCheckmark
%Warum: Unbewusst was gebaut werden soll
Als Entwickler ist einem oft garnicht bewusst was überhaupt gebaut werden soll, da es schwer zu durchschauen und herauszufinden ist was die Software in dem Anwendungsgebieten  leisten soll.\newline%
%Was: Erhebungstechniken Vergleichen
Deshalb wird sich in diesem Unterabschnitt genau dazu Gedanken gemacht. Es wird überlegt wie die Anforderungen für diese Arbeit am besten erhoben werden können. Dazu werden eine Reihe von Erhebungstechniken verglichen. %
%
\begin{itemize}
	\item \textbf{Introspektion}: %
		%Was: def
		Bei der Introspektion wird versucht selbstständig durch das Nachdenken Anforderungen zu erheben. 
		%Auswirkung: Benutzung
		Da jede Entscheidung diese Arbeit [sowieso] gut überdacht sein sollte, wurde diese Erhebungstechnik durchgängig und fast immer benutzt. %
		%Vorteil: Missverständnisse + kein vorbereitungs-aufwand
		Diese Technik hat zwei Vorteile. Erstens können aus eigenen Überlegungen keine Missverständnisse entstehen. Zweitens benötigt es keine großen Vorbereitungs-Aufwand, [da man sofort loslegen kann]. %
		%Nachteil: Domäne auskennen
		Jedoch muss man sich dafür mit der Domäne auskennen. Wenn man diese nicht versteht, können einen auch keine Ideen einfallen.%
	\item \textbf{User Feedback}: %
		%Warum: Schwer alles zu überblicken
		Als einzelne Person ist es eine schwere Aufgabe alle Anforderungen und Wünsche vieler Nutzer zu erraten und überblicken. Daher sind Erhebungstechniken wie das User Feedback nützlich. %
		%Was + Warum: Userfeedback
		Dabei geben einen Nutzer Feedback über die Software. Damit werden nicht nur existierende Funktionen bewertet, sondern es können auch neue Wünsche und Funktionen geäußert und entdeckt werden. %
		%Was: nicht nutzen
		Jedoch wird diese Technik nicht genutzt. %
			%Nachteil: lauffähige software
			Denn dafür benötigt es eine lauffähige Software und es wird erwartet, dass diese erst zum Ende der Bearbeitungszeit bereit steht.%
				%Trimmed
				%Denn einerseits benötigte es dafür eine lauffähige Software und diese nach jeder neuen Iteration neu zu kompilieren und bereitzustellen wäre ein größer Aufwand. %
				%%Nachteil: Testgruppe finden
				%Außerdem wird vermutet, dass sich das finden einer Testgruppe, welche über mehrere Iterationen die App testet als schwer herausstellen könnte. Wahrscheinlich würde das Interesse nach jeder Iteration mehr schwinden und so verfallen auch die Nutzer.%
	\item \textbf{Umfragen}: %
	%Was: Umfrage 
	Von daher wurde sich stattdessen eine Umfrage entschieden. Dabei werden sich einige Fragen ausgedacht und einmalig an die Zielgruppe gestellt. %
	%Vorteil: Nutzer Findung leichter
	Das hat die Vorteile, dass es vermutlich leichter ist freiwillige Nutzer für ein einmalige Frage, statt eines dauerhaften Testens, zu finden. %
	%Vorteil: Fragen\Antworten können gelenkt werden
	Außerdem können Fragen in beliebige Richtungen stellen kann. So bekommt man Feedback zu gewünschten anstatt zu allen möglichen Themen. % 
	%Nachteil: schriftliches Feedback missverstanden %TODO -> in Fazit
	Jedoch hat diese Technik, genau wie die Vorherige, den Nachteil, dass das schriftliche Feedback anhand fehlendes Kontextes leicht missverstanden werden. %
	%Nachteil: einzelnes Feedback nicht als zu wichtig ansehen %TODO -> in Fazit
	Außerdem muss darauf geachtet werden einzelne Nachrichten nicht als zu wichtig einzustufen. Denn sie könnten zwar für einen Nutzer wichtig sein, aber es muss nicht die eigentliche Zielgruppe repräsentieren.%
	\item \textbf{Inspiration durch Vergleiche}: %
		%Warum: ähnliche Funktionen
		Es existiert zwar noch keine App wie jene welche in dieser Arbeit entwickelt werden soll, jedoch wird angenommen, dass es in dieser App trotzdem einige ähnliche Funktionen und Anforderungen zu konventionellen Kalender-Apps geben wird. %
		%Was: simpel -> intuitive Eindrücke?
		Auch wenn die Anforderungen und Funktionen einer normalen Kalender-App zuerst simpel scheinen, so gilt auch hier, dass man sich nicht auf seine intuitiven Eindrücke verlassen sollte. %
			%Warum:
			Es könnte zum Beispiel bereits Eigenheiten und etablierte Standards in Kalender-Apps geben, welche man ohne Vergleiche nicht finden würde. Oder es gibt als \"selbstverständlich\" angesehene Funktionen, welche deshalb von niemanden angesprochen aber trotzdem erwartet werden. %
		%Schlussfolgerung: Sinnvoll
		Von daher scheint es Sinnvoll sich mindestens für die allgemeine und typischen Funktionen Inspiration zu suchen.%
	\item \textbf{Domänenwissen}: %
		%Was: def
		Durch das einarbeiten in die Domäne CLI-Terminkalender kann bewusst werden was für Funktionen und Anforderungen sie besitzen. %
		%Warum: neue Ideen
		Dieses Wissen könnte zu Inspiration von neue Ideen und Anforderungen für die App führen führen. %
		%Was: dagegen Entschieden
		Jedoch wurde sich gegen das Einarbeiten in die Domäne entschieden. %
			%Warum: Einarbeitungszeit
			Einerseits wird die Einarbeitungszeit als zu hoch eingeschätzt. Denn es existieren viele CLI-Terminkalender und diese lassen sich meistens eher komplex und unterschiedlich bedienen und bieten darüberhinaus noch verschiedene Eigenheiten. %
			%Warum: geringer Informationserwerb
			Andererseits wird der Erwerb an Informationen als zu gering eingeschätzt. So wird nämlich durch die Unterschiede des Pc's und des Handys vermutet, dass die zu erstellende App sich sehr von CLI-Terminkalender unterscheiden wird. Immerhin ist das Ziel auch nicht solch ein Programm zu portieren, sondern die Stärken des Handys und Pc's zu nutzen.%
	\item \textbf{Iterative Develompent}: %
		%Was: def
		Das iterative Arbeiten kann auch als Erhebungstechnik bezeichnet werden. Während jeder Iteration bietet sich die Chance die Anforderungen zu überdenken und sein zuvor neu gelerntes darauf anzuwenden. %
		%Warum: passiert nebenbei 
		Da für diese Arbeit eine Agile-Arbeitsweise betrieben wird, wird diese Technik auch [nebensächlich/währenddessen/dabei] angewendet.%
\end{itemize}

\subsubsection{Umfrage}\label{subsection:umfrage}
%Was+ Warum
Um möglichst wertvolle und aussagekräftige Ergebnisse aus der Umfrage zu erzielen, wurden die Durchführungsbedingungen und Fragen sorgfältig überdacht. %
%Was+ Warum
Die Wahl des Standorts und die Formulierung der Fragen wurden dabei als entscheidende Faktoren für den Erfolg der Umfrage identifiziert.%
%
\newline%
\myNewSection
\textbf{Standort}: %
%Was: Auswahlmöglichkeiten
Zur Auswahl stehen die folgenden Möglichkeiten: die Durchführung der Umfrage im Bekanntenkreis, an der Universität oder in Online-Foren.\newline%
%Was+Warum: keine Zielgruppe in 1&2 -> entfallen
Es wird vermutet, dass sich weder im Bekanntenkreis noch an der Universität viele Nutzer der Zielgruppe finden lassen.\newline%
%Auswirkung -> OnlineForums
Dementsprechend fällt die Entscheidung auf die Durchführung der Umfrage in Online-Foren. %
%Was: erneut frage stellen
Jedoch ergibt sich nun die Frage, in welchen Foren die Umfrage veröffentlicht werden soll. %
	%Was/Auswirkung: großes Spektrum an Auswahlmöglichkeiten
	Dadurch ergibt sich ein noch viel größeres Spektrum an Auswahlmöglichkeiten. %
	%Was/Warum: nicht zuviel Zeit verschwenden -> Suchanfragen
	Um nicht zu viel Zeit mit der Suche nach einem maßgeschneiderten Forum zu verschwenden, wurden ungefähr die ersten 20 Ergebnisse einer Google-Suche mit dem Suchbegriff \glqq CLI-Calendar Forum\grqq{} betrachtet. %
	%Aufzählung
	Dabei wurden unter anderem folgende Foren vorgeschlagen: \glqq reddit: r/commandline\grqq{}\cite{forum_rCommandLine} , \glqq Stack Exchange: Unix \& Linux\grqq{}\cite{forum_unixAndLinux}, \glqq archlinux: Forums\grqq{}\cite{forum_archlinux}, \glqq Debian User Forums\grqq{}\cite{forum_debianUserForums}, \glqq Linux Mint Forums\grqq{}\cite{forum_linuxMintForums}, \glqq Puppy Linux Discussion Forum\grqq{}\cite{forum_puppyLinux}. %
	%Warum: andere geringe Reichweite
	Viele der betrachteten Foren beschränken sich auf ein einzelnes Betriebssystem und haben daher vermutlich eine geringere Reichweite. %
		%Was: Wahl auf reddit
		Das einzige Forum, das dabei heraussticht, ist Reddit. Dementsprechend fiel auch die Wahl auf dieses Forum.\newline%
%Was+Warum: kleiner Aufwand -> r/Apps veröffentlichen  
Da der zusätzliche Aufwand für das Veröffentlichen der Umfrage auf einem weiteren Reddit-Forum als gering eingeschätzt wird, wird die gleiche Umfrage auch auf r/androidapps\cite{forum_rAndroidapps} und r/iosapps\cite{forum_rIOSapps} veröffentlicht. 
	%Warum: weiteres feedback
	Das Ziel ist es, so auch Einblicke in die Wünsche und Erwartungen der App-Nutzer zu erhalten.%
%
\newline
\myNewSection
\textbf{Fragen}: %
%Was vordefiniert VS offenes Konstrukt
Für die zu stellenden Fragen wurde überlegt, ob sie vordefiniert oder als offenes Konstrukt gestaltet werden sollen. %
%Was: Vordefiniert
	%Vorteil: gezielte Antworten
	Vordefinierte Fragen haben den Vorteil, dass gezielt Antworten auf bestimmte Fragen erhalten werden können. %
	%Nachteil: zu sehr an Fragen orientieren
	Allerdings besteht hierbei die Möglichkeit, dass sich Nutzer zu sehr an den Fragen orientieren und dadurch wichtige und interessante Ideen nicht zum Vorschein kommen. %
	%Nachteil: fällt schwer Fragen zu überlegen
	Zudem wurde sich zu diesem Zeitpunkt noch keine Anforderungen bezüglich der Anwendung überlegt, was die Erstellung von Fragen erschwert. %
%Auswirkung:
Daher wurde sich letztendlich für offene Freitextfragen und -antworten Konstrukt entschieden.%
%
\myNewSection
Die Umfragen können unter folgenden Links abgerufen werden: %
\newline%
\url{https://www.reddit.com/r/iosapps/comments/10k3d2c/developing_an_app_for_clicalendars_opinion_poll/}
\newline%
\url{https://www.reddit.com/r/androidapps/comments/10k3k7w/developing_an_app_for_clicalendars_opinion_poll/}
\newline%
\url{https://www.reddit.com/r/commandline/comments/10k38bc/developing_an_app_for_clicalendars_opinion_poll/}
\subsubsection{Vergleich}\myCheckmark %
Ähnlich wie bei der \nameref{subsection:umfrage} wird sich zu dieser Erhebungstechnik auch [gesondert] Gedanken gemacht. Ziel dadurch soll es sein möglichst Wertvolle und aussagekräftige Ergebnisse zu erzielen und dabei möglichst zeiteffizient vorzugehen. 

\myNewSection
%Was es bringen soll
Durch den Vergleich sollen lediglich Inspiration sowie allgemeine und \glqq offensichtliche\grqq{} Anforderungen gesammelt werden. %
%Was nicht: Abgrenzung
Sie soll nicht dazu verleiten Funktionen und Designs zu kopieren oder sich beeinflussen zu lassen. %
	%Wie dagegen angekommen wird.
	Deshalb werden nur wenige Apps zum Vergleich herangezogen und diese auch nur kurzweilig getestet.%
		%Warum: weiterer Punkt: Zeitaufwand
		%[Außerdem] würde das testen weiterer App auch zu viel Aufwand und Zeit kosten.\newline%
\newline%
%Was: Auswahl von Apps
Bei der Auswahl der Apps wurde versuche diejenigen zu wählen, welche möglichst nützliche Informationen liefern können. %
	%Apple & Google
	Dabei wurden einmal der native Apple iOS Kalender\cite{A_calendarApple} und Google Kalender\cite{A_calendarGoogle} zum Vergleich ausgewählt. Denn es wird vermutet, dass diese Unternehmen durch ihren Erfolg, ihrer Größe und dadurch dass sie eigene Richtlinien für Apps festgelegt haben\cite{konventionen_guidelinesApple, konventionen_guidelinesGoogle}, besonders Achtsam bei der Entwicklung dieser Apps waren.\newline%
	%Calendars
	Des Weiteren wurde eine App nach Kundenbewertungen ausgewählt. Denn wohlmöglich wurde solch eine App deshalb so positiv bewertet, weil sie über Funktionen verfügt, welche bei den anderen beiden nicht vorhanden sind. Die Wahl viel dementsprechend auf Calendars\cite{A_calendarReviews}.%
\subsection{Funktionale Anforderungen}
Was sind Funktionale Anforderungen \newline
Aufzählung, Alternativen, Entscheidungen, durch welche Erhehbung ...

\myNewSection
Trotzdem wollen wir solch eine App entwickeln. Nur wird die Idee dabei nicht sein die Anwendung eins zu eins vom Pc zum Handy zu Portieren. Stattdessen soll die App als Erweiterung der bereits bestehenden CLI-Kalender dienen. Durch diese Variante kann dann von den Stärken des Handys sowie des Pc's profitiert werden. 
\subsection{Nicht Funktionale Anforderungen} \myCheckmark
Dieser Abschnitt handelt von den nicht funktionalen Anforderungen. Das sind all jene Eigenschaften welche die App besitzen soll. Dabei werden die Anforderungen nicht nur erläutert sondern auch durch eine Aufzählung nach der Wichtigkeit bewertet.\newline%
Einerseits hilft diese Priorisierung nämlich dabei zu erkennen welche Anforderungen für diese Arbeit am wichtigsten sind, und andererseits lässt es die begrenzte Arbeitszeit nicht zu alle Anforderungen gleich intensive Beachtung zu schenken.

\myNewSection %todo remove
\myTextTodo{LEITFADEN: 1.Was ist die Eigenschaft, 2.Warum ist es wichtig, 3.Wie wird es umgesetzt?} %%%

\myTextTodo{\textbf{PcVsPhone Leistung -> n.f.A:} Deshalb wird darauf geachtet, dass die App nicht zu Ressourcen-Aufwändig wird, wie bereits in den nicht funktionalen Anforderungen erwähnt. Da es sich bei der App um einen Kalender und nicht um komplizierte 3D-Darstellungen oder Algorithmen handelt, wird aber davon ausgegangen, dass es zu keinen Performance-Problemen kommen sollte.}

\myNewSection
\myTextTodo{\textbf{PcVsPhone Internet -> n.f.A}: Deshalb wollen wir bei der App eine schnelle Ladezeit sicherstellen. Um das zu erreichen wird es wichtig sein beim Abtausch von Daten über das Internet diese möglichst brandbreiten-effizient zu übertragen, sowie die App zur Leistung zur optimieren}

\myNewSection
\myTextTodo{\textbf{PcVsPhone Speicher -> nfA}: Es wird zwar davon ausgegangen, dass die App keine absurde Menge von Daten benötigen wird. Trotzdem wird darauf geachtet, dass die App keine große Menge an Speicher annimmt, um auch Geräte mit kleineren Kapazität zu unterstützen. Besonders sollten nicht zuviele Daten geladen werden, damit es zu keiner langen Ladezeit kommt.\newline
Pc als Speicherablage/was man irgendwann mal braucht. Handy nur für wichtigstes/was man unterwegs/immer baucht. Pc sehr große Anwendungen (200gb) alleine. Auf Handy muss nach speichere eher optimiert werden.}

\begin{itemize}

	\item \textbf{Stärken von Pcs und Handys}: die wohl wichtigste Eigenschaft und auch eine Hauptaufgabe dieser Arbeit ist es, die Vorzüge von Pc's und Handy's in der App zu nutzen. Funktionen welche auf einen Endgerät besser Funktionieren als auf den anderen, sollten vielleicht überdacht oder Verlagert werden. So sollte zum Beispiel die Tastatureingabe auf dem Handy wahrscheinlich versucht werden zu überdenken. Entweder könnte man alternative Eingabe nutzen, wie zum Beispiel VoiceToSpeech, oder falls möglich wird die Texteingabe auf dem Pc verlegt. \newline%
	Diese Eigenschaft hat große Auswirkungen auf die funktionalen Anforderung haben, da jede Funktion mit dieser Eigenschaft überdenken werden muss. \newline%
	Wie sich die beiden Systeme in Ihren Stärken überhaupt unterscheiden, wurde im Kapitel ... behandelt.
		
	\item \textbf{Wartbarkeit, Erweiterbarkeit, Verständlichkeit}: Da die Bachelorarbeit, wie bereits erwähnt, nur eine kurze Bearbeitungszeit zulässt, muss man damit rechnen, dass nicht alle Funktionen bis zur Abgabeschluss umgesetzt werden können. Daher soll der Quellcode möglichst gut für in der Zukunft liegende und Fremde Weiterentwicklung ausgelegt sein. Um das zu erreichen muss der Quellcode möglichst Wartbar sein, was anders ausgedrückt bedeutet, dass der Quellcode verständlich und erweiterbar sein muss.\newline%
		Des Weiteren sollten späte Änderungen in den Anforderungen nicht allzu umständlich umzusetzen sein, da wir in dieser Arbeit und durch unsere Agile-Arbeitsweise genau diese Veränderungen erwarten. \newline%
		Deswegen werden die die drei stark miteinander verbundenen Anforderungen Wartbarkeit, Erweiterbarkeit und Verständlichkeit als wichtig eingeschätzt. \newline%
		Wie versucht wird diese Anforderung zu ermöglichen wird im Kapitel ... besprochen.
		
	\item \textbf{Benutzbarkeit}: Unter Benutzbarkeit verstehen wir, dass die App intuitiv, einfach und effektiv zu nutzen ist. \newline%
	Die App soll nicht nur nützliches im Konzept, sonder auch nützlich für den Endnutzer sein. Daher soll die App möglichst effektiv zu nutzen sein. Wichtige und oft genutzt Funktionen sollten also zum Beispiel leicht zugänglich sein, anstatt diese hinter mehreren Seiten zu verstecken und damit die Nutzung zu erschweren. \newline%
	Des Weiteren sollen möglichst viele Personen auch ohne große Einarbeitung und Vorwissen die App nutzen können. Dafür muss die App intuitiv und einfach sein. Ein Beispiel wie man das umsetzen könnte ist, das alle Funktionen haben eine klare Bedeutung haben und sind genau dort aufzufinden wo man sie auch erwartet. \newline%
	Die wahrscheinlich wichtigste Variabel um eine gute Benutzbarkeit zu ermöglichen ist das Design, denn sie ist das einzige mit welchem der Nutzer interagiert.\newline%
	Im Kaptitel ... wurde bereits festgestellt, dass das Design sowie Richtlinien dazu sehr wichtig scheinen. So wurde unter anderem Begründet, dass Handynutzer Apps präferieren\cite{pcVsphone_mobileAppVsWebTimeSpent}, da diese passende Richtlinien befolgen, während Websites auch für Pc's ausgelegt sind und daher weniger auf für Handys passende Konventionen achten. Außerdem scheint es, dass wenn eine Anwendung intuitiv und leicht zu nutzen ist, Benutzer eher dazu geneigt sind die Anwendung weiter zu nutzen\cite{pcVsphone_peopleWillRevisitMobileIfEasyToUse}.\newline%
	Daher sind wir ziemlich sicher, dass die Benutzbarkeit eine wichtige Anforderung für diese App darstellt. Deshalb wird sich dem Design ein eigenen Abschnitt im Kapitel ... gewidmet.

%todo remove
\myComment{
	Dass das Design sowie die Richtlinien dafür nicht nur wichtig für uns erscheint sondern es dafür auch Andeutungen gibt, haben wir im Kapitel ... festgestellt. So wurde unter anderem Begründet, dass Handynutzer Apps präferieren\cite{}, da diese passende Richtlinien befolgen, während Websites auch für Pc's ausgelegt sind und daher weniger auf für Handys passende Konventionen achten. Außerdem scheint es, dass wenn eine Anwendung intuitiv und leicht zu nutzen ist, Benutzer eher dazu geneigt sind die Anwendung weiter zu nutzen\cite{pcVsphone_peopleWillRevisitMobileIfEasyToUse}.\newline%
	Wegen diesen Gründen betrachten wir die Benuztbarkeit als eine wichtige Eigenschaft. Deshalb gibt es in ... einen eigenen Abschnitt, welcher sich mit dem Design auseinander setzt.
	
	Für die Benutzbarkeit in Apps ist das Design entscheidend, da dies das einzige ist womit der Nutzer interagiert. Dass das Design sowie Richtlinien für jenes wichtig für Apps sind haben wir im Kapitel ... festgestellt. So wurde unter anderem Begründet, dass Handynutzer Apps präferieren\cite{}, da diese passende Richtlinien befolgen, während Websites auch für Pc's ausgelegt sind und daher weniger auf für Handys passende Konventionen achten. \newline%
	Ein weiteres Indiz was die Wichtigkeit dieser Eigenschaft untermauert ist, dass wenn eine Anwendung intuitiv und leicht zu nutzen ist, die Benutzer eher dazu geneigt sind die Anwendung weiter zu nutzen\cite{pcVsphone_peopleWillRevisitMobileIfEasyToUse}.\newline%
	Daher schätzen wir die Benutzbarkeit als eine wichtige Eigenschaft für die App ein. Die App soll immerhin nicht nur nützliches im Konzept sein, sonder auch nützlich für den Endnutzer sein.
	
	Die App soll nicht nur nützliches im Konzept sein, sonder auch nützlich für den Endnutzer sein. Dafür sollte sie also möglichst effektiv und intuitiv zu benutzen sein.\newline%
	Für die Benutzbarkeit in Apps ist dafür das Design entscheidend, da dies das einzige ist womit der Nutzer interagiert. Dass das Design sowie Richtlinien für jenes wichtig für Apps sind haben wir im Kapitel ... festgestellt. So wurde unter anderem Begründet, dass Handynutzer Apps präferieren\cite{}, da diese passende Richtlinien befolgen, während Websites auch für Pc's ausgelegt sind und daher weniger auf für Handys passende Konventionen achten. \newline%
	Ein weiteres Indiz was die Wichtigkeit dieser Eigenschaft untermauert ist, dass wenn eine Anwendung intuitiv und leicht zu nutzen ist, die Benutzer eher dazu geneigt sind die Anwendung weiter zu nutzen\cite{pcVsphone_peopleWillRevisitMobileIfEasyToUse}.\newline%
	Deshalb gibt es in ... einen eigenen Abschnitt, welcher sich mit dem Design und Richtlinien auseinander setzt. 
	
	 Ein Beispiel dafür: wichtig Funktionen sollten leicht zugänglich sein, anstatt diese hinter mehreren Seiten zu verstecken und damit die Nutzung zu erschweren.\newline%
		Um möglichst viele Personen unserer Zielgruppe auch ohne große Einarbeitung und Vorwissen den Zugang zu ermöglichen, soll die App außerdem möglichst intuitiv sein. Also alle Funktionen haben eine klare Bedeutung und sind genau dort aufzufinden wo man sie auch erwartet.\newline% 
		Diese Anforderungen hängen am meisten von der Grafischenoberfläche der App ab, da dass das einzige ist mit was der Nutzer interagiert. Deswegen gibt es in ... einen eigenen Abschnitt, welcher sich mit dem Design auseinander setzt.
}%%%

		
	\item \textbf{Qualität/Korrektheit}: Mit der Anforderung Qualität soll sichergestellt werden, dass die App sich genau so zu verhalten hat wie zuvor Spezifiziert. Es soll also zu keinen unerwarteten Situationen wie Fehler und Abstürzen kommen.\newline%
	Falls doch würden das nicht nur die Benutzbarkeit einschränken, sondern auch die Nutzer irritieren und möglicherweise dazu bewegen die App nicht weiter zu nutzen. So würden laut einer Umfrage 88\% von Nutzern die App bei einem "Bug" verlassen\cite{nfA_bugsAbandon}. Daten von Google bekräftigen diese Aussage. So handeln 54\% aller 1-Sterne-Bewertungen im Play Store von "Bugs" oder "Stability"\cite{nfA_bugsReview}. Deshalb sehen wir die Anforderung Qualität auch als durchaus wichtig an.\newline%
	Um die Qualität sicherzustellen darf man entweder beim Programmieren keine Fehler machen, was keine Sinnvolle Annahme ist, da Agilen-Arbeitsweisen Fehler erwartet werden, oder man testet die Software ausgiebig genug, sodass man nachweisen kann, dass die Software keine Defekte besitzt. Daher werden wir unsere App testen. Wie genau das vonstatten geht im Kapitel ... erwähnt.
	
	\item \textbf{Reichweite}: Auch wichtig aber im Vergleich zu den anderen Anforderungen eher zweitrangig ist die Reichweite. Zwar sollen möglichst vielen Handy-Nutzern ermöglicht werden die App zu benutzen, jedoch handelt es sich bei um eine nischen-Anwendung und daher sollte nicht zuviel Aufwand in diese Eigenschaft fließen.\newline%
	Am wichtigsten für die Reichweite ist es, dass die App von Android und iOS genutzt werden können, da dies die weitverbeitesten Systeme im Handymarkt sind\cite{}. Andererseits kann auch das Alter von Handys kann hierzu betrachtet werden. Wenn unsere App zu viel Ressourcen benötigt oder eine zu hohe Android oder Apple Version beanspruchen, könnte Sie von älteren Handys nicht benutzt werden.
	%todo Verteilung der Handy os-verionen %mabye doch eher unwichtig, da quelle: viele leute kaufen oft neue handys
	
	\item \textbf{Sicherheit}: Die Anforderung Sicherheit steht dafür, dass die in der App verwendeten Daten\footnote{Zum Beispiel die Kalendereinträge} nicht von dritten mitgelesen werden können. \newline%
	Für dieses Projekt gibt es drei verschiedene Standort-Möglichkeiten der zu schützenden Daten und somit auch drei verschiedene Angriffsflächen für einen Dritten die Daten zu stehlen.\newline%
	Erstens die Situation, dass sich die Daten auf dem Handy befinden. Die Sicherheit hierzu wird jedoch vom Betriebsystem sichergestellt. So bietet Appel zum Beispiel für jede App eine Sandbox, welche verhindert, dass andere Apps auf die zu schützenden Daten zugreifen könnten \cite{nfA_sandbox}.\newline%
	Die zweite und dritte Situation wären einmal, dass sich die Daten auf dem Backend befinden oder während der Kommunikation von Endgerät zu Backend. Beide dieser Situationen werden meist Framework und Datenbank geschützt, da aktuelle Software oft standardmäßig bereits starke Verschlüsselung benutzen. \newline%
	Diese Anforderung wird also nicht als niedrigstes eingestuft, weil sie am unwichtigsten betrachten wird, sonder weil wir für dieses Projekt nur wenig Einfluss darauf ausüben können. Es ist lediglich wichtig ein Backend und Framework zu finden, welche Verschlüsselung anbieten.

\end{itemize}


\section{Technologische Überlegungen}

%%% hidden subsection for a better structure in latex editor: "texifier"
\myComment{\subsection*{Übersicht}}
%%%Einleitung: Was+Warum
In diesem Abschnitt werden sich Gedanken zur Auswahl der Technologien gemacht, da diese Auswahl bereits Auswirkungen auf die funktionalen und nicht-funktionalen Anforderungen haben kann.\newline%
%%%Übersicht: xWas
\textbf{Übersicht:} %
Zunächst wird behandelt, mithilfe welchen Frameworks die App erstellt werden soll. Anschließend wird sich für einen CLI-Terminkalender, für die Darstellungen der Erinnerungen und für das Dateiformat der Konfigurationsdatei entschieden. Danach wird die Wahl der verwendeten Software und Entwicklungsumgebung begründet und schließlich werden für die Arbeit relevante Pakete ausgewählt.\newline%
%%%%Ergebnis
\textbf{Ergebnisse:} %
	%Framework
	Im ersten \secref{subsection:auswahlDesFrameworks} wurde behandelt, ob die Anwendung als App oder Webseite und mithilfe von Native oder Cross Platform Frameworks erstellt werden soll. Die Wahl fiel dabei erstens auf eine App, da diese in der Regel intuitiver ausfallen können und beliebter sind, und zweitens auf Cross Platform Frameworks, da damit Apps für iOS und Android erstellt werden können. Zuletzt wurden noch die beiden Frameworks Flutter und React Native miteinander verglichen. Aufgrund von Beliebtheit, Performance und der Annahme, dass damit effizient gearbeitet werden kann, fiel die Wahl auf Flutter.\newline%
	%Terminkalender & Konfigurationsdatei
	Die \nameref{section:tech:sub:cli_terminkalender} viel auf When und das \nameref{section:tech:sub:konfigurationsdateiformat} auf JSON, da durch dieser Wahl erhofft wird, da durch diese Entscheidungen erhofft wird, während der Entwicklung ein evaluierbares Produkt zu erstellen.%
	%Erinnerungen
	Weiter wurde für die \nameref{section:tech:sub:darstellung_der_erinnerungen} Issues gewählt, da diese ein passendes Format besitzen und zudem die GitHub-Webseite genutzt werden kann, um die Erinnerungen einzusehen.\newline%
	%Entwicklungsumgebung
	Im darauf folgenden \secref{subsection:entwicklungsumgebung} wurde entschieden, MacOS als Betriebssystem, Android Studio als IDE, GitLab zur Versionsverwaltung und Emulatoren sowie Handys zum Testen zu nutzen. Es wurde auch erwähnt, dass die Versionen dieser Software während der Arbeit nicht verändert werden, um so mögliche Komplikationen zu vermeiden.\newline%
	%Pakete
	Im letzten \secref{subsection:auswahlDerPakete} wurden die für die Arbeit benötigten Pakete nach ihrem Alter, ihrer Beliebtheit und ihrem Update-Verlauf ausgewählt. Die Wahl fiel auf json\_serializable, tests und flutter\_lints, da sie die Lesbarkeit des Codes verbessern und nützliche Funktionen bieten. Weiter wurde das Paket github ausgewählt, da es das einzige verfügbare Paket ist, das eine Schnittstelle zum Backend bietet. Schließlich wurde das Paket syncfusion\_flutter\_calendar ausgewählt, da es eine Kalendardarstellung bereitstellt und somit Zeit gespart wird, da diese nicht selbst implementiert werden muss.%
%
%
%
%
%Todo - Remove
%\myComment{
%	%Old AllÜbersicht 	
%	\myNewSection
%	\myTextTodo{
%	\textbf{Abschnitte der Arbeit}\\
%	%Technologische Überlegungen -> wichtig da erfüllt Anforderungen + erst nach der Erhebung
%	Im darauf folgenden \secref{section:technologischeUeberlegungen} wird sich Gedanken über die Auswahl von Technologien gemacht. Das hat den Grund, da bereits die Auswahl von Technologien Auswirkungen auf Funktionale und nicht Funktionale Anforderungen haben können. Daher ist es auch wichtig, diesen Abschnitt erst nach der Anforderungserhebung zu behandeln.
%		%Beispiel
%		Man stelle sich vor es wird zuerst ein Framework welches bekannt für seine langsame Ausführung ist gewählt und erst zu einen späteren Zeitpunkt wird die Anforderung einer \dq schnelle Performance\dq erhoben. Durch diesem Konflikt müssten die Wahl des Framework neu überdacht werden, was wichtige Bearbeitungszeit verschwenden könnte.\newline%
%	}
%
%}
\subsection{Auswahl des Frameworks}\label{subsection:auswahlDesFrameworks}%
%Einleitung
In diesem Abschnitt wird die Frage behandelt, wie die Anwendung erstellt werden soll. %
%Was: App vs Web.
Dabei bestehen die Optionen die Anwendung als eine Webseite oder als Applikation zu erstellen. %
	%Warum: Plattformunabhängig VS UserPreference, Gesten, Design Richtlinien
	Zwar bieten Webseiten einige Vorteile im Vergleich zu Applikationen, zum Beispiel die Plattformunabhängigkeit. Wie zuvor in \secref{section:pcVsPhone} erwähnt, können Apps jedoch durch Richtlinien und Gesten aber um einiges intuitiver sein und Nutzer scheinen diese generell zu präferieren\cite{pcVsphone_mobileAppVsWebTimeSpent}. %
	%Auswirkung: -> App
	Daher soll die Anwendung als Applikation entwickelt werden. %
%
\newline
\myNewSection
%Was: Framework
Nun kann sich für ein Framework entschieden werden. %
%Was+Def: Native vs Crossplatform
Die erste Wahl liegt dabei zwischen einem Native-Framework oder Cross-Platform-Framework.\newline% 
	%Todo: remove useless beispiel:
		%Dabei sind Native-Frameworks diejenigen, welche die für die Platform spezifischen tools benutzt. Während Cross-Platform-Frameworks ihre eigenen tools anbieten.
		%Warum: Native: Performance
	Generell scheinen Native-Frameworks eine bessere Performance als Cross-Platform-Frameworks zu bieten \cite{tech_performanceReactNativeVsFlutter1, tech_performanceReactNativeVsFlutter2}. %
	%Was: Native: Aussehen
	Außerdem sehen und fühlen sich die nativ erstellte Apps einheitlich mit der Plattform an, da für diese Apps plattformspezifischen Funktionen und Komponenten benutze werden.
		%Todo: remove useless beispiel:
			%, wie zum Beispiel die Schieberegler von iOS [\ref{pic:schieberegler}]. 
		%Warum: intuitive
		Das würde wahrscheinlich zu einer intuitiveren Benutzung für den Nutzer führen, da dies für ihm bereits bekannte Muster und Funktionen wären. %
		%Auswirkung: 
		Da Performance und Benutzbarkeit zwei Anforderungen für diese Arbeit sind scheinen native-SDKs eine gute Wahl für diese Arbeit zu sein.\newline%
	%Was: Crossplatform: Relativierung
	Jedoch wurde sich für eine Cross-Platform-Framework entschieden. %
		%Performance
		Einerseits wird der Performance-Verlust als marginal eingeschätzt, da es sich bei der zu erstellenden App wahrscheinlich um eine simple Anwendung ohne schwierige Berechnungen oder aufwändigen Animationen handeln wird. 
		%Aussehen
		Andererseits kann die grafische Oberfläche auch versucht werden mit Cross-Platform-Frameworks passend für das System zu erstellen. Zwar wäre das ein größerer Aufwand als bei Native-Frameworks,
		%Warum: Zeit + Crossplatform
		aber dafür sind die Anwendungen von Cross-Platform-Frameworks mit iOS und Android kompatibel. Wie in \secref{section:anforderungen} besprochen, ist dies der wichtigste Punkt, um eine große Reichweite zu ermöglichen. Zwar wäre dies auch möglich, indem für iOS und Android jeweils eine eigene Codebasis über native SDKs erstellt würde, jedoch müssten dann auch zwei Codebasen gepflegt werden. Durch die relativ kurze Bearbeitungszeit dieser Arbeit erscheint diese Idee weniger sinnvoll. Stattdessen wird versucht, mithilfe von Cross-Platform-Frameworks möglichst schnell und zeiteffizient eine evaluierbare Anwendung für beide Plattformen zu entwickeln. Falls sich die App später als nützlich und beliebt herausstellt, kann immer noch eine Native-Entwicklung gestartet werden.
	
\myNewSection
%Was: Flutter vs React Native
Zuletzt muss eine Entscheidung für ein konkretes Framework getroffen werden. %
	%Warum: Weiterentwickelbarkeit
	Wie in \nameref{section:anforderungen} erwähnt, ist die Weiterentwickelbarkeit eine wichtige Anforderung. Daher ist es einerseits entscheidend, ein Framework auszuwählen, das möglichst lange unterstützt wird, um eine zukünftige Weiterentwicklung zu gewährleisten.\newline%
	%%Warum: Beliebtheit
	Andererseits muss auch berücksichtigt werden, dass das Framework beliebt ist und von vielen Personen genutzt wird, um die Chance zu erhöhen, dass Interessenten gefunden werden können.\newline%
	%Quelle:
	Basierend auf der Update-Historie und der Sternbewertung auf Github sind React-Native und Flutter die beiden beliebtesten Frameworks für die Cross-Platform-App-Entwicklung\cite{tech_flutterStars, tech_reactNativStars}.\newline%
%Warum: Beliebtheit
Wenn man die beiden Frameworks dementsprechend miteinander vergleicht, scheint Flutter beliebter zu sein. Es hat mit 150.000 Sternen auf GitHub etwa 38\% mehr als React Native\cite{tech_flutterStars, tech_reactNativStars}. Ein ähnliches Ergebnis zeigt sich auch bei Google Trends, da Flutter fast doppelt so viele Suchanfragen wie React Native hat\cite{tech_googleTrendsFlutterVsReactNative}.\newline%
%Warum: performance
Im Bereich der Performance verhält es sich ähnlich. Laut zwei Analysen von inVerita scheint Flutter ressourcensparender und schneller als React Native zu sein\cite{tech_performanceReactNativeVsFlutter1, tech_performanceReactNativeVsFlutter2}.\newline%
%Warum: UI
Dafür bietet React Native den Vorteil, plattformspezifische Komponenten und Funktionen nutzen zu können. Wie zuvor erwähnt wurde, könnte dies zu einer verbesserten Benutzbarkeit führen.
\newline%
%Warum: vordefinierte features
Flutter bietet im Vergleich zu React-Native viele vorgefertigte Komponenten und Features, was letztendlich zur Entscheidung für dieses Framework geführt hat. Während React-Native nur 25 Core-Komponenten hat, besitzt Flutter allein für Animationen bereits 22 Komponenten\cite{tech_componentsFlutter, tech_componentsReactNative}. Diese vordefinierten Komponenten können dabei helfen, zeiteffizient vorzugehen, da sie sonst möglicherweise selbst implementiert werden müssten.%
%
%
%
%
\subsection{Entwicklungsumgebung}\label{subsection:entwicklungsumgebung}\myCheckmark%
%Was: Einleitung
In diesem Abschnitt werden die benutzten Software und dessen Versionen genannt. %
%Warum: Einfluss
Denn die Wahl von Software und dessen Versionen kann bereits einen Einfluss auf das Endprodukt ausüben. So wäre es zum Beispiel durch die Wahl von Windows nicht möglich Applikationen für iOS zu entwickelt.\newline%
	%Warum: Nachbilden + Kompatibilität
	Außerdem wird durch die Nennung der Software das Nachbilden der Applikation garantiert. Mit verschiedener Software oder anderen Versionen ist es aus Erfahrung durchaus Vorstellbar, dass es zu Komplikationen kommen kann. So wurde zum Beispiel während der Ausarbeitung Android Studio nach einem Update nicht mehr funktionsfähig. %
		%->Neuste Versionen
		Deswegen wurden zu diesem Zeitpunkt in der Arbeit alle Softwares auf die neuste Version aktualisiert und danach bis zum Ende der Arbeit auch auf diesen Versionen belassen. %Einerseits sollten damit möglichst alle neuen Funktionen und Bugfixes [benutzbar werden]. Aber viel wichtiger noch sollten weitere Komplikationen in der Zukunft damit verhindert werden. \newline%

\begin{enumerate}
	%Was+Warum: Betriebssystem -> iOS Apps
	\item Betriebsystem: Für das Betriebssystem standen Windows und MacOS zur verfügung. Es wurde MacOs entschieden, da auf diesem Betriebsystem für iOS sowie Android Apps entwickelt werden können. Version: macOS Ventura 13.2.1%

	%Was+Warum: IDE
	\item IDE: Flutter empfiehlt unteranderem Visual Studio Code, Android Studio oder Emacs als Editor zu nutzen\cite{tech_ideSuggestion}. Da Android Studio genau wie Flutter beide von Google entwickelt wurden, wird erwartet, dass der Editor besonders gut auf das Framework abgestimmt ist. Deshalb und weil Android Studio für die App-Entwicklung ausgelegt ist, wurde er schlussendlich als Editor ausgewählt. Version: 2022.1.1%

	%Was: Flutter
	\item Framework: Flutter. Mehr dazu in \secref{subsection:auswahlDesFrameworks}. Version: 3.7.3%

	%Was: Versionsverwaltung
	\item Versionsverwaltung: 
		Für die Versionsverwaltung wurde sich für GitLab entschieden. %
		%Warum: Erfahrung und hosting
		Einerseits da im Studium damit bereits Erfahrung gesammelt wurde und andererseits weil die Universität ihre eigene Version dazu bereitstellt.\newline%
		%Was+Warum: Einstellungen -> Verständlich + weiterentwickelndes
		Um ein möglichst Verständliches und einfach zu weiterentwickelndes Projekt zu erstellen, wurden außerdem einige Einstellung in GitLab getroffen. %
			%konventional commits
			So werden einerseits einheitliche und ausschlaggebende Commit-Nachrichten benutzt, damit diese bei späterer Betrachtung verständlich sind. Dabei wurde sich an das bereits existierende Regelwerk \glqq Conventional Commits\grqq{}\cite{tech_conventionalCommits} gehalten. %
			%ci/cd pipelines
			Des Weiteren wurde eine CI/CD pipeline erstellt um Tests automatisch zu überprüfen. Dadurch werden Entwickler automatisch auf mögliche Fehler ihrer Änderungen aufmerksam gemacht. %
			%Version
			Version: git 2.37.1 (Apple Git-137.1)

	%Was+Warum:Emulatoren
	\item Emulatoren: %
		%Warum:Testen
		Zum testen der App werden Emulatoren benutzt. Um dabei sicherzustellen, dass die Anwendung auf Android und iOS läuft sowie auf den neusten und älteren Betriebssystemversionen, wurde jeweils ein iPhone und ein Android Emulator mit der neusten und älteste verfügbaren Version zum testen benutzt. %
		%Was+Warum: Reales Handy -> Performance
		Des Weiteren wurde auch noch ein echtes Handy zum Testen verwendet, da vermutet wird, dass unter realen Bedienungen das testen der Performance aussagekräftiger ist. Dabei wurde mit dem iPhone SE1 gezielt ein relativ altes Handy gewählt. %
			%Warum: Alter
			Dadurch lässt sich nämlich gut prüfen, ob die Anwendung auch von älterer und schwächerer Hardware unterstützt wird. % 
			%Warum: größe
			Außerdem ist das Handy im Gegensatz zu neuen Handys relativ klein mit einer Bildschirmdiagonalen von vier Zoll. Wodurch sich prüfen lässt, ob das Design der App auch auf kleineren Bildschirmen funktioniert. %
		%Versionen
		Versionen: IPhone 14 iOS 16.2 \& 13.7, Pixel 6 Android 13.0 \& 5.0, iPhone SE1 iOS 15.7.3%
		
	\item CLI-Terminalkalender: When. Mehr dazu im \secref{subsections:cli_termincalendar}. Version 1.1.45
		
	%Was+Warum: Weiteres
	\item Weiteres: Die folgende Software wird standardmäßig durch einige der zuvor genanten Technologien benötigt. Dementsprechend wird außer der Nennung der Version nicht näher auf sie eingegangen: Xcode 14.2, Android SDK Platform-Tools: 34.0.0, DevTools: 2.20.1, Dart 2.19.2%
\end{enumerate}

%OS-Umgebung -> Iphone + Android entwickelbar
%Ide: Von Flutter empfohlen + vorherige erfahrung mit intellij umgebungen
%Versionen: neuste aber danach nicht weiter geupdatet... dart, flutter, ide, os, emulatoren
%Emulatoren: echte hardware + viele emulatoren
\input{3_technologischeUeberlegungen/3_3_versionsverwalung.tex}
\subsection{Auswahl der Pakete}\myCheckmark
%Einleitung
In diesem Abschnitt wird sich mit der Auswahl der Pakete auseinandergesetzt. Dabei wird zuerst erklärt warum und nach welchen Kriterien generell die Pakete ausgesucht wurden. Anschließend werden die wichtigsten Pakete aufgelistet und ihre Daseinsberechtigung/Benutzung/Verwendung begründet.

%Generelle Kriterien
\myNewSection
	%Was
	Die Pakete wurden nach einem ähnlichen Verfahren wie auch in \secref{subsection:auswahlDesFrameworks} ausgesucht. So wurden bei Ihnen generell auf ihr Alter, ihre Beliebtheit und den Verlauf ihrer Aktualisierungen geachtet. 
	%Warum -> long time support
	Dabei wird nämlich angenommen, dass bei einer starken Ausprägung dieser drei Kriterien, es wahrscheinlicher ist, dass ein Paket noch für lange Zeit weitere Unterstützung und Aktualisierungen erhält. Und das ist wiederum nützlich wenn die zu erstellende Anwendung auch in der Zukunft noch bestehen soll. 
	%Warum -> 
	Außerdem kann es die Verständlichkeit, einer unserer nicht funktionalen Anforderungen, erleichtern, wenn die gewählten Pakete guten Support haben und etabliert/bereits bekannt sind. 
	%Auswirkung
	Aus diesem Grund wurde, falls mehrere Pakete mit ähnlichen Funktionen zur verfügung stehen, [stets] das ausgewählt, welches in diesem drei Punkten am besten abschneidet.

%Packages
\myNewSection
Pakete:
\begin{itemize}
	%Was
	\item json\_serializable\cite{tech_packageJson}: Dieses Package erstellt Json-Klassen automatisch, welche für die Übersetzung von Daten zu Json-Objekten und umgekehrt benötigt werden. 
		%Warum: Zeit
		Einerseits wird durch die automatische Erstellung etwas Zeit gespart. 
		%Warum: Einheitlich + Popularität -> Weiterentwickelbarkeit
		Andererseits haben die automatisch erstellten Klassen alle das selbe Muster. Dadurch wird der Code und das Projekt einheitlicher, wodurch eine bessere Struktur und Überblick geschaffen wird.
		Außerdem ist das Package mit 2550 Likes und 99\% Popularität\footnote{eine eigene Kategorie auf dart.dev} das beliebteste Paket dieser Liste. Durch diese beiden Punkte wird erwartet, das es andere Programmierer leichter fällt diese Klassen erkennen und verstehen. Im Endeffekt sollte das also die Verständlichkeit erleichtern.
		 
	%Was
	\item tests\cite{tech_packageTest}: Aus dem gleichen Grund wurde das Test Package für Flutter gewählt. Es legt ein einheitliches Muster für das Schreiben von Tests vor, was zu einer übersichtlichen Struktur führt. 
		%Warum: Popularität -> Weiterentwickelbarkeit
		So wird bei diesem Paket, genau wie beim vorherigen, aus der Beliebtheit und des einheitlichen Musters daraus geschlossen, dass die Nutzung davon die Verständlichkeit verbessert.
	%Was
	\item flutter\_lints\cite{tech_packageLints}: Das Package flutter\_lints legt eine Reihe von Regeln für nützliche  Programmierpraktiken vor. So zum Beispiel das eine Textzeile nicht länger als 80 Zeichen lang sein darf oder das die Benennung von Variablen stets mit einem kleinen Buchstaben anfangen.
		%Warum: Zeitsparen
		Durch die Nutzung dieses Paketes wird etwas Zeit gespart, da man sich weniger Gedanken um den Code-Style machen muss. Stattdessen kann man die vordefinierten Regeln vom Ersteller des Paketes benutzen. Beim Ersteller dieses Paketes handelt es sich dabei um \glqq flutter.dev\grqq. Daher wird auch angenommen, dass die ausgewählten Style Entscheidungen und Programmierpraktiken gut durchdacht und passend zum Framework und der Programmiersprache sind. 
		%Warum: Einheitlich
		Außerdem macht es den Quellcode generell übersichtlicher, da durch das Paket in jeder Datei eine einheitliche Formatierung und Style genutzt wird.
		%Warum: Beliebtheit
		Des Weiteren erfreut sich auch dieses Paket erneut großer Beliebtheit. Daher wird auch hier erneut davon ausgegangen, dass durch die Beliebtheit und einheitlichen Muster, auch dieses Package die Leserlichkeit für andere Personen erleichtern.
	
	%Warum: Zeit + Tests
	\item github\cite{tech_packageGithub}: Um mit dem Backend kommunizieren zu können benötigt es einer Schnittstelle von der Programmiersprache zum Backend. Eine solche Schnittstelle zu erstellen ist eine umfangreiche und zeitaufwändiges Umfangen. Es würde also nicht nur den Rahmen sondern auch die Zeit dieser Arbeit sprengen, wenn versucht würde diese Aufgabe eigenständig zu bewältigen. Daher wird stattdessen das Paket \glqq github\grqq benutzt, welche sich genau dieser Aufgabe widmet. 
		%Was
		So stellt das Package \dq github\dq eine API bereit, damit über dart mit GitHub interagiert werden kann. 
		%Alternative
		Dieses Paket ist dabei das einzige, welches eine solche Schnittstelle zur verfügung stellt. Das heißt also, dass es für dieses Package keine alternativen gibt. Das könnte sich als Nachteil herausstellen, da bei Komplikationen, Fehler oder keiner weiteren Unterstützung das Paket nicht gewechselt werden kann. 

\end{itemize}

\section{Design}\label{section:design}

%\input{4_design/stichpunkte4.tex} %todo: remove after this sections completion

%%% hidden subsection for a better structure in latex editor: "texifier"
\myComment{\subsection*{Übersicht}} 

%Einleitung
Wie im \secref{section:anforderungen} beschrieben wurde, ist eine gute Benutzbarkeit ein wichtiger Aspekt für die zu entwickelnde Anwendung. Dabei wurde Insbesondere das Design und die Einhaltung von Designrichtlinien im \secref{section:pcVsPhone} als eine entscheidende Rolle für eine gute Benutzbarkeit hervorgehoben.\newline%
%Übersicht
\textbf{Überblick:} %
	%Richtlinien
	Im ersten Unterabschnitt werden daher diese Designrichtlinien genauer betrachtet. Hierbei wird zunächst eine Richtlinie ausgewählt und ihr Inhalt genauer erläutert, um die für die Arbeit relevanten Regeln zu identifizieren.  %für eine Entschieden -> welcher Inhalt davon wirklich nützlich/verwendbar -> Erkenntnisse/Schlussfolgerung daraus anstatt Blind alles befolgen um so die Entscheidungen zu verstehen und diese auf nicht behandelte themen anzuwenden
	%Design
	Anschließend wird das konkrete Design der App vorgestellt und einige der getroffenen Entscheidungen begründet.\newline%
%Ergebnisse
\textbf{Ergebnisse:}
%AusWahl
Es wurde sich zwischen den Richtlinien Apple und Google für Apples entschieden.
%Ergebnisse
Viele der Regeln in den Designrichtlinien stimmen mit den Stärken des Handys, beziehungsweise den Ergebnissen aus \secref{section:pcVsPhone}, überein. Beispielsweise wird empfohlen, dass die App und ihr Design folgende Merkmale aufweisen sollten: \glqq sofort einsatzbereit sein\grqq{},\glqq Aufgaben vereinfachen\grqq{},\glqq Informationen reduzieren\grqq{},\glqq Informationen indirekt vermitteln\grqq{},\glqq konsistent innerhalb der App und Plattform sein\grqq{} und \glqq intuitiv sein\grqq{}.
%Design
Abschließend wurden die Richtlinien und ihre Regeln sowie die Merkmale auf die verschiedenen Komponenten der App angewendet und erläutert, einschließlich der Kalenderseite, der Erinnerungsseite, der Einstellungsseite, der Appleisten und der Systemtastatur.%
%
%
%
%
%Trimmed
%\myComment{
%
%\myTextTodo{
%%Design -> starke Auswirkung auf Qualität
%\textbf{Abschnitte der Arbeit}\\
%Im \secref{section:design} wird sich Gedanken über das äußere sowie innere Design der App gemacht. Anders gesagt also Überlegungen zu der grafischen Oberfläche sowie der Architektur. Dieser Abschnitt wird behandelt weil, beide dieser Punkte starke Auswirkungen auf Qualität der App ausüben können. Dabei würde die grafischen Oberfläche besonders die Qualität für den Endnutzer beeinflussen, da dies das einzige ist mit dem Benutzer interagiert. Gleicherweise würde die Architektur die Qualität für den Entwickler entscheiden, da der Quelltext seine Schnittstelle darstellt.\newline
%}
%
%}
\subsection{Konventionen}

%Einleitung
%Wie in \secref{section:pcVsPhone} erwähnt, soll für das Design der Anwendung [Konventione/Richtlinienen] befolgt werden. 
% Abgrenzung -> Was nicht (nicht alle Regeln, nur wichtige / interessante) (in Einleitung?)
%Dabei soll es in diesem Abschnitt nicht darum gehen alle Regeln der Konvention zu nennen, sondern einige wichtige welche sich in dem Design der Anwendung wiederfindet und diese begründen...? \myTodo


%Trimmed
%\myComment{
%
%	\myNewSection
%	%Einleitung
%	Wie in \secref{section:pcVsPhone} erwähnt, soll für das Design der Anwendung [Konventione/Richtlinienen] befolgt werden. 
%	
%	\myNewSection
%	% Abgrenzung -> Was nicht (nicht alle Regeln, nur wichtige / interessante) (in Einleitung?)
%	Dabei soll es in diesem Abschnitt nicht darum gehen alle Regeln der Konvention zu nennen, sondern einige wichtige welche sich in dem Design der Anwendung wiederfindet und diese begründen...? \myTodo
%	
%	\myNewSection
%	%Auswahl von Konvention
%		%Was: Mehrere Konventionen
%		Es stehen mehrere verschiedene [Richtlinien] für Apps zur verfügung. So zum Beispiel eine für iOS von Apple und eine für Android von Google \cite{konventionen_guidelinesApple, konventionen_guidelinesGoogle}. %
%		%Was+Warum: Entscheiden
%		Jedoch wird sich für eine einzelne [Richtlinie] entschieden, da unterschiedliche [Richtlinien] verschiedene Regeln nennen können und sich so gegebenenfalls sogar widersprechen könnten. %
%			%Beispiel
%			So bietet Google zum Beispiel eine eigene Seite und Regeln für \glqq Floating action buttons\grqq{} während diese in den Apple [Guidelines] nie erwähnt werden und daher wahrscheinlich auch nicht erwünscht sind. %
%		%Was+Warum: Auswahl Apple -> eigene Präferenz(leichter+intuitiver)
%		Aus eigener Präferenz, da die Apps dieser Platform noch etwas intuitiver und leichter zu bedienen scheinen, wurde sich für die [Richtlinien] von Apple entschieden.%
%		
%	\myNewSection
%	%Richtlinien: Inhalt + Was wurde betrachtet + Was nicht
%		%Was: Woraus besteht die Richtlinie
%		Die [Richtlinie] von Apple bietet fünf Abschnitte welche unteranderem von einzelnen Komponenten handeln, wie zum Beispiel Textfelder und Knöpfe, aber auch von Grundlagen und Mustern, welche allgemeine Regeln für die Erstellung von Apps liefern. Zusammen besteht die [Konvention] aus rund 148 Einträge. %
%		%Was: davon interessant
%		Davon schienen vorerst etwas mehr als die Hälfte als [passend/nützlich/interessant] für die Arbeit. Bei [genauerer/intensiverer] Betrachtung stellten sich 34 dieser Einträge als [wirklich passend] heraus, da sie Funktionen und Anforderungen behandeln, welche in die Arbeit einfließen sollen. %
%		%Was: davon uninteressant
%		Die meisten restlichen Einträge handeln von Funktionen welche nicht in die Anwendung eingebaut werden sollen, wie zum Beispiel eine Tastaturbedienung oder NFC-Funktionalität\cite{konventionen_keyboard, konventionen_nfc}. Einige andere Einträge welche eigentlich nützliche Funktionen für diese Arbeit nennen, wie zum Beispiel ein Nachtmodus oder Siri Unterstützung, wurden [aus Zeitmangel/ wegen der begrenzten Arbeitszeit] vorerst [übersprungen]\cite{konventionen_darkmode,konventionen_siri}.%
%		
%	\myNewSection
%	%Erkenntnisse
%		%Einleitung
%		Durch das Durchlesen der vielen Regeln und Richtlinien konnte sich ein Muster erkennen lassen, was die [Auswirkung/Ziele] dieser Regeln und Richtlinien deutet. %
%			%Konsistenz
%			So ist ein oft erwähntes [Theme] die Konsistenz. In der App benutzte Design entscheidungen sollten konsistent in der ganzen App beibehalten werden und wo möglich sollten [system/apple] [definierte/vorgegebene] [Einstellungen] übernommen werden. So werden zum Beispiel die Übernahme von Gesten\cite{konventionen_accessibility}, Schrift\cite{konventionen_typography} und Farben\cite{konventionen_color} empfohlen.
%			%Little info
%			Ein weiteres oft erwähntes [Theme] ist es die Darstellung von wenig Informationen. Weniger Informationen \glqq hilft dabei Leuten sich bei ihrer Aufgabe zu fokusieren\grqq{}\cite{konventionen_platformIOS}, daher sollten unter anderem \glqq möglichst wenig Wörter genutzt werden\grqq{}\cite{konventionen_writing} und \glqq wichtige Information auch so Dargestellt werden\grqq{}\cite{konventionen_layout}.
%			%simple&intuitive
%			Das Hauptziel der Regeln und auch der beiden zuvor genannten [Themes] scheint es aber zu sein, die App intuitiv und simpel [zu machen]. So lässt sich dieses [Theme] in den Regeln am meisten wiederfinden. Unteranderem wird die effektivste App Erfahrung als intuitiv beschrieben\cite{konventionen_offeringHelp}, das nutzen von Gesten wird empfohlen\cite{konventionen_accessibility} und Größe, Farbe und Font von Text und Icons sollen benutzt werden um dessen Bedeutung zu vermitteln\cite{konventionen_icons, konventionen_typography}. 
%		%->pcVsPhone
%		[Interessant/Nennenswert] hierbei ist es, dass diese drei [Themen] sich mit den Erkenntnissen und Vermutungen aus \secref{section:pcVsPhone} übereinstimmen. So wurde zuvor zum Beispiel die [Simpelheit und Intuitivität] als eine Stärke des Handys benannt und die Darstellung von wenig Informationen als eine Anforderung für Aufgaben auf dem Handy.
%		%Weiteres:
%		Des Weiteren gibt es auch viele Regeln welche weitere dieser Vermutungen und Erkenntnisse weiter [bekräftigen/unterstützen].
%			%Wenig Konfig + Eingabe
%			So ratet Apple unteranderem einerseits von Texteingaben und einer hohen Anzahl an Einstellungen ab\cite{konventionen_enteringDate,konventionen_settings}. 
%			%kurzer einstiegsaufwand
%			Während sie andererseits mit Regeln wie \glqq Ask for initial setup information only when necessary\grqq{}\cite{konventionen_launching}, \glqq Show content as soon as possible\grqq{}\cite{konventionen_loading} und \glqq Delay sign-in for as long as possible\grqq{}\cite{konventionen_managing-accounts} der Einstiegsaufwand und damit auch die Kürze von Aufgaben auf dem Handys verringern.
%
%}
\subsubsection{Auswahl}


\myNewSection
%Auswahl von Konvention
	%Was: Mehrere Konventionen
	Es stehen mehrere verschiedene Richtlinien für Apps zur Verfügung, z.B. eine für iOS von Apple und eine für Android von Google\cite{konventionen_guidelinesApple, konventionen_guidelinesGoogle}. %
	%Was+Warum: Entscheiden
	Es wird jedoch eine einzige Richtlinie gewählt, da unterschiedliche Richtlinien unterschiedliche Regeln enthalten und sich gegebenenfalls sogar widersprechen können. %
		%Beispiel
		Google bietet beispielsweise eigene Seiten und Regeln für \glqq Floating Action Buttons\grqq{}\cite{konventionen_floatingActionButton}, während diese in den Apple Richtlinien nie erwähnt werden und daher wahrscheinlich auch nicht erwünscht sind. %
	%Was+Warum: Auswahl Apple -> eigene Präferenz(leichter+intuitiver)
	Da aufgrund persönlicher Präferenz die Apps von Apple als noch intuitiver und leichter zu bedienen empfunden werden und dies für die Anforderungen der App von Vorteil ist, wurde sich für die Richtlinien von Apple entschieden.%
	% Aus eigener Präferenz wurde aufgrund der Tatsache, dass die Apps dieser Plattform intuitiver und leichter zu bedienen sind und dies den Anforderungen der App entspricht, entschieden, die Richtlinien von Apple zu verwenden.
	
\myNewSection
%Richtlinien: Inhalt + Was wurde betrachtet + Was nicht
	%Was: Woraus besteht die Richtlinie
	Die Richtlinien von Apple bestehen aus fünf Abschnitten, die unter anderem die Gestaltung einzelner Komponenten wie Textfelder und Knöpfe sowie allgemeine Regeln und Muster für die Erstellung von Apps behandeln. Insgesamt umfasst die Richtlinie etwa 148 Regelungen. %
	%Was: davon interessant
	Davon schienen anfangs etwas mehr als die Hälfte als nützlich für die Arbeit. Bei genauerer Betrachtung stellten sich 34 dieser Einträge als tatsächlich passend heraus, da sie Funktionen und Anforderungen behandeln, die in die Arbeit einfließen sollen. %
	%Was: davon uninteressant
	Die meisten restlichen Einträge behandeln Funktionen, die nicht in die Anwendung integriert werden sollen, wie zum Beispiel die Bedienung per Hardwaretastatur oder NFC-Funktionalität\cite{konventionen_keyboard, konventionen_nfc}. Einige andere Einträge, die eigentlich nützliche Funktionen für diese Arbeit beschreiben, wie zum Beispiel ein Nachtmodus oder Siri-Unterstützung, wurden aufgrund des begrenzten Zeitrahmens vorerst übersprungen\cite{konventionen_darkmode,konventionen_siri}.%
\subsubsection{Auswertung \& Erkenntnisse}\label{subsection:design:erkenntnisse}%
%Was
In diesem Abschnitt werden Erkenntnisse und Muster beschrieben, die während der Ausarbeitung der Richtlinien ersichtlich wurden.\newline%
%Warum
Diese Erkenntnisse haben einen allgemeineren Charakter als die Richtlinien und können dementsprechend auf die gesamte App und auch auf Bereiche angewendet werden, für die keine Regeln vorhanden sind.\newline%
Dabei ist zu beachten, dass einige der Erkenntnisse eng miteinander verknüpft sind und deshalb einen fließenden Übergang zwischen ihnen herrscht.%
%
\newline
\myNewSection
Das Design und die App sollten gemäß den Richtlinien folgende Merkmale aufweisen:
\begin{enumerate}%[noitemsep,topsep=0pt,parsep=0pt,partopsep=0pt]
	%%% kurzweilig sind oderwenig Zeit benötigen.
	\item Sofort einsatzbereit sein.%
	\begin{itemize}%[noitemsep,topsep=0pt,parsep=0pt,partopsep=0pt]
		\item[] Denn es wird unter anderem empfohlen:% 
		\item den Inhalt der App möglichst früh zu zeigen\cite{konventionen_patterns_loading}%
		\item die Notwendigkeit für eine Anmeldung möglichst lang zu Verzögern\cite{konventionen_patterns_managingAccounts}%
		\item sowie initiale und unterbrechende Informationsanfragen, wie zum Beispiel nach Bewertungen oder Zugriffsrechten, nur wenn wirklich nötig anzufordern\cite{konventionen_patterns_launching}.%
	\end{itemize}


	%%% einfach und intuitiv seien sollen
	\item Aufgaben vereinfachen.%
	\begin{itemize}%[noitemsep,topsep=0pt,parsep=0pt,partopsep=0pt]
		\item[] Denn es wird unter anderem empfohlen:% 
		\item Texte so klar wie möglich zu verfassen\cite{konventionen_foundations_writing}
		\item wo möglich, Informationen automatisch vom System zu entnehmen, anstatt den Nutzer danach zu fragen\cite{konventionen_patterns_enteringData, konventionen_platforms_ios}
		\item sowie Alternativen zur Texteingabe, wie zum Beispiel \glqq drag \& drop\grqq{} oder eine Liste von Optionen, anzubieten\cite{konventionen_patterns_enteringData}.
		%\item choices instead of text entries
	\end{itemize}

	%%% nur wenig Informationen darstellen oder benötigen
	\item Informationen reduzieren.%
	\begin{itemize}%[noitemsep,topsep=0pt,parsep=0pt,partopsep=0pt]
		\item[] Denn es wird unter anderem empfohlen:% 
		\item Texte so kurz wie möglich zu verfassen\cite{konventionen_foundations_writing}
		\item die Anzahl an Steuerelementen zu begrenzen\cite{konventionen_platforms_ios}
		\item simplifizierte Designs für Icons zu verwenden\cite{konventionen_foundations_icons}
		\item sowie wichtigen Informationen genügend Platz zu geben, indem zum Beispiel eher unwichtige Details ausgelassen werden\cite{konventionen_foundations_layout}.
	\end{itemize}	

	%%% einfach und intuitiv seien sollen
	\item Informationen indirekt vermitteln.%
	\begin{itemize}%[noitemsep,topsep=0pt,parsep=0pt,partopsep=0pt]
		\item[] Denn es wird unter anderem empfohlen:% 
		\item Farbe zu nutzen, um Bedeutung zu vermitteln (beispielsweise indem grün für bestätigende Aktionen genutzt wird) \cite{konventionen_foundations_color}.
		\item Icons zu nutzen, da diese bei richtiger Verwendung ein Konzept sofort verständlich machen können\cite{konventionen_foundations_icons} 
		\item Textgröße, Schriftart und Farbe zu nutzen, um Wichtigkeit zu vermitteln\cite{konventionen_foundations_typography}.
		\item Platzierung von Objekten zu nutzen, um Wichtigkeit zu vermitteln\cite{konventionen_foundations_layout}
		\item sowie zur Situation passende Sprache zu verwenden\cite{konventionen_foundations_writing}.
	\end{itemize}
	
	%%% mit der Annahme, dass es sich lohnt bestehende Richtlinien zu folgen
	\item Konsistent innerhalb der App und Plattform sein.% 
	\begin{itemize}%[noitemsep,topsep=0pt,parsep=0pt,partopsep=0pt]
		\item[] Denn es wird unter anderem empfohlen:% 
		\item die Plattformfarben zu nutzen und bei einmal definierten Farben konsistent zu bleiben\cite{konventionen_foundations_color}
		\item die Plattformdefinierten Textstile zu nutzen\cite{konventionen_foundations_typography}
		\item die Plattformdefinierten Gesten zu nutzen\cite{konventionen_foundations_accessibility} 
		\item sowie auf allen Seiten ein konsistentes Icondesign zu verwenden\cite{konventionen_foundations_icons}.
	\end{itemize}
	
	
	%%% einfach und intuitiv seien sollen
	\item Intuitiv sein.%
	\begin{itemize}
		\item Denn alle vorherigen Punkte haben dies als Ziel.
	\end{itemize}
	

\end{enumerate}
%
%
%was: Erkentnisse überschneiden mit PcVsPhone Stärken
Interessant hierbei ist, dass sich diese Erkenntnisse mit den Ergebnissen aus \secref{section:pcVsPhone} überschneiden. %
	%Beispiel:
	So passt beispielsweise die Erkenntnis \glqq Sofort einsatzbereit sein\grqq{} zur Stärke des Handys, gut für kurzweilige Aufgaben zu sein. Die Erkenntnisse \glqq Aufgaben vereinfachen\grqq{}, \glqq Informationen indirekt vermitteln\grqq{} und \glqq Intuitiv sein\grqq{} passen hingegen zum Ergebnis, dass einfache und intuitive Aufgaben gut zum Handy passen.  \glqq Informationen reduzieren\grqq{} passt hingegen zum Ergebnis, dass auf Handys Aufgaben besser funktionieren, die wenig Informationen darstellen oder benötigen.\newline%
	%Auf in Einzelnen Regeln
	Weiter gibt es in den Richtlinien auch einzelne Regeln, bei denen es zu einer ähnlichen Überschneidung kommt. 
		%Wenig Konfig + Eingabe	
		So rät Apple beispielsweise sowohl von Texteingaben als auch von einer hohen Anzahl an Einstellungen ab\cite{konventionen_patterns_enteringData,konventionen_settings}. Dies überschneidet sich mit den Ergebnissen, dass Handys keine schnelle, präzise und vielfältige Eingabe erfordern sollten und ohne viele Optionen und Konfigurationen auskommen sollten.
\subsection{App Design}%
%Einleitung
In diesem Abschnitt werden einige konkrete Designentscheidungen vorgestellt. Generell wurde sich dabei an die Erkenntnisse aus \secref{subsection:design:erkenntnisse} gehalten.%
\subsubsection{Appleisten}
%Was
Bei den Appleisten wurde generell versucht, die Funktionen der Seite und der Tasten so klar wie möglich darzustellen. %
	%Wie: deutlichkeit
	Dazu wurden für die Seiten und Tasten möglichst kurze und deutliche Wörter und Icons verwendet. %
	%Konsistenz:
	Außerdem wurde versucht, die Leisten möglichst konsistent mit bereits bestehenden Konventionen zu entwerfen. So befindet sich der Zurückknopf, wie man es erwarten würden, immer am oberen linken Rand. Zudem wurde vorerst ein \glqq Tab-Design\grqq{} gewählt \ref{} \myTodo, da dieses auch in vielen Apps vorkommt.\newline%
	%Design änderung
	Jedoch wurde später beschlossen, von diesem ursprünglichen Design abzuweichen. Die Funktion, eine Erinnerung hinzuzufügen, wird als eine der häufig genutzten Hauptfunktionen der Anwendung angesehen. Allerdings ist sie in diesem Design relativ schwer zu erreichen (Plus-Icon oben rechts). Es wäre besser, diese Aktion unten oder in der Mitte anzubieten, da dieser Bereich für die meisten Nutzer besser erreichbar ist\cite{konventionen_platforms_ios}. Zusätzlich dazu ist der Tab zu der Einstellungsseite relativ groß und immer sichtbar, obwohl davon ausgegangen wird, dass dieser Seite nur einmal oder sehr selten benötigt wird.\newline%
	%Auswirkung
	Dementsprechend wurde das Design abgeändert\ref{}\myTodo. Die Hauptaktion eine Erinnerung zu erstellen ist nun verständlicher erklärt und besser erreichbar. Die Aktion zur Einstellungsseite ist nun kleiner und wurde aus dem meistgenutzten Bereich (unten und mittig) entfernt. Auch die Navigationstaste zur Kalenderseite wurde entfernt, da diese als Hauptseite gilt und dementsprechend immer über den Zurückknopf erreichbar ist.
	%\myTextTodo{anders als das Tab Design verändert sich unser design je nach seite, dies kann das design zwar etwas komplizierter machen da nicht statisch/konsistent, aber dafür kann für jede seite genau die benötigte }
	
%\pic{old} \pic{new}
%Konvention: Use the standard back button
%Text: Use the title area to describe the current screen if it provides useful context
%Keep tabs visible even when their content is unavailable
%Use concrete nouns or verbs as tab titles
\subsubsection{Kalenderseite}%
Wie bereits im \secref{subsection:auswahlDerPakete} erwähnt, wird für die Darstellung des Kalenders ein bestimmtes Paket genutzt. Daher ließ sich das Design dieser Seite nur begrenzt verändern. Allerdings verhält sich das Design des genutzten Pakets konsistent zu anderen Kalender-Apps und wird daher als passend für diese App bewertet.\newline%
%Ladezeit
Um die sofortige Nutzung der App zu ermöglichen, ist weder eine initiale Konfiguration noch ein Login erforderlich, um auf den Kalender oder andere Seiten zuzugreifen. Weiter wird während des Ladens von Daten aus der Datenbank eine Ladeanimation angezeigt, um so dem Benutzer visuelles Feedback zu geben und eine intuitivere Benutzererfahrung zu schaffen.%
%pic{ohne anmeldung} pic{ladevorgang + angezeigte daten}
%
%Des Weiteren wird um eine schnelle Ladezeit für den Kalender zu ermöglichen vorerst nur 4 Monate berechnet als das ganze Jahr. Weitere Einträge werden sobald benötigt nachgeladen.
%\pic{}
\subsubsection{Erinnerungenseite}
%Liste
Die Erinnerungsseite zeichnet sich hauptsächlich durch ihre Liste von Erinnerungen aus. Diese Liste nutzt Animationen, um einen fließenden Übergang beim Erstellen und Abschließen von Erinnerungen zu gewährleisten. Ohne diese Animationen könnten abrupte Änderungen in der Darstellung auftreten und möglicherweise zu Verwirrung führen.\newline%
%Einträge
Die Erinnerungseinträge bestehen aus einem Titel und enthalten optional eine Beschreibung sowie Dateien. %
	%Warum
	Denn es wird angenommen, dass für einfache Erinnerungen wenige Wörter oder ein kurzer Satz ausreichen. Für komplexere Erinnerungen kann stattdessen die Beschreibung genutzt werden. In Fällen, in denen sich die Erinnerung hingegen leichter über eine Datei erklären lässt, soll auch dafür eine Möglichkeit bestehen.\newline%
%Kompakt
Das dazu erstellte Design wurde versucht so kompakt wie möglich zu gestalten, um so auf dem relativ kleinen Bildschirm eines Handys alle notwendigen Informationen angemessen darzustellen. %
	%Optional
	Des Weiteren wurde das Design so gestaltet, dass es ansprechend aussieht, unabhängig davon, ob die optionalen Eingaben vorhanden sind oder nicht.\newline%
%Was: bestätigungsknopf
Das Abschließen einer Erinnerung wird durch einen Knopf neben der entsprechenden Erinnerung ermöglicht. %
	%Warum
	Da der Knopf immer sichtbar ist, lässt er sich deshalb mit minimalem Aufwand betätigen, ohne dass weitere Aktionen und Zwischenschritte erforderlich sind. %
	%Links knopf
	Der Knopf wurde bewusst links von der Erinnerung platziert, da an dieser Position davon ausgegangen wird, dass Nutzer\footnote{gilt nur für Nutzer, die das Handy mit der rechten Hand nutzen} ihn seltener versehentlich betätigen.\newline%
%Erstellung
Außerdem wird, falls bei der Erstellung einer neuen Erinnerung kein Titel gewählt wurde, dieser automatisch erstellt, um dem Nutzer Zeit und Aufwand zu ersparen.%
%
%pic{leere seite} pic{mit 4 Einträgen, nur titel, titel und beschreibung, titel datein, titel beschreibung dateien}
%
%
\subsubsection{Einstellungsseite}%
%Was
Die Hauptmerkmale dieser Seite sind die Textfelder zum Einstellen der Login-Optione. %
	%Konsistenz
	Dabei wurde versucht, sie so zu gestalten, dass sie möglichst konsistent mit den iOS-Plattformtextfeldern sind.\newline%
%Intuitiv
Um die Benutzerfreundlichkeit zu erhöhen, werden Abhängigkeiten zwischen den Textfeldern durch Farben und Fehlermeldungen und Hinweise zur Eingabe indirekt angezeigt. Zudem wird eine Ladeanimation eingeblendet, um den Fortschritt bei der Überprüfung der Eingaben zu signalisieren.
	%Deaktiviert, Hints, Error
	%Um möglichst intuitiv zu sein, wird durch Farbe indirekt angedeutet welche Felder voneinander abhängig sind. Weiter wird gezeigt welche Art von Eingaben erwartet werden und warum Eingaben gegebenenfalls nicht funktionieren indem Tipps zur Eingabe sowie Fehlermeldungen angezeigt werden.\newline%
	%Laden
	%Au Während die Eingaben von der Datenbank überprüft werden, wird dem Nutzer eine Ladeanimation angezeigt, um ihn über den Fortschritt zu informieren.\newline%
%automatisches überprüfen
Um Nutzern Arbeit abzunehmen, werden bei der Änderung eines Textfelds die davon abhängigen Textfelder automatisch überprüft. Andernfalls müsste der Benutzer beispielsweise nach der Änderung des Tokens auch das Repository und den Pfad zur Konfigurationsdatei neu festlegen.\newline% Da diese jedoch automatisch mit getestet werden, spart sich der Nutzer mehrere Aktionen.
%AutoSetup
Die Hauptaktion dieser Seite ist die \glqq autoSetup-Aktion\grqq{} und ersetzt dementsprechend die Hauptaktion eine neue Erinnerung zu erstellen. Diese Aktion erstellt ein neues Repository mit der benötigten Konfigurationsdatei sowie Vorschaueinträgen und stellt die entsprechenden Optionen in der App automatisch ein. Um die Funktion dieser Aktion zu erläutern und da davon ausgegangen wird, dass diese Aktion durchaus aus Versehen betätigt werden kann, wird vorher mithilfe eines \glqq Alerts\grqq{} die Aktion erklärt und nach Bestätigung gefragt.\newline%
%Informationen
Außerdem werden am Ende der Seite einige für den Nutzer möglicherweise interessante Informationen angezeigt.\newline%
Während Aktionen ausgeführt werden, die von der Datenbank abhängen, wie beispielsweise dem Erstellen, Beenden und Bearbeiten von Erinnerungen, werden kleine Ladeanimationen neben der ausgeführten Aktion angezeigt. Dadurch soll der Nutzer auch ein Feedback über den Status der Datenbank erhalten. 
\newline%
Für eine Beispieldarstellung siehe Abbildung \ref{fig:settings_page} und \ref{fig:settings_page_alert}.


\subsubsection{Tastatur}%
%Was
Wie zuvor erwähnt, ist geplant, alternative Eingabemöglichkeiten für die Erstellung von Erinnerungen anzubieten. Es wurde entschieden, Fotos, Videos, Sprachnachrichten oder andere auf dem Handy bereits vorhandene Dateien als Optionen anzubieten, da angenommen wird, dass diese Dateitypen nützlich für die Erinnerungen sein könnten und einfach über das Handy zugänglich sind.\newline%
%Wo Platzieren
Anstatt diese Alternativen statisch auf einer Seite anzuzeigen, sollten sie nur dann angezeigt werden, wenn sie benötigt werden. Dadurch werden die dargestellten Informationen reduziert und eine bessere Übersichtlichkeit gewährleistet. Gleichzeitig wird die Bedeutung der Aktionen indirekt vermittelt, da dadurch gezeigt wird, wann und wofür die Aktionen genutzt werden können. Der dafür passende Platz ist die Systemtastatur.\newline%
%Kosistenz
Um konsistent mit dem Betriebssystem zu bleiben, soll die Standardtastatur beibehalten werden. Anstatt eine eigene Tastatur zu entwerfen, werden die Alternativen stattdessen in die bereits existierende Standardtastatur integriert.\newline%
%Icons
Die Aktionen werden wiederum als Icons dargestellt, um mit möglichst wenig Platz ihre Bedeutung zu vermitteln. 
\newline%
Für eine Beispieldarstellung siehe Abbildung \ref{pic:tastatur}.

%\input{4_design/4_3_architektur.tex}
%\input{4_design/4_4_objektDesign.tex}
\section{Implementierung}\label{section:implementierung}

\input{5_implementierung/stichpunkte5.tex} %todo: remove after this sections completion

%%% hidden subsection for a better structure in latex editor: "texifier"
\myComment{\subsection*{Übersicht}}%
%Einleitung
	Da nun die Anforderungen und das Design der Anwendung bekannt sind sowie die Technologie, mit der sie erstellt werden soll, kann sich der vorliegende Abschnitt mit der Implementierung befassen.
	%Begrenzung: Was&WasNicht
	Dabei wird nicht die gesamte Implementierung dargestellt, da dies den Umfang dieser Arbeit sprengen würde. Stattdessen werden die wichtigsten Komponenten der Anwendung sowie die während der Implementierung getroffenen Entscheidungen und Erkenntnisse behandelt.%
%Übersicht
\newline%
\textbf{Übersicht:}\newline%
Zunächst werden einige Eigenheiten und Besonderheiten während der Implementation des Parsers und der Verbindung zur Datenbank genannt. Anschließend wird erläutert, wie die Qualität der Anwendung und des Programmcodes sichergestellt wurde. Zum Schluss werden einige gezielt genutzte Entwurfsmuster genannt und ihre Verwendung begründet.%
%Zusammenfassung
\newline%
\textbf{Erkenntnisse:}\newline%
	%Parser
	Für den Parser wurde generell versucht, sich möglichst immer an die Vorgaben der Dokumentation zu halten, damit Nutzer ihre bereits bestehenden Kalenderdateien nicht anpassen müssen. Dabei gab es jedoch zwei Fälle, in denen von der Dokumentation abgewichen wurde. % 
		%Zeitangabe
		Zum einen bei der Zeitangabe, da angenommen wurde, dass die Anwendung von dieser Änderung profitieren würde. So bietet die Anwendung zusätzlich zur Interpretation eines Startzeitpunkts auch die Funktion, den Endzeitpunkt zu verstehen und anzuzeigen. %
		%Fehler
		Zum anderen wurden Syntax und Fehleingaben der Kalenderdatei anders interpretiert, da sie beim When-Programm inkonsistent zu sein scheinen. %
		%Variabeln&Operationen
		Zuletzt wurde sich aufgrund der begrenzten Zeit vorerst darauf beschränkt, nur die wichtigsten Operationen und Variablen des CLI-Terminkalenders zu unterstützen.%
	\newline%
	%Datenbank
	Bei der Verbindung zur Datenbank mussten besondere Vorkehrungen für die Authentifizierung und das Herunterladen von Dateien getroffen werden, da die Endpunkte der API diese Funktionen nicht zufriedenstellend erfüllen. Zudem wurde bei der Darstellung der Erinnerungen in der Datenbank versucht, ein Format zu wählen, das sowohl für die WebView als auch das CLI geeignet ist.%
	\newline%
	%Qualität
	Um die Stabilität der Anwendung sicherzustellen, wurden automatische Rückfalltests für den Parser und die Datenbankverbindung durchgeführt, während die grafische Oberfläche manuell getestet wurde %
	Zur Verbesserung der Lesbarkeit des Quellcodes wurden Modularisierung, Zugriffsmodifikatoren und eine eigene Kommentarstruktur angewendet.%
	\newline
	%Entwurfsmuster
	Für die Implementierung der Datenbankverbindung und des Parsers wurden die Singleton- und Strategie-Patterns sowie eine ablagebasierte Struktur verwendet, um den Programmcode anpassungsfähiger und übersichtlicher zu gestalten. Darüber hinaus wurde das Proxy-Pattern eingesetzt, um geladene Dateien zwischenzuspeichern und somit das ständige Neuladen und Neuberechnen durch Datenbank und Parser zu verhindern.%
%
%
%
%
%---Old-Rephrased---
%\myTextTodo{
%%Implementierung - Was + Warum wenig Implementierung wiedergegeben
%\textbf{Abschnitte der Arbeit}\\
%An dieser Stelle sollten die wichtigsten Erkenntnisse erhoben worden sein. Nun gehen wir über zum \secref{section:implementierung}. Hier wird betrachtet wie die Software aufgebaut sein soll. Also geht es auch hier unter anderem um das \dq innere Design\dq.\newline%
%Die Implementierung, ein Großteil der eigentlichen Arbeit von diesem Kapitel, wird nicht wiedergegeben. Jede einzelne Entscheidung und Erkenntnis der Implementierung zu erwähnen und begründen würde nicht nur den Rahmen sprengen, sonder auch sehr ermüdend für den Leser werden. Daher werden nur die wichtigsten Entscheidungen, Schwierigkeiten und Erkenntnisse erwähnt.\newline%
%}
%
%\myTextTodo{
%Aufzählung, Alternativen, Entscheidungen, \\
%LEITFADEN: 1.Was ist die Eigenschaft, 2.Warum ist es wichtig, 3.Wie wird es umgesetzt\\
%} %%%
%
%
%\myNewSection
%auto setup -> einfachheit für das handy -> anforderungen\newline
%
%\myNewSection
%Github config file:\newline
%json weil vs text. erst json aus einfachheit später auch noch txt files möglich da dies besser zu cli terminkalendar passen würde
%mögliche optionen...
%konfiguration welche auf dem handy nötig ist: token+repoName+configPath
%kommenatre in json nicht möglich zum erklären der variabeln. Lösung: erklärung in den variabel namen vs extra kommentarVariabeln = 'string'.
%\newline
%config file: besonders lange variabeln namen. normalerweise nicht sinnvoll und da kürzere beim programmieren genau so verständlich seien können. da aber der enduser nicht das gleiche wissen wie ein interner entwickler der app hat, wurden lange variabel namen genutzt, damit die bedeutung klar wird. + json akzeptiert keine commantare.
\subsection{Parser}\label{subsection:imp:parser}
%%
%%
%%
\subsubsection{Modell}
%Einleitung
Um einen Parser für den WhenKalendar zu erstellen, musste dieser zunächst verstanden werden. Dafür wurde die Dokumentation ausführlich [gelesen/studiert] und intensiv manuell getestet.\newline%
%Was: WhenAppointment Erklärung 
Dabei wurde festgestellt, dass ein Termin in When aus drei Teilen besteht: dem Datum, einer optionalen Zeitangabe und einer Beschreibung.%
%Was: WhenDates
Die Herausforderung beim Parsen der Termine liegt insbesondere im Datum. Während die Beschreibung sowie die Zeitangabe fast eins zu eins übernommen werden können, kann das Datum in variabler Darstellung vorliegen. So würde zum Beispiel \glqq m=2\grqq{} bedeuten, dass der Termin jeden Tag im Februar stattfindet.
%Kombinationen
Darüber hinaus können diese variablen Datumsangaben miteinander über Operationen kombiniert werden. Der Termin \glqq m=2\&(d=1|d=25)\grqq{} würde zum Beispiel immer am ersten und am 25. Februar stattfinden.
\newline
\myNewSection
%Vorraussetzung
Um diese Daten zu parsen, wird vorerst angenommen, dass Daten immer in Klammern eingeschlossen sind. Dadurch wird der Aufwand für die Implementierung des Parsers reduziert, da die verschiedenen Bindungsstärken der Operationen nicht berücksichtigt werden müssen. %
%Reruksiv
Anschließend wird das Datum rekursiv durchgegangen, bis ein einzelner Ausdruck ohne Operation gefunden wird. Aus diesem Ausdruck wird ein modelliertes WhenDate-Objekt erstellt. Dieses Objekt kann mit anderen Objekten seiner Art über die When-Operationen kombiniert werden. Dadurch kann die Rekursion schlussendlich ein Ergebnis liefern. %
\newline
\myNewSection
%Lösung
Nun verfügt der Parser über ein WhenDate-Objekt. Dieses kann jedoch immer noch variabel sein. Um Termine im Kalender anzuzeigen, benötigt es jedoch konkrete Daten. Dazu wird von einem gewählten Startdatum bis zu einem Enddatum überprüft, welche konkreten Daten das variable WhenDate-Objekt annehmen kann. Diese Funktion wird auch nützlich sein, um Termine dynamisch im Kalender nachzuladen. So werden zunächst nur wenige Monate im Voraus und in der Vergangenheit berechnet, um so eine möglichst schnelle Ladezeit zu gewährleisten. Wenn der Benutzer Termine über die initial geladenen Einträge hinaus ansehen möchte, werden diese automatisch nachgeladen und angezeigt.\newline%
%Kalender dynamisch laden
Diese Funktion ist auch nützlich, um Termine dynamisch im Kalender nachzuladen und wird dementsprechend in der Kalenderimplementierung genutzt. So werden zunächst nur wenige Monate im Voraus und in der Vergangenheit berechnet, um eine möglichst schnelle Ladezeit zu gewährleisten. Wenn der Benutzer Termine über die initial geladenen Einträge hinaus ansehen möchte, werden diese automatisch nachgeladen und angezeigt.%
\myTextTodo{Pic:Whendate Modiliert}
%
%
%
%
%
\subsubsection{Zeitangabe}
Dadurch, dass die App eine grafische Oberfläche hat, besteht die Möglichkeit, die Zeitspanne von Terminen anzuzeigen. Obwohl diese Option nicht von When unterstützt wird - bei Terminen wie \glqq..., 17:00-18:00 ...\grqq{} wird immer nur die erste Zeitangabe berücksichtigt -, wird diese Funktion in die App integriert, da angenommen wird, dass sie eine nützliche grafische Information darstellt.%
\myTextTodo{Bild von Zeitspanne: none-daily-17bis18}
%
%
%
%
%
\subsubsection{Variabeln \& Operationen}
%Was
When bietet eine Vielzahl von Variabeln und Operationen an, um daraus Termine zu erstellen. Dazu gehören klassischen Vergleichsoperatoren wie \glqq =,!=,>,<,>=,<=\grqq{} sowie Logikoperatoren wie \glqq|,\&,!\grqq{}. %
%Was: Zeitbeschränkugn + Menge an Optionen-> Auswirkung
Aufgrund der begrenzten Zeit und der Vielzahl an Optionen wurde sich zunächst auf die als am wichtigsten und elementarsten eingeschätzten Variablen und Operationen konzentriert. Die Wahl viel dabei auf die Operationen \glqq|,\&\grqq{} und die Variablen \glqq d,m,w,y\grqq{} für die Angabe eines Tages, eines Monats, eines Wochentages und eines Jahres.%
%
%
%
%
%
\subsubsection{Eingabe- \& Formatfehler}%
%Was
Während des Testens von When wurden einige Inkonsistenzen festgestellt. %
%Konstant vs Variable
Wenn ungültige Daten wie beispielsweise \glqq 2000 1 0 \grqq{} als Konstanten eingegeben werden, wird ein Fehler angezeigt. Jedoch tritt dieser Fehler nicht auf, wenn das gleiche Datum über Variablen und Operationen eingegeben wird, z.B. \glqq y=2000 \& m=1 \& d=0\grqq{}.\newline%
%erstige Annahme
Vorerst wurde angenommen, dass bei der Verwendung von Variablen keine Fehler auftreten, da auch bei ungültigen Eingaben ein gültiger Termin erzeugt werden kann. Zum Beispiel ist die Eingabe von \glqq d=0\grqq{} ungültig, aber der folgende Ausdruck führt dennoch zu einem gültigen Termin: \glqq d=0|y=2000\grqq{}.
	%Jedoch
	Jedoch hat sich herausgestellt, dass dies nicht immer der Fall ist, da Eingaben wie \glqq d=-1\grqq{} oder \glqq m=Marchz\grqq{} trotz Verwendung von Variablen zu Fehlermeldungen führen. %
%Auswirkung
Dieses Verhalten ist inkonsistent und scheint eher willkürlich zu sein. Es wäre besser, wenn Eingaben immer das gleiche Ergebnis liefern würden. Entweder sollten also alle genannten Beispiele zu Fehlern führen oder keines. Aus diesem Grund wurde sich im Sinne der Konsistenz leicht von der ursprünglichen Vorlage abgewendet. Es wurde entschieden, keines der oben genannten Daten als Fehler zu betrachten, da Variable Daten auch bei ungültigen Eingaben immer noch einen gültigen Termin erzeugen können.%
%
%
%
%
%
\myComment{
%- jedoch unterscheiden wir zwischen eingabe fehlern oder format fehlern, bei format fehlern wird trotzdew die ganze zeile ignoriert, (bei input fehlern nur die equation). 
%-bei format fehlern könnte man zwar noch einige teil equations interpretieren/verstehen, jedoch ist das chance sehr hoch dass diese durch den format fehler falsch interpretiert werden und so auf falsche daten landen. + würde fehlerkontrolle erwschwären. daher lieber garnicht anzeigen als falsch. ODER anderer grund?
%-außerdem wird alles was wir als formatfehler interpretieren auch von when so gesehen / bzw wurde von when so definiert (bei diesen (format r fehlern ist when nicht willkührlich), daher werden diese fehler bei when im terminal angezeigt (müssen nicht in der app extra erwähnt werden / niemand wird sich wundern wenn etwas nicht angezeigt wird da when es im terminal erwähnt)
%- wir zum akzeptieren zumbeispiel nicht nur yyyy mm dd sondern y+ m+ d+
}
\subsection{Ansprüche an den Quelltext}
\subsection{Qualitätssicherung}\label{section:qualitaetssicherung}
%
%
%
%
\subsubsection{Tests}
%Einleitung:
Um sicherzustellen, dass die Anwendung stabil läuft und keine Defekte enthält, werden mehrere Tests durchgeführt. %
%Was + Warum: Parser+Db viel getestet -> wichtig
Dabei wird der Parsers und die Verbindung zur Datenbank am ausgiebigsten getestet, da sie die Grundfunktionen der Anwendung darstellen und somit als kritische Bereiche angesehen werden. %
%Warum+Was: ManuellesTesten
Da diese beiden Funktionen viele Komponenten enthalten, die oft geändert werden, wäre das manuelle Testen zu zeitaufwendig und müsste nach jeder kleinen Änderung erneut durchgeführt werden. %
	%Auswirkung: Rückfalltests
	Deshalb wurde stattdessen beschlossen, automatisierte Rückfalltests durchzuführen. %
		%Pro/Con:
		Obwohl die Erstellung von automatisierten Rückfalltests anfangs mehr Aufwand erfordert, lohnt es sich, sobald die Komponenten häufiger getestet werden müssen. Außerdem bieten die Tests eine Art von Dokumentation, da sie zeigen, wie die Komponenten funktionieren sollten. Deshalb werden die Rückfalltests auch als nützlich für die Weiterentwicklung der Anwendung eingeschätzt.\newline%
%Was+Warum: Bottom-Up-Integrationstests
Um sicherzustellen, dass die Tests zeigen, ob ein Defekt von der getesteten Komponente oder von einer von ihr aufgerufenen Funktion verursacht wird, werden die Tests nach dem Schema der Bottom-Up-Integrationstests geschrieben. Das bedeutet, dass zuerst die Module und Komponenten getestet werden, für die alles, was sie aufrufen, bereits getestet wurde.\newline%
%Was:Eingabe
Damit die Tests die Komponenten möglichst vollständig auf Defekte prüft, wurde bei der Erstellung der Tests versucht durch die Eingaben das Abdeckungskriterium der Bedingungsüberdeckung zu erreichen. Außerdem wurden stets Randfälle wie leere Eingaben oder völlig unsinnige Eingaben getestet, da dies aus Erfahrung oft zu Defekten führt.\newline%
%Coverage
Die Ausführung der Tests mit Code-Coverage-Analyse ergab, dass im Parser 70\% und bei der Datenbankverbindung 90\% aller Codezeilen getestet werden.\newline%
%Grafische Oberfläche
Die Funktionen der grafischen Oberfläche wurden hingegen manuell getestet, da sich diese besser für manuelle Tests eignen und somit zeiteffektiver sind als automatisierte Tests.%
%
%
%
%
%
\subsubsection{Verständlichkeit}
%Einleitung
%Was: Trennung von Belangen und Modularität
Um den Quelltext möglichst verständlich zu gestalten, wurde generell versucht, Komponenten möglichst modular zu gestalten und so aufzuteilen, dass jede Komponente immer genau eine einzige einzigartige Aufgabe hat. %
%Visability
Des Weiteren wurden Zugriffsmodifikatoren für Methoden gesetzt. Dadurch soll deutlich werden, welche Methoden der Komponente ausschließlich für interne Funktionen genutzt werden und welche dem Klienten zur Verfügung stehen.%
%Was: Kommentare
Außerdem wurden für wichtige Funktionen, wie beispielsweise alle Funktionen aus der Datenbankverbindung und dem Parser, Kommentare verfasst. %
	%Nachteile + Vorteil:
	Kommentare haben zwar den Nachteil, dass sie bei Änderungen des betreffenden Codes aktualisiert werden müssen, jedoch können sie die Verständlichkeit verbessern. Ein kurzer Satz kann beispielsweise die Funktion einer ansonsten großen und schwer lesbaren Methode erläutern. %
	%Warum + Was: Format
	Um genau solche nützliche Kommentare zu schreiben, wurde ein selbst vorgegebenes Format verwendet, das sich während der Programmierung als hilfreich erwiesen hat. Das Format besteht aus den Feldern \glqq def\grqq{}, \glqq purpose\grqq{}, \glqq assert\grqq{}, \glqq expect\grqq{}, \glqq return\grqq{} und \glqq example\grqq{}.\glqq Def\grqq{} beschreibt die Funktion, \glqq purpose\grqq{} gibt an, wofür die Funktion benötigt wird, \glqq assert\grqq{} zeigt was die Funktion voraussetzt, \glqq expect\grqq{} gibt an, was die Funktion erwartet, aber nicht unbedingt voraussetzt und \glqq example\grqq{} enthält ein Beispiel. Die Felder werden nicht immer alle benutzt, sondern nur dann, wenn es als nützlich erscheint.\newline%
		%assert (meistens preconditions?)
		%Da mit dem kommentar assert preconditions gegeben sind, wurde dies gleichzeitig auch mit der mithilfe der Flutter gegebenen assert Funktion während der laufzeit zugesichert. %

\myComment{

%\subsubsection{Exceptions}
%Um dem Endbenutzer eine gute Qualität 
%exceptions & errors: für die verbindung mit der api wurde versucht möglichst alle möglichen errors/exceptions abzufangen. dies wurde durch ausgiebieges testen versucht zu bestätigen. wenn es nämlich während des handy nutzens zu einen error kommen würde, würde dass die benutzung der app behindern / große fehler abbilden. z.b. würde die api nach 1000 abrufen fehler werfen, bis eine neue verbindung/ip hergestellt würde/ softcap (link). lieber sollten diese fehler geplant im handy angezeigt werden anstatt dass diese abstürzt oder einen flutter code error angezeigt bekommt. dementsprechend werden alle api calls exeptions mit try()catch() abgefangen. Jedoch sollte die möglichkeit für eine solche exception ziemlich niedrig sein, da beim testen sehr viele dieser calls gemacht wurden, es aber insgesamt während der test und programmieren nur einmal dazu kam (art von stresstest). zum beispiel würde die github api bei ... eine exception werfen, dabei sollte die app aber nicht abstürzen, dementsprechend gibt die implementierte verbindung zur datenbank einfach einen bool false aus anstatt die expection zu werfen. ratelimit: the rate limit is 1,000: https://docs.github.com/en/rest/overview/resources-in-the-rest-api?apiVersion=2022-11-28
%


%%Repo
%Auch das Repository wurde versucht Lesbar zu gestalten, damit auch dieses von anderen gelesen und nachvollzogen werden kann. Dafür wurde einerseits für die Darstellung der Commit-History die Lineare Option eingestellt und befolgt und andererseits für die Commit-Nachrichten ein einheitliches format gehalten\cite{}.
%%pipeline+linting
%Des Weiteren wurde wie in \nameref{subsection:anforderung:nichtFunktionaleAnforderungen} erwähnt ein Linter benutzt um den Quellcode in ein einheitliches Format zu bringen. Damit die Regeln auch durchgängig eingehalten werden und nichts ausversehen auf das Repository gelangt, wurde eine Pipeline zu GitLab hinzugefügt, welches [gepushte] Neuerungen überprüft und ablehnt wenn Regeln vom linter nicht befolgt werden. %
}


\subsection{Entwurfsmuster}%
%Was+Warum:
Während der Implementation wurden sich bewusst für die Verwendung einiger Entwurfsmuster entschieden, da sie in den jeweiligen Situationen nützlich erscheinen. %
%Übersicht:
Folgend werden die Entwurfsmuster aufgezählt und dessen Verwendungsgrund begründet.%
%
%
%
%
%
%
%
%
%
%
\subsubsection{Fassade}
Durch das Umschließen der API mit einem Fassadenobjekt wird die Schnittstelle für den Nutzer der Datenbankverbindung vereinfacht und irrelevante Informationen werden verborgen. Dadurch konnten beispielsweise die erforderlichen Zwischenschritte zum Herunterladen einer Datei von GitHub verborgen werden.
%
%
%
%
%
%
\subsubsection{Singelton}%
%Was: 
Während der Implementierung wurde versucht, nicht auf das Singleton-Pattern zurückzugreifen. 
	%Warum: Vor/Nachteile:
	Obwohl es die Programmierung erleichtert, indem es den Zugriff auf ein Objekt von überall aus ermöglicht, hat es auch Nachteile. %
	Durch Der globale Zugriff auf das Objekt erschwert die Nachvollziehbarkeit, welche Komponenten Zugriff auf das Objekt benötigen, und führt so unter anderem zu einer Verkomplizierung der Fehlerbehebung. %
%Auswirkung:
Trotzdem wurde für die Datenbank das Singleton-Pattern verwendet, da die Datenbank sonst zwischen vielen Komponenten übergeben werden müsste und die Struktur des Programmcodes darunter leiden würde.\newline%
Es folgt die konkrete Situation welche während der Implementierung aufgetreten, als Beispiel. Siehe Abbildung \myTextTodo{Pic Singelton 123} für die grafische Darstellung des Beispiels. Im Grunde benötigen nur drei Seiten Zugriff auf die Datenbank. Allerdings erstellt die Navigation in Flutter ein Seitenobjekt und benötigt daher ebenfalls Zugriff auf die Datenbank. Es gibt zwei Möglichkeiten: Entweder man gibt die Datenbank bis zum Navigationsknopf weiter oder man definiert die Navigationsfunktion bereits auf der Seite und gibt sie bis zum Knopf weiter. Beide Optionen würden jedoch dazu führen, dass viele Zwischenklassen die Datenbank oder Funktion als Parameter übergeben müssten, obwohl sie diese gar nicht benötigen. Das Singleton-Entwurfsmuster löst dieses Problem.
%
%
%
%
%
\subsubsection{Strategie}
%Was+ Warum: Parser
Das Strategie Pattern würde für den Parser verwendet werden, da es in Zukunft geplant ist, dass die App auch weitere CLI-Terminkalender unterstützt. Das Strategie Pattern ermöglicht, dass der When-Parser einfach gegen einen anderen Parser ausgetauscht werden kann. %
%Was: Datenbank
Für die Datenbank wurde ebenfalls das Strategie Pattern verwendet. %
	%Warum: Was nicht
	Die Idee dahinter war nicht nur, die Datenbank gegen eine alternative auszutauschen - obwohl das auch möglich wäre, beispielsweise durch eine GitLab-API anstelle der verwendeten GitHub-API. %
	%Warum: Stattdessen
	Vielmehr ermöglicht das Strategie Pattern auch den einfachen temporären Austausch einer Mock-Datenbank\footnote{Eine Datenbank, welche die Funktionen einer echten simuliert.}. Eine simulierte Datenbank kann sich nämlich als hilfreich beim Testen herausstellen, da sie unabhängig von einer API ist. Entsprechend wurde auch für diese Anwendung eine solche Mock-Datenbank implementiert. %
%Warum
%Außerdem half das Strategy Pattern auch dabei zu verstehen welche Funktionen sichtbar für den Nutzer sein müssen und bei welchen funktionen es sich um interne Funktionen der jeweiligen Datenbank implementation handelt. 
%
%
%
%
%
\subsubsection{Proxy}%
%Warum
Es wird angenommen, dass die Kalender- und Konfigurationsdateien sowie die Issues von GitHub während der Benutzung der App selten extern geändert werden. Deshalb ist es ausreichend, die Daten einmalig in der App zu laden und sie nur bei Bedarf, d.h. bei einer gezielten Anfrage nach neuen Dateien, erneut vom Server abzurufen. Derzeit lädt die App jedoch bei jeder Navigation auf eine Seite die Daten erneut von der Datenbank.\newline%
%Was
Dementsprechend wird für den Parser sowie der Datenbank das Remote Proxy Pattern verwendet. Dabei umschließe ein Proxy-Objekt das eigentliche Datenbank- bzw. Parser-Objekt und speichert die Ergebnisse in einem Cache. Wenn dieselben Daten erneut angefragt werden, werden sie aus dem Cache zurückgegeben. Bei einer gezielten Anfrage auf neue Dateien wird hingegen der Cache geleert und neue Dateien von der Datenbank geladen.
%
%
%
%
%
\subsubsection{Datenflussnetze \& Ablagebasiert}
%Was: Datenflussnetze
Zu Beginn wurde aus Einfachheit für den Parser eine Struktur ähnlich einem Datenflussnetz erstellt. %
	%Was: unflexibel
	Im weiteren Verlauf der Entwicklung stellte sich jedoch heraus, dass diese Struktur zu unflexibel ist. %
		%Warum:
		Denn bei Änderungen an einzelnen Methoden mussten auch die unmittelbar vorherigen und nachfolgenden Methoden angepasst werden. %
	%Was+Warum: unübersichtlich
	Darüber hinaus wurde der Parser durch die lange Verkettung von Methoden immer unübersichtlicher.
%Folgerung: Was:
Deshalb wurde die Struktur des Parsers zu einer ablagebasierten Struktur geändert. %
	%Warum:
	Dadurch wird der Parser einerseits übersichtlicher und andererseits flexibler und änderungsfreundlicher, da die einzelnen Funktionen nicht direkt miteinander kommunizieren. %
%Was: Abbildungen
Die Abbildungen \myTodo und \myTodo zeigen die Schritte des WhenParsers bei der Eingabe einer Datei bis zum erlangten Kalendareintrag in der zuvor genutzten datenflussnetzähnlichen Struktur und in der verbesserten ablagebasierten Struktur.
%
%
%
%
%
%\subsection{(Weitere erwähnenswerte Besonderheiten)}
% !TeX encoding = UTF-8
\section{Evaluation}

%%% hidden subsection for a better structure in latex editor: "texifier"
\myComment{\subsection*{Übersicht}} 
%
%Einleitung
In diesem Abschnitt wird versucht [festzustellen,herrauszufinden], ob die erstellte Anwendung auch das erfüllt was sich zuvor vorgenommen wurde.\newline%
%
%Übersicht
\textbf{Übersicht:} %
Dazu wird eine Anforderungsverifizierung mit den funktionalen und nicht funktionalen Anforderungen aus dem \secref{section:anforderungen} durchgeführt.\newline%
%
%Zusammenfassung
\textbf{Erkenntnisse:}\myTodo %


\myComment{
	
	%Einleitung: Warum
	\textbf{Abschnitte der Arbeit}\\
	Um zuletzt festzustellen ob wir auch wirklich das erschaffen haben, was wir uns zuvor als Ziel setzten, wird die Software mithilfe verschiedener Methoden im \secref{section:evaluation} Validiert.\newline%

}
\input{6_validierung/6_1_appErgebnisse.tex}
\subsection{Anforderungsverifizierung}%
%
%Einleitung
In diesem Abschnitt werden noch einmal alle Anforderungen aus \secref{section:anforderungen} aufgelistet, um sie einzeln mit dem Endprodukt vergleichen und so verifizieren zu können. %
%Funktionalen
Dabei werden die funktionalen Anforderungen darauf überprüft, ob sie vollständig oder teilweise umgesetzt wurden, oder lediglich in der Konzeptionsphase verblieben sind. %
%Nicht funktionalen
Bei vielen der nicht funktionalen Anforderungen ist es hingegen schwieriger zu bewerten, ob sie wie gewünscht umgesetzt wurden. Daher könnten bei diesen Anforderungen lediglich Vermutungen angestellt werden. %
	%Nutzertest
	Es wäre wahrscheinlich besser gewesen, einen Nutzertest durchzuführen, um aussagekräftigere Schlüsse ziehen zu können. Insbesondere hätten die Nützlichkeit und Benutzerfreundlichkeit besser eingeschätzt werden können. Aber auch die Wartbarkeit und Leistung hätten mithilfe einer Änderungsfreundlichkeitsanalyse und Benchmarks besser beurteilt werden können. Leider blieb dafür jedoch keine Zeit mehr übrig.%
\newline%
\myNewSection%
\textbf{Funktionale Anforderungen:}%
\begin{enumerate}%
	\item \textbf{M Verbindung mit Backend:} %
		Vollständig implementiert. Alle benötigten Datenbankfunktionen stehen zur Verfügung.%
		%
	\item \textbf{M (+C) Übersetzer für CLI-Terminkalender:} %
		Teilweise implementiert. Wie in \secref{subsection:imp:parser} erwähnt versteht der Übersetzer derzeit nicht alle Operanden und setzt vorerst Klammersetzung voraus. Des Weiteren werden drei Terminaloptionen\footnote{Die besagten when Terminaloptionen sind: monday\_first, ampm, auto\_pm\cite{cli_when}.}, die \glqq when\grqq{} bietet und die Interpretation von Terminen verändert, noch nicht unterstützt.%
		%
	\item \textbf{M (+C) Kalender Darstellung:} %
		Vollständig implementiert. Alle benötigten Kalenderansichten stehen zur Verfügung. %Weitere könnten jedoch noch hinzugefügt werden.%
		%
	\item \textbf{S Einträge erstellen, bearbeiten, löschen:} %
		%Was
		Teilweise implementiert. Erinnerungseinträge können erstellt, bearbeitet und abgeschlossen werden. Jedoch können noch keine Dateien wie Bilder, Audios oder Videos hinzugefügt werden.\newline
		%Was nicht
		Außerdem wurde sich gegen das Löschen entschieden, da die Abschließen-Funktion bereits eine ähnliche Funktionalität bietet und daher angenommen wird, dass die Darstellung der Löschaktion die App nur unübersichtlicher machen würde.
		%
	\item \textbf{S Einschränkungen:} %
		%Was
		Vollständig implementiert. Die Anwendung hat mehrere Einschränkungen, die darauf abzielen, die Nutzung auf die Stärke des Handys zu lenken, wie etwa die kurzweiligen und einfachen Aufgaben. %
		%Beispiele
		So werden unter anderem bei der Erstellung neuer Erinnerungen die Länge des Titels und der Beschreibung sowie die Anzahl der Dateien begrenzt. Zudem ist auch die Anzahl der in der App darstellbaren Erinnerungen begrenzt.%
		%
	\item \textbf{S Benachrichtigungen:} %
		Vollständig implementiert. Erinnerungen zu Terminen werden automatisch erstellt.%
		%
	\item \textbf{S Konfiguration auf dem Pc:}
		%Was
		Vollständig implementiert. Für die Anwendung existiert eine Konfigurationsdatei welche über den PC angepasst werden kann. %
		%Was nicht
		Jedoch könnte diese noch weiter verbessert werden, indem das Dateiformat auf das Textformat umgewandelt wird und gegebenenfalls weitere Optionen hinzugefügt werden.%
		% 
	\item \textbf{C Suchfunktion:} %
		Nicht implementiert.%
		%
	\item \textbf{C Weitere Kalender Abonnieren \& Teilen:} %
		%Was nicht
		Nicht implementiert. Zurzeit kann nur ein Kalender gleichzeitig Angezeigt werden. %
		%Was schon
		Jedoch sollte es möglich sein, dadurch dass GitHub als Datenbank verwendet wird, diese zum teilen von Terminkalendern zu nutzen.%
		%
	\item \textbf{C Offline Funktionen:} %
		Nicht implementiert.%
		%
	\item \textbf{W Anleitung:} %
		%Was nicht
		Teilweise implementiert. Wie zuvor geplant wurde keine Anleitung implementiert, da es stattdessen besser wäre eine Anwendung zu erstellen, die so intuitiv ist, dass keine Anleitung benötigt wird. %
		%Was schon
		Jedoch würde sich für das erste aufsetzen des Repositories mit Kalendar sowie Konfigurationsdatei eine Anleitung lohnen, da dies ein relatives komplexes und aufwändiges Unterfangen darstellt. Statt jedoch eine Anleitung zu entwerfen, wurde beschlossen, eine Funktion bereitzustellen, welche dem Benutzer diesen Aufwand abnimmt. Daher verfügt die App über eine Auto-Setup-Taste, die automatisch ein Repository, eine Konfigurationsdatei und Beispieleinträge erstellt.%
		%
	\item \textbf{W Commit-History:}
		Nicht implementiert.%
\end{enumerate}%
%
\myNewSection
\textbf{Nicht funktionale Anforderungen:}
\begin{enumerate}
	\item \textbf{M Stärken von PCs und Handys:}\newline%
		%Was:
		Umgesetzt. Da die Anwendung nach den Stärken des PCs und Handys aus \secref{section:pcVsPhone} erstellt wurde. %
			%Hauptaufgaben
			So sind für diese Anwendung die Hauptaufgaben des PCs das Erstellen von Terminen und das Vollenden von Erinnerungen sowie die Konfiguration der Anwendung und des Repositories. Im Gegensatz dazu sind die primären Aufgaben des Handys das Anzeigen von Terminen, Benachrichtigungen und Erinnerungen sowie das Erstellen von Erinnerungen.\newline%
			%Auflistung
			Im Folgenden werden die Stärken des PCs und Handys erneut aufgelistet, um diese so einzeln [im Hinblick] auf die erstellte Anwendung zu bewerten.\newline%
		%PC
		\myNewSection
		Die Aufgaben des PCs:%
  		\begin{enumerate}[label*={\arabic*}]
			\item benötigen viel Leistung.\newline%
				Nicht umgesetzt. Die Anwendung beinhaltet grundsätzlich keine Aufgaben oder Funktionen, die viel Leistung benötigen.%
				%
			\item erfordern schnelle, präzise oder vielfältige Eingaben.\newline%
				Umgesetzt. Die Erstellung von Terminen über das Terminal profitiert von schneller und präziser Eingabe. Außerdem benötigt der CLI-Terminkalender spezielle Symbole, und dementsprechend erfordert diese Aufgabe auch eine vielfältige Eingabe.%
				%
			\item stellen viele Informationen dar oder benötigen viele Informationen.\newline%
				Umgesetzt. Es wird angenommen, dass es beim Erstellen von Termineinträgen durchaus hilfreich sein kann, Informationen aus anderen Quellen zu beziehen. Eine solche Informationsquelle wird beispielsweise durch die Erinnerungen bereitgestellt, die von dieser Anwendung erstellt werden.
				%
			\item bieten viele Optionen und Konfigurationen an.\newline%
				Umgesetzt. Es besteht die Möglichkeit, die Anwendung, das Repository und gegebenenfalls den CLI-Terminkalender zu konfigurieren.%
				%
			\item sind langwierig oder benötigen viel Zeit.\newline%
				Überwiegend umgesetzt. Der Zeitaufwand für das Erstellen und Organisieren von Terminen auf dem Terminal ist nutzerabhängig. Einige Nutzer benötigen möglicherweise viel Zeit, da sie ihre Termine sorgfältig organisieren und komplexe Syntax für neue Termine verwenden, während andere Nutzer dafür nur den minimalen Aufwand aufbringen. Es wird jedoch vermutet, dass die Erstellung und Organisation von Terminen auf dem Terminal eher eine zeitaufwändige Aufgabe ist.%
				%
		\end{enumerate}
		%
		%HANDY
		\myNewSection%
		Die Aufgaben des Handys:%
		\begin{enumerate}[label*={\arabic*}]
		 	%RESSOURCEN
			\item sind ressourcenschonend und benötigen nicht viel Leistung.\newline%
				Überwiegend umgesetzt. Weder das Einsehen von Terminen und Benachrichtigungen noch das Erstellen von Erinnerungen sind leistungsaufwändige Aufgaben. Weiteres siehe: 5. C Leistung.
				%
			%EINGABE
			\item erfordern keine schnelle, präzise oder vielfältige Eingaben.\newline%
				Umgesetzt. Lediglich das Erstellen von Erinnerungen erfordert wiederholte Eingaben und für diese werden alternative Eingabemöglichkeiten angeboten, um den Aufwand dabei zu verringern. 
				%
			%INFORMATIONEN
			\item stellen weder viele Informationen dar noch benötigen sie viele.\newline%
				Umgesetzt. Das [Einsehen/Ansehen] von Terminen und Benachrichtigungen benötigt keine weiteren Informationen. Auch das Erstellen von Erinnerungen sollte hingegen nur wenig weiteren Informationen benötigen, wie zum Beispiel Bilder, Videos, Text, Sprachnachrichten oder andere Dateien auf dem Handy. Darüberhinaus wurde bei der Gestaltung der grafischen Oberfläche darauf geachtet, dass die Anwendung generell übersichtlich ist und keine überflüssigen Informationen enthält.%
				%
			%OPTIONEN
			\item kommen ohne viel Optionen und Konfigurationen aus.\newline%
				Umgesetzt. Für das Handy wurden bereits passende Voreinstellungen getroffen. Für jene Einstellungen, bei denen angenommen wurde, dass Nutzer sie gegebenenfalls selbstständig anpassen möchten, wurden die Option zum Konfigurieren auf den PC verlegt.
				%
			%KURZWEILIG
			\item sind kurzweilig oder benötigen wenig Zeit.\newline%
				Umgesetzt. Einerseits wird eingeschätzt, dass die Aufgaben des Einsehens von Terminen und Benachrichtigungen sowie das Erstellen von Erinnerungen relativ kurzweilig sind. Darüber hinaus wurde der Zeitaufwand für die Erstellung von Erinnerungen durch die Bereitstellung alternativer Eingabemethoden und die Begrenzung der Textlänge und maximalen Anzahl von Erinnerungen reduziert. Andererseits erfordert die Anwendung wenig bis keinen Einstiegsaufwand, da sie schnell geladen wird (siehe Punkt 5. C Performance) und keine initiale Konfiguration erforderlich ist, um die Anwendung anzusehen. Um [jedoch] die Funktionen der Anwendung zu nutzen, sind nur wenige Konfigurationsschritte in Form von drei Feldern erforderlich. Diese wurde durch die Funktion des \glqq Automatischen Setups\grqq{} noch weiter vereinfacht.
				%
			%UNTERWEGS
			\item sind lohnenswert Unterwegs zu lösen.\newline%
				Umgesetzt. Die Funktionen der Anwendung (Anzeigen von Terminen und Benachrichtigungen sowie Erstellen von Erinnerungen) wurden im \secref{section:anforderungen} ausgewählt, da unter anderem davon ausgegangen wurde, dass es sich lohnt, diese Funktionen auch unterwegs nutzen zu können.
				%%%
			%INTUITIV
			\item sind einfach und intuitiv.\newline%
				Überwiegend umgesetzt. Siehe: 2. M Benutzbarkeit. 
				%
		\end{enumerate}
	
	\item \textbf{M Benutzbarkeit:}\newline%
		%Was
		Überwiegend umgesetzt. Es wird angenommen, dass die App einfach, intuitiv und effektiv zu nutzen ist, da sich beim Design an die Regeln und Erkenntnisse der von Apple gegebenen Richtlinien gehalten wurde.% und andererseits  sich bei den Anforderungen an die Stärken des Handys und Pc gehalten wurde.%
		%Was nicht
		\newline%
		Jedoch gibt es beim Design noch einige wenige Entscheidungen, wie zum Beispiel die Navigationsleisten, welche noch ähnlicher zur iOS-Plattform aussehen könnten und daher verbesserungswürdig sind.%
		%
	\item \textbf{S Wartbarkeit, Erweiterbarkeit, Verständlichkeit:}\newline%
		%Was
		Teilweise umgesetzt. Wie im \secref{section:implementierung} erwähnt, sind ausreichend Rückfalltests und Dokumentation für die grundlegenden Komponenten vorhanden. Weiter steht eine Mocked-Datenbank zur Verfügung, mit der die Funktionalität unabhängig von der GitHub-API getestet werden kann. Darüber hinaus wurde versucht, möglichst so modular zu programmieren und an geeigneten Stellen wurden Entwurfsmuster verwendet. All dies sollte sich positiv auf die Wartbarkeit, Erweiterbarkeit und Verständlichkeit auswirken.%
		%Was nicht
		Jedoch konnte dieser Qualitätsstandard nicht auf den gesamten Programmcode angewendet werden. Denn zum Ende der Bearbeitungszeit wurde der der Schwerpunkt eher darauf gelegt, alle erforderlichen Funktionen zu in die Benutzeroberfläche zu integrieren, um so ein möglichst validierbares Produkt zu erhalten. Dadurch litt jedoch die Modularität und Dokumentation des Programmcodes etwas.%
		%
	\item \textbf{S Qualität \& Korrektheit:}\newline%
		Umgesetzt. Es wird mit großer Sicherheit vermutet, dass die Anwendung sich so verhält, wie zuvor spezifiziert wurde. Dies liegt daran, dass es, wie im \secref{section:qualitaetssicherung} erwähnt, für die Grundkomponenten der Anwendung reichlich Rückfalltests gibt und die grafische Oberfläche ausgiebig manuell getestet wurde.%
		%
	\item \textbf{C Leistung:}\newline%
		Überwiegend umgesetzt: %
		%Flüssigkeit/CPu
		So lief die App während des Tests flüssig und es konnte keine hohe CPU-Auslastung festgestellt werden. %
		%Größe + Neztwerkauslastung & Ladezeit (Proxy, Nachlade Kalender)
		Außerdem ist die App mit rund 73 MB relativ klein, und durch Funktionen wie den dynamisch nachladenden Kalender und den Proxy wurden Ladezeiten sowie Netzwerkauslastung reduziert. %
		%Jedoch: CLI Parser Dateigröße
		Die Ladezeit der Terminkalenderseite hängt jedoch von der Bearbeitungszeit des Parsers ab. Dieser ist wiederum abhängig vom Inhalt der CLI-Terminkalenderdatei und folglich auch vom Nutzer. Aus diesem Grund sollte der Parser zunächst einem Last-, Stress- und Leistungstest unterzogen werden, bevor die Ladezeit endgültig eingeschätzt werden kann. Es wird jedoch [erwartet/geschätzt], dass der Rechenaufwand des Parsers vernachlässigbar ist.%
		%
	\item \textbf{C Reichweite:}\newline%
		Teilweise umgesetzt: %
		%OS Version + Performance
		Die App sollte einerseits auf recht alten Handys nutzbar sein, da sie verhältnismäßig alte Betriebssystemversionen wie iOS 11.0 (von 2017) und Android 5.0 (von 2014) unterstützt. Außerdem benötigt sie, wie zuvor erwähnt, relativ wenig Leistung. %
		%Flutter/Vorerst nur Android
		Jedoch lässt sich die Anwendung vorerst trotz der Verwendung des Cross-Frameworks Flutter nur auf iOS installieren. Aufgrund von zeitlichen Beschränkungen wurde sich zunächst eher auf eine Plattform konzentriert, in diesem Fall iOS. Dabei wurden einige Darstellungskomponenten verwendet die inkompatibel mit Android sind. Jedoch wird der Aufwand, die Anwendung auch für Android zu ermöglichen, als relativ gering eingeschätzt. Es müssen lediglich die Komponenten durch eine Abfrage gegen die Android-Kompliment ersetzt werden.%
\end{enumerate}
\input{6_validierung/6_3_usertests.tex}


% !TeX encoding = UTF-8
\section{Fazit \& Ausblick}

\input{7_fazitUndAusblick/stichpunkte7.tex} %todo: remove after this sections completion

%%% hidden subsection for a better structure in latex editor: "texifier"
\myComment{\subsection*{Übersicht}} 
\input{7_fazitUndAusblick/7_1_fazit.tex}
\subsection{Ausblick}
\cite{test1} \cite{test2} \cite{test3}
%\include{9_latex-beispiele} %TODO remove later
\printbibliography

\appendix
% !TeX encoding = UTF-8
\section{Anhang}

%\myNewSection
%\begin{figure}[h]
%    \centering
%    \includegraphics[width=0.7\textwidth]{res/IMG_A9A2347E0F7D-1.jpeg} 
%    \caption{iOS Schieberegler} 
%    \label{pic:schieberegler}
%\end{figure}
%\myNewSection
%\textbf{Stichpunkte:} 
%Ermüdende Informationsberge sollten in einen Anhang verbannt werden. (Wichtige Begriffe definieren)

%Beispiel: passende Lösung für dieses Problem zu finden
%\myNewSection \label{anhang:einleitung:passendeLösung}
%Um diese Idee \dq Stärken des Pc's sowie des Handys zu nutzen\dq noch etwas verständlicher zu machen, folgt eine kurzes Beispiel, wie diese Idee zu einem erfolgreichen Produkt geführt hat.
%Der erste Ipod\cite{einleitung_ipod} wurde 2001 vorgestellt und als \glqq tragbarer digitaler Medienabspielgeräte\grqq{} entworfen. Der Kleinheit und Portabilität zugunsten wurden nur die für das Gerät passenden und nötigen Funktionen implementiert. So zum Beispiel das Abspielen und Auswählen von Musik. Funktionen welche auf solch einem kleinen Gerät nicht gut funktionieren, wurden auf den Pc verlagert. So war der Pc für eine seiner Stärken zuständig, und zwar der Konfiguration, also dem Hinzufügen, Bearbeiten und Löschen von Musik. 

Link zum Repository: \url{https://git.imp.fu-berlin.de/thob97/bachelor-cli_calendar_app}
\newline
Commit-Hath: ab2bf74a810a521c61af575b9af284b2dc07eca0

\newpage
\myNewSection
%Appleiste
\begin{figure}
    \centering
    \includegraphics[width=0.5\textwidth]{res/appbar_old.png} 
    \caption{Erstes Appleisten Design} 
    \label{pic:appleiste_old}
\end{figure}
\begin{figure}
    \centering
    \includegraphics[width=0.5\textwidth]{res/appbar_new_home.png} 
    \caption{Aktuelles Appleisten Design} 
    \label{pic:appleiste_new}
\end{figure}

%Calendar
\begin{figure}
\centering
\begin{minipage}{.5\textwidth}
  \centering
  \includegraphics[width=0.8\linewidth]{res/calendar_view1.png}
  \captionof{figure}{Monatsansicht}
  \label{fig:calendar_view1}
\end{minipage}%
\begin{minipage}{.5\textwidth}
  \centering
  \includegraphics[width=0.8\linewidth]{res/calendar_view2.png}
  \captionof{figure}{Tagesansicht}
  \label{fig:calendar_view2}
\end{minipage}%
\end{figure}

%Reminder Page
\begin{figure}
    \centering
    \includegraphics[width=0.5\textwidth]{res/reminders_page.png} 
    \caption{Erinnerungsseite mit Beispiel Erinnerungen} 
    \label{pic:reminders_page}
\end{figure}


%Settings
\begin{figure}
\centering
\begin{minipage}{.5\textwidth}
  \centering
  \includegraphics[width=0.9\linewidth]{res/settings_page.png}
  \captionof{figure}{Loginseite}
  \label{fig:settings_page}
\end{minipage}%
\begin{minipage}{.5\textwidth}
  \centering
  \includegraphics[width=0.9\linewidth]{res/settings_page_alert.png}
  \captionof{figure}{AutoSetup Alert}
  \label{fig:settings_page_alert}
\end{minipage}%
\end{figure}

%tastatur
\begin{figure}
    \centering
    \includegraphics[width=0.6\textwidth]{res/tastatur.png} 
    \caption{Systemtastatur mit Icons} 
    \label{pic:tastatur}
\end{figure}

%WhenDate
\begin{figure}
    \centering
    \includegraphics[width=0.9\textwidth]{res/when_termin.png} 
    \caption{Modellierter WhenTermin} 
    \label{pic:when_termin}
\end{figure}

%Zeitspanne
\begin{figure}
    \centering
    \includegraphics[width=0.6\textwidth]{res/zeitspanne.png} 
    \caption{Beispiel Darstellung einer Zeitangabe} 
    \label{pic:zeitspanne}
\end{figure}

%Fassade
\begin{figure}
    \centering
    \includegraphics[width=0.9\textwidth]{res/fassade_database_connection.png} 
    \caption{Vereinfachte Darstellung der Fassade} 
    \label{pic:fassade_database_connection}
\end{figure}

%Singelton
\begin{figure}
    \centering
    \includegraphics[width=1.1\textwidth]{res/global_database.png} 
    \caption{Globale Datenbank durch Singelton} 
    \label{pic:global_database}
\end{figure}
\begin{figure}
  \centering
  \includegraphics[width=1.1\linewidth]{res/weitergegebene_datasabe.png}
  \captionof{figure}{Weitergereichte Datenbank}
  \label{fig:weitergegebene_datasabe}
\end{figure}
\begin{figure}
  \centering
  \includegraphics[width=1.1\linewidth]{res/hochgeschobene_nav_funktion.png}
  \captionof{figure}{Nach oben geschobene Navigationsfunktionen}
  \label{fig:hochgeschobene_nav_funktion}
\end{figure}


%Strategie
\begin{figure}
  \centering
  \includegraphics[width=1.1\linewidth]{res/strategie_parser.png}
  \captionof{figure}{Beispiel der Strategiedarstellung für den Parser}
  \label{fig:strategie_parser}
\end{figure}
\begin{figure}
  \centering
  \includegraphics[width=1.1\linewidth]{res/strategie_database.png}
  \captionof{figure}{Beispiel der Strategiedarstellung für die Datenbank}
  \label{fig:strategie_database}
\end{figure}

%Proxy
\begin{figure}
    \centering
    \includegraphics[width=0.8\textwidth]{res/proxy.png} 
    \caption{Vereinfachte Darstellung der Proxydatenbank} 
    \label{pic:proxy}
\end{figure}

%AblageBasiert
\begin{figure}
    \centering
    \includegraphics[width=1.2\textwidth]{res/datenfluss.png} 
    \caption{Ablauf mit der datenflussnetz ähnlichen Struktur} 
    \label{pic:datenfluss}
\end{figure}

\begin{figure}
  \centering
  \includegraphics[width=1.2\linewidth]{res/Ablage-basiert1.png}
  \captionof{figure}{Ablauf mit der ablagebasierten ähnlichen Struktur Teil 1}
  \label{fig:ablage_basiert1}
\end{figure}
\begin{figure}
  \centering
  \includegraphics[width=1.2\linewidth]{res/Ablage-basiert2.png}
  \captionof{figure}{Ablauf mit der ablagebasierten ähnlichen Struktur Teil 2}
  \label{fig:ablage_basiert2}
\end{figure}
%



%reminders views comparison

\begin{figure}
    \centering
    \includegraphics[width=.5\textwidth]{res/reminder_handy.png} 
    \caption{Handyansicht der Erinnerungen} 
    \label{pic:reminder_handy}
\end{figure}

\begin{figure}
    \centering
    \includegraphics[width=1.1\textwidth]{res/reminders_webview.png} 
    \caption{Webseitenansicht der Erinnerungen} 
    \label{pic_webview}
\end{figure}

\begin{figure}
    \centering
    \includegraphics[width=1\textwidth]{res/reminder_cli.png} 
    \caption{CLI-Ansicht der Erinnerungen} 
    \label{pic_reminder_cli}
\end{figure}


\end{document}
