% !TeX encoding = UTF-8

\documentclass[serif,article,noparskip,de]{config-agse-thesis}

%Todo: Maybe Titel anpassen:
%Entwicklung einer Kalender-App für Geeks: Ein Versuch die Lücke zwischen Terminal und Smartphone zu überbrücken
\newcommand{\thesisTitle}{Entwicklung einer Kalender-App für Geeks: Ein Versuch wie man die Lücke zwischen Terminal und Smartphone überbrücken könnte}
% -> You may use \par (but not \\) to format the title. If you do so, you'll

\newcommand{\studentName}{Thore Brehmer}
%===============================================================================

\hypersetup{pdftitle={\thesisTitle}}
\hypersetup{pdfauthor={\studentName}}

\addbibresource{quellen.bib}

%%% multiline comment
\newcommand{\myComment}[1]{}
\newcommand{\myNewSection}[0]{\ \\}

%TODO Remove later after refactoring demonstration texts
\usepackage{lipsum}

%TODO remove after removing all stichpunkte
%Für farbige Checkmarks
\usepackage{tikz}
\newcommand{\myCheckmark}{}%
\DeclareRobustCommand{\myCheckmark}{%
  \tikz\fill[scale=0.7, color=red]
  (0,.35) -- (.25,0) -- (1,.7) -- (.25,.15) -- cycle;%
}

%für farbige rote circles
\newcommand{\myTodo}{}%
\DeclareRobustCommand{\myTodo}{
	\tikz\draw[red,fill=red] (0,0) circle (.5ex);
}

%für roten Text
\newcommand{\myTextTodo}[1]{\textcolor{red}{#1}}%

%für nameref
\usepackage{hyperref}
\newcommand{\secref}[1]{\autoref{#1}. \nameref{#1}}

\begin{document}

\coverpage[
    student/id=5216879,
    student/mail=thob97@zedat.fu-berlin.de,
    thesis/type=Bachelorarbeit,            % optional, default: Bachelorarbeit
    thesis/group={}, % TODO
                                           % optional, default: AGSE
    thesis/advisor={Prof. Dr.-Ing. Volker Roth},           % optional TODO
    thesis/examiner={Prof. Dr.-Ing. Volker Roth},  % TODO
    %thesis/examiner/2={}, % optional TODO
    thesis/date=\today,                    % optional, default: \today
   %title/size=\LARGE,      % set this value to overwrite automatic font size
   %abstract/separate       % toggle this to move the abstract to its own page
]
{ % Your abstract/Zusammenfassung here:
    %%% hidden subsection for a better structure in latex editor: "texifier"
\myComment{\section*{Zusammenfassung}} 

\input{stichpunkte0.tex} %todo: remove after this sections completion

\myNewSection
\textbf{WisschenschaftlichesSchreiben:}
\\ Zusammenfassung des Inhalts
\\ - wird am häufigsten gelesen
\\ - max länge vorgegeben (150-300 Wörter)
}

% !TeX encoding = UTF-8
\subsection*{Eidesstattliche Erklärung}

Ich versichere hiermit an Eides Statt, dass diese Arbeit von niemand anderem
als meiner Person verfasst worden ist. Alle verwendeten Hilfsmittel wie
Berichte, Bücher, Internetseiten oder ähnliches sind im Literaturverzeichnis
angegeben, Zitate aus fremden Arbeiten sind als solche kenntlich gemacht. Die
Arbeit wurde bisher in gleicher oder ähnlicher Form keiner anderen
Prüfungskommission vorgelegt und auch nicht veröffentlicht.\\

\thesisDate \\

\studentName


\cleardoublepage

\tableofcontents

\cleardoublepage

\pagestyle{fancy}

% Actual content starts here

% !TeX encoding = UTF-8
\section{Einleitung}

\myComment{\subsection*{Stichpunkte1}} 

\myComment{

	\myNewSection
	\textbf{Website:}
	\\Was ist das Problem? Warum ist es ein Problem? Wie bettet es sich in andere Arbeiten ein? Was ist nicht das Problem? Was wird nicht gelöst mit dieser Arbeit? \myTodo

	\myNewSection 
	\textbf{My Notes (Also Website)}: \myTodo
	\\ <Hauptteil> Gewählter Lösungsansatz, Alternativen, Abwägungen
	\\<Hauptteil> Beschreibung besonderer Schwierigkeiten und wie sie gelöst, umgangen oder vermieden wurden (oder warum nicht)
	\\<Hauptteil> Dokumentation der Durchführung und der entstandenen Artefakte
	\\ Alle Behauptungen müssen belegt werden, sei es mit einer Literaturquelle, einem sorgfältigen Argument oder mit eigenen empirischen Daten.

	\myNewSection
	\textbf{Purpose}: Dient dem Autor zur Orientierung, aber findet sich normalerweise später in der Einleitung des Dokumentes wieder.
	\begin{itemize}
		\item 1. Eine Beschreibung des größeren Zusammenhangs, in dem das Dokument angesiedelt ist.
		\item 2. Die Beschreibung des konkreten Problems, das im Dokument behandelt wird. \myCheckmark
		\item 3. Die Charakterisierung der Ziele, die das Dokument erreichen soll (z.B. der Information, die es liefern soll). \myCheckmark
		\item 4. Eine Begründung, warum und für wen das Dokument wichtig ist. \myCheckmark
		\item 5. Die Charakterisierung des Vorwissens der angepeilten Leserschaft. \myCheckmark
		\item 6. Eine Auflistung \underline{relevanter Randbedingungen: Zeitbeschränkungen}, Umfangsbeschränkungen, technische Randbedingungen (Medien etc.), äußere Vorgaben (Standards) für Stil, Organisation oder Format. \myCheckmark
		\item Einführung: \underline{Was ist das Problem? Warum ist es ein Problem?} Wie bettet es sich in andere Arbeiten ein? Was ist nicht das Problem? \underline{Was wird nicht gelöst mit dieser Arbeit?} \myCheckmark
	\end{itemize}



	\myNewSection
	\textbf{Prezi:}
	\begin{enumerate}
		\item Zuerst erzähle ich also etwas über die Motivation bzw. der Aufgabenstellung der Arbeit. Also warum mir das Thema relevant erscheint [und was es leisten kann]. \myCheckmark
		\item Bei der Vorgehensweise erkläre ich mit welcher Grundlegenden Herangehensweise ich versuche das Produkt zu erstellen welches die genannten Erwartungen und Ziele aus der Motivation erfüllt. \myCheckmark
		\item Wie gerade erwähnt wird ein agile Arbeitsstil befolgt -> die \underline{Dargestellten Schritte} lassen sich also nicht wie eigentlich abgebildet voneinander trennen und nacheinander lösen, sondern es herrscht ein fließender Übergang während der Bachelorarbeit.
		\\ \underline{Als Beispiel dazu}… Während der Implementation kommt es bestimmt zu Anforderungensveränderungen. Einfach weil man währenddessen neue Ideen hat oder weil einem klar wird, dass die Anforderungen derzeit nicht umsetzbar oder zu Aufwändig sind.
		\\ Jedoch macht es auch Sinn die \underline{Schritte getrennt zu betrachten}. Undzwar jetzt für den Vortrag sowie später in meiner Ausarbeitung. Damit lässt sich nämlich die Komplexität senken. bzw: es macht die Arbeit etwas Verständlicher \myCheckmark
		\item Obwohl ich mir hierzu also schon ein Paar Gedanken gemacht habe. Würde ich diesen Abschnitt, trotzdem während der Arbeit, nochmal detailierter bearbeiten wollen. Einfach weil ich glaube, dass es nützlich und interessant sein könnet. Aber trotzdem versuche ich auch während der Arbeit ständig den Prozess weiter zu verbessern und zu Überdenken. \myCheckmark
	\end{enumerate}
	
	
	
	\textbf{Prof:} Generell bei jeder Übersicht: (Was in diesem Kapitel gemacht wird) + Entscheidungen +-> Ergebnisse dieses Kapitels. \myCheckmark
	
	
	
	\myNewSection
	\textbf{Motivation:} \myCheckmark
	\begin{enumerate}
		\item Für die Motivation ist es vorerst wichtig zu Wissen was \textbf{CLI-Terminkalender} eigentlich sind. Einfach erklärt sind das Terminkalender ohne Grafische Oberfläche. Die Interaktion mit dem Kalender erfolgt stattdessen über das Terminal.
		\item Ein \textbf{Vorteil} von solchen Kalendern ist zum Beispiel: dass man sich nicht an der Vorgegebenen Grafischen Oberfläche anpassen muss. Bei normalen Kalendern mit Grafischer Oberfläche hätte man bei unerwünschte Funktionen oder Design Entscheidungen, nämlich einfach Pech gehabt. Bei den CLI-Kalendern hat man hingegen sehr viele Freiheiten zur Konfiguration und kann sich dadurch selbst sein gewünschtes Umfeld einstellen. 
		\item \textbf{Deswegen} gibt es auch genügend Benutzer welche sich auf den Terminal und damit auch auf CLI-Kalendern wohler fühlen. 
		\item \textbf{Beispiele} für solche Programme wären “when, remind, khal, calcurse, calendar”.
		\item Nun zum \textbf{Problem bzw. der Motivation}: Es gibt derzeitig keine App welche sich zufriedenstellend mit CLI-Kalendern Verbinden lässt. Denn am Wünschenswertesten wäre eine Lösung, welche die Vorzüge des PCs und die Vorzüge der Handys ausschöpft.
		\item Ich meinte ja grade, dass versucht werden sollte die Stärken der jeweiligen Systeme möglichst effektiv zu nutzen, damit sich die beiden Welten Verbinden lassen. Eine Stärke des PC’s ist es zum \textbf{Beispiel}, dass sie eine Tastatur besitzt und sich dadurch schnell und bequem tippen lässt. Ein Handy besitzt zwar auch eine Software Tastatur, diese ist aber viel kleiner und Sondersymbole sind hinter mehreren Layers versteckt. Daher sollte einen die Erkenntnis kommen, dass sich auf Handys weniger gut tippen lässt. Die aller einfachste Lösung dieses Themas und der Arbeit, einen CLI-Terminkalender also einfach 1 zu 1 auf ein Handy zu Portieren scheint also nicht als sinnvoll.
		\item Ein \textbf{Beispiel} wie solch ein Problem gut gelöst wurde, ist der Ipod Nano. Es wurden nämlich nur die benötigten und für das Gerät passende Funktionen implementiert. Das wären zum Beispiel das Abspielen von Musik. Die Restlichen Funktionen konnten nur mithilfe des Pc’s erledigt werde. Zum Beispiel das hinzufügen oder Löschen von Musik.
		\item Mit der Erkenntniss dass Funktionen die gut auf dem PC funktionieren nicht gut auf dem Handy funktionieren müssen 
		\item Bsp. wie man es nicht macht: ssh, Terminal App -> tippen klobig weil benötigte Sonderzeichen nicht leicht erreichbar und Tastatur sehr klein
		\item Und \textbf{noch ein Grund} warum es solch eine App benötigt, wäre, dass es sicherlich nochmehr Leute gibt, welche sich auch auf dem Terminal wohlfühlen, aber trotzdem gerne die Vorzüge einen Handys nutzen würden (Bsp. Mobilität)
		\item \textbf{Das Ziel} ist es also, sich zu Überlegen, wie die beiden Welten miteinander verbunden werden können, und dabei die passende App zu erstellen.
	\end{enumerate}
	
	
	
	\myNewSection
	\textbf{Allgemeine Vorgehensweise} \myCheckmark
	\begin{enumerate}
		\item \textbf{Generell} gilt… also für alle Folgenden Schritte dieses Vortrags, dass während der Bachelorarbeit immer wieder Entscheidungen getroffen werden. Und jede dieser Entscheidungen sollte sorgfältig gewählt und gut Begründet sein.
		\begin{enumerate}
			\item Grund dafür ist erstens, dass es einerseits Interessant für den Leser und andererseits wichtig für die Evaluation ist: den Gedankengang hinter den getroffenen Entscheidungen nachvollziehen zu können.
			\item Und zweitens, weil diese Fragen einen selbst dabei helfen… eine bessere oder sogar die beste Entscheidung, aus vielen Möglichkeiten, zu finden.
		\end{enumerate}
		
		\item Für den \textbf{Prozess} habe ich mir eine Agile-Arbeitsweise vorgenommen. Und zwar unteranderem wegen den Folgenden drei, eng miteinander verbundenen, Prinzipien:
		\begin{enumerate}
			\item Das erste ist dabei die Arbeit in \textbf{Iterationen}. Denn die Schritte zur bau einer Software lassen sich nicht nacheinander und in einem Rutsch abarbeiten. Daher werde ich mich stattdessen in mehreren kleinen Schritten zum Ziel herantasten. Vorteile davon sind, dass man steht's ein Evaluierbares Produkt hat… und dies ist natürlich sehr nützlich für die Abgabe. Außerdem hilft Iteration auch bei Anforderungsunsicherheiten. Denn ich gehe davon aus, dass sich getroffene Entscheidungen während der Arbeit / bzw. der Implementierung noch ändern könnten, obwohl man zuvor dachte, dass diese Entscheidung die richtige Wahl wäre.
			\item Das zweite Prinzip ist der Kurzer \textbf{Planungshorizont}: Anfangs würde ich nämlich einmal Oberflächlich einige Allgemeine Ziele raussuchen. Danach gilt immer nur für eine oder zwei Iteration im Vorraus im Detail zu Planen. Das hat den Vorteil das Änderungen weniger verheerend sind, da die Planung leicht geändert werden kann und nicht zu viel Zeit Anfangs darin gesteckt wurde.
			\item ODER: Ein kurzer \textbf{Planungshorizont}: würde dabei heflen, dass Änderungen weniger verheerend sind. Undzwar weil nicht zu viel Zeit in die Planung gesteckt wird.
			\item Das letzte Prinzip sind die \textbf{Retrospektiven}. Dabei würde ich mich wöchentlich, also nach je einer Iteration, Fragen was gut und was schlecht lief. Diese Reflexion kann nämlich dabei helfen den Prozess zu Verbessern und Ursache von Mängeln zu beseitigen (konstruktive Qualitätssicherung) (ständige wird sich Schrittweise verbessert)
		\end{enumerate}
	\end{enumerate}
	
	
}%%% %todo: remove after this sections completion

%%% hidden subsection for a better structure in latex editor: "texifier"
\myComment{\subsection*{Übersicht}}

%Einleitung \ Überblick
\textbf{Überblick:}
Dieser Abschnitt behandelt die kronkrete Beschreibung des Problems sowie die Charakterisierung des Ziels. Dabei soll unter anderem auch der Nutzen der Arbeit deutlich werden.\newline
Zusätzlich wird die Struktur des Dokuments erläutert und das erforderliche Vorwissen der angepeilten Leserschaft genannt. Dadurch soll verdeutlicht werden, ob und in welcher Weise die Ausarbeitung gelesen werden kann.\newline
%Result
\textbf{Ergebnisse:}
Für diese Ausarbeitung werden allgemeine Informatikkenntnisse vorausgesetzt. 
Zudem enthält jeder Abschnitt dieser Arbeit eine Übersicht und Zusammenfassung, um das Lesen erleichtern und Querlesen zu ermöglichen.\newline
Während der Beschreibung des Problems wurden mehrere Erkenntnisse gewonnen. Zum einen erfreuen sich Handys Relevanz und Beliebtheit und dementsprechend lohnt es sich Anwendungen auch als App anzubieten. Zum anderen besitzen Handys und Pc's verschiedene Stärken und Schwächen und CLI-Terminkalender sind diesbezüglich eine Interessante Anwendung.
%Warum: Um die Vorraussetzungen für den Leser bewusst zu machen
\subsection{Voraussetzungen}
%Vorraussetzungen für den Leser
Im Rahmen dieser Ausarbeitung werden wiederholt Begriffe aus dem Jargon der Informatik genutzt. Auf jene Begriffe, die mit einem Informatikstudium als informatisches Allgemeinwissen bezeichnet werden, wird nicht näher eingegangen. Das bedeutet, dass ein gewisses informatisches Vorwissen seitens des Lesers vorausgesetzt wird%
\footnote{Beispiele Begriffe zur Orientierung:
	\begin{itemize}[noitemsep,topsep=0pt,parsep=0pt,partopsep=0pt]
		\item wird Vorausgesetzt: Framework, Package, API (Application Programming Interface)
	\end{itemize}
	\nointerlineskip %removes space between itemize and next footnote
}. \newline%
%Begründung
Dies hat den Grund, ein Abschweifen während der Ausarbeitung und damit verbundene Störungen des Leseflusses zu verhindern.%

%%%Alternativ:%%%
%\myComment{
%
%%Begründung
%Das hat den Grund, dass die Zielgruppe dieses Dokumentes genau diese Vorwissen bereits besitzt. Darüberhinaus wird die Leserlichkeit verbessert, wenn der Lesefluss nicht ständig durch Definitionen unterbrochen wird. \newline%
%%Was stattdessen gemacht wird - Wichtige Begriffe definiert
%Nicht allgemeine Begriffe werden hingegen erläutert. Auch wird es Beispiele geben um das Verständnis weiter zu vertiefen.\newline%
%Für die Arbeit eher unwichtige Begriffe, sowie nebensächliche Beispiele werden aber meistens nur im Anhang oder der Fußnote erwähnt.
%
%}
\subsection{Struktur}\label{kapitel_struktur}
%
%ÜBERSICHT
%\myComment{
%%Überblick + Warum: Um zu erklären wie die Ausarbeitung gelesen werden kann
%Im folgenden wird kurz die Struktur dieses Dokumentes erläutert. Dadurch soll bewusst werden, auf welche Arten die Ausarbeitung gelesen werden kann.
%}
%
%TRIMMED
%\myComment{
%%Was wurde gemacht: Ausarbeitung Abschnitte Modularisiert
%\myNewSection%
%Die Abschnitte dieser Ausarbeit wurden so gewählt, wie man sich den Ablauf eines 'naiven' Software Projektes vorstellet \footnote{Für eine Beispieldarstellung siehe Abbildung [\ref{fig:wasserfallmodell}]}. Diese Abschnitte lassen sich während eines Softwareprojektes und damit auch während dieser Bachelorarbeit eigentlich so gut wie nie getrennt voneinander betrachten und lösen, sondern es herrschte steht's ein fließender Übergang. Trotzdem wurden genau das Versucht in der Ausarbeitung so darzustellen. Es wurde also versucht die Abschnitte und auch alle ihrer Unterpunkte möglichst gut getrennt voneinander zu betrachten und gewissermaßen so zu modularisieren.\newline%
%%Warum: Überschaubar und Querlesen
%Durch die Aufteilung in verschiedene Kapitel wirkt die Bachelorarbeit nämlich mehr überschaubar und auch mehr schaffbar. Nicht umsonst ist eins der wichtigsten Ziele in der Softwaretechnik Projekte in verdauliche Happen aufzuteilen. Des Weiteren ermöglicht die Modularisierung der Abschnitte das \dq Querlesen\dq. Damit ist gemeint, dass Textabschnitte auch ohne das Lesen des kompletten Dokumentes verstanden werden können. 
%}
%
%
%Was + Warum: Überblick für jeden Abschnitt
Jeder Abschnitt enthält eine Einleitung, in der der Inhalt und dessen Bedeutung für das Projekt erläutert werden sowie die verwendete Methodik beschrieben wird.\newline%
%Was + Warum: Zusammenfassung für jeden Abschnitt -> Querlesen
Darüber hinaus gibt es am Ende jedes Abschnitts eine kurze Zusammenfassung, die wichtige Entscheidungen und Ergebnisse des Kapitels zusammenfasst. Das soll dem Leser die Möglichkeit geben, Abschnitte schnell zu überfliegen und trotzdem die wichtigsten Aspekte und Ergebnisse der Arbeit zu verstehen. Zusammenfassend soll diese Strukturierung beim Querlesen helfen.
\subsection{Motivation}

\subsection{Zielsetzung}\label{section:zielsetzung} \myCheckmark

%Wiederholung der wichtigsten Punkte aus dem Letzten Kapitel
Im letzten Abschnitt haben wir folgende Erkenntnisse gewonnen.
\begin{enumerate}
	\item Handys erfreuen sich großer Relevanz und Beliebtheit
	\item Dementsprechend lohnt es sich Anwendungen auch als Apps anzubieten
	\item Handys und Pc's müssen verschiedene Stärken und Schwächen besitzen
	\item CLI-Terminkalender sind in dieser Hinsicht eine Interessante Anwendung
\end{enumerate}

%Nennung des Zieles
\myNewSection
Ziel ist es daher eine passende App für die CLI-Terminkalender zu entwickeln. Passend bedeutet in diesem Sinn, das die Stärken des Pc's und Handys beachtet werden und so mit in die zu erstellende Anwendung einfließen.\newline%
Oder anders formuliert ist das Ziel die \glqq Entwicklung einer Kalender-App für Geeks: Ein Versuch wie man die Lücke zwischen Terminal und Smartphone überbrücken könnte\grqq{}.\newline%
Die simpelste Lösung, einen CLI-Terminkalender auf das Handy zu portieren, ist also nicht [zielführend/zufriedenstellend]. Stattdessen wird versucht eine passende Lösung für dieses Problem zu finden\footnote{Beispiel dazu im Anhang \ref{anhang:einleitung:passendeLösung}}.

%Abgrenzung: Versuch anstatt Lösung + begrenzte Bearbeitungszeit
\myNewSection
Sehr wichtig nochmal zu betonen ist, dass es sich lediglich um einen Versuch handelt. Die in der Arbeit ermittelten Informationen und Ergebnisse sollen weder als allgemeines Beispiel noch Lösung dienen.\newline%
Da die Bearbeitungszeit während einer Bachelorarbeit auch äußerst beschränkt ist, soll das Ziel auch nicht sein ein fertiges Produkt zu liefern, sondern eher sich schrittweise dem fertigen Produkt zu nähern.


%Todo - Old Version - Remove
\myComment{
	%Zur Struktur
	Da nicht nur irgendeine App erstellt werden soll sondern eine möglichst sinnvolle \footnote{funktionale Anforderungen} und hochwertige\footnote{nichtfunktionale Anforderungen}. Wie Versucht wird die beiden Forderungen sicherzustellen kann im Kapitel \myTodo Anforderungen betrachtet werden.
	
	Ziel ist es eine Kalender-App zu erstellen welche die Stärken des Pcs und des Handys ausschöpfen. Oder anders formuliert ist das Ziel die Entwicklung einer Kalender-App "für Geeks", mit dem Versuch die Lücke zwischen Terminal und Smartphone zu überbrücken. Ziel ist es also offensichtlich nicht einen CLI-Kalender auf das Handy zu portieren, sondern eher der Versuch eine passend Lösung für dieses Problem zu finden. \newline
	Natürlich soll nicht nur irgendeine App erstellt werden sondern eine möglichst sinnvolle (funktionale Anforderungen) und hochwertige (nichtfunktionale Anforderungen). Wie Versucht wird die beiden Forderungen sicherzustellen kann im Kapitel \myTodo Anforderungen betrachtet werden.

}
%\subsection{Allgemeine Vorgehensweise}

%%% hidden subsection for a better structure in latex editor: "texifier"
\myComment{\subsubsection*{Übersicht}} 

\subsubsection{Abschnitte der Arbeit}

%Allgemein Arbeitsweise: Herangehensweise an die Arbeit
\subsubsection{Allgemein Arbeitsweise} \myTodo
Generell gilt, also für alle Folgenden Schritte dieses Vortrags, dass während der Bachelorarbeit immer wieder Entscheidungen getroffen werden. Und jede dieser Entscheidungen sollte sorgfältig gewählt und gut Begründet sein. \newline
Grund dafür ist erstens, dass es einerseits Interessant für den Leser und andererseits wichtig für die Evaluation ist, den Gedankengang hinter den getroffenen Entscheidungen nachvollziehen zu können. \newline
Und zweitens, weil diese Fragen einen selbst dabei helfe eine bessere oder sogar die beste Entscheidung, aus vielen Möglichkeiten, zu finden.
\subsubsection{Prozess}
Eine wichtige Annahme ist, dass man in der Regel genau dann hohe Qualität erhalt, wenn man geeignete/sinnvolle Arbeitsweisen verwendet und diese sorgfältig/gezielt durchführt. Zum Beispiel was für Themen betrachtet werden sollen, wie Umfangreich man diese betrachtet und wann man zum nächsten Thema übergeht.
Da dies ein sehr großer und somit auch wichtiger Teil für die Arbeit ist, sollte man diese Gesamt-Vorgehensweise nicht neu erfinden, sondern sich stattdessen auf vorhandene Erfahrungen halten.
Das erste Modell was wir dazu betrachten ist das Wasserfall-Modell. Dies hätte den Vorteil, dass man Schritte zum Bau der Software, genau wie in dieser Arbeit, voneinander trennen und nacheinander lösen kann. Dadurch würde die Zeitplanung auch einfacher ausfallen, was sich sehr gut bei einer vorgegebenen Zeitangabe, wie bei dieser Arbeit, macht.
Jedoch wurde sich doch dagegen entschieden. Die Annahme "alle Aufgaben lassen sich getrennt voneinander lösen", ist bei dieser Arbeit nicht Sinnvoll. Es ist sehr Wahrscheinlichkeit, dass sich während der Implementierung und dem Design neue Erkenntnisse zu den Anforderungen ergeben. Man stelle sich zum Beispiel vor, dass sich erst bei der eigentlichen Implementierung bewusst wird, dass eine Funktion, aus Schwierigkeit und Zeitgründen, nicht umzusetzen ist. Es wird also stattdessen davon ausgegangen, dass sich die Schritte in diesem Softwareprojekt nicht nacheinander und getrennt voneinander lösen lassen, sondern das zwischen Ihnen ein fließender Übergang herrscht.
Auch die Zeitplanung lässt sich nicht, wie im Wasserfall-Modell vorgesehen, einhalten. Dafür benötigt es ein wohldefiniertes Resultat. In dieser Arbeit herrscht aber große Unsicherheit. Anforderungen, Design sowie die Art der Implementierung müssen alle zuerst überlegt/erhoben werden.

\myNewSection
Stattdessen wurde sich dann doch für eine Agile-Arbeitsweise entschieden. Und zwar unteranderem wegen den Folgenden, eng miteinander verbundenen, Prinzipien:
	\begin{enumerate}
		\item Das erste ist dabei die Arbeit in \textbf{Iterationen}. Es wird sich also in vielen kleinen Schritten zum Ziel herangetastet. Einerseits hat man dadurch steht's ein Evaluierbares Produkt, was sehr nützlich für die Evaluierung ist. Außerdem die Arbeit in kleinen überschaubaren Schritten auch bei der zuvor erwähnten Anforderungsunsicherheiten dieser Arbeit. Jedoch gibt es auch Nachteile bei der Arbeit in Iterationen. So könnte diese bei schlechter Planung oder sehr großer Unsicherheit zu Doppelarbeit führen. Zum Beispiel weil das zuletzt erst erstellte und für richtig gehaltene Produkt nun doch nicht benötigt wird.
		\item Das zweite Prinzip ist der Kurzer \textbf{Planungshorizont}. Auch dies würde dabei helfen, dass Änderungen weniger verheerend sind, weil sich nicht strikt an einen Plan gehalten werden muss.
		\item Das letzte Prinzip sind die \textbf{Korrekturen}. Dabei geht das Prinzip sinnvoll davon aus, das man während eines Projektes Fehler machen wird. Deswegen wird der Prozess so gestaltet, dass Auftretende Fehler gut behoben werden können. Offensichtlich erfüllen die beiden zuvor genannten Prinzipien bereits diese "Regel".
		\item Prozessmodelle passen verschieden gut zu Projekten. So könnte das Wasserfall-Modell gut zu einem bereits sehr definierten Projekt passen und Agile zu einer Revolutionären Idee. Auch das einhalten der Prinzipien und Regeln kann misslingen. Von daher ist es wichtig das Prinzip \textbf{ständige reflektion} zu erwähnen. Dabei werden zum Beispiel wöchentlich, also nach je einer Iteration, Überlegungen über den Prozess gemacht. Diese Reflexion helfen einen dabei, schrittweise, da wo es hapert oder wo es sehr gut läuft, den Prozess zu Verbessern und Ursache von Mängeln zu beseitigen.
	\end{enumerate}











\section{PcVsPhoneTemp}\label{section:pcVsPhone}

\myTextTodo{
\textbf{Pc Vokabular: Trennung von Desktop, Laptop, Server, miniPcs:}\\
Aus einfachheit wird bei der nennung von Pc's immer auf Desktops refereziert. Es gibt zwar auch sehr kleine Laptops, welche die gleiche größe wie Handys haben, und auch server mit sehr viel stärkeren Komponenten, aber da wir eine App für normale Nutzer erstellen wollen und wir nicht erwarten, dass diese beiden systeme unter denen häufig verbreitet sind, schließen wir diese aus. Auch betrachten/beziehen wir der einfachkeits halber auch nur Desktops und lassen Laptops eher außen vor. Dies sollte aber keine große Auswirkungen ausüben. Denn die meisten Punkte sollten auch für Laptops gelten.
}

\myNewSection
%%% hidden subsection for a better structure in latex editor: "texifier"
\myComment{\subsection*{Übersicht}} 

In diesem Abschnitt geht es nicht darum möglichst vollständig alle Unterschiede zwischen Pc und Handy aufzulisten. Stattdessen sollen wichtige Erkenntnisse gewonnen werden, indem wir einige (ausschlaggebende/wichtige) Unterschiede betrachten. Denn genau diese Erkenntnisse sollen uns dabei helfen die Stärken und Schwächen des Handys und Pc's zu identifizieren.

Diese erkenntnisse werden uns bei der Wahl von funktionalen und nicht funktionalen Anforderungen begleiten, weil...


\myTextTodo{Da diese nicht funktionale Anforderung starke Einflüsse auf die funktionalen Anforderung haben kann, wird sie hier nochmal gesondert und im Detail betrachtet. Dabei folgt eine Aufzählung der Unterschiede und Stärken von Smartphones und Pc's, sowie die Bewertung dieser nach der Wichtigkeit für diese Arbeit. \myTextTodo{Aufzählung Reihenfolge beachten?}} \newline%

\myTextTodo{In general, every product of technology excels in some key parameters while showing limitations in others, as is the case with laptops and desktop computers}\newline

\myTextTodo{When mobile phones were first introduced to the public, they were meant to become a portable version of telephones, allowing people to communicate without the constraints of wires or phone holders. Over the years, they have evolved into comprehensive and advanced tools that serve various purposes—communication, entertainment, emergency, storage, and applications.}\newline
\subsection{Leistung}\label{PcVsPhone:Leistung}
%Was: Pc
In Bezug auf Leistung gelten PCs im Allgemeinen als schneller und stärker im Vergleich zu Smartphones. 
	%Warum: 
	Das liegt an verschiedenen Faktoren. Zum einen haben die meisten PCs einen konstanten Zugang zu Strom, während Smartphones häufig nur über Batterien verfügen. Darüber hinaus können PCs aufgrund ihres größeren Volumens leistungsstärkere Hardware-Komponenten verbauen und größere Kühlsysteme nutzen, was die Verwendung dieser stärkeren Hardware überhaupt erst ermöglicht.\newline%
%Was + Warum: Handy
Im Gegensatz dazu ist die Hardware von Smartphones in der Regel auf Portabilität und Energieeffizienz ausgelegt, um eine möglichst lange Akkulaufzeit zu ermöglichen.\newline%
%
%
%
%
%
%
%\myComment{
%
%	%%%->Benutzung%%%
%	%Zusammenfassung: Leistung VS Portabilität
%	Während Pc's also auf möglichst performante Leistung ausgelegt sind, wird stattdessen beim Handy eher auf die Portabilität geachtet.\newline%
%	
%	\myNewSection
%	\myTextTodo{
%	-> Leistung vs Portabilität\\
%	- Aufwändige Hardware vs Optimierung(Leistung + Batterie) 
%	}
%	
%	
%	\myNewSection
%	%Auswirkungen
%	-Eine mögliche Auswirkung davon ist, dass aufwändige Anwendungen öfters nur auf dem genutzt(da schneller) oder gar nur auf dem Pc unterstützt werden. Solch eine aufwändige Anwendung wäre zum Beispiel das exportieren von Videodateien. 
%	-Währenddessen wird/muss auf dem Handy wahrscheinlich eher auf die Optimierung geachtet, um wenig Strom zu verbrauchen und so eine längere Betriebszeit zu ermöglichen. Und um handys schneller zu gestalten (3 sekunden regel)
%
%}
\subsection{Internet}\label{PcVsPhone:Internet}\myCheckmark
%Was: Pc+Handy
Während Pc's oft mit schnellen und stabilen, aber dafür auch stationären, Ethernet-Anschlüssen verbunden sind, werden Handys oft mit dem WLAN, der mobileren aber dafür häufig langsameren Alternative, genutzt. 
%Zusammenfassung
Also wird auch hier, wie im Vergleich der \nameref{PcVsPhone:Leistung}, Performance, in diesem Fall in Form von Schnelligkeit und Stabilität, gegen Mobilität getauscht. \newline%
%Warum: internet wichtig für Handy
Jedoch scheint besonders diese Schnelligkeit und Stabilität auf dem Handy wichtig zu sein, vielleicht genau weil sie auf dem Handys oft fehlt. So verlassen die hälfte aller Handy Nutzer laut Google die Website wenn sie länger als drei Sekunden braucht zu laden\cite{pcVsphone_threeSeconds}.\newline%
%Was+Warum: Handy -> Mobile Daten
Außerdem besitzen Handys Zugang zu \dq Mobile-Daten\dq. Dabei ähnelt sich die Funktion mit der des WLAN's, mit dem Unterschied, dass Sie die Stärken und Schwächen des WLAN's noch weiter in die extreme ziehen. So erweitern sich die mobile Nutzung vom Haus hin zu fast jedem Standort. Jedoch gelingt dies erneut nur durch den Austausch von Leistung. So ist das Datennetz für die \dq Mobilen-Daten\dq häufig langsamer und instabiler als die des Heiminternets. Das liegt wahrscheinlich zuteil an dem bereits fehleranfälligeren Übertragungsmedium der Mobilen-Daten. So wird wenn es um Stabilität geht die Verbindung über ein Kabel über die Verbindung über die Luft bevorzugt.\newline%
%Auswirkungen
Das führt zum Beispiel dazu, dass für Anwendungen welche von eine schnelle und stabile Internetverbindung profitieren, eher der Pc präferiert wird. So zum Beispiel bei online Spielen oder Videoübertragungen.




%old - todo remove
\myComment{
		Handys bieten trotzdem auch hier einige Vorzüge. Das WLAN macht das Handy um einiges mobiler, dass es kein Kabel benötigt. Während Laptops auch WLAN besitzen, sind viele Desktops nicht damit ausgestattet. Außerdem besitzen die meisten Handys einen dauerhaften Internetzugang durch 'Mobile Daten'. Diese sind zwar oft langsamer und nur begrenzt verfügbar, jedoch ermöglicht das einen Internetzugang von fast jedem Standort.
		}
\subsection{Speicher}
%Warum: Speicher
Was den Speicher betrifft, so müssen die Speichermodule in Handys im Vergleich zu Festplatten für PCs kleiner sein, um in den Formfaktor des Geräts zu passen. Dem entsprechend bieten Handy-Speichermodule in der Regel auch eine geringere Kapazität als die größeren Festplatten von PCs. %   
	%Quellen
	Laut einer Studie von Counterpoint besitzen Handys im Durchschnitt rund 118 Gigabyte Speicherkapazität\cite{pcVsphone_storageSmartphone}. Im Gegensatz dazu zeigt ein Report von Seagate, dass sie im vierten Quartal 2022 im Durchschnitt PC-Festplatten mit einer Größe von rund 8 Terabyte verkauft haben\cite{pcVsphone_storageSeagate}\footnote{Die hohe Festplattengröße kommt vermutlich durch Server zustande. Der Speicher von Heimcomputern fällt dementsprechend wahrscheinlich etwas kleiner aus.}.% 
%
%\myComment{
%
%	%%%->Benutzung%%%
%	%Zusammenfassung
%	Also wird, wie bereits im Vergleich der \nameref{PcVsPhone:Leistung} und dem \nameref{PcVsPhone:Internet}, auch beim Speicher die Leistung zugunsten der Mobilität eingeschränkt.
%		
%	\myNewSection
%	\myTextTodo{
%	-> Leistung(Kapazität) vs Portabilität\\
%	- große erweiterbare Festplatten + Datengrab vs kleine Festplatten + Optimierung und nur wichtigstes\\ 
%	}
%	
%	\myNewSection	
%	%Auswirkung
%	-Dadurch dass der Speicher auf Handys also relativ klein ist, wird vermutet, dass Handy-Nutzer eher auf ihren Speicher achten. Während der Pc gerne als Speicherablage genutzt wird, wird auf dem Handy eher wichtige Daten und welche man häufiger braucht gespeichert.
%	 -Optimierung für App nötig. Wenn Speicher knapp wird wahrscheinlich erst eine App mit Großem Speicherbedarf gelöscht/nicht benutzt
%
%}
\subsection{Eingabe} 
%Was: Maus
PCs werden in der Regel mithilfe einer Maus bedient. Mit ihr können Objekte auf der Oberfläche präzise ausgewählt werden. Außerdem verfügt die Maus über verschiedene Tasten, mit denen eine Interaktion mit den Objekten auf verschiedene Arten möglich ist.\newline%
%Was: Handy: Zeigen+Gesten
Auf dem Handy wird hingegen der Touchscreen als Mausersatz benutzt. Durch Berühren des Bildschirms lassen sich Mausfunktionen simulieren. Anstatt beispielsweise das gewünschte Objekt mit der Maus auszuwählen, wird es mit dem Finger berührt. Weitere Funktionen wie Zoomen, Scrollen und Rechtsklick können durch Gesten ausgeführt werden.\newline%
	%Warum: Intuitive
	Dadurch wird die Bedienung auf dem Handy intuitiver, insbesondere das Auswählen von Objekten fühlt sich natürlicher an.\newline%
		%Geste
		Erfahrungsgemäß gilt das gleiche für die Gesten. Scheinbar fühlen sie sich intuitiv an und haben sich bereits als Standard in den Handymarkt eingegliedert, dass aus eigener Erfahrung viele Menschen lieber das Handy nutzen statt den Pc um Bilder anzuschauen. Die Möglichkeiten mithilfe von Gesten zu wischen und vergrößern fühlt sich wohl angenehmer an, als über Verwendung von Maus und Tastatur die Bilder zu betrachten.\newline%
	%Warum: Unpräzise -> Design
	Dafür ist die Eingabe über den Touchscreen im Vergleich zur Maus ungenauer. Da der Finger deutlich größer ist als der Mauszeiger, lassen sich kleine Objekte nicht so konsistent auswählen. Um diesem Problem entgegenzuwirken, ist ein passendes Design erforderlich. Beispielsweise wird für Apps eine Mindestgröße für Buttons vorgeschrieben: \glqq On a touchscreen, buttons need a hit target of at least 44x44 points to accommodate a fingertip\grqq{}\cite{konventionen_buttonSize}.%
\myNewSection%
%Was: Keyboard unterschiede
Zur Eingabe von Text auf einem PC wird üblicherweise eine physische Tastatur verwendet, während auf Handys eine Softwaretastatur zum Einsatz kommt. Aufgrund der unterschiedlichen Größe und Bedienung bieten beide Varianten verschiedene Vor- und Nachteile.\newline%
	%Warum Pc: schneller und präziser + Tasten Anzahl und Tastenkombinationen
	Eine Hardware-Tastatur ermöglicht durch ihre Größe und das Tippgefühl schnelleres und präziseres Schreiben im Vergleich zur Touchscreen-Tastatur auf Handys. Außerdem kann sie im Vergleich zur mobilen Variante ungefähr das Dreifache an Tasten darstellen.\footnote{Full-size Tastaturen können bis zu 108 Tasten haben\cite{pcVsphone_pcKeyboardSize}. Die Standard deutsche Texttastatur auf iOS 16.1.2 hat hingegen rund 33 Tasten. Selbst nachgezählt und kontrolliert auf iPhone SE1, SE2 und 13 mini.}. Des Weiteren unterstützt Hardwaretastaturen Tastenkombinationen. % 
		%Auswirkung: TODO ~maybe~ remove
		Dadurch können Programme, die lange Eingaben, viele verschiedene Symbole oder Tastenkombinationen erfordern, in der Regel auf einem PC besser bedient werden. Ein Beispiel dafür sind Programmiereditoren.\newline
	%Warum Handy: Mobiler und intuitiver
	Die Handy-Tastatur hat den Vorteil der Mobilität und das Tippen fühlt sich oft intuitiver an, da sich die Tastatur je nach Anwendung variabel darstellen lässt. %
		%Beispiel:
		Ein Beispiel dafür ist die Taschenrechner-Anwendung, bei der nur die benötigten Zahlen und Symbole auf der Tastatur angezeigt werden und dementsprechend keine Buchstaben.%


%\myComment{
%
%	%%%->Benutzung%%%
%	\myNewSection
%	%Zusammenfassung
%	Zusammenfassend hat das Handy also auch in diesem Vergleich an Leistung, in diesem Fall die präzise fehlerfreie und schnelle Eingabe, verloren und dafür an Mobilität gewonnen. Dafür fühlt es sich aber in vielen Anwendungen intuitiver und damit leichter zu benutzen an.
%	
%	\myNewSection
%	\myTextTodo{
%	-> Leistung(schnellere präzisere fehlerfreie eingabe) vs Portabilität\\
%	-> um den entgegenzuwirken Optimierung(Design-Eingabe auf Anwendung Anpassen + Gesten + Touch) -> Intuitive
%	}
%	
%	\myNewSection	
%	%Auswirkung
%	- Da die Handy-Tastatur also Fehleranfälliger und schwerer zu benutzen ist, nehmen wir an, dass Handy Nutzer weniger gern lange texte auf dem Handy schreiben. \newline%
%	- Andersherum benutzen Handy Nutzer für kurze und einfachere Aufgaben lieber das Handy. Denn bei diesen Aufgaben benötigt es nicht viele verschiedene Tasten, Präzise Eingabe, schnelle Eingabe (-> bedienung wird nicht erschwert) und ist daher mit der Handy Eingabe (Touch, Geste + Anwendungstatsatur) (-> bedienung wird verbessert) intuitiver\newline%
%	- Dafür muss die App aber dementsprechend auch optimiert werden. (leichte Anwendung, Touch + Gesten Unterstützung + richtige Tastatur)
%	- Schwierige und Komplexe Aufgaben funktionieren dafür besser auf dem Pc, da lange Eingabe + verschiedene Symbole + Tastenkombinationen in der Regel für diese benötigt werden. 
%
%}
\subsection{Bildschirm}
%Warum: Bildschirmgröße
Pcs bieten die Möglichkeit, mehrere Monitore gleichzeitig zu nutzen und ihre Bildschirme sind in der Regel größer als die von Handys \cite{pcVsphone_screenResolutionStats}\cite{pcVsphone_screenResolutionToSize}\footnote{Die Monitorgröße wurde anhand der Monitorauflösung abgeschätzt; die meisten Monitore sind zwischen 14 und 23 Zoll groß.}.\newline%
	%Was: Anzahl an darstellbaren Informationen
	Dadurch kann der PC viele Informationen gleichzeitig darstellen.\newline%
%Warum: Hoch- vs Querformat
Handys und PCs unterscheiden sich auch in der Ausrichtung ihrer Bildschirme. Während PC-Bildschirme meist im Landschaftsmodus betrieben werden, werden Handys meist im Porträtmodus verwendet.\newline%
	%Was: Verschiedene Darstellungen/Anwendungen
	Dies bedeutet, dass Anwendungen unterschiedlich gut auf den beiden Geräten funktionieren, da sie sich aufgrund der verschiedenen Ausrichtung und Bildschirmgröße unterschiedlich darstellen. %
		%Beispiel
		Zum Beispiel ist die Darstellung einer großen, detaillierten Tabelle auf einem PC wahrscheinlich einfacher, während lange vertikale Listen mit einfachen Details sich besser auf Handys überfliegen lassen.\newline%
%Auswirkung: Design
Dementsprechend wird angenommen, dass bei der Entwicklung von Anwendungen für Mobilgeräte besonderes Augenmerk auf das \nameref{section:design} gelegt werden sollte. %	%Begründung :Richtlinien + Google/Apple
	Diese Annahme wird weiter dadurch bestärkt, dass Apple und Google für die beiden beliebtesten Plattformen iOS und Android\cite{pcVsphone_mobileOperatingSystem} ihre eigenen Designrichtlinien veröffentlicht haben\cite{konventionen_guidelinesApple, konventionen_guidelinesGoogle}.\newline%
	%Warum: intuitiver + einfacher
	Interessant ist, dass durch die Anwendung dieser Richtlinien, die App intuitiver und einfacher werden kann. %
		%Begründung: Regeln abgestimmt für die Eigenschaften des Handys
		Dies liegt einerseits daran, dass diese Richtlinien Regeln enthalten, die speziell auf die Eigenschaften von Mobilgeräten abgestimmt sind. Eine Regel besagt beispielsweise, dass Schaltflächen eine Mindestgröße von 44x44 Pixel haben müssen, um die Größe der Finger zu berücksichtigen.\cite{konventionen_buttonSize}.\newline%
		%Begründung: Reichweite + Einfluss -> Verbreitung -> Konvention -> ähnliches Design+Zurechtfinden
		Andererseits spielt wahrscheinlich auch die Verbreitung dieser Richtlinien eine Rolle. So wirkt sich die Reichweite und der Einfluss von Apple und Google wahrscheinlich auch auf die Bekanntheit und Verbreitung ihrer Richtlinien aus. %
		%Konvention
		Durch eine hohe Bekanntheit und Verbreitung würden immer mehr Apps diesen Richtlinien folgen. %
		%ähnliches Design -> intuitiver
		Dadurch würden sich im Umkehrschluss verschiedene Apps ähnlich bedienen lassen und Nutzer müssten nicht für jede Anwendung eine neue Bedienung erlernen. Dementsprechend finden sich Nutzer in neuen Apps schneller zurecht.\newline%
		%Quelle: Richtlinien sind nützlich :TODO: quelle wirklich hier verwenden? oder in auswertung oder design oder nfA?
		Ein Indiz für die Annahme, dass Richtlinien und deren Verbreitung Anwendungen einfacher und intuitiver machen, wäre die größere Popularität von Handy-Apps im Vergleich zu Webseiten\cite{pcVsphone_mobileAppVsWebTimeSpent}\footnote{Laut der Studie werden rund 88\% der Nutzung von Handys für Apps und 12\% für den Browser verwendet.}. Denn während für Apps zumindest einige der Regeln aus den Richtlinien, wie zum Beispiel die Knopfgröße, zwingend angewendet werden müssen\cite{konventionen_buttonSize}, sind für Webseiten diese Regeln nicht Vorschrift.%
%
%
%
%\myComment{
%
%		%Old Backup: TODB Remove
%		%Die Richtlinien sowie dessen Verbreitung sind wahrscheinlich ein [großer] Grund dafür, warum Handys als leichter zu bedienen bewertet werden\cite{pcVsphone_easyUseVsImportantTask}. Außerdem gibt das auch ein Indiz warum auf dem Handy Apps sehr viel beliebter sind als Websites\cite{pcVsphone_mobileAppVsWebTimeSpent}, denn für Webseiten muss keine der Appeigenen Richtlinien befolgt werden. Bei Apps gibt es einige Regeln, welche zwingend für die Veröffentlichung der App benötigt werden.\newline%
%
%
%	%%%->Benutzung%%%
%	
%	%Zusammenfassung
%	Die Bildschirme von Handy und Pc unterscheiden sich. So ist der Bildschirm vom Handy kleiner und dadurch mobiler, während der Pc Monitor durch seine Größe viele Informationen gleichzeitig darstellen kann. Daher kam es zu der Erkenntnis, dass unterschiedliche Anwendungen und Darstellungen verschieden gut auf den beiden Geräten funktionieren.
%	Um der Informationsarmut des Handys entgegenzuwirken muss muss sich nun detaillierte Gedanken über die Darstellung der App gemacht werden. Wenn einem dies gelingt, kann die App im Vergleich zu Pc Anwendungen sogar um einiges intuitiver und leichter werden.	
%		
%	
%	
%	
%			
%	\myNewSection
%	%Auswirkung
%	- Verschiedene Darstellungen und Anwendungen können unterschiedlich gut abgebildet werden.\\
%	-> Das Handy braucht um gut zu funktionieren andere Designentscheidungen als Pc's\\
%	-> Schwere Aufgaben (viele Details) funktionieren besser auf dem Pc\\
%	-> Nutzer benutzen Handys gerne, da sie durch das Design einfach und intuitive zu nutzen sind\\
%	-> Einfache Aufgaben funktionieren gut auf dem Handy, da genau diese Aufgaben oft nicht viele Informationen Darstellen müssen + Handy durch andere Punkte bereits intuitive und einfach, was diesen Punkt noch mehr verstärkt. 
%	- 
%	
%	%Was
%	%Warum
%	%Zusammenfassung
%	%Auswirkung
%	
%	\myNewSection
%	\myTextTodo{
%	-> Leistung(Darstellbare Informationen) vs Portabilität\\
%	- um dem entgegenzuwirken Optimierung(Design) -> intuitive + einfach\\ 
%	}
%
%}
%
%
%
%
%\myComment{Da diese Konventionen bereits sehr etabliert sind, immerhin werden sie von Android und Apple empfohlen, die beiden größten Handy Hersteller \cite{}, werden sie auch sehr häufig in Apps eingesetzt + werden in eigenen Apps verwendet. -> nutzer gewöhnt sich an konventionen -> neue nutzer finden apps simpler (da alle das gleiche navigaions muster) verwenden -> auf pc's gibt es keine solche konventionen/regeln -> schlechtes design lässt sich auf dem pc allgemein eher verzeihen, da großer bildschirm + präzisere eingabe -> apps simpler / pc anwendungen oft komplizierter quotate? hier benuzten oder in "hier, usage, nfA, einleitung konventionen"?}
\subsection{Mobilität}

\myNewSection
\textbf{Portability}: Einer und wenn nicht sogar der wichtigste Unterschied zwischen Pc's und Handys ist die Mobilität. Alle zuvor erwähnten Unterschiede sind also eigentlich nur eine Auswirkung dieser von Handys gewünschten Eigenschaft. Wenn es nicht mobil sein müsste, wäre mehr Leistung, ein größerer Bildschirm, eine stabilere Internetverbindung usw möglich. Jedoch wären wahrscheinlich auch die entstandenen Vorzüge durch die Design-Richtlinien und der Eingabegesten verfallen. \newline%
Durch die kleine Größe der Smartphones genießen es also eine hohe Mobilität. Zwar sind Laptops auch schon um einiges mobiler als Stand-Pc's, jedoch kommen sie in dem Aspekt nicht an Handys heran. Handys passen in die meisten Hosentaschen, während man für Laptops oft eine größere Tasche oder einen Rucksack benötigt. \newline%

\myTextTodo{Überlegen ob hier oder in Benutzung: durch mobil -> tasks on the go -> ...
Zwar ist dies keine Stärke welche wir direkt in der App einbauen und nutzen können. Jedoch vielleicht die größte Stärke des Handys und eine Deutung dafür, dass Smartphone apps beleibt sind und sich diese Arbeit lohnt.}
\subsection{Benutzung}\myCheckmark
%Was: Einstiegsaufwand
	%Pc
	Um Pc's zu benutzen, bedarf es eines gewissen Einstiegsaufwands. %
		%Warum: Stationär
		So muss sich einerseits, da Desktops stationär sind, zuerst zum Standort des Pc's begeben werden. %
		%Warum: Startuptime
		Darüber hinaus muss der Pc vor jeder Nutzung eingeschaltet werden. Denn es wird davon ausgegangen, dass Pc's normalerweise ausgeschaltet sind, wenn sie derzeit nicht in Benutzung sind. Das Anschalten dauert in der Regel ein paar Sekunden. Laut dem Benchmark \glqq Startup Timer\grqq{} beträgt die schnellste aufgezeichnete Startzeit bis der Pc nutzbar ist elf Sekunden\cite{pcVsphone_boottime}.\newline%
	%Handys
	Bei Handys existiert dieser Aufwand hingegen nicht. %
		%Warum: Mobil + Relevanz -> Immer bei einem
		So wird davon ausgegangen, dass die meisten Nutzer ihr Handy immer bei sich tragen. Das liegt einerseits an der zuvor erwähnten Mobilität, aber andererseits auch in der \nameref{subsection:motivation} erwähnten Relevanz und Beliebtheit von Handys. %
		%Warum:
		Weiterhin wird davon ausgegangen, dass Handys normalerweise nicht ausgeschalten werden, um unter anderem für wichtige Anrufe oder Nachrichten erreichbar zu sein. Das Handy ist also immer an und benötigt dementsprechend keine Startzeit.\newline%		
%Was: Multitasking%-------------------------Reword-------------------------------------
Ein weiterer Aspekt bei der Verwendung ist, wie Anwendungen auf den jeweiligen Geräten genutzt werden können. %
	%Handy: 
	Dabei können Handys immer nur eine Anwendung gleichzeitig darstellen. Das könnte eine Limitation des Betriebssystems oder der Leistung sein. Vermutlich wird das Multitasking aufgrund des kleinen Displays nicht unterstützt.\newline%
	%Pc:
	Währenddessen unterstützt der Pc genau diese Funktion. Er kann mehrere Anwendungen gleichzeitig ausführen und darstellen.%
	
	
\myComment{

\myTextTodo{\textbf{One task at a time}: Although many mobile operating systems now offer a split-screen mode, the small screen size limits its usefulness. The fact is, in most cases, users on mobile devices must focus on one window at a time. This limitation means that it’s difficult to combine multiple sources of information and carry out complex tasks. These mobile constraints are no problem if the task is simple, unimportant, or open-ended. However, when the task is goal-based and has high stakes, these constraints are reason enough to save the task for another device}

}
\subsection{Hardware}\myCheckmark
%Was: Modulär
Die meisten Desktop sind in Hinsicht ihrer Hardware modular. %
	%Warum: Konfigurierbarkeit/Anpassungsfreiheit 
	So wird dem Benutzer die Option gegeben viele Hardwarekomponenten nach belieben auszutauschen. %
	%Warum: Auswirkung: Größe
	Dementsprechend muss das Gehäuse des Pc's auf diese Anpassung aber auch ausgelegt sein, denn es gibt Hardwarekomponenten in einer Varietät von Größen und Formen. Die Modularität kommt also mit dem Nachteil, dass das Gehäuse groß genug sein muss um auch die verschiedenen Hardware-Optionen zu ermöglichen.\newline%
%Was: Handy allInOne 	
Das Handy unterstützen dementsprechend diese Konfigurierbarkeit nicht. Dafür bietet es sich als all-in-one-Gerät an. %
	%Warum: Ease of use
	Während man sich beim Pc' auch Gedanken um Monitor und Eingabegeräte machen muss, fällt die Wahl bei einem Handy leichter aus. Hier wird alles kompakt zusammen als ein System angeboten. Unteranderem besitzen sie sogar, anders als die meisten Pc's, eine Kamera, GPS, Gyroskop und die Option für biometrisches login Verfahren.\newline%
	
	
	
\myComment{

	%%%->Benutzung%%%
	
	%Zusammenfassung
	Hier erkennen wir also einen Abtausch zwischen Anpassungsfähigkeit und Einfachheit. Während die meisten Pc's sehr viele Optionen zur Konfiguration geben wurde beim Handy für einen bereits alle Entscheidungen getroffen.%
	
	%Was
	%Warum
	%Zusammenfassung
	%Auswirkung
	- Handys werden als leichter angesehen (all in one - ease of use) -> gut für Simple Aufgaben VS Pc eher für Leute die sich auskennen oder konfigurieren wollen
	- Dafür sind Pc's sehr anpassungsfähig. -> Gut für Arbeit / wichtige und Komplexe Aufgaben. Da manche Aufgaben zum Beispiel sehr spezifische Komponenten benötigen. Zum Beispiel einen schnellen Prozessor, oder eine schnelle Grafikkarte, oder viel Speicher, eine soundkarte, sehr schnelle netzwerkgeschwindigkeit, oder viele Monitore usw...
	
	\myNewSection
	\myTextTodo{
	-> Leistuns(Modulär) vs Portabilität\\
	-> Aber auch Modularität vs Abgeschlossenheit -> Easy of use \\
	- 
	}


}
\subsection{Betriebssystem}\myCheckmark
%Was: Pc: Vielfalt + Entscheidungsfreiheit
Bei Desktops gibt es eine Vielfalt an zu wählenden Betriebssystemen und dem Nutzer ist die Option überlassen sich für eins oder mehrere davon zu entscheiden. %
	%Was: lange Unterstützung
	Selbst mit alter Hardware lassen sich oft aktuelle Betriebssysteme installieren. So unterstützt der knapp zehn Jahre alter Prozessor Intel Pentium J1750 das weltweit meist genutzte und immer noch aktuelle Betriebssystem Windows 10\cite{pcVsphone_intelWindowsSupport, pcVsphone_destkopOperatingSystem, pcVsphone_windowsVersions}\newline%
%Was: Handy Betriebssysteme
Für Handys wird hingegen ein festes Betriebssystem vorgesetzt. %
	%Was: keine Entscheidungsfreiheit + kurze Unterstützung
	Das Betriebssystem lässt sich nicht ohne weiteres ändern und wird meist nur drei bis fünf Jahre unterstützt \cite{pcVsphone_deviceSupportGoogle}\cite{pcVsphone_deviceSupportApple}. %
		%Nachteil: Sicherheit und Leitsung
		Zwar kann das Handy nach dieser Zeitspanne noch weiter betrieben werden, jedoch leidet ohne weitere Softwareunterstützung die Sicherheit und Leistung des Handys. Das hat den Nachteil, dass alle drei bis fünf Jahre das Handy gewechselt werden sollte.\newline%
		%Vorteil: Optimierung
		Die kürzere Unterstützungszeit der Handy Betriebssysteme könnte aber auch einen Vorteil bieten. Dadurch könnten sich die Betriebssystem Entwickler auf eine kleinere Anzahl von Handys konzentrieren und die Software dementsprechend gut auf diese optimieren. Daher wird angenommen, dass Handys in dem unterstützten Zeitraum zum Beispiel mit weniger Defekten und Rucklern auskommen sowie eine flüssigere grafische Oberfläche besitzen.\newline%








\myComment{

%%%->Benutzung%%%

	%Zusammenfassung
	Während der Pc einen also die Freiheiten für Optionen und Wahlmöglichkeiten lässt, ist das Handy auf Einfachheit und Benutzerfreundlichkeit ausgelegt. (Immerhin muss man sich keine Gedanken um ein passendes Betriebssystem machen und die Leistung wird durch Optimierung garantiert). -> Handy Betriebssystem ist auf genau auf die Hardware ausgelegt.
		
	%Auswirkung
	-> Handys wirken auf Nutzer einfacher. Man muss sich weniger Gedanken um OS machen,  'it just works'. -> passend für simple Aufgaben, diese sollten auch einfach zu lösen sein
	-> Pcs sind mehr konfigurierbar und erweiterbar -> anpassbarkeit ist gut für komplexe und schwierige aufgaben, denn diese brauchen wohlmöglich komplexe und spezifische umgebungen
	
	\myNewSection
	\myTextTodo{
	-> Konfiguration VS Ease of use\\
	- 
	}

}
\subsection{Software}\myCheckmark
%Was: Pc Software -> Konfigurierbar
Software für den Pc sind oft konfigurierbar. %
	%Beispiel: Installation
	So fängt zum Beispiel die Auswahl verschiedener Optionen bereits bei der Installation der Software an. Unter anderem werden Fragen zum Speicherort, der automatischen Aktualisierung, der Desktop Verknüpfung und dem automatischen Starten gestellt.\newline%
%Was: Handy Software -> Benutzerfreundilchkeit
Handys hingegen scheinen weniger auf Konfigurierbarkeit, sondern eher auf Benutzerfreundlichkeit ausgelegt zu sein. %
	%Warum: Begründung + Beispiel(Installation)
	So benötigt die Installation von neuen Apps oft nur einen Knopfdruck. Rudimentäre Fragen wie der Speicherstandort oder die Erstellung einer Verknüpfung werden hingegen vom Betriebssystem übernommen. Das Abarbeiten einer Liste von Optionen wird also auf dem Handy übersprungen und dadurch wird angenommen, dass sich der Übergang von der Installation bis zum Nutzen der App flüssiger für den Nutzer anfühlt.\newline%
%Was: Pc Software Allgemein -> Pc besser Konfigurierbar 
Dieses Verhalten gilt auch über die Installation hinaus. Generell scheinen Anwendungen mit vielen Optionen besser auf dem Pc zu funktionieren. %
	%Begründung: Aus Erfahrung besser auf Pc
	So werden Aufgaben wie das Editieren von Videos, Entwicklungsumgebungen oder das Erstellen von Steuererklärungen werden erfahrungsgemäß zumeist auf dem Computer bearbeitet. %
		%Begründung: Zuvor Erwähnten Unterschieden -> viele Optionen schwer zu überschauen + klicken
		Womöglich liegt das an den zuvor erwähnten Unterschieden und Limitierungen des Handys. Eines davon wäre zum Beispiel die Displaygröße. Wenn eine Anwendung viele Optionen bietet, lassen sich diese bei einem kleinen Display mit ungenauerer Eingabe schwieriger auswählen und überschauen.\newline%
%Was: Pc Software Allgemein -> Weniger Optionen %---------TODO re check this part----------
Aus Erfahrung spiegelt sich dieser Sachverhalt auch in Apps wieder. So werden für Handys meist nur die relevanten Optionen dargestellt. Das würde bedeuten, dass die restlichen eher nebensächlichen Optionen bereits vom Entwickler oder Betriebssystem getroffen worden.\newline% 
	%Was: Entwickler machen sich mehr Gedanken: Optionen
	Daher wird vermutet, dass sich Entwickler bei der Erstellung von Apps mehr Gedanken über passende Entscheidungen machen, da ihnen bewusst ist, dass die Nutzer die Optionen später nicht selbst anpassen können.\newline%
	%Warum: Vorteil: Userexperience
	Das hätte im besten Fall den Vorteil, dass sich Handynutzer weniger bis keine Gedanken um die Auswahl von Optionen machen müssen und sich dementsprechend die App intuitiver anfühlt. Schließlich wurde bereits die passende Auswahl vom Entwickler getroffen.%
%
%
%
%
%
\myComment{

%%%->Benutzung%%%

	%Zusammenfassung
	Während der Pc also erneut Optionsfreiheit anbietet scheint das Handy wieder Benutzerfreundlicher zu sein. So lässt die Pc Software einen sehr viele Optionen zum selbst konfigurieren, während Apps sofort einsatzbereit sind. 
	
	%Was
	%Warum
	%Zusammenfassung
	%Auswirkung
	- Handy Benutzerfreundlicher\\
	- Handy Besser für einfache Aufgaben, da man dort meistens eh nicht viel konfigurieren möchte
	- Generell scheinen Komplexe Aufgaben mit vielen Optionen besser auf dem Pc zu funktionieren.
	
	\myNewSection
	\myTextTodo{
	-> Konfiguration VS Einfachheit -> Optimierung
	}

}

\subsection{Auswertung}\myCheckmark
%Einleitung
Nun folgt eine Aufzählung der gesammelten Erkenntnisse und Schlussfolgerungen dieses Abschnittes. % 
%Alles was zuvor in den vorherigen Unterabschnitten begründet wurde, wird hier nicht erneut [gleich detailreich/nur oberflächlich] [begründet/aufgesagt/erwähnt].%todo mabye warum?

% erfordert/bedarf/voraussetzt/benötigt/brauchen
% denn/darüberhinaus/außerdem/des weiteren/zusätzlich/daneben
\myNewSection
Pc's scheinen für jene Aufgaben gut zu funktionieren welche:
\begin{enumerate}%
	\item viel Leistung benötigt.\newline%
	Denn durch den dauerhaften Zugang zu Strom kann performante Hardware benutzt werden. Das Ethernet bietet eine schnelle und stabile Internetverbindung. Und durch die Größe des Pc's kann große Hardware, wie zum Beispiel Festplatten mit viel Kapazität, verbaut werden.%
	%
	\item schnelle, präzise oder vielfältige Eingaben erfordern.\newline%
	Die Maus und die Tastatur lassen sich durch ihre Größe und dem physischen Feedback schneller und präziser bedienen. Die vielfältigen Eingaben werden durch die hohe Anzahl an Tasten und die Möglichkeit für Tastenkombinationen ermöglicht.%
	%
	\item viele Informationen gleichzeitig darstellen oder benötigt.\newline%
	Denn mithilfe des großen Displays können viele Details dargestellt werden. Darüber hinaus können durch das Multitasking weitere Informationen von anderen Anwendungen dargestellt werden.%
	%
	\item viele Optionen und Konfigurationen anbieten oder benötigt.\newline%
	Die Hardware und das Betriebssystem kann bei Desktops beliebig ausgetauscht und konfiguriert werden. Außerdem ist ein großer Bildschirm hilfreich wenn viele Optionen dargestellt werden sollen.% 
	%
	\item langwierig sind oder viel Zeit benötigt.\newline% 
	Einerseits benötigen Aufgaben auf dem Pc generell mehr Startzeit, da der Einstiegsaufwand größer ist. Andererseits helfen die schnellen und präzisen Eingaben dabei lange Aufgaben schneller bewältigen zu können.%   
	%
\end{enumerate}%
%
\myNewSection
Handys scheinen hingegen für jene Aufgaben gut zu funktionieren welche:
\begin{enumerate}
	\item Ressourcenschonend sind und nicht viel Leistung benötigen.\newline%
	Denn das Handy besitzt durch die Batterie nur begrenzt Strom und die Komponenten sind daher eher auf Effizienz ausgelegt. Das WLAN oder die mobilen Daten sind oft langsamer und instabiler im Vergleich zum Ethernet. Und durch die kleine Größe sind auch nur kleine Festplatten mit begrenzter Kapazität möglich.%
	%
	\item keine schnelle, präzise und vielfältige Eingabe erfordern.\newline%
	Das kleine Display kann nur wenig Tasten gleichzeitig darstellen. Der Finger ist größer und damit zum Auswählen unpräziser als ein Mauszeiger. Dementsprechend ist die Eingabe auch langsamer, da sie sonst zu fehleranfällig werden würde.%
	%
	\item nur wenig Informationen darstellen oder benötigen.\newline%
	Das Display vom Handy ist relativ klein und Multitasking wird meistens auch nicht unterstützt.%
	%
	\item ohne viele Optionen und Konfiguration auskommen.\newline%
	Das Betriebssystem und die Hardware sind nicht änderbar. Außerdem übernimmt das Betriebssystem viele eigentlich optionale Entscheidungen, wie zum Beispiel das Installationsverzeichnis von Anwendungen. Des Weiteren lassen sich durch das relativ kleine Display nicht viele Optionen gleichzeitig darstellen.%
	%
	\item kurzweilig sind oder wenig Zeit benötigen.\newline%kurze Aufgaben welche man schnell lösen will
	Das Handy steht stets zur Disposition des Nutzers und es erfordert nur einen geringen Einstiegsaufwand. Des weiteren fällt das Abarbeiten von langen Aufgaben auf dem Handy schwerer, da die Eingabe langsamer und unpräziser ist.%
	%
	\item lohnenswert sind unterwegs zu lösen\newline%
	Da das Handy mobil ist, sich meistens beim Eigentümer befindet und grundsätzlich über Internet verfügt, ist es möglich Aufgaben unterwegs zu bearbeiten.%
	%
	\item einfach und intuitiv seien sollen.\myTodo\newline%
	Denn erstens wird die Darstellung der App mithilfe von Richtlinien benutzerfreundlicher. Zweitens ist die Bedienung des Handys durch unteranderem die Gesten intuitiver. Und zuletzt muss sich der Nutzer keine Gedanken um die Konfiguration machen, das Betriebssystem, die Software und Hardware wurden bereits passend für das Handy konfiguriert.%
	%
\end{enumerate}

\myNewSection
%Was: Quellen
Diese gesammelten Erkenntnisse lassen sich auch durch Statistiken und Studien deuten. %
%Warum: Studie -> Pc wichtig + Handy simpel
So nutzen zum Beispiel Befragte laut einer Studie, Pc's lieber für wichtige Aufgaben. Währenddessen werden Handys als leichter zu benutzen bewertet, was zu den einfachen und simplen Aufgaben des Handys passen würde\cite{pcVsphone_easyUseVsImportantTask}.\newline%
%allgemeine Nutzerverhalten
Aber auch das allgemeine Nutzerverhalten im Internet deutet darauf hin.
	%Warum: Googel Suche -> Handy simpel + Pc Komplex
	So sind einerseits die Suchanfragen auf Google je nach Gerät anders. Auf dem Pc werden zum Beispiel eher aufwändige und komplexe Kategorien wie Computer, Elektronik, Arbeit, Ausbildung und Wissenschaft angefragt. Während auf dem Handy oft eher nach simplen und kurzen Themen wie Essen, Nachrichten und Sport gesucht wird\cite{pcVsphone_onWebsites_DevicesDistrubition_TimeSpent_Bouncrate_PageViews_Categories}.\newline%
		%Warum: Website Visits
		Das diese Aufgaben auch wirklich einfacher und kurzweiliges sind und Handys dafür benutzt werden, kann anhand der Aufrufe und Dauer von Webseiten beobachten werden. So kommen zum Beispiel 68\% alles Webseiten-Aufrufe von Handys, aber sie machen nur 33\% der Zeit die auf Webseiten verbracht werden aus\cite{pcVsphone_onWebsites_DevicesDistrubition_TimeSpent_Bouncrate_PageViews_Categories}. Das bedeutet also, das Handys für sehr viele kleine Suchanfragen genutzt werden. Pc's werden [bei diesem Thema] hingegen für wenigere Lange Aufgaben genutzt.\newline% 
		%Warum: Handy threeSeconds
		Ein weiteres Indiz dafür, dass die kürze und dementsprechend die Einfachheit von Aufgaben für Handys wichtig sind, ist eine Messung von Google. So verlassen rund 53\% von Handy Nutzern eine Webseite, wenn sie länger als drei Sekunden braucht zu laden\cite{pcVsphone_threeSeconds}.\newline%
		%Warum: Emails
		Ein ähnliches Verhalten lässt sich auch bei Emails feststellen. Laut einem Survey von Adobe werden für Emails mit Arbeitsthemen lieber der Pc genutzt während für private Emails stattdessen zum Handy gegriffen wird\cite{pcVsphone_personalEmailsVsWorkEmails}. Da angenommen wird, dass Arbeit eher mit langwierige Aufgaben welche viele Informationen benötigen verbunden werden, während private Aufgaben eher kurzweilig und einfach sind, wie zum Beispiel das aussuchen von Essen oder dem Schreiben einer Nachricht, stimmt auch dieses [Indiz] mit den Erkenntnissen überein.

%%%%%%%%%%%%%Backup
\myComment{

	\myNewSection%
	\myTextTodo{Maybe nicht zusammenfassen zu Einfache/Schwere sondern einfach nur belegen? + threeSecondLoad erwähnen?}\newline%
	%Was: Schlussfolgerung -> Handys leichte Aufgaben vs Pc's komplexe Aufgaben
	(Schlussfolgernd/Zusammengefasst) und angesichts der Punkte der Aufzählung scheinen Handys für jene Aufgaben gut zu funktionieren welche als einfach, simpel oder leicht zu beschreiben sind. Während die Aufgaben des Pc's eher als aufwändig, komplex oder wichtig zu beschreiben sind.\newline%
	%Begründung/Quellen
		%Warum: Studie -> Pc wichtig + Handy simpel
		Die Erkenntnisse lassen sich auch durch eine Studie bekräftigen. So scheinen die Befragten, Pc's lieber für wichtige Aufgaben zu nutzen. Währenddessen werden Handys als leichter zu benutzen bewertet, was zu den einfachen und simplen Aufgaben des Handys passen würde\cite{pcVsphone_easyUseVsImportantTask}.\newline%
		%allgemeine Nutzerverhalten
		Aber auch das allgemeine Nutzerverhalten im Internet deutet darauf hin.
			%Warum: Googel Suche -> Handy simpel + Pc Komplex
			So sind einerseits die Suchanfragen auf Google je nach Gerät anders. Auf dem Pc werden zum Beispiel eher aufwändige und komplexe Kategorien wie Computer, Elektronik, Arbeit, Ausbildung und Wissenschaft angefragt. Während auf dem Handy oft eher nach simplen und kurzen Aufgaben wie nach Essen, Nachrichten und Sport gesucht wird\cite{pcVsphone_onWebsites_DevicesDistrubition_TimeSpent_Bouncrate_PageViews_Categories}.\newline%
				%Warum: Website Visits
				Das diese Aufgaben auch wirklich einfacher und kurzweiliges sind, kann anhand der Aufrufe und Dauer von Webseiten beobachten werden. So kommen zum Beispiel 68\% alles Webseiten-Aufrufe von Handys, aber sie machen nur 33\% der Zeit die auf Webseiten verbracht werden aus\cite{pcVsphone_onWebsites_DevicesDistrubition_TimeSpent_Bouncrate_PageViews_Categories}.\newline% 
			%Auswirkung:
			Das Handy wird beim Surfen also eher für kurzweilige und schnelle Aufgaben benutzt, während sich auf dem Pc mehr Zeit gelassen wird.\newline%
			%Warum: Emails
			Ein ähnliches Verhalten lässt sich auch bei Emails feststellen. Laut einem Survey von Adobe werden für Emails mit Arbeitsthemen lieber der Pc genutzt während für private Emails stattdessen zum Handy gegriffen wird. Da Arbeit oft mit aufwändigen, komplexen und wichtigen Aufgaben verbunden wird, deutet auch diese Aussage mit der Erkenntnis überein\cite{pcVsphone_personalEmailsVsWorkEmails}.

}
% !TeX encoding = UTF-8
\section{Anforderungen}\label{section:anforderungen}

%\myComment{\subsection*{Stichpunkte2}} 

\myComment{

	\myNewSection
	\textbf{Prezi:}
	\begin{enumerate}
		\item Hier geht es Allgemein um die Frage “was die App überhaupt können soll”. Und sobald ich einige wichtige Anforderungen gesammelt habe, würde ich diese nach Wichtigkeit bewerten, damit ,falls es am Ende Zeitlich knapp wird, die wichtigsten Funktionen bereits Implementiert und konzipiert sind.
		\item Beim Punkt “Vorgehensweise”… geht es darum, dass ich mir Gedanken darüber mache, wie ich die Anforderungen überhaupt erhebe. Zum Beispiel über Typische Features, eigene Ideen, kleine Umfragen, Inspiration durch andere Apps oder über das einlesen in die CLI-Kalender Domäne.
		\item ...
	\end{enumerate}
	
	\myNewSection
	\textbf{Funktionale Anforderungen}
	\begin{enumerate}
		\item Einmal wäre da die \textbf{Verbindung mit Github}. Damit ist gemeint, dass Github als Backend genutzt wird um die CLI-Kalender mit dem App-Kalender zu synchonisieren. Ich könnte mir gut vorstellen, dass ich es als Backendserver wähle, weil Github beliebt, kostenlos und zugänglich ist. Außerdem bietet es auch interessante und nützliche features, welche auch in der App nützlich sein könnten. Wie zum Beispiel Versionskontrolle, branches und die Möglichkeit repos zu teilen)
		\item Die zweite funktionale Anforderung ist: “\textbf{Adapter für etablierte CLI-Kalender}”. Damit ist gemeint, dass die App die Sprache der bereits existierenden CLI-Kalender verstehen soll. Also dass eine Synchronisation zwischen App und Pc möglich sein soll. Das sehe ich einfach mal als Grundanforderung, für die App an.
		\item ...
	\end{enumerate}
	
	\myNewSection
	\textbf{Nicht Funktionale Anforderungen}
	\begin{enumerate}
		\item Dabei wäre einmal \textbf{Wartbarkeit und Erweiterbarkeit} wichtig. Denn Änderungen sollen leicht durchzuführen sein und das Projekt soll auch möglichst Verständlich sein. Ziel davon ist es damit fremde Weiterentwicklung zu ermöglicht und vereinfachen.
		\item Bei der zweiten n.f.Anforderung habe ich die \textbf{Bedienbarkeit} ausgewählt. Denn ich könnte mir gut Vorstellen, dass es wichtig ist, dass die App für Benutzer leicht zu bedienen ist und dass sie die App möglichst effektiv nutzen können  [-> um es möglichst Zugänglich und nützlich zu machen]
	\end{enumerate}
	
} %todo: remove after this sections completion

%%% hidden subsection for a better structure in latex editor: "texifier"
\myComment{\subsection*{Übersicht}}\myCheckmark
%Einleitung
	%Was + Warum:
	Da wir nun wissen, welche Aufgaben auf dem Pc und welche auf dem Handy gut funktionieren, kann als nächstes die Frage behandelt werden \glqq was die zu erstellende App überhaupt können und leisten soll\grqq{} und wie diese Eigenschaften auf die App übertragen werden können. %
	%Warum:
	%Da das [Resultat,Erkenntnisse,Antwort] dieser Frage die darauf folgenden Abschnitte stark beeinflusst, wurde sich damit so früh wie möglich befasst.%
%Übersicht
\newline
\textbf{Übersicht:}
	%Vorgehensweise
	Dabei wird im \secref{subsection:anforderung:vorgehensweise} behandelt auf welche Arten und mit welchen Techniken versucht wird dies Frage zu beantworten. %
	%fA +nfA
	In \secref{subsection:anforderung:nichtFunktionaleAnforderungen} und \secref{subsection:anforderung:funktionaleAnforderungen} werden die gewünschten Eigenschaften sowie einige der Funktionen der App aufgezählt, begründet und bewertet.%
%Ergebnisse
\newline
\textbf{Ergebnisse:} %
%Was: 1: Vorgehensweise
Für die Erhebung wurden sich die Techniken Introspektion, Umfrage und Vergleich entschieden, denn diese schienen für die Arbeit den besten Ausgleich zwischen Informationen und Zeitkosten zu liefern. %
%Was 2: n.f.A
Bei den Anforderungen wurde stets versucht Entscheidung entsprechend der Stärken des Handys und Pc's zu treffen. Dementsprechend wurde die nicht funktionalen Anforderungen \glqq Stärken von Pcs und Handys\grqq{} als am wichtigsten bewertet. Eine Stärke des Handys ist die \glqq Benutzbarkeit\grqq{}, weswegen sie auch nochmal getrennt betrachtet und als zweit wichtigste nicht funktionale Anforderung bewertet wurde. %
%Was 3: f.A.
Zuletzt wurden die Funktionen einer \glqq Verbindung zum Backend\grqq{}, einer \glqq grafische Darstellung für den Kalender\grqq{} sowie einen \glqq Übersetzer für CLI-Terminkalender\grqq{} als [unbedingt] erforderlich eingeschätzt. Aus ihnen besteht also das Grundgerüst der App. Aber auch Funktionen wie das \glqq Erstellen, Bearbeiten und Löschen von Einträgen \grqq{}, \glqq Benachrichtigungen \grqq{} und \glqq Konfigurationen auf dem Pc \grqq{} werden für diese Anwendung als wichtig eingeschätzt.


%Maybe AbschlussPrezi
\myComment{
(---Außerdem sind viele Vorzüge des Handys erst durch optimierung entstanden, zum Beispiel die intuitive Benutzung durch Gesten und Design, daher muss sich überlegt werden wie diese umzusetzen ist (n.f.A.)---)

}

\subsection{Vorgehensweise}\label{subsection:anforderung:vorgehensweise}\myCheckmark
%Warum: Unbewusst was gebaut werden soll
Als Entwickler ist einem oft garnicht bewusst was überhaupt gebaut werden soll, da es schwer zu durchschauen und herauszufinden ist was die Software in dem Anwendungsgebieten  leisten soll.\newline%
%Was: Erhebungstechniken Vergleichen
Deshalb wird sich in diesem Unterabschnitt genau dazu Gedanken gemacht. Es wird überlegt wie die Anforderungen für diese Arbeit am besten erhoben werden können. Dazu werden eine Reihe von Erhebungstechniken verglichen. %
%
\begin{itemize}
	\item \textbf{Introspektion}: %
		%Was: def
		Bei der Introspektion wird versucht selbstständig durch das Nachdenken Anforderungen zu erheben. 
		%Auswirkung: Benutzung
		Da jede Entscheidung diese Arbeit [sowieso] gut überdacht sein sollte, wurde diese Erhebungstechnik durchgängig und fast immer benutzt. %
		%Vorteil: Missverständnisse + kein vorbereitungs-aufwand
		Diese Technik hat zwei Vorteile. Erstens können aus eigenen Überlegungen keine Missverständnisse entstehen. Zweitens benötigt es keine großen Vorbereitungs-Aufwand, [da man sofort loslegen kann]. %
		%Nachteil: Domäne auskennen
		Jedoch muss man sich dafür mit der Domäne auskennen. Wenn man diese nicht versteht, können einen auch keine Ideen einfallen.%
	\item \textbf{User Feedback}: %
		%Warum: Schwer alles zu überblicken
		Als einzelne Person ist es eine schwere Aufgabe alle Anforderungen und Wünsche vieler Nutzer zu erraten und überblicken. Daher sind Erhebungstechniken wie das User Feedback nützlich. %
		%Was + Warum: Userfeedback
		Dabei geben einen Nutzer Feedback über die Software. Damit werden nicht nur existierende Funktionen bewertet, sondern es können auch neue Wünsche und Funktionen geäußert und entdeckt werden. %
		%Was: nicht nutzen
		Jedoch wird diese Technik nicht genutzt. %
			%Nachteil: lauffähige software
			Denn dafür benötigt es eine lauffähige Software und es wird erwartet, dass diese erst zum Ende der Bearbeitungszeit bereit steht.%
				%Trimmed
				%Denn einerseits benötigte es dafür eine lauffähige Software und diese nach jeder neuen Iteration neu zu kompilieren und bereitzustellen wäre ein größer Aufwand. %
				%%Nachteil: Testgruppe finden
				%Außerdem wird vermutet, dass sich das finden einer Testgruppe, welche über mehrere Iterationen die App testet als schwer herausstellen könnte. Wahrscheinlich würde das Interesse nach jeder Iteration mehr schwinden und so verfallen auch die Nutzer.%
	\item \textbf{Umfragen}: %
	%Was: Umfrage 
	Von daher wurde sich stattdessen eine Umfrage entschieden. Dabei werden sich einige Fragen ausgedacht und einmalig an die Zielgruppe gestellt. %
	%Vorteil: Nutzer Findung leichter
	Das hat die Vorteile, dass es vermutlich leichter ist freiwillige Nutzer für ein einmalige Frage, statt eines dauerhaften Testens, zu finden. %
	%Vorteil: Fragen\Antworten können gelenkt werden
	Außerdem können Fragen in beliebige Richtungen stellen kann. So bekommt man Feedback zu gewünschten anstatt zu allen möglichen Themen. % 
	%Nachteil: schriftliches Feedback missverstanden %TODO -> in Fazit
	Jedoch hat diese Technik, genau wie die Vorherige, den Nachteil, dass das schriftliche Feedback anhand fehlendes Kontextes leicht missverstanden werden. %
	%Nachteil: einzelnes Feedback nicht als zu wichtig ansehen %TODO -> in Fazit
	Außerdem muss darauf geachtet werden einzelne Nachrichten nicht als zu wichtig einzustufen. Denn sie könnten zwar für einen Nutzer wichtig sein, aber es muss nicht die eigentliche Zielgruppe repräsentieren.%
	\item \textbf{Inspiration durch Vergleiche}: %
		%Warum: ähnliche Funktionen
		Es existiert zwar noch keine App wie jene welche in dieser Arbeit entwickelt werden soll, jedoch wird angenommen, dass es in dieser App trotzdem einige ähnliche Funktionen und Anforderungen zu konventionellen Kalender-Apps geben wird. %
		%Was: simpel -> intuitive Eindrücke?
		Auch wenn die Anforderungen und Funktionen einer normalen Kalender-App zuerst simpel scheinen, so gilt auch hier, dass man sich nicht auf seine intuitiven Eindrücke verlassen sollte. %
			%Warum:
			Es könnte zum Beispiel bereits Eigenheiten und etablierte Standards in Kalender-Apps geben, welche man ohne Vergleiche nicht finden würde. Oder es gibt als \"selbstverständlich\" angesehene Funktionen, welche deshalb von niemanden angesprochen aber trotzdem erwartet werden. %
		%Schlussfolgerung: Sinnvoll
		Von daher scheint es Sinnvoll sich mindestens für die allgemeine und typischen Funktionen Inspiration zu suchen.%
	\item \textbf{Domänenwissen}: %
		%Was: def
		Durch das einarbeiten in die Domäne CLI-Terminkalender kann bewusst werden was für Funktionen und Anforderungen sie besitzen. %
		%Warum: neue Ideen
		Dieses Wissen könnte zu Inspiration von neue Ideen und Anforderungen für die App führen führen. %
		%Was: dagegen Entschieden
		Jedoch wurde sich gegen das Einarbeiten in die Domäne entschieden. %
			%Warum: Einarbeitungszeit
			Einerseits wird die Einarbeitungszeit als zu hoch eingeschätzt. Denn es existieren viele CLI-Terminkalender und diese lassen sich meistens eher komplex und unterschiedlich bedienen und bieten darüberhinaus noch verschiedene Eigenheiten. %
			%Warum: geringer Informationserwerb
			Andererseits wird der Erwerb an Informationen als zu gering eingeschätzt. So wird nämlich durch die Unterschiede des Pc's und des Handys vermutet, dass die zu erstellende App sich sehr von CLI-Terminkalender unterscheiden wird. Immerhin ist das Ziel auch nicht solch ein Programm zu portieren, sondern die Stärken des Handys und Pc's zu nutzen.%
	\item \textbf{Iterative Develompent}: %
		%Was: def
		Das iterative Arbeiten kann auch als Erhebungstechnik bezeichnet werden. Während jeder Iteration bietet sich die Chance die Anforderungen zu überdenken und sein zuvor neu gelerntes darauf anzuwenden. %
		%Warum: passiert nebenbei 
		Da für diese Arbeit eine Agile-Arbeitsweise betrieben wird, wird diese Technik auch [nebensächlich/währenddessen/dabei] angewendet.%
\end{itemize}

\subsubsection{Umfrage}
Um möglichst Wertvolle und Aussagekräftige Ergebnisse aus der Umfrage zu erzielen haben wir folgende zwei Themen betrachtet: der Standort der Durchführung und die zu stellenden Fragen.

\myNewSection
\textbf{Standort}: Zur Auswahl stehen uns hier die folgenden Möglichkeiten: Die Umfrage in Bekanntenkreis, der Universität, Online-Forums. \newline
Da es sehr wichtig ist, dass wir eine möglichst große Reichweite auf die Zielgruppe haben, fallen die ersten beiden Auswahlmöglichkeiten weg. \newline
Also bleiben uns nur noch die Online-Forums. Auch hier muss sich jedoch erneut die Frage über den Standort gestellt werden: 'In welchen Foren soll die Umfrage veröffentlicht werden'. Dadurch resultiert ein noch viel größeres Spektrum an Auswahlmöglichkeiten. Um dadurch nicht zuviel Zeit in die Suche eines maßgeschneiderten Forums zu verschwenden, wurden ungefähr die ersten 20-Google-Ergebnisse, der Suche 'CLI-Calendar Forum', betrachtet. Dabei erhielten wir folgende Forum Vorschläge: reddit: r/commandline\cite{rCommandLine} , Stack Exchange: Unix \& Linux\cite{unixAndLinux}, archlinux: Forums\cite{archlinux}, Debian User Forums\cite{debianUserForums}, Linux Mint Forums\cite{linuxMintForums}, Puppy Linux Discussion Forum\cite{puppyLinux}. Da sich viele dieser Foren auf ein einzelnes Betriebssystem beschränken schätzen wir die Reichweite als eher gering ein. Das einzige Forum dabei was hinaussticht und allgemeiner angesiegelt ist, ist reddit. \newline
Um die Zielgruppe nicht nur bei CLI Nutzern zu belassen, sondern auch App-liebhaber anzusprechen, wird die gleiche Umfrage auch auf r/androidapps\cite{rAndroidapps} und r/iosapps\cite{rIOSapps} veröffentlicht.

\myNewSection
\textbf{Fragen}: Bei den zu stellenden Fragen wurde überlegt ob sie vordefiniert werden oder ein Offenes Konstrukt genutzt wird. Vordefinierte Fragen haben zwar den Vorteil, dass man gezielte Antworten bekommt, jedoch besteht hier die Möglichkeit, dass sich die Nutzer zu sehr an die Fragen orientieren und dadurch beeinflusst werden und wichtige und interessante Ideen nicht zum Vorschein kommen. Außerdem wurden sich in diesem Punkt der Arbeit noch keine Anforderungen überlegt, was die Erstellung von Fragen erschwierigt und zeitlich kostspielig macht. Daher haben wir uns für Frei-Text Fragen/Antworten entschieden.

\myNewSection
Die Umfragen wurden auf folgenden Seiten veröffentlicht: \myTodo
\subsubsection{Vergleich} %
Ähnlich wie bei der \nameref{subsection:umfrage} werden auch für diese Erhebungstechnik weitere Überlegungen angestellt. Das Ziel dabei ist es, möglichst wertvolle und aussagekräftige Ergebnisse zu erzielen und dabei zugleich zeiteffizient vorzugehen..%
%
%
\newline%
\myNewSection%
%Was es bringen soll
Durch den Vergleich sollen lediglich Inspiration sowie allgemeine und offensichtliche Anforderungen gesammelt werden. %
%Was nicht: Abgrenzung
Das Ziel ist es jedoch nicht, Funktionen und Designs von anderen Apps zu kopieren oder sich von ihnen beeinflussen zu lassen. %
	%Wie dagegen angekommen wird.
	Aus diesem Grund werden nur wenige Apps zum Vergleich herangezogen und diese auch nur kurzzeitig getestet.%
		%Warum: weiterer Punkt: Zeitaufwand
		%[Außerdem] würde das testen weiterer App auch zu viel Aufwand und Zeit kosten.\newline%
\newline%
%Was: Auswahl von Apps
Bei der Auswahl der Apps wurde versucht, solche auszuwählen, die möglichst nützliche Informationen liefern können. %
	%Apple & Google
	Dazu wurden der native Apple iOS Kalender\cite{A_calendarApple} und der Google Kalender\cite{A_calendarGoogle} ausgewählt, da angenommen wird, dass diese Unternehmen aufgrund ihres Erfolgs, ihrer Größe und da sie eigenen Richtlinien für Apps haben \cite{konventionen_guidelinesApple, konventionen_guidelinesGoogle}, besonders sorgfältig bei der Entwicklung dieser Apps vorgegangen sind.\newline%
	%Calendars
	Darüber hinaus wurde eine App aufgrund von Kundenbewertungen ausgewählt. Bei einer solchen App wäre es nämlich beispielsweise möglich, dass sie aufgrund von Funktionen, die in den anderen beiden Apps nicht vorhanden sind, so positiv bewertet wurde. Aus diesem Grund wurde sich für Calendars\cite{A_calendarReviews} entschieden.%
\subsection{Nicht Funktionale Anforderungen}

\subsection{Funktionale Anforderungen}\label{subsection:anforderung:funktionaleAnforderungen}
%Was
Dieser Abschnitt beschäftigt sich mit den funktionalen Anforderungen, also dem, was das System können soll. Wie bereits im \secref{subsection:anforderung:nichtFunktionaleAnforderungen} , wird auch hier aus denselben Gründen die MoSCoW-Priorisierung verwendet.





%Wiederholung
%\myComment{
%
%	\textbf{Vorab/Kurze wiederholung:} In der \nameref{section:zielsetzung} wurde das Ziel mit \glqq das Erstellen einer App welche die Stärken des Pc's und Handys nutzt\grqq beschrieben. Um auch die Vorzüge des Pc's benutzen zu können, wird keine Standalone App entwickelt. Stattdessen baut die zu erstellende App auf die bereits existierenden CLI-Terminkalender auf und kommuniziert mit dem Pc. \myTextTodo{-> beliebtheit+bekanntheit+es muss nur eine App für das Handy entworfen werden, da für den Pc schon alles verfügbar/existiert - damit erste beide fA Sinnvoller/weniger überfüllt}
%
%}


\begin{itemize}
%Must haves
	\item \textbf{M Verbindung mit Backend:} %
		%Was
		Damit das Handy und der Pc zusammen genutzt werden können, müssen sich diese miteinander verbinden können. Dazu soll ein Backend verwendet werden. Die Wahl fiel auf GitHub, da es für diese Arbeit mehrere Vorteile bietet. %
		%Warum: 
			%Funktionen
			Es bietet es einerseits viele interessante Funktionen, die in die App integriert werden könnten, wie beispielsweise die Versionsverwaltung, Zugriffskontrolle und die Möglichkeit von Entwicklungszweigen. %
			%Beliebt
			Darüber hinaus wird vermutet, dass CLI-Terminkalender-Nutzer auch gerne Git nutzen, da es sich dabei um eine beliebte CLI-Anwendung handelt\footnote{Beispielsweise ist Git bereits standardmäßig in Ubuntu vorinstalliert.\cite{nfA_ubuntuManifestGIT}}. Dementsprechend müssten viele Nutzer keine neue Anwendung erlernen, um die App zu nutzen.%
		%
	\item \textbf{M [+ C] Übersetzer für CLI-Terminkalender:} %
		%Was
		Damit Daten verarbeitet und ausgetauscht werden können, ist neben der Verbindung der Geräte auch eine Kommunikation erforderlich. %
			%Parser vs neue CLI Anwendung
			Zur Ermöglichung dieser Kommunikation bieten sich zwei Optionen an: die Erstellung eines neuen CLI-Terminkalenders, der mit der App kommunizieren kann, oder die Erstellung eines Parsers bzw. Übersetzers für bereits existierende CLI-Terminkalender.
				%Entscheidung+Warum:
				Es wurde sich für die zweite Option entschieden, da sich dadurch einerseits auf die App konzentriert werden kann und andererseits mehrere verschiedene CLI-Terminkalender mit der App funktionieren können. Dadurch würde sich die Zielgruppe und der Nutzen der App erhöhen.\newline%
		%Was + Warum: fokus auf einen -> zeit
		Zunächst wird sich jedoch nur auf die Übersetzung für einen CLI-Terminkalender konzentriert, da einer ausreicht, um die restlichen Funktionen der App zu testen und vorzuführen.%
		%Was+Warum: Später weitere -> reichweite
		Weitere CLI-Terminkalender können immer noch zu einem späteren Zeitpunkt hinzugefügt werden, um so die Zielgruppe zu erweitern.%
		%
	\item \textbf{M [+ C] Kalender Darstellung:} %
		%Was+Warum: simpel+kurz+unterwegs -> ansehen von Terminen 
		Eine Aufgabe, die gut zum Handy passt, da es eine kurzweilige und einfache Aufgabe ist, die man durchaus unterwegs lösen möchte, ist das Ansehen von anstehenden Terminen. Dafür benötigt es eine Darstellung für den Kalender. %
		%Was+Warum: Design
		Dabei wurde sich von bereits existierenden und konventionellen Darstellungen inspiriert, um eine möglichst einfache und intuitive Benutzung zu ermöglichen. Die verglichenen Apps zeigten dabei oft mehrere verschiedene Darstellungen für den Kalender an. %
		%Was+Warum: Monatsansicht -> Zeit + benötigt
		Um Zeit zu sparen und da nicht alle Darstellungen zum Benutzen des Kalenders notwendig sind, wurden vorerst nur zwei Darstellungen fokussiert. Es wurde sich für die Monats- und Tagesansicht entschieden, da sie zusammen einen guten Ausgleich zwischen Details und Übersicht bieten.\newline%
		%Was+Warum:  Weitere später -> Details+Übersicht
		Um verschiedene Darstellungen an Details und Übersicht zu ermöglichen können nachwirkend noch weitere Ansichten wie eine Listen-, Wochen- oder Jahresansicht hinzugefügt werden.%

%should haves
	\item \textbf{S Einträge erstellen, bearbeiten, löschen:} %
		%Was+Warum: simpel+kurz+unterwegs -> Einträge hinzufügen, bearbeiten, löschen
		Eine weitere Aufgabe, die gut zum Handy passt, da es eine kurzweilige und einfache Aufgabe ist, die man durchaus unterwegs lösen möchte, ist das Erstellen, Bearbeiten und Löschen von Terminen. %
		%Was+Warum: nicht 1zu1 -> da Syntax mit vielen Sonderzeichen
		Da die Erstellung von CLI-Terminkalendereinträgen eine eigene Syntax mit vielen Sonderzeichen erfordert, sollte diese Funktion jedoch nicht genau wie auf dem PC über ein Terminal und Texteingabe gelöst werden. %
		%Was: Todoliste
		Stattdessen sollen neue Einträge auf dem Handy in einer separaten Erinnerungsliste erstellt werden. Für den PC sollen die Erinnerungen mit Hilfe von GitHub sichtbar gemacht werden. %
			%Nachteil: Doppelarbeit
			Zwar entsteht hierbei der Nachteil, dass eine gewisse Doppelarbeit entsteht, da die Einträge auf dem Handy lediglich als Erinnerung dienen und zu einem späteren Zeitpunkt auf dem PC vervollständigt werden müssen. %
			%Vorteil: Sinnvolle Aufteilung
			Jedoch hat diese Methode den Vorteil, dass die Aufgabe passend den Stärken der Geräte aufgeteilt wird. Das Handy wird für einfache und kleine Einträge genutzt, während auf dem PC die Einträge vervollständigt werden können, um ausführliche Beschreibungen mit komplexen Zeichen und Syntax zu erstellen.%
			%Warum: Einfachheit -> weniger komplexe Gedanken
			%Außerdem ist ein weiterer Vorteil die Einfachheit dieser Lösung. So muss sich dadurch zum Beispiel keine [komplexen] Gedanken über wie mit Push-Konflikten umgegangen wird [gemacht werden].%
			%
	\item \textbf{S Einschränkungen:}	%
		%Einleitung
		Im \secref{section:pcVsPhone} wurden wiederholt Einschränkungen erwähnt, die dem Handy bewusst auferlegt wurden, um so positiven Eigenschaften wie Einfachheit zu betonen %
		%Was+Warum: Begrenzen -> zu kurzweilige und einfache Aufgaben lenken 
		Eine ähnliche Idee wird auch hier verfolgt. Einige Funktionen sollen begrenzt werden, um die Nutzung der App auf die Stärken des Handys zu beschränken, beispielsweise auf kurzweilige und einfache Aufgaben. %
		%Beispiele:
		So könnten zum Beispiel eine maximale Textlänge für neue Einträge auf dem Handy festgelegt werden, um zu vermeiden, dass zu viel auf dem Handy geschrieben wird. Außerdem könnte die maximale Anzahl der Erinnerungseinträge begrenzt werden, um so eine unübersichtlich lange Liste auf dem Handy zu vermeiden. %
		%
	\item \textbf{S Benachrichtigungen:} %
		%Was
		Eine Aufgabe, die sich besser für das Handy eignet als für den PC, sind Erinnerungsbenachrichtigungen für Termine. 
			%Warum:Handy
			Das Handy kann den Nutzer fast immer benachrichtigen, da wie in \secref{section:pcVsPhone} erwähnt wurde, dass davon ausgegangen wird, dass das Handy fast immer beim Nutzer angeschaltet ist. Im Gegensatz dazu ist der PC stationär und wird in der Regel ausgeschaltet, wenn er nicht benutzt wird. Dadurch hat er weniger Möglichkeiten, den Nutzer zu benachrichtigen.\newline%
		%Auswirkungen
		Daher wird in der App die Möglichkeit für Benachrichtigungen von Terminen angeboten. % 
		Allerdings wurde entschieden, diese Funktion nicht für die zuvor erwähnten Erinnerungseinträge anzubieten, da dies dazu führen könnte, dass Erinnerungen als Termine genutzt werden. Dies würde dem zuvor festgelegten Design widersprechen, das vorsieht, dass das Handy für kurze Aufgaben wie Erinnerungen genutzt wird, während auf dem PC komplexere Aufgaben erledigt werden.%
		%
		%
	\item \textbf{S Konfiguration auf dem Pc:} %
		%Was
		Für Optionen, die von möglicherweise Nutzern ändern wollen, wird eine Konfigurationsmöglichkeit bereitgestellt. %
			%Beispiel
			Dazu gehören beispielsweise die Standardzeit für Erinnerungen oder Einstellungen zu den zuvor erwähnten Einschränkungen. %
		%Was+ Warum
		Da in \secref{section:pcVsPhone} festgestellt wurde, dass sich PCs eher als Handys zum Konfigurieren eignen, soll diese Funktion über den PC ermöglicht werden.%
	%
%could haves
	\item \textbf{C Suchfunktion:} %
		%Was
		Um unterwegs schnell gezielte Informationen zu Terminen abrufen zu können, soll eine Suchfunktion implementiert werden. %
		%Warum
		Da die Informationen auch über die grafische Oberfläche oder den PC abgerufen werden können, wird diese Aufgabe jedoch als zweitrangig betrachtet.%
	%
	\item \textbf{C Weitere Kalender Abonnieren \& Teilen:} %
		%Was + Warum:
		Die Funktionen, mehrere Terminkalender als einen darzustellen sowie eigene Terminkalender zu teilen und variable Zugriffsrechte zu bestimmen, sind in allen drei getesteten Apps wiederkehrende Funktionen und scheinen dementsprechend Standardfunktionen in Kalendern zu sein. %Falls die CLI-Terminkalender diese Funktion nicht bereits unterstützen, kann diese mithilfe dieser Anwendung ermöglicht werden.%
		
	\item \textbf{C Offline Funktionen:} % 
		%Was
		Da die App lediglich zum Herunterladen und Hochladen von Dateien eine Internetverbindung benötigt, ist eine permanente Verbindung nicht zwingend erforderlich. %
		%Warum
		Da jedoch fast alle Smartphones heutzutage über eine Internetverbindung verfügen, wird dieser Funktion vorerst weniger Aufmerksamkeit geschenkt.%

%wont haves
	\item \textbf{W Anleitung:} %
		%Was
		Um die Nutzung der App zu vereinfachen, könnte es beim ersten Start eine Einführung mittels Anleitung geben. %
		%Warum:
		Noch besser wäre jedoch, wenn die App so intuitiv gestaltet ist, dass keine Anleitung benötigt wird.%


	\item \textbf{W Commit-History:} %
		%Was
		Eine möglicherweise interessante Funktion, die durch GitHub ermöglicht wird, ist die Darstellung der Commit-History als Graph in der App. Dadurch können die letzten Veränderungen und Versionen betrachtet werden. %
		%Warum
		Da unter anderem für die Darstellung lange Listen und viele Details erforderlich sind, wird diese Funktion jedoch nicht als besonders geeignet für die Nutzung auf dem Handy angesehen. %
		%Auswirkung:
		Daher wird diese Funktion vorerst nicht implementiert.%



\end{itemize}



\section{Technologische Überlegungen}\label{section:technologischeUeberlegungen}

%\input{3_technologischeUeberlegungen/stichpunkte3.tex} %todo: remove after this sections completion

%%% hidden subsection for a better structure in latex editor: "texifier"
\myComment{\subsection*{Übersicht}}
%%%Einleitung: Was+Warum
In diesem Abschnitt werden sich Gedanken zur Auswahl der Technologien gemacht, da diese Auswahl bereits Auswirkungen auf die funktionalen und nicht-funktionalen Anforderungen haben kann.\newline%
%%%Übersicht: xWas
\textbf{Überblick:} %
Zunächst wird behandelt, mithilfe welchen Frameworks die App erstellt werden soll. Anschließend wird sich für einen CLI-Terminkalender, für die Darstellungen der Erinnerungen und für das Dateiformat der Konfigurationsdatei entschieden. Danach wird die Wahl der verwendeten Software und Entwicklungsumgebung begründet und schließlich werden für die Arbeit relevante Pakete ausgewählt.\newline%
%%%%Ergebnis
\textbf{Ergebnisse:} %
	%Framework
	Im ersten \secref{subsection:auswahlDesFrameworks} wurde behandelt, ob die Anwendung als App oder Webseite und mithilfe von Native oder Cross Platform Frameworks erstellt werden soll. Die Wahl fiel dabei erstens auf eine App, da diese in der Regel intuitiver ausfallen können und beliebter sind, und zweitens auf Cross Platform Frameworks, da damit Apps für iOS und Android erstellt werden können. Zuletzt wurden noch die beiden Frameworks Flutter und React Native miteinander verglichen. Aufgrund von Beliebtheit, Performance und der Annahme, dass damit effizient gearbeitet werden kann, fiel die Wahl auf Flutter.\newline%
	%Terminkalender & Konfigurationsdatei
	Die \nameref{section:tech:sub:cli_terminkalender} viel auf When und das \nameref{section:tech:sub:konfigurationsdateiformat} auf JSON, da durch dieser Wahl erhofft wird, da durch diese Entscheidungen erhofft wird, während der Entwicklung ein evaluierbares Produkt zu erstellen.%
	%Erinnerungen
	Weiter wurde für die \nameref{section:tech:sub:darstellung_der_erinnerungen} Issues gewählt, da diese ein passendes Format besitzen und zudem die GitHub-Webseite genutzt werden kann, um die Erinnerungen einzusehen.\newline%
	%Entwicklungsumgebung
	Im darauf folgenden \secref{subsection:entwicklungsumgebung} wurde entschieden, MacOS als Betriebssystem, Android Studio als IDE, GitLab zur Versionsverwaltung und Emulatoren sowie Handys zum Testen zu nutzen. Es wurde auch erwähnt, dass die Versionen dieser Software während der Arbeit nicht verändert werden, um so mögliche Komplikationen zu vermeiden.\newline%
	%Pakete
	Im letzten \secref{subsection:auswahlDerPakete} wurden die für die Arbeit benötigten Pakete nach ihrem Alter, ihrer Beliebtheit und ihrem Update-Verlauf ausgewählt. Die Wahl fiel auf json\_serializable, tests und flutter\_lints, da sie die Lesbarkeit des Codes verbessern und nützliche Funktionen bieten. Weiter wurde das Paket github ausgewählt, da es das einzige verfügbare Paket ist, das eine Schnittstelle zum Backend bietet. Schließlich wurde das Paket syncfusion\_flutter\_calendar ausgewählt, da es eine Kalendardarstellung bereitstellt und somit Zeit gespart wird, da diese nicht selbst implementiert werden muss.%
%
%
%
%
%Todo - Remove
%\myComment{
%	%Old AllÜbersicht 	
%	\myNewSection
%	\myTextTodo{
%	\textbf{Abschnitte der Arbeit}\\
%	%Technologische Überlegungen -> wichtig da erfüllt Anforderungen + erst nach der Erhebung
%	Im darauf folgenden \secref{section:technologischeUeberlegungen} wird sich Gedanken über die Auswahl von Technologien gemacht. Das hat den Grund, da bereits die Auswahl von Technologien Auswirkungen auf Funktionale und nicht Funktionale Anforderungen haben können. Daher ist es auch wichtig, diesen Abschnitt erst nach der Anforderungserhebung zu behandeln.
%		%Beispiel
%		Man stelle sich vor es wird zuerst ein Framework welches bekannt für seine langsame Ausführung ist gewählt und erst zu einen späteren Zeitpunkt wird die Anforderung einer \dq schnelle Performance\dq erhoben. Durch diesem Konflikt müssten die Wahl des Framework neu überdacht werden, was wichtige Bearbeitungszeit verschwenden könnte.\newline%
%	}
%
%}
\subsection{Auswahl des Frameworks}\label{subsection:auswahlDesFrameworks}%
%Einleitung
In diesem Abschnitt wird die Frage behandelt, wie die Anwendung erstellt werden soll. %
%Was: App vs Web.
Dabei bestehen die Optionen die Anwendung als eine Webseite oder als Applikation zu erstellen. %
	%Warum: Plattformunabhängig VS UserPreference, Gesten, Design Richtlinien
	Zwar bieten Webseiten einige Vorteile im Vergleich zu Applikationen, zum Beispiel die Plattformunabhängigkeit. Wie zuvor in \secref{section:pcVsPhone} erwähnt, können Apps jedoch durch Richtlinien und Gesten aber um einiges intuitiver sein und Nutzer scheinen diese generell zu präferieren\cite{pcVsphone_mobileAppVsWebTimeSpent}. %
	%Auswirkung: -> App
	Daher soll die Anwendung als Applikation entwickelt werden. %
%
\newline
\myNewSection
%Was: Framework
Nun kann sich für ein Framework entschieden werden. %
%Was+Def: Native vs Crossplatform
Die erste Wahl liegt dabei zwischen einem Native-Framework oder Cross-Platform-Framework.\newline% 
	%Todo: remove useless beispiel:
		%Dabei sind Native-Frameworks diejenigen, welche die für die Platform spezifischen tools benutzt. Während Cross-Platform-Frameworks ihre eigenen tools anbieten.
		%Warum: Native: Performance
	Generell scheinen Native-Frameworks eine bessere Performance als Cross-Platform-Frameworks zu bieten \cite{tech_performanceReactNativeVsFlutter1, tech_performanceReactNativeVsFlutter2}. %
	%Was: Native: Aussehen
	Außerdem sehen und fühlen sich die nativ erstellte Apps einheitlich mit der Plattform an, da für diese Apps plattformspezifischen Funktionen und Komponenten benutze werden.
		%Todo: remove useless beispiel:
			%, wie zum Beispiel die Schieberegler von iOS [\ref{pic:schieberegler}]. 
		%Warum: intuitive
		Das würde wahrscheinlich zu einer intuitiveren Benutzung für den Nutzer führen, da dies für ihm bereits bekannte Muster und Funktionen wären. %
		%Auswirkung: 
		Da Performance und Benutzbarkeit zwei Anforderungen für diese Arbeit sind scheinen native-SDKs eine gute Wahl für diese Arbeit zu sein.\newline%
	%Was: Crossplatform: Relativierung
	Jedoch wurde sich für eine Cross-Platform-Framework entschieden. %
		%Performance
		Einerseits wird der Performance-Verlust als marginal eingeschätzt, da es sich bei der zu erstellenden App wahrscheinlich um eine simple Anwendung ohne schwierige Berechnungen oder aufwändigen Animationen handeln wird. 
		%Aussehen
		Andererseits kann die grafische Oberfläche auch versucht werden mit Cross-Platform-Frameworks passend für das System zu erstellen. Zwar wäre das ein größerer Aufwand als bei Native-Frameworks,
		%Warum: Zeit + Crossplatform
		aber dafür sind die Anwendungen von Cross-Platform-Frameworks mit iOS und Android kompatibel. Wie in \secref{section:anforderungen} besprochen, ist dies der wichtigste Punkt, um eine große Reichweite zu ermöglichen. Zwar wäre dies auch möglich, indem für iOS und Android jeweils eine eigene Codebasis über native SDKs erstellt würde, jedoch müssten dann auch zwei Codebasen gepflegt werden. Durch die relativ kurze Bearbeitungszeit dieser Arbeit erscheint diese Idee weniger sinnvoll. Stattdessen wird versucht, mithilfe von Cross-Platform-Frameworks möglichst schnell und zeiteffizient eine evaluierbare Anwendung für beide Plattformen zu entwickeln. Falls sich die App später als nützlich und beliebt herausstellt, kann immer noch eine Native-Entwicklung gestartet werden.
	
\myNewSection
%Was: Flutter vs React Native
Zuletzt muss eine Entscheidung für ein konkretes Framework getroffen werden. %
	%Warum: Weiterentwickelbarkeit
	Wie in \nameref{section:anforderungen} erwähnt, ist die Weiterentwickelbarkeit eine wichtige Anforderung. Daher ist es einerseits entscheidend, ein Framework auszuwählen, das möglichst lange unterstützt wird, um eine zukünftige Weiterentwicklung zu gewährleisten.\newline%
	%%Warum: Beliebtheit
	Andererseits muss auch berücksichtigt werden, dass das Framework beliebt ist und von vielen Personen genutzt wird, um die Chance zu erhöhen, dass Interessenten gefunden werden können.\newline%
	%Quelle:
	Basierend auf der Update-Historie und der Sternbewertung auf Github sind React-Native und Flutter die beiden beliebtesten Frameworks für die Cross-Platform-App-Entwicklung\cite{tech_flutterStars, tech_reactNativStars}.\newline%
%Warum: Beliebtheit
Wenn man die beiden Frameworks dementsprechend miteinander vergleicht, scheint Flutter beliebter zu sein. Es hat mit 150.000 Sternen auf GitHub etwa 38\% mehr als React Native\cite{tech_flutterStars, tech_reactNativStars}. Ein ähnliches Ergebnis zeigt sich auch bei Google Trends, da Flutter fast doppelt so viele Suchanfragen wie React Native hat\cite{tech_googleTrendsFlutterVsReactNative}.\newline%
%Warum: performance
Im Bereich der Performance verhält es sich ähnlich. Laut zwei Analysen von inVerita scheint Flutter ressourcensparender und schneller als React Native zu sein\cite{tech_performanceReactNativeVsFlutter1, tech_performanceReactNativeVsFlutter2}.\newline%
%Warum: UI
Dafür bietet React Native den Vorteil, plattformspezifische Komponenten und Funktionen nutzen zu können. Wie zuvor erwähnt wurde, könnte dies zu einer verbesserten Benutzbarkeit führen.
\newline%
%Warum: vordefinierte features
Flutter bietet im Vergleich zu React-Native viele vorgefertigte Komponenten und Features, was letztendlich zur Entscheidung für dieses Framework geführt hat. Während React-Native nur 25 Core-Komponenten hat, besitzt Flutter allein für Animationen bereits 22 Komponenten\cite{tech_componentsFlutter, tech_componentsReactNative}. Diese vordefinierten Komponenten können dabei helfen, zeiteffizient vorzugehen, da sie sonst möglicherweise selbst implementiert werden müssten.%
%
%
%
%
\subsection{CLI-Terminkalenderwahl}\label{section:tech:sub:cli_terminkalender}%
Wie im \secref{subsection:anforderung:funktionaleAnforderungen} erwähnt, soll ein Übersetzer für einen CLI-Terminkalender geschrieben werden. Dazu muss sich vorerst für einen entschieden werden.\newline%
Für die Wahl des CLI-Terminkalenders wurden calcurse\cite{cli_calcurse}, khal\cite{cli_khal}, remind\cite{cli_remind}, When\cite{cli_when} in betracht gezogen und sich [letztendlich] für When entschieden. Dieser beschreibt sich in seiner Dokumentation mit \glqq minimalistic\grqq{} und \glqq It's a very short and simple program\grqq{}\cite{cli_when}. Dementsprechend wird angenommen das es für diesen Terminkalender am leichtesten ist einen passenden Übersetzer zu entwickeln und dies hilft wiederum dabei in der Bearbeitungszeit ein evaluierbares Produkt zu entwickeln.\newline%
%
Falls sich die Anwendung nach der Evaluierung wie Angenommen als nützlich herausstellt, können immer noch weitere Übersetzer hinzugefügt werden. 
\subsection{Darstellung der Erinnerungen}\label{section:tech:sub:darstellung_der_erinnerungen}%
%Was
Wie im \secref{subsection:anforderung:funktionaleAnforderungen} erwähnt, soll es die Funktion geben Erinnerungen zu erstellen. Diese Erinnerungen sollen mithilfe von GitHub zwischen dem Handy und den Pc übertragen werden.\newline%
%Frage
%Die nun zu stellende Frage ist, wie genau die Erinnerungen über GitHub übertragen werden sollen.\newline%
%as Files
Eine Möglichkeit dafür wäre es, die Erinnerungen als Dateien auf das GitHub Repository zu speichern. Die jeweiligen Dateien müssten darauf nur von beiden Geräten herunterladen werden.\newline%
Eine andere Möglichkeit, für welches sich letztendlich auch entschieden wurde, ist es die Erinnerungen als GitHub Issues darzustellen. Dies hätte den Vorteil, dass die Erinnerung neben der App und dem Terminal auch über die Webseite [betrachtbar/ersichtlich] sind. Des Weiteren bietet sich das Format von Issues auch für Erinnerungen an. So bieten sie die Möglichkeit ein Titel, eine Beschreibung, Dateien sowie verschiedene Statuse hinzuzufügen.\newline%
Ein Nachteil davon ist es jedoch, das es nicht möglich ist über das Programm git Issues anzuzeigen. Deshalb benötigt es eine weitere CLI-Anwendung, welche diese Funktion unterstützt. Da GitHub aber bereits ein passendes Programm anbietet\cite{tech_github-cli}, wird dies als kein zu [verheerender] Nachteil eingeschätzt.
\subsection{Konfigurationsdateiformat}\label{section:tech:sub:konfigurationsdateiformat}%
%Was
Wie im \secref{subsection:anforderung:funktionaleAnforderungen} erwähnt, soll es die Möglichkeit geben Einstellung für die Anwendung über den Pc zu konfigurieren. Dazu soll die Konfigurationsdatei mithilfe von GitHub zwischen dem Handy und den Pc übertragen werden.\newline%
%Frage
Eine Frage die sich hier gestellt werden kann ist, was für ein Format die Datei haben soll.\newline%
Passen zu den CLI-Terminkalendern wäre es wahrscheinlich eine Textdatei zu verwenden, da diese zur Unix-Philosophie\cite{tech_unix-philosophie} und damit zu CLI-Programmen passen würde. %
Ferner wurde sich zunächst für eine JSON-Datei entschieden, da sich diese einfacher mit Flutter verwenden lässt. Dies wiederum spart Arbeitszeit ein. %
Weiterhin lässt sich das Dateiformat nach der Bearbeitungszeit zu einem späteren Zeitpunkt immer noch auf eine Textdatei umwandeln.%
\subsection{Entwicklungsumgebung}\label{subsection:entwicklungsumgebung}\myCheckmark%
%Was: Einleitung
In diesem Abschnitt werden die benutzten Software und dessen Versionen genannt. %
%Warum: Einfluss
Denn die Wahl von Software und dessen Versionen kann bereits einen Einfluss auf das Endprodukt ausüben. So wäre es zum Beispiel durch die Wahl von Windows nicht möglich Applikationen für iOS zu entwickelt.\newline%
	%Warum: Nachbilden + Kompatibilität
	Außerdem wird durch die Nennung der Software das Nachbilden der Applikation garantiert. Mit verschiedener Software oder anderen Versionen ist es aus Erfahrung durchaus Vorstellbar, dass es zu Komplikationen kommen kann. So wurde zum Beispiel während der Ausarbeitung Android Studio nach einem Update nicht mehr funktionsfähig. %
		%->Neuste Versionen
		Deswegen wurden zu diesem Zeitpunkt in der Arbeit alle Softwares auf die neuste Version aktualisiert und danach bis zum Ende der Arbeit auch auf diesen Versionen belassen. %Einerseits sollten damit möglichst alle neuen Funktionen und Bugfixes [benutzbar werden]. Aber viel wichtiger noch sollten weitere Komplikationen in der Zukunft damit verhindert werden. \newline%

\begin{enumerate}
	%Was+Warum: Betriebssystem -> iOS Apps
	\item Betriebsystem: Für das Betriebssystem standen Windows und MacOS zur verfügung. Es wurde MacOs entschieden, da auf diesem Betriebsystem für iOS sowie Android Apps entwickelt werden können. Version: macOS Ventura 13.2.1%

	%Was+Warum: IDE
	\item IDE: Flutter empfiehlt unteranderem Visual Studio Code, Android Studio oder Emacs als Editor zu nutzen\cite{tech_ideSuggestion}. Da Android Studio genau wie Flutter beide von Google entwickelt wurden, wird erwartet, dass der Editor besonders gut auf das Framework abgestimmt ist. Deshalb und weil Android Studio für die App-Entwicklung ausgelegt ist, wurde er schlussendlich als Editor ausgewählt. Version: 2022.1.1%

	%Was: Flutter
	\item Framework: Flutter. Mehr dazu in \secref{subsection:auswahlDesFrameworks}. Version: 3.7.3%

	%Was: Versionsverwaltung
	\item Versionsverwaltung: 
		Für die Versionsverwaltung wurde sich für GitLab entschieden. %
		%Warum: Erfahrung und hosting
		Einerseits da im Studium damit bereits Erfahrung gesammelt wurde und andererseits weil die Universität ihre eigene Version dazu bereitstellt.\newline%
		%Was+Warum: Einstellungen -> Verständlich + weiterentwickelndes
		Um ein möglichst Verständliches und einfach zu weiterentwickelndes Projekt zu erstellen, wurden außerdem einige Einstellung in GitLab getroffen. %
			%konventional commits
			So werden einerseits einheitliche und ausschlaggebende Commit-Nachrichten benutzt, damit diese bei späterer Betrachtung verständlich sind. Dabei wurde sich an das bereits existierende Regelwerk \glqq Conventional Commits\grqq{}\cite{tech_conventionalCommits} gehalten. %
			%ci/cd pipelines
			Des Weiteren wurde eine CI/CD pipeline erstellt um Tests automatisch zu überprüfen. Dadurch werden Entwickler automatisch auf mögliche Fehler ihrer Änderungen aufmerksam gemacht. %
			%Version
			Version: git 2.37.1 (Apple Git-137.1)

	%Was+Warum:Emulatoren
	\item Emulatoren: %
		%Warum:Testen
		Zum testen der App werden Emulatoren benutzt. Um dabei sicherzustellen, dass die Anwendung auf Android und iOS läuft sowie auf den neusten und älteren Betriebssystemversionen, wurde jeweils ein iPhone und ein Android Emulator mit der neusten und älteste verfügbaren Version zum testen benutzt. %
		%Was+Warum: Reales Handy -> Performance
		Des Weiteren wurde auch noch ein echtes Handy zum Testen verwendet, da vermutet wird, dass unter realen Bedienungen das testen der Performance aussagekräftiger ist. Dabei wurde mit dem iPhone SE1 gezielt ein relativ altes Handy gewählt. %
			%Warum: Alter
			Dadurch lässt sich nämlich gut prüfen, ob die Anwendung auch von älterer und schwächerer Hardware unterstützt wird. % 
			%Warum: größe
			Außerdem ist das Handy im Gegensatz zu neuen Handys relativ klein mit einer Bildschirmdiagonalen von vier Zoll. Wodurch sich prüfen lässt, ob das Design der App auch auf kleineren Bildschirmen funktioniert. %
		%Versionen
		Versionen: IPhone 14 iOS 16.2 \& 13.7, Pixel 6 Android 13.0 \& 5.0, iPhone SE1 iOS 15.7.3%
		
	\item CLI-Terminalkalender: When. Mehr dazu im \secref{subsections:cli_termincalendar}. Version 1.1.45
		
	%Was+Warum: Weiteres
	\item Weiteres: Die folgende Software wird standardmäßig durch einige der zuvor genanten Technologien benötigt. Dementsprechend wird außer der Nennung der Version nicht näher auf sie eingegangen: Xcode 14.2, Android SDK Platform-Tools: 34.0.0, DevTools: 2.20.1, Dart 2.19.2%
\end{enumerate}

%OS-Umgebung -> Iphone + Android entwickelbar
%Ide: Von Flutter empfohlen + vorherige erfahrung mit intellij umgebungen
%Versionen: neuste aber danach nicht weiter geupdatet... dart, flutter, ide, os, emulatoren
%Emulatoren: echte hardware + viele emulatoren
\subsection{Auswahl der Packages}

%\subsection{Versionsverwaltung}\myCheckmark

\section{Design}\label{section:design}

\input{4_design/stichpunkte4.tex} %todo: remove after this sections completion

%%% hidden subsection for a better structure in latex editor: "texifier"
\myComment{\subsection*{Übersicht}} 
\myTextTodo{
%Design -> starke Auswirkung auf Qualität
\textbf{Abschnitte der Arbeit}\\
Im \secref{section:design} wird sich Gedanken über das äußere sowie innere Design der App gemacht. Anders gesagt also Überlegungen zu der grafischen Oberfläche sowie der Architektur. Dieser Abschnitt wird behandelt weil, beide dieser Punkte starke Auswirkungen auf Qualität der App ausüben können. Dabei würde die grafischen Oberfläche besonders die Qualität für den Endnutzer beeinflussen, da dies das einzige ist mit dem Benutzer interagiert. Gleicherweise würde die Architektur die Qualität für den Entwickler entscheiden, da der Quelltext seine Schnittstelle darstellt.\newline
}
%Einleitung
Wie in \secref{section:pcVsPhone} erwähnt spielt das Design und die Befolgung von Richtlinien für diese einer Anwendung für das Handy eine [wichtige/signifikante] Rolle.\newline%
%Übersicht
\textbf{Übersicht:} %
	%Richtlinien
	Dementsprechend wird sich im ersten Unterabschnitt mit genau solchen Richtlinien beschäftigt. Dabei wird sich zuerst für eine Richtlinie entschieden und sich mit dem Inhalt befasst um so die /nützlichen Abschnitte für die Arbeit zu nennen  %für eine Entschieden -> welcher Inhalt davon wirklich nützlich/verwendbar -> Erkenntnisse/Schlussfolgerung daraus anstatt Blind alles befolgen um so die Entscheidungen zu verstehen und diese auf nicht behandelte themen anzuwenden
	%Design
	Anschließend wird das Design der App vorgestellt und einige Entscheidungen begründet.\newline%
%Ergebnisse
\textbf{Ergebnisse:}\myTodo
\subsection{Konventionen}

%Einleitung
%Wie in \secref{section:pcVsPhone} erwähnt, soll für das Design der Anwendung [Konventione/Richtlinienen] befolgt werden. 
% Abgrenzung -> Was nicht (nicht alle Regeln, nur wichtige / interessante) (in Einleitung?)
%Dabei soll es in diesem Abschnitt nicht darum gehen alle Regeln der Konvention zu nennen, sondern einige wichtige welche sich in dem Design der Anwendung wiederfindet und diese begründen...? \myTodo


%Trimmed
\myComment{

	\myNewSection
	%Einleitung
	Wie in \secref{section:pcVsPhone} erwähnt, soll für das Design der Anwendung [Konventione/Richtlinienen] befolgt werden. 
	
	\myNewSection
	% Abgrenzung -> Was nicht (nicht alle Regeln, nur wichtige / interessante) (in Einleitung?)
	Dabei soll es in diesem Abschnitt nicht darum gehen alle Regeln der Konvention zu nennen, sondern einige wichtige welche sich in dem Design der Anwendung wiederfindet und diese begründen...? \myTodo
	
	\myNewSection
	%Auswahl von Konvention
		%Was: Mehrere Konventionen
		Es stehen mehrere verschiedene [Richtlinien] für Apps zur verfügung. So zum Beispiel eine für iOS von Apple und eine für Android von Google \cite{konventionen_guidelinesApple, konventionen_guidelinesGoogle}. %
		%Was+Warum: Entscheiden
		Jedoch wird sich für eine einzelne [Richtlinie] entschieden, da unterschiedliche [Richtlinien] verschiedene Regeln nennen können und sich so gegebenenfalls sogar widersprechen könnten. %
			%Beispiel
			So bietet Google zum Beispiel eine eigene Seite und Regeln für \glqq Floating action buttons\grqq{} während diese in den Apple [Guidelines] nie erwähnt werden und daher wahrscheinlich auch nicht erwünscht sind. %
		%Was+Warum: Auswahl Apple -> eigene Präferenz(leichter+intuitiver)
		Aus eigener Präferenz, da die Apps dieser Platform noch etwas intuitiver und leichter zu bedienen scheinen, wurde sich für die [Richtlinien] von Apple entschieden.%
		
	\myNewSection
	%Richtlinien: Inhalt + Was wurde betrachtet + Was nicht
		%Was: Woraus besteht die Richtlinie
		Die [Richtlinie] von Apple bietet fünf Abschnitte welche unteranderem von einzelnen Komponenten handeln, wie zum Beispiel Textfelder und Knöpfe, aber auch von Grundlagen und Mustern, welche allgemeine Regeln für die Erstellung von Apps liefern. Zusammen besteht die [Konvention] aus rund 148 Einträge. %
		%Was: davon interessant
		Davon schienen vorerst etwas mehr als die Hälfte als [passend/nützlich/interessant] für die Arbeit. Bei [genauerer/intensiverer] Betrachtung stellten sich 34 dieser Einträge als [wirklich passend] heraus, da sie Funktionen und Anforderungen behandeln, welche in die Arbeit einfließen sollen. %
		%Was: davon uninteressant
		Die meisten restlichen Einträge handeln von Funktionen welche nicht in die Anwendung eingebaut werden sollen, wie zum Beispiel eine Tastaturbedienung oder NFC-Funktionalität\cite{konventionen_keyboard, konventionen_nfc}. Einige andere Einträge welche eigentlich nützliche Funktionen für diese Arbeit nennen, wie zum Beispiel ein Nachtmodus oder Siri Unterstützung, wurden [aus Zeitmangel/ wegen der begrenzten Arbeitszeit] vorerst [übersprungen]\cite{konventionen_darkmode,konventionen_siri}.%
		
	\myNewSection
	%Erkenntnisse
		%Einleitung
		Durch das Durchlesen der vielen Regeln und Richtlinien konnte sich ein Muster erkennen lassen, was die [Auswirkung/Ziele] dieser Regeln und Richtlinien deutet. %
			%Konsistenz
			So ist ein oft erwähntes [Theme] die Konsistenz. In der App benutzte Design entscheidungen sollten konsistent in der ganzen App beibehalten werden und wo möglich sollten [system/apple] [definierte/vorgegebene] [Einstellungen] übernommen werden. So werden zum Beispiel die Übernahme von Gesten\cite{konventionen_accessibility}, Schrift\cite{konventionen_typography} und Farben\cite{konventionen_color} empfohlen.
			%Little info
			Ein weiteres oft erwähntes [Theme] ist es die Darstellung von wenig Informationen. Weniger Informationen \glqq hilft dabei Leuten sich bei ihrer Aufgabe zu fokusieren\grqq{}\cite{konventionen_platformIOS}, daher sollten unter anderem \glqq möglichst wenig Wörter genutzt werden\grqq{}\cite{konventionen_writing} und \glqq wichtige Information auch so Dargestellt werden\grqq{}\cite{konventionen_layout}.
			%simple&intuitive
			Das Hauptziel der Regeln und auch der beiden zuvor genannten [Themes] scheint es aber zu sein, die App intuitiv und simpel [zu machen]. So lässt sich dieses [Theme] in den Regeln am meisten wiederfinden. Unteranderem wird die effektivste App Erfahrung als intuitiv beschrieben\cite{konventionen_offeringHelp}, das nutzen von Gesten wird empfohlen\cite{konventionen_accessibility} und Größe, Farbe und Font von Text und Icons sollen benutzt werden um dessen Bedeutung zu vermitteln\cite{konventionen_icons, konventionen_typography}. 
		%->pcVsPhone
		[Interessant/Nennenswert] hierbei ist es, dass diese drei [Themen] sich mit den Erkenntnissen und Vermutungen aus \secref{section:pcVsPhone} übereinstimmen. So wurde zuvor zum Beispiel die [Simpelheit und Intuitivität] als eine Stärke des Handys benannt und die Darstellung von wenig Informationen als eine Anforderung für Aufgaben auf dem Handy.
		%Weiteres:
		Des Weiteren gibt es auch viele Regeln welche weitere dieser Vermutungen und Erkenntnisse weiter [bekräftigen/unterstützen].
			%Wenig Konfig + Eingabe
			So ratet Apple unteranderem einerseits von Texteingaben und einer hohen Anzahl an Einstellungen ab\cite{konventionen_enteringDate,konventionen_settings}. 
			%kurzer einstiegsaufwand
			Während sie andererseits mit Regeln wie \glqq Ask for initial setup information only when necessary\grqq{}\cite{konventionen_launching}, \glqq Show content as soon as possible\grqq{}\cite{konventionen_loading} und \glqq Delay sign-in for as long as possible\grqq{}\cite{konventionen_managing-accounts} der Einstiegsaufwand und damit auch die Kürze von Aufgaben auf dem Handys verringern.

}
\subsection{App Design}
\subsection{Architektur}
\subsection{Objekt Design}
\section{Implementierung}

\input{5_implementierung/stichpunkte5.tex} %todo: remove after this sections completion

%%% hidden subsection for a better structure in latex editor: "texifier"
\myComment{\subsection*{Übersicht}}%
%Einleitung
	Da nun die Anforderungen und das Design der Anwendung bekannt sind sowie die Technologie, mit der sie erstellt werden soll, kann sich der vorliegende Abschnitt mit der Implementierung befassen.
	%Begrenzung: Was&WasNicht
	Dabei wird nicht die gesamte Implementierung dargestellt, da dies den Umfang dieser Arbeit sprengen würde. Stattdessen werden die wichtigsten Komponenten der Anwendung sowie die während der Implementierung getroffenen Entscheidungen und Erkenntnisse behandelt.%
%Übersicht
\newline%
\textbf{Überblick:}\newline%
Zunächst werden einige Eigenheiten und Besonderheiten während der Implementation des Parsers und der Verbindung zur Datenbank genannt. Anschließend wird erläutert, wie die Qualität der Anwendung und des Programmcodes sichergestellt wurde. Zum Schluss werden einige gezielt genutzte Entwurfsmuster genannt und ihre Verwendung begründet.%
%Zusammenfassung
\newline%
\textbf{Erkenntnisse:}\newline%
	%Parser
	Für den Parser wurde generell versucht, sich möglichst immer an die Vorgaben der Dokumentation zu halten, damit Nutzer ihre bereits bestehenden Kalenderdateien nicht anpassen müssen. Dabei gab es jedoch zwei Fälle, in denen von der Dokumentation abgewichen wurde. % 
		%Zeitangabe
		Zum einen bei der Zeitangabe, da angenommen wurde, dass die Anwendung von dieser Änderung profitieren würde. So bietet die Anwendung zusätzlich zur Interpretation eines Startzeitpunkts auch die Funktion, den Endzeitpunkt zu verstehen und anzuzeigen. %
		%Fehler
		Zum anderen wurden Syntax und Fehleingaben der Kalenderdatei anders interpretiert, da sie beim When-Programm inkonsistent zu sein scheinen. %
		%Variabeln&Operationen
		Zuletzt wurde sich aufgrund der begrenzten Zeit vorerst darauf beschränkt, nur die wichtigsten Operationen und Variablen des CLI-Terminkalenders zu unterstützen.%
	\newline%
	%Datenbank
	Bei der Verbindung zur Datenbank mussten besondere Vorkehrungen für die Authentifizierung und das Herunterladen von Dateien getroffen werden, da die Endpunkte der API diese Funktionen nicht zufriedenstellend erfüllen. Zudem wurde bei der Darstellung der Erinnerungen in der Datenbank versucht, ein Format zu wählen, das sowohl für die WebView als auch das CLI geeignet ist.%
	\newline%
	%Qualität
	Um die Stabilität der Anwendung sicherzustellen, wurden automatische Rückfalltests für den Parser und die Datenbankverbindung durchgeführt, während die grafische Oberfläche manuell getestet wurde %
	Zur Verbesserung der Lesbarkeit des Quellcodes wurden Modularisierung, Zugriffsmodifikatoren und eine eigene Kommentarstruktur angewendet.%
	\newline
	%Entwurfsmuster
	Für die Implementierung der Datenbankverbindung und des Parsers wurden die Singleton- und Strategie-Patterns sowie eine ablagebasierte Struktur verwendet, um den Programmcode anpassungsfähiger und übersichtlicher zu gestalten. Darüber hinaus wurde das Proxy-Pattern eingesetzt, um geladene Dateien zwischenzuspeichern und somit das ständige Neuladen und Neuberechnen durch Datenbank und Parser zu verhindern.%
%
%
%
%
%---Old-Rephrased---
%\myTextTodo{
%%Implementierung - Was + Warum wenig Implementierung wiedergegeben
%\textbf{Abschnitte der Arbeit}\\
%An dieser Stelle sollten die wichtigsten Erkenntnisse erhoben worden sein. Nun gehen wir über zum \secref{section:implementierung}. Hier wird betrachtet wie die Software aufgebaut sein soll. Also geht es auch hier unter anderem um das \dq innere Design\dq.\newline%
%Die Implementierung, ein Großteil der eigentlichen Arbeit von diesem Kapitel, wird nicht wiedergegeben. Jede einzelne Entscheidung und Erkenntnis der Implementierung zu erwähnen und begründen würde nicht nur den Rahmen sprengen, sonder auch sehr ermüdend für den Leser werden. Daher werden nur die wichtigsten Entscheidungen, Schwierigkeiten und Erkenntnisse erwähnt.\newline%
%}
%
%\myTextTodo{
%Aufzählung, Alternativen, Entscheidungen, \\
%LEITFADEN: 1.Was ist die Eigenschaft, 2.Warum ist es wichtig, 3.Wie wird es umgesetzt\\
%} %%%
%
%
%\myNewSection
%auto setup -> einfachheit für das handy -> anforderungen\newline
%
%\myNewSection
%Github config file:\newline
%json weil vs text. erst json aus einfachheit später auch noch txt files möglich da dies besser zu cli terminkalendar passen würde
%mögliche optionen...
%konfiguration welche auf dem handy nötig ist: token+repoName+configPath
%kommenatre in json nicht möglich zum erklären der variabeln. Lösung: erklärung in den variabel namen vs extra kommentarVariabeln = 'string'.
%\newline
%config file: besonders lange variabeln namen. normalerweise nicht sinnvoll und da kürzere beim programmieren genau so verständlich seien können. da aber der enduser nicht das gleiche wissen wie ein interner entwickler der app hat, wurden lange variabel namen genutzt, damit die bedeutung klar wird. + json akzeptiert keine commantare.
\subsection{Verbindung zur Datenbank} %
%Einleitung: 
Wie bereits erwähnt, wurde für die Datenbankverbindung eine vorhandene API verwendet. Im Folgenden werden Schwierigkeiten und Besonderheiten bei der Verwendung dieser aufgeführt.
%
%
%Zusammenfassung: Für die Authentizierung und das herrunterladen von Dateien mussten besondere vorkehrungen unternommen werden, da die Endpunkte der Api diese Funktionen nicht zufriedenstellend erfüllen. Bei der Darstellung der Todos bzw Issues wurde versucht eine für die WebView und das CLI passendes Format zu wählen.
%
%
%
%
%
\subsubsection{Authentikation \& Berechtigungen}%
%
Die Authentizierung über einen Benutzernamen und ein Passwort wird von der Api nicht unterstützt\cite{imp_github_userPasswordAuthentication}. Deshalb wird die authentifizierung mit tokens benutzt. %
Dies stellte aber kein Problem dar, da die Verwendung von Tokens ohnehin bevorzugt wird. Denn dadurch kann der Nutzer sicher sein, dass die Anwendung mit den gegebenen Zugriffsrechten nichts Schädliches ausführen kann.%
%
\newline%
Zwei der verwendeten Funktionen der API erfordern besondere Zugriffsrechte. Das Erstellen eines Repositorys erfordert die \glqq repo\grqq{} Berechtigung\cite{imp_github_createRepo}, während das Erstellen und Aktualisieren von Dateien die \glqq workflow\grqq{} Berechtigung erfordert\cite{imp_github_createFile}. Um den Benutzer über diese Informationen zu informieren, muss das Token auf diese Berechtigungen getestet werden. Jedoch unterstützt die API diese Funktion nicht mehr\cite{imp_github_tokensScopeDiscontinued}. Stattdessen bietet GitHub jedoch die Möglichkeit, sich die Berechtigungen über die Antwort einer URL anzeigen zu lassen\cite{imp_github_tokensScopeCurl}. Um dies in der Anwendung zu nutzen, wird die URL mit curl aufgerufen und die Antwort nach den passenden Berechtigungen geparst.%
\newline%
In In der Dokumentation konnte zwar nichts dazu gefunden werden, jedoch hat sich beim Testen herausgestellt, dass diese Methode, um die Berechtigungen zu erlangen, nur von den \glqq classic tokens\grqq{} unterstützt wird. Für die neueren \glqq beta tokens\grqq{} sendet der Server keine Antwort zurück. Daher werden derzeit \glqq beta tokens\grqq{} in der App nicht unterstützt.%
%%
%%
%%
%%
%%
\subsubsection{Dateien herrunterladen}%
%Was: Endpoint
Die API bietet einen Endpunkt an, mit dem Dateien direkt aus einem Repository heruntergeladen werden können\cite{imp_github_1mb100mbDownloadFile}. %
	%Warum
	Allerdings fügt dieser Endpunkt in Flutter einige nicht konventionelle Zeichen hinzu, was zu ungültigen heruntergeladenen Dateien führt. Durch Tests wurde festgestellt, dass es sich dabei um Leerzeichen und ein \textbackslash n-Zeichen handelt. Wenn diese entfernt werden, kann die Datei normal verwendet werden.%
\newline%
%Was:Filesize
Beim weiteren Testen wurde festgestellt, dass dieser Endpunkt nur Dateien bis zu einer Größe von maximal 1 MB unterstützt. Diese Beschränkung ist aus Erfahrung zu klein, wenn Fotos, Videos und Audio heruntergeladen werden sollen. %
%Auswirkung
Dementsprechend wird der Endpunkt nicht mehr direkt genutzt. Stattdessen werden die Dateien über den GitHub-Link heruntergeladen. Dadurch können Dateien mit einer Größe von bis zu 100 MB heruntergeladen werden.%
%%
%%
%%
%%
%%
\subsubsection{Issues}%
%Was: Issues -> App, CLI, Webview
Wie zuvor erwähnt, sollen Erinnerungen über GitHub Issues ermöglicht werden. Dadurch ist es möglich, die Listen über die App, das CLI oder die Webview anzuzeigen und zu bearbeiten.%
%Was:Unsupported
Leider ist es nicht möglich, über die API Dateien zu Issues hinzuzufügen\cite{imp_github_issueFilesUnsupported}. %
	%Auswirkung: herrausfinden
	Um herauszufinden, wie Dateien dennoch zu Issues hinzugefügt werden können, wurde versucht herrauszufinden wie Daten über die Webansicht zu issues hinzugefügt werden. Es stellte sich heraus, dass die Dateien nicht direkt in den Issues gespeichert werden, sondern lediglich durch einen Link referenziert werden. %
	%Auswirkung: Funktion erstellen
	Mit dieser Information konnte eine passende Funktion erstellt werden: Zunächst wird die GitHub API genutzt, um ein Issue zu erstellen. Anschließend wird die Datei, die dem Issue hinzugefügt werden soll, auf das entsprechende Repository hochgeladen. Schließlich wird ein Link zur Datei dem Issue hinzugefügt, wodurch die Datei im Webview als solche angezeigt wird. %
\newline%
%Was: Embedded
Die GitHub-Webansicht unterstützt Dateien eingebettet darzustellen, anstatt sie nur als Link anzuzeigen. Beispielsweise wird bei einer PNG-Datei direkt das Bild angezeigt und bei einer MP4-Datei wird das entsprechende Video wiedergegeben. %
%Warum: (Zeitersparnis)
Dies hat den Vorteil, dass es dem Nutzer Aufwand spart, da sie die Datei direkt in der Webansicht betrachten können, ohne sie zuvor herunterladen zu müssen.%
%Auswirkung:
Daher sollen auch die mit der App erstellten Issues von dieser Funktion profitieren. Um herauszufinden, wie die WebView Dateien einbetten kann, wurden zahlreiche Tests durchgeführt und die Dokumentation zurate gezogen\cite{imp_github_syntaxing}. Dabei stellte sich heraus, dass einige der in der Dokumentation erwähnten Funktionen für Issues nicht funktionieren, wie zum Beispiel \glqq Relative Link\grqq{}, die in Markdown-Dateien\footnote{Eine reine Textdatei welche auf GitHub syntax unterstützt um den Text zu formatieren} verwendet werden können, aber nicht in Issues.
Bei den Tests konnte erfolgreich die Einbindung von Bildern in die WebView reproduziert werden, jedoch war es nicht möglich, Videos einzubetten. Es scheint, dass Videos nur dann in die WebView eingebettet werden können, wenn sie über eine interne GitHub-URL der Form \glqq https://user-images.githubusercontent.com/...\grqq{} verfügbar sind. Diese URLs können jedoch anscheinend nur generiert werden, wenn die Dateien über die WebView hinzugefügt wurden. Daher ist es derzeit nicht möglich, eingebettete Videos in der WebView mit der App zu erstellen.%
\newline
\myNewSection
%Was+Warum:CLI
Um die Issues auch im CLI ansprechend darzustellen, wurde versucht, bei deren Erstellung ein geeignetes Format zu verwenden.%
\newline%
Ein Issue besteht immer aus einem Titel und einer Beschreibung. Die Beschreibung wird so formatiert, dass die erste Zeile die Beschreibung der Erinnerung enthält und die folgenden Zeilen die Dateien.%
\newline%
Die Dateien müssen für die WebView im Format \glqq [ ]\{url\}\grqq{} gekennzeichnet werden. Innerhalb der \glqq[ ]\grqq{} wird der Dateipfad angegeben, um eine weitere Referenz auf die Datei für die CLI bereitzustellen. %
Um eine klare Struktur im Repository zu gewährleisten, werden die Dateien der Issues dabei in Ordnern der Form \glqq issue\{issueNummer\}\grqq{} gespeichert.%
%Beispiel
\myTextTodo{pic beispiel issue WebView vs CLI vs App}
%%
%%
%%
%%
%%
\myComment{

%\subsubsection{Endpunkte}
%Von der Api wurden folgende Endpunkte benutzt. createRepository getRepository
%repo: für create repository: https://docs.github.com/en/rest/repos/repos?apiVersion=2022-11-28#create-a-repository-for-the-authenticated-user
%workflow: für create und update files in repo: https://docs.github.com/en/rest/repos/contents?apiVersion=2022-11-28#create-a-file
%
%Weitere benutzte funktionen der Api(benötigen keine extra scopes):
%get repository content: https://docs.github.com/en/rest/repos/contents?apiVersion=2022-11-28#get-repository-content
%List issues assigned to the authenticated user :https://docs.github.com/en/rest/issues/issues?apiVersion=2022-11-28#list-issues-assigned-to-the-authenticated-user
%create an issue: https://docs.github.com/en/rest/issues/issues?apiVersion=2022-11-28#create-an-issue
%update an issue: https://docs.github.com/en/rest/issues/issues?apiVersion=2022-11-28#update-an-issue
 

}
\subsection{Qualitätssicherung}\label{section:qualitaetssicherung}
%
%
%
%
\subsubsection{Tests}
%Einleitung:
Um sicherzustellen, dass die Anwendung stabil läuft und keine Defekte enthält, werden mehrere Tests durchgeführt. %
%Was + Warum: Parser+Db viel getestet -> wichtig
Dabei wird der Parsers und die Verbindung zur Datenbank am ausgiebigsten getestet, da sie die Grundfunktionen der Anwendung darstellen und somit als kritische Bereiche angesehen werden. %
%Warum+Was: ManuellesTesten
Da diese beiden Funktionen viele Komponenten enthalten, die oft geändert werden, wäre das manuelle Testen zu zeitaufwendig und müsste nach jeder kleinen Änderung erneut durchgeführt werden. %
	%Auswirkung: Rückfalltests
	Deshalb wurde stattdessen beschlossen, automatisierte Rückfalltests durchzuführen. %
		%Pro/Con:
		Obwohl die Erstellung von automatisierten Rückfalltests anfangs mehr Aufwand erfordert, lohnt es sich, sobald die Komponenten häufiger getestet werden müssen. Außerdem bieten die Tests eine Art von Dokumentation, da sie zeigen, wie die Komponenten funktionieren sollten. Deshalb werden die Rückfalltests auch als nützlich für die Weiterentwicklung der Anwendung eingeschätzt.\newline%
%Was+Warum: Bottom-Up-Integrationstests
Um sicherzustellen, dass die Tests zeigen, ob ein Defekt von der getesteten Komponente oder von einer von ihr aufgerufenen Funktion verursacht wird, werden die Tests nach dem Schema der Bottom-Up-Integrationstests geschrieben. Das bedeutet, dass zuerst die Module und Komponenten getestet werden, für die alles, was sie aufrufen, bereits getestet wurde.\newline%
%Was:Eingabe
Damit die Tests die Komponenten möglichst vollständig auf Defekte prüft, wurde bei der Erstellung der Tests versucht durch die Eingaben das Abdeckungskriterium der Bedingungsüberdeckung zu erreichen. Außerdem wurden stets Randfälle wie leere Eingaben oder völlig unsinnige Eingaben getestet, da dies aus Erfahrung oft zu Defekten führt.\newline%
%Coverage
Die Ausführung der Tests mit Code-Coverage-Analyse ergab, dass im Parser 70\% und bei der Datenbankverbindung 90\% aller Codezeilen getestet werden.\newline%
%Grafische Oberfläche
Die Funktionen der grafischen Oberfläche wurden hingegen manuell getestet, da sich diese besser für manuelle Tests eignen und somit zeiteffektiver sind als automatisierte Tests.%
%
%
%
%
%
\subsubsection{Verständlichkeit}
%Einleitung
%Was: Trennung von Belangen und Modularität
Um den Quelltext möglichst verständlich zu gestalten, wurde generell versucht, Komponenten möglichst modular zu gestalten und so aufzuteilen, dass jede Komponente immer genau eine einzige einzigartige Aufgabe hat. %
%Visability
Des Weiteren wurden Zugriffsmodifikatoren für Methoden gesetzt. Dadurch soll deutlich werden, welche Methoden der Komponente ausschließlich für interne Funktionen genutzt werden und welche dem Klienten zur Verfügung stehen.%
%Was: Kommentare
Außerdem wurden für wichtige Funktionen, wie beispielsweise alle Funktionen aus der Datenbankverbindung und dem Parser, Kommentare verfasst. %
	%Nachteile + Vorteil:
	Kommentare haben zwar den Nachteil, dass sie bei Änderungen des betreffenden Codes aktualisiert werden müssen, jedoch können sie die Verständlichkeit verbessern. Ein kurzer Satz kann beispielsweise die Funktion einer ansonsten großen und schwer lesbaren Methode erläutern. %
	%Warum + Was: Format
	Um genau solche nützliche Kommentare zu schreiben, wurde ein selbst vorgegebenes Format verwendet, das sich während der Programmierung als hilfreich erwiesen hat. Das Format besteht aus den Feldern \glqq def\grqq{}, \glqq purpose\grqq{}, \glqq assert\grqq{}, \glqq expect\grqq{}, \glqq return\grqq{} und \glqq example\grqq{}.\glqq Def\grqq{} beschreibt die Funktion, \glqq purpose\grqq{} gibt an, wofür die Funktion benötigt wird, \glqq assert\grqq{} zeigt was die Funktion voraussetzt, \glqq expect\grqq{} gibt an, was die Funktion erwartet, aber nicht unbedingt voraussetzt und \glqq example\grqq{} enthält ein Beispiel. Die Felder werden nicht immer alle benutzt, sondern nur dann, wenn es als nützlich erscheint.\newline%
		%assert (meistens preconditions?)
		%Da mit dem kommentar assert preconditions gegeben sind, wurde dies gleichzeitig auch mit der mithilfe der Flutter gegebenen assert Funktion während der laufzeit zugesichert. %

\myComment{

%\subsubsection{Exceptions}
%Um dem Endbenutzer eine gute Qualität 
%exceptions & errors: für die verbindung mit der api wurde versucht möglichst alle möglichen errors/exceptions abzufangen. dies wurde durch ausgiebieges testen versucht zu bestätigen. wenn es nämlich während des handy nutzens zu einen error kommen würde, würde dass die benutzung der app behindern / große fehler abbilden. z.b. würde die api nach 1000 abrufen fehler werfen, bis eine neue verbindung/ip hergestellt würde/ softcap (link). lieber sollten diese fehler geplant im handy angezeigt werden anstatt dass diese abstürzt oder einen flutter code error angezeigt bekommt. dementsprechend werden alle api calls exeptions mit try()catch() abgefangen. Jedoch sollte die möglichkeit für eine solche exception ziemlich niedrig sein, da beim testen sehr viele dieser calls gemacht wurden, es aber insgesamt während der test und programmieren nur einmal dazu kam (art von stresstest). zum beispiel würde die github api bei ... eine exception werfen, dabei sollte die app aber nicht abstürzen, dementsprechend gibt die implementierte verbindung zur datenbank einfach einen bool false aus anstatt die expection zu werfen. ratelimit: the rate limit is 1,000: https://docs.github.com/en/rest/overview/resources-in-the-rest-api?apiVersion=2022-11-28
%


%%Repo
%Auch das Repository wurde versucht Lesbar zu gestalten, damit auch dieses von anderen gelesen und nachvollzogen werden kann. Dafür wurde einerseits für die Darstellung der Commit-History die Lineare Option eingestellt und befolgt und andererseits für die Commit-Nachrichten ein einheitliches format gehalten\cite{}.
%%pipeline+linting
%Des Weiteren wurde wie in \nameref{subsection:anforderung:nichtFunktionaleAnforderungen} erwähnt ein Linter benutzt um den Quellcode in ein einheitliches Format zu bringen. Damit die Regeln auch durchgängig eingehalten werden und nichts ausversehen auf das Repository gelangt, wurde eine Pipeline zu GitLab hinzugefügt, welches [gepushte] Neuerungen überprüft und ablehnt wenn Regeln vom linter nicht befolgt werden. %
}


\subsection{Entwurfsmuster}: %
%Was+Warum:
Während der Implementation wurden sich bewusst für die Verwendung einiger Entwurfsmuster entschieden, da sie in den jeweiligen Situationen nützlich erscheinen. %
%Übersicht:
Folgend werden die Entwurfsmuster aufgezählt und dessen Verwendungsgrund begründet.%
%
%
%
%
%
%Zusammenfassung: Für die Implementation der Verbindung zur Datenbank sowie dem Parser wurden Singelton und Strategie patterns genutzt, sowie eine Ablage-basiert Struktur verwendet um so den Programmcode anpassungsfähiger und übersichtlicher zu gestalten. Darüber hinaus wurde das Proxy Pattern verwendet um geladene Dateien zwischen zu speichern und so das ständige neu laden und berechnen von der Datenbank und dem Parser zu verhindern.
%
%
%
%
%
\subsubsection{Fassade}
Durch das Umschließen der API mit einem Fassadenobjekt wird die Schnittstelle für den Nutzer der Datenbankverbindung vereinfacht und irrelevante Informationen werden verborgen. Dadurch konnten beispielsweise die erforderlichen Zwischenschritte zum Herunterladen einer Datei von GitHub verborgen werden.
%
%
%
%
%
%
\subsubsection{Singelton}%
%Was: 
Während der Implementierung wurde versucht, nicht auf das Singleton-Pattern zurückzugreifen. 
	%Warum: Vor/Nachteile:
	Obwohl es die Programmierung erleichtert, indem es den Zugriff auf ein Objekt von überall aus ermöglicht, hat es auch Nachteile. %
	Durch Der globale Zugriff auf das Objekt erschwert die Nachvollziehbarkeit, welche Komponenten Zugriff auf das Objekt benötigen, und führt so unter anderem zu einer Verkomplizierung der Fehlerbehebung. %
%Auswirkung:
Trotzdem wurde für die Datenbank das Singleton-Pattern verwendet, da die Datenbank sonst zwischen vielen Komponenten übergeben werden müsste und die Struktur des Programmcodes darunter leiden würde.\newline%
Es folgt die konkrete Situation welche während der Implementierung aufgetreten, als Beispiel. Siehe Abbildung \myTextTodo{Pic Singelton 123} für die grafische Darstellung des Beispiels. Im Grunde benötigen nur drei Seiten Zugriff auf die Datenbank. Allerdings erstellt die Navigation in Flutter ein Seitenobjekt und benötigt daher ebenfalls Zugriff auf die Datenbank. Es gibt zwei Möglichkeiten: Entweder man gibt die Datenbank bis zum Navigationsknopf weiter oder man definiert die Navigationsfunktion bereits auf der Seite und gibt sie bis zum Knopf weiter. Beide Optionen würden jedoch dazu führen, dass viele Zwischenklassen die Datenbank oder Funktion als Parameter übergeben müssten, obwohl sie diese gar nicht benötigen. Das Singleton-Entwurfsmuster löst dieses Problem.
%
%
%
%
%
\subsubsection{Strategie}
%Was+ Warum: Parser
Das Strategie Pattern würde für den Parser verwendet werden, da es in Zukunft geplant ist, dass die App auch weitere CLI-Terminkalender unterstützt. Das Strategie Pattern ermöglicht, dass der When-Parser einfach gegen einen anderen Parser ausgetauscht werden kann. %
%Was: Datenbank
Für die Datenbank wurde ebenfalls das Strategie Pattern verwendet. %
	%Warum: Was nicht
	Die Idee dahinter war nicht nur, die Datenbank gegen eine alternative auszutauschen - obwohl das auch möglich wäre, beispielsweise durch eine GitLab-API anstelle der verwendeten GitHub-API. %
	%Warum: Stattdessen
	Vielmehr ermöglicht das Strategie Pattern auch den einfachen temporären Austausch einer Mock-Datenbank\footnote{Eine Datenbank welche die Funktionen einer echten Simultiert}. Eine simulierte Datenbank kann sich nämlich als hilfreich beim Testen herausstellen, da sie unabhängig von einer API ist. Entsprechend wurde auch für diese Anwendung eine solche Mock-Datenbank implementiert. %
%Warum
%Außerdem half das Strategy Pattern auch dabei zu verstehen welche Funktionen sichtbar für den Nutzer sein müssen und bei welchen funktionen es sich um interne Funktionen der jeweiligen Datenbank implementation handelt. 
%
%
%
%
%
\subsubsection{Proxy}%
%Warum
Es wird angenommen, dass die Kalender- und Konfigurationsdateien sowie die Issues von GitHub während der Benutzung der App selten extern geändert werden. Deshalb ist es ausreichend, die Daten einmalig in der App zu laden und sie nur bei Bedarf, d.h. bei einer gezielten Anfrage nach neuen Dateien, erneut vom Server abzurufen. Derzeit lädt die App jedoch bei jeder Navigation auf eine Seite die Daten erneut von der Datenbank.\newline%
%Was
Dementsprechend wird für den Parser sowie der Datenbank das Remote Proxy Pattern verwendet. Dabei umschließe ein Proxy-Objekt das eigentliche Datenbank- bzw. Parser-Objekt und speichert die Ergebnisse in einem Cache. Wenn dieselben Daten erneut angefragt werden, werden sie aus dem Cache zurückgegeben. Bei einer gezielten Anfrage auf neue Dateien wird hingegen der Cache geleert und neue Dateien von der Datenbank geladen.
%
%
%
%
%
\subsubsection{Datenflussnetze \& Ablagebasiert}
%Was: Datenflussnetze
Zu Beginn wurde aus Einfachheit für den Parser eine Struktur ähnlich einem Datenflussnetz erstellt. %
	%Was: unflexibel
	Im weiteren Verlauf der Entwicklung stellte sich jedoch heraus, dass diese Struktur zu unflexibel ist. %
		%Warum:
		Denn bei Änderungen an einzelnen Methoden mussten auch die unmittelbar vorherigen und nachfolgenden Methoden angepasst werden. %
	%Was+Warum: unübersichtlich
	Darüber hinaus wurde der Parser durch die lange Verkettung von Methoden immer unübersichtlicher.
%Folgerung: Was:
Deshalb wurde die Struktur des Parsers zu einer ablagebasierten Struktur geändert. %
	%Warum:
	Dadurch wird der Parser einerseits übersichtlicher und andererseits flexibler und änderungsfreundlicher, da die einzelnen Funktionen nicht direkt miteinander kommunizieren. %
%Was: Abbildungen
Die Abbildungen \myTodo und \myTodo zeigen die Schritte des WhenParsers bei der Eingabe einer Datei bis zum erlangten Kalendareintrag in der zuvor genutzten datenflussnetzähnlichen Struktur und in der verbesserten ablagebasierten Struktur.
%
%
%
%
%
\subsection{(Weitere erwähnenswerte Besonderheiten)}
% !TeX encoding = UTF-8
\section{Evaluation}

\input{6_validierung/stichpunkte6.tex} %todo: remove after this sections completion

%%% hidden subsection for a better structure in latex editor: "texifier"
\myComment{\subsection*{Übersicht}} 
%
%Einleitung
In diesem Abschnitt wird versucht festzustellen, ob die erstellte Anwendung auch das erfüllt was sich zuvor vorgenommen wurde.\newline%
%
%Übersicht
\textbf{Übersicht:} %
Dazu wird eine Anforderungsverifizierung mit den funktionalen und nicht funktionalen Anforderungen aus dem \secref{section:anforderungen} durchgeführt.\newline%
%
%Zusammenfassung
\textbf{Erkenntnisse:}\newline%
	%f.A.
	Im Hinblick auf die funktionalen Anforderungen konnten alle \glqq Must-haves\grqq{} und \glqq Should-haves\grqq{} erfolgreich umgesetzt werden. Lediglich der Parser blieb unvollständig, da andere Funktionen aufgrund von Zeitmangel priorisiert wurden. %
	%n.f.A
	Die nicht funktionalen Eigenschaften \glqq Wartbarkeit\grqq{} und \glqq Reichweite\grqq{} konnten aufgrund derselben Einschränkungen nicht vollständig implementiert werden. Jedoch konnten die übrigen nicht funktionalen Anforderungen überwiegend bis vollständig erfüllt werden.%
%
%
%
%\myComment{
%	
%	%Einleitung: Warum
%	\textbf{Abschnitte der Arbeit}\\
%	Um zuletzt festzustellen ob wir auch wirklich das erschaffen haben, was wir uns zuvor als Ziel setzten, wird die Software mithilfe verschiedener Methoden im \secref{section:evaluation} Validiert.\newline%
%
%}
\subsection{Ergebnisse}
\subsection{Anforderungsvalidierung}
\subsection{Usertests}


% !TeX encoding = UTF-8
\section{Fazit \& Ausblick}

\input{7_fazitUndAusblick/stichpunkte7.tex} %todo: remove after this sections completion

%%% hidden subsection for a better structure in latex editor: "texifier"
\myComment{\subsection*{Übersicht}} 
\subsection{Fazit}
\subsection{Ausblick}
\cite{test1} \cite{test2} \cite{test3}
%\include{9_latex-beispiele} %TODO remove later
\printbibliography

\appendix
% !TeX encoding = UTF-8
\section{Anhang}

\end{document}
