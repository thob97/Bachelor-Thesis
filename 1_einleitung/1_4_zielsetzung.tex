\subsection{Zielsetzung}\label{section:zielsetzung} \myCheckmark

%Wiederholung der wichtigsten Punkte aus dem Letzten Kapitel
Im letzten Abschnitt haben wir folgende Erkenntnisse gewonnen.
\begin{enumerate}
	\item Handys erfreuen sich großer Relevanz und Beliebtheit
	\item Dementsprechend lohnt es sich Anwendungen auch als Apps anzubieten
	\item Handys und Pc's müssen verschiedene Stärken und Schwächen besitzen
	\item CLI-Terminkalender sind in dieser Hinsicht eine Interessante Anwendung
\end{enumerate}

%Nennung des Zieles
\myNewSection
Ziel ist es daher eine passende App für die CLI-Terminkalender zu entwickeln. Passend bedeutet in diesem Sinn, das die Stärken des Pc's und Handys beachtet werden und so mit in die zu erstellende Anwendung einfließen.\newline%
Oder anders formuliert ist das Ziel die \glqq Entwicklung einer Kalender-App für Geeks: Ein Versuch wie man die Lücke zwischen Terminal und Smartphone überbrücken könnte\grqq{}.\newline%
Die simpelste Lösung, einen CLI-Terminkalender auf das Handy zu portieren, ist also nicht [zielführend/zufriedenstellend]. Stattdessen wird versucht eine passende Lösung für dieses Problem zu finden.%

%Abgrenzung: Versuch anstatt Lösung + begrenzte Bearbeitungszeit
\myNewSection
Sehr wichtig nochmal zu betonen ist, dass es sich lediglich um einen Versuch handelt. Die in der Arbeit ermittelten Informationen und Ergebnisse sollen weder als allgemeines Beispiel noch Lösung dienen.\newline%
Da die Bearbeitungszeit während einer Bachelorarbeit auch äußerst beschränkt ist, soll das Ziel auch nicht sein ein fertiges Produkt zu liefern, sondern eher sich schrittweise dem fertigen Produkt zu nähern.


%Beispiel - Maybe Remove - Todo	
\myNewSection
Um diese Idee \dq Stärken des Pc's sowie des Handys zu nutzen\dq noch etwas verständlicher zu machen, folgt eine kurzes Beispiel, wie diese Idee zu einem erfolgreichen Produkt geführt hat.
Der erste Ipod\cite{einleitung_ipod} wurde 2001 vorgestellt und als \glqq tragbarer digitaler Medienabspielgeräte\grqq{} entworfen. Der Kleinheit und Portabilität zugunsten wurden nur die für das Gerät passenden und nötigen Funktionen implementiert. So zum Beispiel das Abspielen und Auswählen von Musik. Funktionen welche auf solch einem kleinen Gerät nicht gut funktionieren, wurden auf den Pc verlagert. So war der Pc für eine seiner Stärken zuständig, und zwar der Konfiguration, also dem Hinzufügen, Bearbeiten und Löschen von Musik. 

%Todo - Old Version - Remove
\myComment{
	%Zur Struktur
	Da nicht nur irgendeine App erstellt werden soll sondern eine möglichst sinnvolle \footnote{funktionale Anforderungen} und hochwertige\footnote{nichtfunktionale Anforderungen}. Wie Versucht wird die beiden Forderungen sicherzustellen kann im Kapitel \myTodo Anforderungen betrachtet werden.
	
	Ziel ist es eine Kalender-App zu erstellen welche die Stärken des Pcs und des Handys ausschöpfen. Oder anders formuliert ist das Ziel die Entwicklung einer Kalender-App "für Geeks", mit dem Versuch die Lücke zwischen Terminal und Smartphone zu überbrücken. Ziel ist es also offensichtlich nicht einen CLI-Kalender auf das Handy zu portieren, sondern eher der Versuch eine passend Lösung für dieses Problem zu finden. \newline
	Natürlich soll nicht nur irgendeine App erstellt werden sondern eine möglichst sinnvolle (funktionale Anforderungen) und hochwertige (nichtfunktionale Anforderungen). Wie Versucht wird die beiden Forderungen sicherzustellen kann im Kapitel \myTodo Anforderungen betrachtet werden.

}