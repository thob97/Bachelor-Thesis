\subsection{Zielsetzung}\label{section:zielsetzung}

%Wiederholung der wichtigsten Punkte aus dem Letzten Kapitel
Im letzten Abschnitt haben wir folgende Erkenntnisse gewonnen.
\begin{enumerate}
	\item Handys erfreuen sich großer Relevanz und Beliebtheit
	\item Daher lohnt es sich, Anwendungen auch als Apps anzubieten.
	\item Handys und PCs haben unterschiedliche Stärken und Schwächen.
	\item CLI-Terminkalender sind in dieser Hinsicht eine interessante Anwendung, da es noch keine passende App gibt und die Anwendung eher für den PC ausgelegt zu sein scheint.
\end{enumerate}

%Nennung des Zieles
\myNewSection
Ziel ist es daher, eine passende App für die CLI-Terminkalender zu entwickeln. \glqq Passend\grqq{} bedeutet in diesem Sinne, dass die Stärken von PC und Handy berücksichtigt werden und somit in die zu erstellende Anwendung einfließen.\newline%
Oder anders formuliert ist das Ziel die \glqq Entwicklung einer Kalender-App für Geeks: Ein Versuch wie man die Lücke zwischen Terminal und Smartphone überbrücken könnte\grqq{}.\newline%
Die simple Lösung einen CLI-Terminkalender auf das Handy zu portieren, ist also nicht zufriedenstellend. Stattdessen soll eine App entwickelt werden, welche auf existierende CLI-Terminkalender aufbaut, denn so könnten beide Geräte und dementsprechend auch dessen Stärken genutzt werden.
%Stattdessen wird versucht eine passende Lösung für dieses Problem zu finden\footnote{Beispiel dazu im Anhang \ref{anhang:einleitung:passendeLösung}}.

%Abgrenzung: Versuch anstatt Lösung + begrenzte Bearbeitungszeit
\myNewSection
Es ist wichtig noch einmal zu betonen, dass es sich hierbei lediglich um einen Versuch handelt. Die in dieser Arbeit gewonnenen Informationen und Ergebnisse sollen weder als allgemeines Beispiel noch als Lösung dienen.\newline%
Da die Bearbeitungszeit für eine Bachelorarbeit begrenzt ist, ist es außerdem nicht das Ziel, ein vollständiges Produkt zu entwickeln. Stattdessen wird versucht, sich schrittweise dem Ziel zu nähern.%
%
%
%Todo - Old Version - Remove
%\myComment{
%	%Zur Struktur
%	Da nicht nur irgendeine App erstellt werden soll sondern eine möglichst sinnvolle \footnote{funktionale Anforderungen} und hochwertige\footnote{nichtfunktionale Anforderungen}. Wie Versucht wird die beiden Forderungen sicherzustellen kann im Kapitel  Anforderungen betrachtet werden.
%	
%	Ziel ist es eine Kalender-App zu erstellen welche die Stärken des Pcs und des Handys ausschöpfen. Oder anders formuliert ist das Ziel die Entwicklung einer Kalender-App "für Geeks", mit dem Versuch die Lücke zwischen Terminal und Smartphone zu überbrücken. Ziel ist es also offensichtlich nicht einen CLI-Kalender auf das Handy zu portieren, sondern eher der Versuch eine passend Lösung für dieses Problem zu finden. \newline
%	Natürlich soll nicht nur irgendeine App erstellt werden sondern eine möglichst sinnvolle (funktionale Anforderungen) und hochwertige (nichtfunktionale Anforderungen). Wie Versucht wird die beiden Forderungen sicherzustellen kann im Kapitel  Anforderungen betrachtet werden.
%
%}