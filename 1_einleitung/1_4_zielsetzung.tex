\subsection{Zielsetzung}\label{section:zielsetzung} \myCheckmark

%Wiederholung der wichtigsten Punkte aus dem Letzten Kapitel
Im letzten Abschnitt haben wir folgende Erkenntnisse gewonnen.
\begin{enumerate}
	\item Handys großer erfreuen sich Relevanz und Beliebtheit
	\item Dementsprechend lohnt es sich Anwendungen auch als Apps anzubieten
	\item Handys und Pc's müssen verschiedene Stärken und Schwächen besitzen
	\item CLI-Terminkalender sind in dieser Hinsicht eine interessante Anwendung, da noch keine passende App existiert und die Anwendung eher für den Pc ausgelegt zu seien scheint.
\end{enumerate}

%Nennung des Zieles
\myNewSection
Ziel ist es daher, eine passende App für die CLI-Terminkalender zu entwickeln. Passend bedeutet in diesem Sinn, dass die Stärken des Pc's und Handys beachtet werden und so mit in die zu erstellende Anwendung einfließen.\newline%
Oder anders formuliert ist das Ziel die \glqq Entwicklung einer Kalender-App für Geeks: Ein Versuch wie man die Lücke zwischen Terminal und Smartphone überbrücken könnte\grqq{}.\newline%
Die simple Lösung einen CLI-Terminkalender auf das Handy zu portieren, ist also nicht zufriedenstellend. Stattdessen soll eine App entwickelt werden, welche auf existierende CLI-Terminkalender aufbaut, denn so könnten beide Geräte und dementsprechend auch dessen Stärken genutzt werden.
%Stattdessen wird versucht eine passende Lösung für dieses Problem zu finden\footnote{Beispiel dazu im Anhang \ref{anhang:einleitung:passendeLösung}}.

%Abgrenzung: Versuch anstatt Lösung + begrenzte Bearbeitungszeit
\myNewSection
Wichtig nochmal zu betonen ist, dass es sich lediglich um einen Versuch handelt. Die in der Arbeit ermittelten Informationen und Ergebnisse sollen weder als allgemeines Beispiel noch Lösung dienen.\newline%
Da die Bearbeitungszeit während einer Bachelorarbeit begrenzt ist, ist es nicht das Ziel ein fertiges Produkt zu verwirklichen. Stattdessen wird versucht sich den Produkt schrittweise anzunähern.%

%Todo - Old Version - Remove
\myComment{
	%Zur Struktur
	Da nicht nur irgendeine App erstellt werden soll sondern eine möglichst sinnvolle \footnote{funktionale Anforderungen} und hochwertige\footnote{nichtfunktionale Anforderungen}. Wie Versucht wird die beiden Forderungen sicherzustellen kann im Kapitel \myTodo Anforderungen betrachtet werden.
	
	Ziel ist es eine Kalender-App zu erstellen welche die Stärken des Pcs und des Handys ausschöpfen. Oder anders formuliert ist das Ziel die Entwicklung einer Kalender-App "für Geeks", mit dem Versuch die Lücke zwischen Terminal und Smartphone zu überbrücken. Ziel ist es also offensichtlich nicht einen CLI-Kalender auf das Handy zu portieren, sondern eher der Versuch eine passend Lösung für dieses Problem zu finden. \newline
	Natürlich soll nicht nur irgendeine App erstellt werden sondern eine möglichst sinnvolle (funktionale Anforderungen) und hochwertige (nichtfunktionale Anforderungen). Wie Versucht wird die beiden Forderungen sicherzustellen kann im Kapitel \myTodo Anforderungen betrachtet werden.

}