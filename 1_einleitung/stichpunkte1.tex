\myComment{\subsection*{Stichpunkte1}} 

\myComment{

	\myNewSection
	\textbf{Website:}
	\\Was ist das Problem? Warum ist es ein Problem? Wie bettet es sich in andere Arbeiten ein? Was ist nicht das Problem? Was wird nicht gelöst mit dieser Arbeit? \myTodo

	\myNewSection 
	\textbf{My Notes (Also Website)}: \myTodo
	\\ <Hauptteil> Gewählter Lösungsansatz, Alternativen, Abwägungen
	\\<Hauptteil> Beschreibung besonderer Schwierigkeiten und wie sie gelöst, umgangen oder vermieden wurden (oder warum nicht)
	\\<Hauptteil> Dokumentation der Durchführung und der entstandenen Artefakte
	\\ Alle Behauptungen müssen belegt werden, sei es mit einer Literaturquelle, einem sorgfältigen Argument oder mit eigenen empirischen Daten.

	\myNewSection
	\textbf{Purpose}: Dient dem Autor zur Orientierung, aber findet sich normalerweise später in der Einleitung des Dokumentes wieder.
	\begin{itemize}
		\item 1. Eine Beschreibung des größeren Zusammenhangs, in dem das Dokument angesiedelt ist.
		\item 2. Die Beschreibung des konkreten Problems, das im Dokument behandelt wird. \myCheckmark
		\item 3. Die Charakterisierung der Ziele, die das Dokument erreichen soll (z.B. der Information, die es liefern soll). \myCheckmark
		\item 4. Eine Begründung, warum und für wen das Dokument wichtig ist. \myCheckmark
		\item 5. Die Charakterisierung des Vorwissens der angepeilten Leserschaft. \myCheckmark
		\item 6. Eine Auflistung \underline{relevanter Randbedingungen: Zeitbeschränkungen}, Umfangsbeschränkungen, technische Randbedingungen (Medien etc.), äußere Vorgaben (Standards) für Stil, Organisation oder Format. \myCheckmark
		\item Einführung: \underline{Was ist das Problem? Warum ist es ein Problem?} Wie bettet es sich in andere Arbeiten ein? Was ist nicht das Problem? \underline{Was wird nicht gelöst mit dieser Arbeit?} \myCheckmark
	\end{itemize}



	\myNewSection
	\textbf{Prezi:}
	\begin{enumerate}
		\item Zuerst erzähle ich also etwas über die Motivation bzw. der Aufgabenstellung der Arbeit. Also warum mir das Thema relevant erscheint [und was es leisten kann]. \myCheckmark
		\item Bei der Vorgehensweise erkläre ich mit welcher Grundlegenden Herangehensweise ich versuche das Produkt zu erstellen welches die genannten Erwartungen und Ziele aus der Motivation erfüllt. \myCheckmark
		\item Wie gerade erwähnt wird ein agile Arbeitsstil befolgt -> die \underline{Dargestellten Schritte} lassen sich also nicht wie eigentlich abgebildet voneinander trennen und nacheinander lösen, sondern es herrscht ein fließender Übergang während der Bachelorarbeit.
		\\ \underline{Als Beispiel dazu}… Während der Implementation kommt es bestimmt zu Anforderungensveränderungen. Einfach weil man währenddessen neue Ideen hat oder weil einem klar wird, dass die Anforderungen derzeit nicht umsetzbar oder zu Aufwändig sind.
		\\ Jedoch macht es auch Sinn die \underline{Schritte getrennt zu betrachten}. Undzwar jetzt für den Vortrag sowie später in meiner Ausarbeitung. Damit lässt sich nämlich die Komplexität senken. bzw: es macht die Arbeit etwas Verständlicher \myCheckmark
		\item Obwohl ich mir hierzu also schon ein Paar Gedanken gemacht habe. Würde ich diesen Abschnitt, trotzdem während der Arbeit, nochmal detailierter bearbeiten wollen. Einfach weil ich glaube, dass es nützlich und interessant sein könnet. Aber trotzdem versuche ich auch während der Arbeit ständig den Prozess weiter zu verbessern und zu Überdenken. \myCheckmark
	\end{enumerate}
	
	
	
	\textbf{Prof:} Generell bei jeder Übersicht: (Was in diesem Kapitel gemacht wird) + Entscheidungen +-> Ergebnisse dieses Kapitels. \myCheckmark
	
	
	
	\myNewSection
	\textbf{Motivation:} \myCheckmark
	\begin{enumerate}
		\item Für die Motivation ist es vorerst wichtig zu Wissen was \textbf{CLI-Terminkalender} eigentlich sind. Einfach erklärt sind das Terminkalender ohne Grafische Oberfläche. Die Interaktion mit dem Kalender erfolgt stattdessen über das Terminal.
		\item Ein \textbf{Vorteil} von solchen Kalendern ist zum Beispiel: dass man sich nicht an der Vorgegebenen Grafischen Oberfläche anpassen muss. Bei normalen Kalendern mit Grafischer Oberfläche hätte man bei unerwünschte Funktionen oder Design Entscheidungen, nämlich einfach Pech gehabt. Bei den CLI-Kalendern hat man hingegen sehr viele Freiheiten zur Konfiguration und kann sich dadurch selbst sein gewünschtes Umfeld einstellen. 
		\item \textbf{Deswegen} gibt es auch genügend Benutzer welche sich auf den Terminal und damit auch auf CLI-Kalendern wohler fühlen. 
		\item \textbf{Beispiele} für solche Programme wären “when, remind, khal, calcurse, calendar”.
		\item Nun zum \textbf{Problem bzw. der Motivation}: Es gibt derzeitig keine App welche sich zufriedenstellend mit CLI-Kalendern Verbinden lässt. Denn am Wünschenswertesten wäre eine Lösung, welche die Vorzüge des PCs und die Vorzüge der Handys ausschöpft.
		\item Ich meinte ja grade, dass versucht werden sollte die Stärken der jeweiligen Systeme möglichst effektiv zu nutzen, damit sich die beiden Welten Verbinden lassen. Eine Stärke des PC’s ist es zum \textbf{Beispiel}, dass sie eine Tastatur besitzt und sich dadurch schnell und bequem tippen lässt. Ein Handy besitzt zwar auch eine Software Tastatur, diese ist aber viel kleiner und Sondersymbole sind hinter mehreren Layers versteckt. Daher sollte einen die Erkenntnis kommen, dass sich auf Handys weniger gut tippen lässt. Die aller einfachste Lösung dieses Themas und der Arbeit, einen CLI-Terminkalender also einfach 1 zu 1 auf ein Handy zu Portieren scheint also nicht als sinnvoll.
		\item Ein \textbf{Beispiel} wie solch ein Problem gut gelöst wurde, ist der Ipod Nano. Es wurden nämlich nur die benötigten und für das Gerät passende Funktionen implementiert. Das wären zum Beispiel das Abspielen von Musik. Die Restlichen Funktionen konnten nur mithilfe des Pc’s erledigt werde. Zum Beispiel das hinzufügen oder Löschen von Musik.
		\item Mit der Erkenntniss dass Funktionen die gut auf dem PC funktionieren nicht gut auf dem Handy funktionieren müssen 
		\item Bsp. wie man es nicht macht: ssh, Terminal App -> tippen klobig weil benötigte Sonderzeichen nicht leicht erreichbar und Tastatur sehr klein
		\item Und \textbf{noch ein Grund} warum es solch eine App benötigt, wäre, dass es sicherlich nochmehr Leute gibt, welche sich auch auf dem Terminal wohlfühlen, aber trotzdem gerne die Vorzüge einen Handys nutzen würden (Bsp. Mobilität)
		\item \textbf{Das Ziel} ist es also, sich zu Überlegen, wie die beiden Welten miteinander verbunden werden können, und dabei die passende App zu erstellen.
	\end{enumerate}
	
	
	
	\myNewSection
	\textbf{Allgemeine Vorgehensweise} \myCheckmark
	\begin{enumerate}
		\item \textbf{Generell} gilt… also für alle Folgenden Schritte dieses Vortrags, dass während der Bachelorarbeit immer wieder Entscheidungen getroffen werden. Und jede dieser Entscheidungen sollte sorgfältig gewählt und gut Begründet sein.
		\begin{enumerate}
			\item Grund dafür ist erstens, dass es einerseits Interessant für den Leser und andererseits wichtig für die Evaluation ist: den Gedankengang hinter den getroffenen Entscheidungen nachvollziehen zu können.
			\item Und zweitens, weil diese Fragen einen selbst dabei helfen… eine bessere oder sogar die beste Entscheidung, aus vielen Möglichkeiten, zu finden.
		\end{enumerate}
		
		\item Für den \textbf{Prozess} habe ich mir eine Agile-Arbeitsweise vorgenommen. Und zwar unteranderem wegen den Folgenden drei, eng miteinander verbundenen, Prinzipien:
		\begin{enumerate}
			\item Das erste ist dabei die Arbeit in \textbf{Iterationen}. Denn die Schritte zur bau einer Software lassen sich nicht nacheinander und in einem Rutsch abarbeiten. Daher werde ich mich stattdessen in mehreren kleinen Schritten zum Ziel herantasten. Vorteile davon sind, dass man steht's ein Evaluierbares Produkt hat… und dies ist natürlich sehr nützlich für die Abgabe. Außerdem hilft Iteration auch bei Anforderungsunsicherheiten. Denn ich gehe davon aus, dass sich getroffene Entscheidungen während der Arbeit / bzw. der Implementierung noch ändern könnten, obwohl man zuvor dachte, dass diese Entscheidung die richtige Wahl wäre.
			\item Das zweite Prinzip ist der Kurzer \textbf{Planungshorizont}: Anfangs würde ich nämlich einmal Oberflächlich einige Allgemeine Ziele raussuchen. Danach gilt immer nur für eine oder zwei Iteration im Vorraus im Detail zu Planen. Das hat den Vorteil das Änderungen weniger verheerend sind, da die Planung leicht geändert werden kann und nicht zu viel Zeit Anfangs darin gesteckt wurde.
			\item ODER: Ein kurzer \textbf{Planungshorizont}: würde dabei heflen, dass Änderungen weniger verheerend sind. Undzwar weil nicht zu viel Zeit in die Planung gesteckt wird.
			\item Das letzte Prinzip sind die \textbf{Retrospektiven}. Dabei würde ich mich wöchentlich, also nach je einer Iteration, Fragen was gut und was schlecht lief. Diese Reflexion kann nämlich dabei helfen den Prozess zu Verbessern und Ursache von Mängeln zu beseitigen (konstruktive Qualitätssicherung) (ständige wird sich Schrittweise verbessert)
		\end{enumerate}
	\end{enumerate}
	
	
}%%%