\subsection{Motivation}\label{subsection:motivation}
%
%Relevanz von Handys steigt
Als Handys zum ersten mal auf den Markt kamen, waren sie als ein mobiler Telefonersatz gedacht\myTextTodo{Quelle nötig?}. Heutzutage scheinen Handys aber einen viel wichtigeren Teil unseres Lebens eingenommen zu haben. Anscheinend fühlt sich bereits die Mehrzahl an Leuten unwohl das Haus ohne ihr Handy zu verlassen\cite{pcVsphone_feelingUneasyWhenLeavingPhoneHome}. Des Weiteren werden Handys mittlerweile täglich 40 Minuten mehr als Pc's benutzt\cite{pcVsphone_phoneScreenTime,pcVsphone_totalScreenTime,pcVsphone_totalScreenTime2}\footnote{Handy: 3.44 Stunden, PC:2.59 Stunden, zusammen: 6.43 Stunden.}. Das zeigt eine starken Wandel, wenn man bedenkt, dass Computer im Jahr 2011 noch rund 94\% des Marktanteils ausmachten\cite{pcVsphone_smartphoneWebTrafficHigherThanPc}. Diese Veränderung scheinen auch Entwickler und Unternehmen zu bemerken. So kündigte zum Beispiel Google bereits in 2017 den Umstieg auf Mobile-Indexing\footnote{Suchergebnisse werden besser bewertet, wenn Seiten eine mobile Ansicht anbieten.}an\cite{pcVsphone_mobileFirstIndexing} und Unternehmen wie Facebook und Pinterest beziehen den der Großteil ihrer Benutzerschaft aus Applikationen\cite{pcVsphone_socialMediaFacebookMobileUsage,pcVsphone_socialMediaPinterestMobileUsage}.\newline%
Generell scheinen Handys also immer beliebter und wichtiger zu werden. Das spiegelt sich auch in den Nutzerzahlen wieder. Während laut einer Studie 96\% der Befragten ein Handy besitzen ist die Anzahl bei Pc's von 73\% auf 59\% innerhalb der letzten vier Jahre\footnote{2018 bis 2022} gesunken\cite{pcVsphone_deviceOwnership}. Auch der Marktanteil des Handys zeigt ein Indiz darauf, denn dieser steht mit 53\% knapp über den des Computers\cite{pcVsphone_smartphoneWebTrafficHigherThanPc}.%
%
%Folgerungen: Apps machen Sinn + Es Handys und Pc's haben verschiedene Stärken 
\newline
\myNewSection%
Basierend auf der Erkenntnis, dass das Handy einen wichtigen Stellenwert eingenommen hat, lassen sich für diese Arbeit zwei interessante Schlussfolgerungen ziehen.\newline
Zum einen, dass es lohnt sich lohnt, Anwendungen nicht nur für den PC, sondern auch für Handys anzubieten.\newline%
Zum anderen, dass das Handy gewisse Vorteile und Stärken im Vergleich zu PCs bieten muss, was dazu führt, dass es an Beliebtheit und Nutzung gewinnt. Dies scheint jedoch beidseitig zu gelten. So gibt es Anwendungen, die trotz der steigenden Handy-Popularität besser auf dem Computer funktionieren. Beispielsweise möchten wahrscheinlich nur wenige Menschen eine wissenschaftliche Arbeit oder Steuererklärung auf einem Handy schreiben.%
%
%Begründung für CLI-Kalendar Anwendung: auf Stärken des Pc's ausgelegt + existiert keine App
\newline
\myNewSection
Eine weitere Gruppe von Anwendungen, die sich besser auf dem Computer als auf dem Handy nutzen lassen, sind die CLI-Terminkalender\footnote{Beispiel CLI-Terminkalender: remind\cite{cli_remind}, khal\cite{cli_khal}, calcurse\cite{cli_calcurse}}. Das Kürzel \glqq CLI\grqq{} steht hierbei für \glqq Command Line Interface\grqq{}, was gleichbedeutend mit einer Kommandozeile oder einem Terminal ist. Bei diesen Terminkalendern gibt es keine grafische Benutzeroberfläche, stattdessen wird über das Terminal interagiert. Die Anwendung erfordert es, unter anderem lange Texte, Befehle und Sonderzeichen zu schreiben. Durch die Tastatur ist dies auf dem Computer einfach auszuführen. Im Gegensatz dazu würde die begrenzte Softwaretastatur des Handys die Bedienung dieser Anwendung erschweren. Wahrscheinlich ist dies auch der Grund, warum es bisher keine erfolgreichen CLI-Terminkalender-Anwendungen für das Handy gibt.%
%
%%%---Kommentare---%%%
%
%%Warum CLI-Kalender gut zu Pc's passen: Keyboard + config Möglichkeit
%\myComment{Diese Eigenheit bringt einige Vorteile mit sich. Zum Beispiel, dass man sich nicht an der vorgegebenen Grafischenoberfläche anpassen muss. Bei "normalen" Kalendern mit Grafischeroberfläche kann man gegen unerwünschte Funktionen oder Design Entscheidungen nichts machen. Bei den CLI-Kalendern hat man hingegen sehr viele Freiheiten zur Konfiguration und kann sich dadurch selbst sein gewünschtes Umfeld einstellen. Besonders die Commandos, die einzige Interaktion mit dem Kalender, lassen sich oft nach belieben anpassen.}
%
%%Alt: keine zufriedenstellende Lösung
%\myComment{Jedoch gibt es keine Zufriedenstellende Lösung für die Verbindung dieser Kalender mit dem Handy. Denn am Wünschenswertesten wäre eine Lösung, welche die Vorzüge des PCs und des Handys betrachten. }