\subsection{Motivation}\label{subsection:motivation} \myCheckmark
%
%Relevanz von Handys steigt
Als Handys zum ersten mal auf den Markt kamen, waren sie als ein mobiler Telefonersatz gedacht\myTextTodo{Quelle nötig?}. Heutzutage scheinen Handys aber einen viel wichtigeren Teil unseres Lebens eingenommen zu haben. Anscheinend fühlt sich bereits die Mehrzahl an Leuten unwohl das Haus ohne ihr Handy zu verlassen\cite{pcVsphone_feelingUneasyWhenLeavingPhoneHome}. Des Weiteren werden Handys mittlerweile täglich 40 Minuten mehr als Pc's benutzt\cite{pcVsphone_phoneScreenTime,pcVsphone_totalScreenTime,pcVsphone_totalScreenTime2}\footnote{Handy: 3.44h, Pc:2.59h, total: 6.43h}. Das zeigt eine starken Wandel, wenn man bedenkt, dass Computer im Jahr 2011 noch rund 94\% des Marktanteils ausmachten\cite{pcVsphone_smartphoneWebTrafficHigherThanPc}. Diese Veränderung scheinen auch Entwickler und Unternehmen zu bemerken. So kündigte zum Beispiel Google bereits in 2017 den Umstieg auf Mobile-Indexing\footnote{Suchergebnisse werden besser Bewertet wenn Seiten eine Mobile-ansicht anbieten.}an\cite{pcVsphone_mobileFirstIndexing} und Unternehmen wie Facebook und Pinterest beziehen den der Großteil ihrer Benutzerschaft aus Applikationen\cite{pcVsphone_socialMediaFacebookMobileUsage,pcVsphone_socialMediaPinterestMobileUsage}.\newline%
Generell scheinen Handys also immer beliebter und wichtiger zu werden. Das spiegelt sich auch in den Nutzerzahlen wieder. Während laut einer Studie 96\% der Befragten ein Handy besitzen ist die Anzahl bei Pc's von 73\% auf 59\% innerhalb der letzten vier Jahre\footnote{2018 zu 2022} gesunken\cite{pcVsphone_deviceOwnership}. Auch der Marktanteil des Handys zeigt ein Indiz darauf, denn dieser steht mit 53\% knapp über den des Computers\cite{pcVsphone_smartphoneWebTrafficHigherThanPc}.%
%
%Folgerungen: Apps machen Sinn + Es Handys und Pc's haben verschiedene Stärken 
\newline
\myNewSection%
Aus der Erkenntnis dass das Handy einen wichtigen Stellenwert eingenommen hat, folgen für diese Arbeit zwei interessante Schlussfolgerungen.\newline
Einerseits, dass sich Anwendung nicht nur ausschließlich für den Pc, sondern auch für Handys lohnen.\newline%
Andererseits, dass Handys gewisse Vorteile und Stärken gegenüber Pc's bieten müssen was dazu führt, dass sie an Beliebtheit und Nutzung gewinnen. Dies scheint aber beidseitig zu gelten. So gibt es trotzdem Anwendungen, welche trotz der Steigenden Handy Popularität, besser auf dem Computer funktionieren. Zum Beispiel möchten wahrscheinlich nur die wenigsten eine Ausarbeitung oder Steuererklärung auf einem Handy schreiben.%
%
%Begründung für CLI-Kalendar Anwendung: auf Stärken des Pc's ausgelegt + existiert keine App
\newline
\myNewSection
Eine weitere solche Anwendungsgruppe sind die CLI-Terminkalender\footnote{Beispiel CLI-Terminkalender: remind\cite{cli_remind}, khal\cite{cli_khal}, calcurse\cite{cli_calcurse}}. Das CLI steht dabei für "Command line interface", was gleich zu setzen mit einer Kommandozeile oder einem Terminal ist. Es handelt sich dabei also um Terminkalender ohne grafische Benutzeroberfläche, mit welchen stattdessen über das Terminal interagiert wird. Die Anwendung erfordert es unter anderem lange Texte, Kommandos und Sonderzeichen zu schreiben. Durch die Tastatur sind diese Funktionen auf dem Pc leicht auszuführen. Im Gegensatz dazu würde die begrenzte Softwaretastatur des Handys die Bedienung dieser Anwendung erschweren. Wahrscheinlich ist dies auch der Grund warum es noch keine erfolgreichen CLI-Terminkalenderanwendungen auf dem Handy gibt.%
%
%%%---Kommentare---%%%
%
%Warum CLI-Kalender gut zu Pc's passen: Keyboard + config Möglichkeit
\myComment{Diese Eigenheit bringt einige Vorteile mit sich. Zum Beispiel, dass man sich nicht an der vorgegebenen Grafischenoberfläche anpassen muss. Bei "normalen" Kalendern mit Grafischeroberfläche kann man gegen unerwünschte Funktionen oder Design Entscheidungen nichts machen. Bei den CLI-Kalendern hat man hingegen sehr viele Freiheiten zur Konfiguration und kann sich dadurch selbst sein gewünschtes Umfeld einstellen. Besonders die Commandos, die einzige Interaktion mit dem Kalender, lassen sich oft nach belieben anpassen.}

%Alt: keine zufriedenstellende Lösung
\myComment{Jedoch gibt es keine Zufriedenstellende Lösung für die Verbindung dieser Kalender mit dem Handy. Denn am Wünschenswertesten wäre eine Lösung, welche die Vorzüge des PCs und des Handys betrachten. }