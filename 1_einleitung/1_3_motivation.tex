\subsection{Motivation}\label{subsection:motivation} \myCheckmark

%Relevanz von Handys steigt
Als Handys zum ersten mal veröffentlicht wurden waren sie als ein mobiler Telefonersatz gedacht. Heutzutage scheinen Handys aber einen viel wichtigeren Teil unseres Lebens eingenommen zu haben. Anscheinend fühlt sich bereits die Mehrzahl an Leuten unwohl das Haus ohne ihr Handy zu verlassen\cite{pcVsphone_feelingUneasyWhenLeavingPhoneHome}. Des Weiteren werden Handys mittlerweile täglich 40 Minuten mehr als Pc's benutzt\cite{pcVsphone_phoneScreenTime,pcVsphone_totalScreenTime,pcVsphone_totalScreenTime2}\footnote{Handy: 3.44h, Pc:2.59h, total: 6.43h}. Das ist eine erstaunlicher Wandel, wenn man bedenkt, dass Pc's im Jahr 2011 noch rund 94\% des Marktanteils ausmachten\cite{pcVsphone_smartphoneWebTrafficHigherThanPc}. Diese Veränderung scheinen auch viele Entwickler und Unternehmen zu bemerken. So kündigte Google bereits in 2017 den Umstieg auf Mobile-Indexing\footnote{Suchergebnisse werden besser Bewertet wenn Seiten eine Mobile-ansicht anbieten.}an\cite{pcVsphone_mobileFirstIndexing}. Aber auch Unternehmen wie Facebook und Pinterest fokussieren sich wahrscheinlich eher auf ihre Apps als auf Desktopanwendungen, da der Großteil ihrer Benutzerschaft aus der App kommt\cite{pcVsphone_socialMediaFacebookMobileUsage,pcVsphone_socialMediaPinterestMobileUsage}.\newline%
Generell scheinen Handys also immer beliebter und wichtiger zu werden. Das spiegelt sich auch in den Nutzerzahlen wider. Während laut einer Studie 96\% der Befragten ein Handy besitzen ist die Anzahl bei Pc's von 73\% auf 59\% innerhalb der letzten vier Jahre\footnote{2018 zu 2022} gesunken\cite{pcVsphone_deviceOwnership}. Aber auch der Marktanteil des Handys gibt ein Indiz darauf, denn dieser steht mit 53\% knapp über den des Pc's\cite{pcVsphone_smartphoneWebTrafficHigherThanPc}.%

%Folgerungen: Apps machen Sinn + Es Handys und Pc's haben verschiedene Stärken 
\myNewSection%
Aus der Erkenntnis das dass das Handy einen wichtigen Stellenwert eingenommen hat folgen für diese Arbeit zwei interessante Schlussfolgerungen.\newline
Erstens, dass man seine Anwendung nicht nur ausschließlich für den Pc sondern auch für Handys ermöglichen sollte.\newline%
Und Zweitens, dass Handys gewisse Vorteile und stärken gegenüber Pc's bieten muss, was dazu führt, das sie an Beliebtheit und Nutzung gewinnen. Dies scheint aber beidseitig zu gelten, denn es gibt viele Anwendungen, welche trotz der Steigenden Handy Popularität, besser auf dem Pc funktionieren. So möchten wahrscheinlich nur die wenigsten eine Ausarbeitung oder Steuererklärung auf einem Handy schreiben.%

%Begründung für CLI-Kalendar Anwendung: auf Stärken des Pc's ausgelegt + existiert keine App
\myNewSection
Eine weitere solche [Anwendungsgruppe] sind die CLI-Terminkalender\footnote{Beispiel CLI-Terminkalender: remind\cite{cli_remind}, khal\cite{cli_khal}, calcurse\cite{cli_calcurse}}. Das CLI steht dabei für "Command line interface", was gleich zu setzen mit einer Kommandozeile oder einem Terminal ist. Es handelt sich dabei also um Terminkalender ohne grafische Benutzeroberfläche, mit welchen stattdessen über das Terminal interagiert wird. Die Anwendung erfordert es unter anderem lange Texte, Kommandos und Sonderzeichen zu schreiben. Das funktioniert auf dem Pc, durch die Tastatur, sehr gut. Es scheint sogar so, als wurde sich beim erstellen der Anwendung gezielt Gedanken um die Stärken des Pc's gemacht. Andererseits würde die begrenzte Softwaretastatur des Handys die Bedienung dieser Anwendung um einiges erschweren. Genau deshalb ist diese Anwendung besonders interessant. Während sie auf dem Pc sehr gut funktioniert, ist sie auf dem Handy fast nicht vorstellbar. Aus diesem Grund gibt es wahrscheinlich auch noch keine CLI-Kalender-Anwendungen auf dem Handy\newline%



%%%---Kommentare---%%%

%Warum CLI-Kalender gut zu Pc's passen: Keyboard + config Möglichkeit
\myComment{Diese Eigenheit bringt einige Vorteile mit sich. Zum Beispiel, dass man sich nicht an der vorgegebenen Grafischenoberfläche anpassen muss. Bei "normalen" Kalendern mit Grafischeroberfläche kann man gegen unerwünschte Funktionen oder Design Entscheidungen nichts machen. Bei den CLI-Kalendern hat man hingegen sehr viele Freiheiten zur Konfiguration und kann sich dadurch selbst sein gewünschtes Umfeld einstellen. Besonders die Commandos, die einzige Interaktion mit dem Kalender, lassen sich oft nach belieben anpassen.}

%Alt: keine zufriedenstellende Lösung
\myComment{Jedoch gibt es keine Zufriedenstellende Lösung für die Verbindung dieser Kalender mit dem Handy. Denn am Wünschenswertesten wäre eine Lösung, welche die Vorzüge des PCs und des Handys betrachten. }