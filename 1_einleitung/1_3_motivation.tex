\subsection{Motivation}

Bevor es mit der Nennung der Motivation und dem Zielen losgeht, folgt ein kurzer aber wichtiger Exkurs, wie wir zu diesem Thema kamen. \newline
Auf die Idee dieser Arbeit kamen wir, als wir uns CLI-Terminkalender betrachteten und überlegten was diese ausmachen und wie man sie Verbessen könnte. CLI steht dabei für "Command line interface", was gleich zu setzen mit einer Kommandozeile oder einem Terminal ist. Also handelt es sich dabei um Terminkalender ohne grafische Oberfläche, mit welchen stattdessen über das Terminal interagiert wird. \newline
Diese Eigenheit bringt einige Vorteile mit sich. Zum Beispiel, dass man sich nicht an der vorgegebenen Grafischenoberfläche anpassen muss. Bei "normalen" Kalendern mit Grafischeroberfläche kann man gegen unerwünschte Funktionen oder Design Entscheidungen nichts machen. Bei den CLI-Kalendern hat man hingegen sehr viele Freiheiten zur Konfiguration und kann sich dadurch selbst sein gewünschtes Umfeld einstellen. Besonders die Commandos, die einzige Interaktion mit dem Kalender, lassen sich oft nach belieben anpassen. \newline
Aus diesen, aber auch weiteren Gründen, gibt es viele Benutzer <\myTodo Quelle> welche sich auf den Terminal und damit auch auf CLI-Kalendern wohler fühlen.

\myNewSection
Nun zur eigentlichen Motivation [bzw. einer Möglichen Verbesserung von CLI-Kalendern]. 
Wir nehmen an, dass die Fangemeinde der CLI-Kalender diese genau deswegen bevorzugen, weil Sie die Vorzüge des Pcs gut ausschöpfen. Das ein Pc eine Tastatur besitze hilft zum Beispiel beim schnellem Tippen und die "offenheit" eines Pcs ermögliche viel konfiguration. Beide dieser Beispiel-Vorzüge werden von einer Reihe von CLI-Kalendern gut umgesetzt.%
\footnote{Beispiel CLI-Terminkalender: when, remind, khal, calcurse, calendar \myTodo}
Jedoch gibt es keine Zufriedenstellende Lösung für die Verbindung dieser Kalender mit dem Handy. Denn am Wünschenswertesten wäre eine Lösung, welche die Vorzüge des PCs und des Handys betrachten. 

\myNewSection (\myTodo Der Umkehrschluss unser vorherigen Annahme wäre nun, dass wenn die Vorzüge des Handys gut in der App umgesetzt werden, sich auf viele dieser Nutzer dafür interessieren könnten. Und weil ein Handy andere Stärken hat als Pcs, wie zum Beispiel die Mobilität, es eine Lücke schließt/ eine Hilfe wäre)

\myNewSection
Um diese Idee/Sachverhalt noch etwas verständlicher zu machen, folgt eine kurzes und bekannte Nennung eines Systems, wie solch ein Problem bereits einmal gelöst wurde.
Der Ipod Nano\cite{ipodNano} wurde 2005 vorgestellt und als Portable-Mediaplayer entworfen. Der kleinheit und portabilität zugunsten wurden nur die für das Gerät passenden und nötigen funktionen implementiert. So zum Beispiel das Abspielen und Auswählen von Musik. Funktionen welche auf solch einem kleinen Gerät nicht gut funktionieren, wurden auf den Pc verlagert. So war der Pc für eine seiner Stärken zuständig, undzwar der Konfiguration, also dem Hinzufügen, Bearbeiten und Löschen von Musik.