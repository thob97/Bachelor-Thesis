\subsection{Struktur}
Im folgenden wird kurz die Struktur dieses Dokumentes erläutert. Dadurch soll bewusst werden, auf welche Arten die Ausarbeitung gelesen werden kann.

\myNewSection Die Abschnitte 1-7 wurden so gewählt, wie man sich den Ablauf eines 'naiven' Software Projektes vorstellet \footnote{Für eine Beispieldarstellung siehe Abbildung [\ref{fig:wasserfallmodell}]}. 
Zwar ließen sich die Abschnitte während der Bachelorarbeit so gut wie nie getrennt voneinander betrachten und lösen, sondern es herrschte steht's ein fließender Übergang. Trotzdem wurden versucht, diese Abschnitte und auch alle Unterpunkte möglichst gut getrennt voneinander zu betrachten und (modularisieren). 
Durch die Aufteilung in verschiedene Kapitel wirk die Bachelorarbeit sehr viel überschaubarer und auch schaffbarer. (Ganz im Sinne "die Lösung wird in vielen kleinen Schritten erreicht") (Nicht umsonst ist eins der wichtigsten Ziele in der Softwaretechnik Projekte in verdauliche Happen aufzuteilen (keep? + Quelle)). Die Trennung der Abschnitte, bzw. anders gesagt die Modularizierung der Abschnitte ermöglicht das 'Querlesen'. Damit ist gemeint, dass Textabschnitte auch ohne das Lesen des kompletten Dokumentes verstanden werden können. 

\myNewSection
Damit ein gewisser Überblick über alle Kapitel geschaffen wird, enthält jeder Abschnitt am Anfang eines Kapitels eine kurze Einleitung über die Unterpunkte und über was sie handeln.
Außerdem haben alle Abschnitte eine kurze Zusammenfassung. Darin werden wichtige Entscheidungen und Ergebnisse des Kapitels genannt. Das soll einen die Möglichkeit geben ganze Abschnitte oder sogar Kapitel überfliegen zu können und trotzdem die wichtigsten Aspekte und Resultate der Arbeit zu verstehen. Anders gesagt hilft diese Strukturentscheidung erneut beim Querlesen