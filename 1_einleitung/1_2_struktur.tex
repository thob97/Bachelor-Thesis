\subsection{Struktur}\label{kapitel_struktur}
%
%ÜBERSICHT
%\myComment{
%%Überblick + Warum: Um zu erklären wie die Ausarbeitung gelesen werden kann
%Im folgenden wird kurz die Struktur dieses Dokumentes erläutert. Dadurch soll bewusst werden, auf welche Arten die Ausarbeitung gelesen werden kann.
%}
%
%TRIMMED
%\myComment{
%%Was wurde gemacht: Ausarbeitung Abschnitte Modularisiert
%\myNewSection%
%Die Abschnitte dieser Ausarbeit wurden so gewählt, wie man sich den Ablauf eines 'naiven' Software Projektes vorstellet \footnote{Für eine Beispieldarstellung siehe Abbildung [\ref{fig:wasserfallmodell}]}. Diese Abschnitte lassen sich während eines Softwareprojektes und damit auch während dieser Bachelorarbeit eigentlich so gut wie nie getrennt voneinander betrachten und lösen, sondern es herrschte steht's ein fließender Übergang. Trotzdem wurden genau das Versucht in der Ausarbeitung so darzustellen. Es wurde also versucht die Abschnitte und auch alle ihrer Unterpunkte möglichst gut getrennt voneinander zu betrachten und gewissermaßen so zu modularisieren.\newline%
%%Warum: Überschaubar und Querlesen
%Durch die Aufteilung in verschiedene Kapitel wirkt die Bachelorarbeit nämlich mehr überschaubar und auch mehr schaffbar. Nicht umsonst ist eins der wichtigsten Ziele in der Softwaretechnik Projekte in verdauliche Happen aufzuteilen. Des Weiteren ermöglicht die Modularisierung der Abschnitte das \dq Querlesen\dq. Damit ist gemeint, dass Textabschnitte auch ohne das Lesen des kompletten Dokumentes verstanden werden können. 
%}
%
%
%Was + Warum: Überblick für jeden Abschnitt
Jeder Abschnitt enthält eine Einleitung, in der der Inhalt und dessen Bedeutung für das Projekt erläutert werden sowie die verwendete Methodik beschrieben wird.\newline%
%Was + Warum: Zusammenfassung für jeden Abschnitt -> Querlesen
Darüber hinaus gibt es am Ende jedes Abschnitts eine kurze Zusammenfassung, die wichtige Entscheidungen und Ergebnisse des Kapitels zusammenfasst. Das soll dem Leser die Möglichkeit geben, Abschnitte schnell zu überfliegen und trotzdem die wichtigsten Aspekte und Ergebnisse der Arbeit zu verstehen. Zusammenfassend soll diese Strukturierung beim Querlesen helfen.