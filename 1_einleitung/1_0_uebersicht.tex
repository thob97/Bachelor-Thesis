%%% hidden subsection for a better structure in latex editor: "texifier"
\myComment{\subsection*{Übersicht}}\myCheckmark

%Einleitung \ Überblick
\textbf{Überblick:}
Dieser Abschnitt handelt von der konkreten Beschreibung des Problems sowie der Charakterisierung des Ziels. Dabei soll unter anderem auch der Nutzen [sowie ...] der Arbeit bewusst werden.\newline
Außerdem wird die Struktur dieses Dokumentes erläutert, sowie das vorausgesetzte Vorwissen der angepeilten Leserschaft genannt. Dadurch soll verdeutlicht werden ob und wie die Ausarbeitung gelesen werden kann.\newline
%Result
\textbf{Ergebnisse:}
Um diese Arbeit zu verstehen werden allgemeine Informatik Kenntnisse vorausgesetzt. Dafür enthält jeder Abschnitt dieser Arbeit, um das Lesen zu vereinfachen und das Querlesen zu ermöglichen, einen Überblick und eine Zusammenfassung.\newline
Während der Beschreibung des Problems kam es zu mehreren Erkenntnissen. Einerseits erfreuen sich Handys großer Relevanz und Beliebtheit und dementsprechend lohnt es sich Anwendungen auch als App anzubieten. Weiterhin besitzen Handys und Pc's verschiedene Stärken und Schwächen und CLI-Terminkalender sind diesbezüglich eine Interessante Anwendung.