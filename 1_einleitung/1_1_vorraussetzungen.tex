%Warum: Um die Vorraussetzungen für den Leser bewusst zu machen
\subsection{Voraussetzungen}
%Vorraussetzungen für den Leser
Im Rahmen dieser Ausarbeitung werden wiederholt Begriffe aus dem Jargon der Informatik genutzt. Auf jene Begriffe, die mit einem Informatikstudium als informatisches Allgemeinwissen bezeichnet werden, wird nicht näher eingegangen. Das bedeutet, dass ein gewisses informatisches Vorwissen seitens des Lesers vorausgesetzt wird%
\footnote{Beispiele Begriffe zur Orientierung:
	\begin{itemize}[noitemsep,topsep=0pt,parsep=0pt,partopsep=0pt]
		\item wird Vorausgesetzt: Framework, Package, API (Application Programming Interface)
	\end{itemize}
	\nointerlineskip %removes space between itemize and next footnote
}. \newline%
%Begründung
Dies hat den Grund, ein Abschweifen während der Ausarbeitung und damit verbundene Störungen des Leseflusses zu verhindern.%

%%%Alternativ:%%%
%\myComment{
%
%%Begründung
%Das hat den Grund, dass die Zielgruppe dieses Dokumentes genau diese Vorwissen bereits besitzt. Darüberhinaus wird die Leserlichkeit verbessert, wenn der Lesefluss nicht ständig durch Definitionen unterbrochen wird. \newline%
%%Was stattdessen gemacht wird - Wichtige Begriffe definiert
%Nicht allgemeine Begriffe werden hingegen erläutert. Auch wird es Beispiele geben um das Verständnis weiter zu vertiefen.\newline%
%Für die Arbeit eher unwichtige Begriffe, sowie nebensächliche Beispiele werden aber meistens nur im Anhang oder der Fußnote erwähnt.
%
%}