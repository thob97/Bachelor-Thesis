\subsection{Vorraussetzungen}

Im Rahmen dieser Ausarbeitung werden wiederholt begriffe aus dem Jargon Informatik genutzt. Auf jene Begriffe welche mit einem Informatikstudium als Informatischesallgemeinwissen/Allgemeinwissen bezeichnet werden, wird nicht näher eingegangen. Das heißt also im Umkehrschluss, dass ein gewisses informatisches Vorwissen vom Leser vorausgesetzt/erwartet wird%
\footnote{Beispiele Begriffe zur Orientierung:
	\begin{itemize}[noitemsep,topsep=0pt,parsep=0pt,partopsep=0pt]
		\item wird Vorausgesetzt: Framework, Package, Byte
		\item wird nicht Vorausgesetzt: \myTodo
	\end{itemize}
	\nointerlineskip %removes space between itemize and next footnote
}. \newline
Das hat den Grund, dass die Zielgruppe dieses Dokumentes genau diese Vorwissen bereits besitzt. Darüberhinaus wird die Leserlichkeit stark verbessert, wenn der Lesefluss nicht ständig durch Definitionen unterbrochen wird. \newline
Um ein möglichst großes Spektrum der Zielgruppe zu erreichen, werden wichtige und nicht Allgemeine Begriffe erläutert und es werden genügend Beispiele gegeben. Für die Arbeit eher unwichtige Begriffe, sowie nebensächliche Beispiele werden aber meistens nur im [Anhang] oder der Fußnote erwähnt. Andererseits würde dies den leseflus zu sehr beeinträchtigen.