\subsubsection{Abschnitte der Arbeit}\myCheckmark
%Warum -> Überblick + wie wird das Problem gelöst -> Inhalt + Relevanz + Reihenfolge
Um einen Überblick zu schaffen und zu vermitteln, wie versucht wird sich an die Lösung des Problems heranzuarbeiten, folgt eine Übersicht zu den Abschnitten der Arbeit. Dabei soll die Übersicht den Inhalt, die Relevanz und die Reihenfolge der Abschnitte erläutern.%

\myNewSection%
%PcVsPhone -> erste zu stellende Frage
Um das in \secref{section:zielsetzung} erwähnte Ziel umzusetzen muss sich zuerst die Frage gestellt werden, was überhaupt die Unterschiede zwischen Pc und Handy sind. Daher wird in \secref{section:pcVsPhone} auch versucht genau diese Frage passend für diese Arbeit zu beantworten.\newline%
%Anforderungen -> wichtig früh zu stellen
Anschließend entsteht die Frage, was die zu entwerfende Software überhaupt alles leisten soll. Mit dieser Aufgabe wird sich in \secref{section:anforderungen} auch relativ früh befasst, da sich besonders hier Schwierigkeiten und Ziele für die nächsten Teilschritte schnell auffinden lassen. Dadurch fällt das planen und bearbeiten der nächsten Schritte also leichter.\newline%
%Technologische Überlegungen -> wichtig da erfüllt Anforderungen + erst nach der Erhebung
Im darauf folgenden \secref{section:technologischeUeberlegungen} wird sich Gedanken über die Auswahl von Technologien gemacht. Das hat den Grund, da bereits die Auswahl von Technologien Auswirkungen auf Funktionale und nicht Funktionale Anforderungen haben können. Daher ist es auch wichtig, diesen Abschnitt erst nach der Anforderungserhebung zu behandeln.
	%Beispiel
	Man stelle sich vor es wird zuerst ein Framework welches bekannt für seine langsame Ausführung ist gewählt und erst zu einen späteren Zeitpunkt wird die Anforderung einer \dq schnelle Performance\dq erhoben. Durch diesem Konflikt müssten die Wahl des Framework neu überdacht werden, was wichtige Bearbeitungszeit verschwenden könnte.\newline%
%Design -> starke Auswirkung auf Qualität
Im \secref{section:design} wird sich Gedanken über das äußere sowie innere Design der App gemacht. Anders gesagt also Überlegungen zu der grafischen Oberfläche sowie der Architektur. Dieser Abschnitt wird behandelt weil, beide dieser Punkte starke Auswirkungen auf Qualität der App ausüben können. Dabei würde die grafischen Oberfläche besonders die Qualität für den Endnutzer beeinflussen, da dies das einzige ist mit dem Benutzer interagiert. Gleicherweise würde die Architektur die Qualität für den Entwickler entscheiden, da der Quelltext seine Schnittstelle darstellt.\newline
%Implementierung - Was + Warum wenig Implementierung wiedergegeben
An dieser Stelle sollten die wichtigsten Erkenntnisse erhoben worden sein. Nun gehen wir über zum \secref{section:implementierung}. Hier wird betrachtet wie die Software aufgebaut sein soll. Also geht es auch hier unter anderem um das \dq innere Design\dq.\newline%
Die Implementierung, ein Großteil der eigentlichen Arbeit von diesem Kapitel, wird nicht wiedergegeben. Jede einzelne Entscheidung und Erkenntnis der Implementierung zu erwähnen und begründen würde nicht nur den Rahmen sprengen, sonder auch sehr ermüdend für den Leser werden. Daher werden nur die wichtigsten Entscheidungen, Schwierigkeiten und Erkenntnisse erwähnt.\newline%
%Evaluation - Warum
Um zuletzt festzustellen ob wir auch wirklich das erschaffen haben, was wir uns zuvor als Ziel setzten, wird die Software mithilfe verschiedener Methoden im \secref{section:evaluation} Validiert.\newline%
%Fazit - Was
Zum Schluss im \secref{section:fazit} folgt ein Fazit und der Ausblick für diese Arbeit. Dabei werden unteranderem die während der Arbeit getroffenen Schwierigkeiten erwähnt, das Resultat bewertet und es wird ein Blick in die Zukunft geworfen.


%todo -old version- remove
\myComment{
	Um das in ... erwähnte Ziel umzusetzen werden eine Vielzahl von Ingenieurmäßigen-Techniken angewandt. \newline
	In diesem Kapitel ... wurde geklärt was die Software überhaupt leisten könnte und was für einen Nutzen das bringt. \newline
	So behandelt das Kapitel ... die Auseinandersetzung mit der Frage, was solch eine Software überhaupt alles leisten soll. Diese Aufgabe sollte möglichst früh bedacht werden, da sich besonders hier Schwierigkeiten und Ziele für die nächsten Teilschritte schnell auffinden lassen. Dadurch fällt das gestalten der nächsten Schritte also sehr viel leichter. \newline
	Im darauf folgende Kapitel ... wird sich Gedanken über die Auswahl von Technologien gemacht. Dabei wird es gezielt nach der Anforderungserhebung behandelt, weil sich bereits dieser Schritt Auswirkungen auf Funktionale und nicht Funktionale Anforderungen haben kann. Man stelle sich vor wir wählen zuerst ein Framework welches bekannt für seine langsame Ausführung ist und erst danach erheben wir die nicht Funktionale Anforderung "schnelle Performance", dann entsteht dadurch ein Konflikt und wir müssten die Wahl über das Framework neu überdenken. \newline
	Das nächste Kapitel ... ist sehr wichtig. Es werden sich Gedanken über das äußere sowie innere Design der App gemacht. Also Überlegungen des Grafischenoberfläche sowie der Architektur. Beide dieser Punkte können enorme Auswirkungen auf Qualität der App ausüben. Dabei würde die Grafischenoberfläche besonders die Qualität für den Endnutzer beinflussen und das 'innere Design' die Qualität für den Entwickler.\newline
	 An dieser Stelle sollten die wichtigsten Erkenntnisse erhoben worden sein. Nun gehen wir über zum Kapitel ... . Hier wird Überdacht wie die Software aufgebaut sein soll. Also auch hier geht es unter anderem wieder um das 'innere Design'. Ein Großteil der Arbeit in diesem Kapitel wird aber nicht reflektiert/wiedergegeben, undzwar die eigentliche Implementierung. Jede einzelne Implementierungs-Entscheidung zu erwähnen und Begründen würde nicht nur den Rahmen sprengen, sonder auch sehr ermüdent sein. Daher werden nur die wichtisten und allgemeinsten(für viele Punkte zutreffenden) Entscheidungen und Schwierigkeiten erwähnt \newline
	 Um zuletzt festzustellen ob wir auch wirklich das erschaffen haben, was wir uns zuvor als Ziel setzten, wird die Software mithilfe verschiedener Methoden Validiert. <ref>
}
 

 