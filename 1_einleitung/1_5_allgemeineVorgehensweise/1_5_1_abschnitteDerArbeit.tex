\subsubsection{Abschnitte der Arbeit}
Um das in ... erwähnte Ziel umzusetzen werden eine Vielzahl von Ingenieurmäßigen-Techniken angewandt. \newline
In diesem Kapitel ... wurde geklärt was die Software überhaupt leisten könnte und was für einen Nutzen das bringt. \newline
So behandelt das Kapitel ... die Auseinandersetzung mit der Frage, was solch eine Software überhaupt alles leisten soll. Diese Aufgabe sollte möglichst früh bedacht werden, da sich besonders hier Schwierigkeiten und Ziele für die nächsten Teilschritte schnell auffinden lassen. Dadurch fällt das gestalten der nächsten Schritte also sehr viel leichter. \newline
Im darauf folgende Kapitel ... wird sich Gedanken über die Auswahl von Technologien gemacht. Dabei wird es gezielt nach der Anforderungserhebung behandelt, weil sich bereits dieser Schritt Auswirkungen auf Funktionale und nicht Funktionale Anforderungen haben kann. Man stelle sich vor wir wählen zuerst ein Framework welches bekannt für seine langsame Ausführung ist und erst danach erheben wir die nicht Funktionale Anforderung "schnelle Performance", dann entsteht dadurch ein Konflikt und wir müssten die Wahl über das Framework neu überdenken. \newline
Das nächste Kapitel ... ist sehr wichtig. Es werden sich Gedanken über das äußere sowie innere Design der App gemacht. Also Überlegungen des Grafischenoberfläche sowie der Architektur. Beide dieser Punkte können enorme Auswirkungen auf Qualität der App ausüben. Dabei würde die Grafischenoberfläche besonders die Qualität für den Endnutzer beinflussen und das 'innere Design' die Qualität für den Entwickler.\newline
 An dieser Stelle sollten die wichtigsten Erkenntnisse erhoben worden sein. Nun gehen wir über zum Kapitel ... . Hier wird Überdacht wie die Software aufgebaut sein soll. Also auch hier geht es unter anderem wieder um das 'innere Design'. Ein Großteil der Arbeit in diesem Kapitel wird aber nicht reflektiert/wiedergegeben, undzwar die eigentliche Implementierung. Jede einzelne Implementierungs-Entscheidung zu erwähnen und Begründen würde nicht nur den Rahmen sprengen, sonder auch sehr ermüdent sein. Daher werden nur die wichtisten und allgemeinsten(für viele Punkte zutreffenden) Entscheidungen und Schwierigkeiten erwähnt \newline
 Um zuletzt festzustellen ob wir auch wirklich das erschaffen haben, was wir uns zuvor als Ziel setzten, wird die Software mithilfe verschiedener Methoden Validiert. <ref>