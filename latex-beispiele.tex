% !TeX encoding = UTF-8
 
\section{Beispiele} 
 
%sections
\subsection{Zu dieser \LaTeX{}-Vorlage}
\subsubsection{Optionen der Dokumentenklasse}
\subsubsection{Tabelle}

%cite
Siehe \zb \cite{Dje06, DjeOezSal07, CocWil00}.

%tabelle
\begin{table}[ht]
\begin{center}
    \begin{tabular}{|l|L{5.5cm}|L{4cm}|}
        \hline
        \textbf{Schlüssel} & \textbf{Funktion} & \textbf{Default-Wert} \\
        \hline
        \texttt{student/id} & Matrikel-Nummer & -- \\
        \texttt{student/mail} & E-Mail-Adresse & -- \\
        \texttt{thesis/type} & Art der Abschlussarbeit & "`Bachelorarbeit"' \\
        \texttt{thesis/group} & Arbeitsgruppe in der die Arbeit geschrieben
        wurde & "`Arbeitsgruppe Software Engineering"' \\
        \texttt{thesis/advisor} & \emph{optional:} Betreuer der Abschlussarbeit
        & -- \\
        \texttt{thesis/examiner} & Erstgutachter der Arbeit & -- \\
        \texttt{thesis/examiner/2} & \emph{optional:} Zweitgutachter der Arbeit
        & -- \\
        \texttt{thesis/date} & \emph{optional:} Datum der Abgabe & aktuelles
        Datum\\
        \texttt{title/size} & \emph{optional:} \LaTeX-Schriftgröße für den
        Titel (\zb \texttt{\textbackslash{}LARGE}) & wird automatisch gesetzt \\
        \texttt{abstract/separate} & \emph{optional:} Schlüssel ohne Wert;
        falls gesetzt, wird der Abstract auf eine eigene Seite gesetzt und die
        Titelseite ist "`luftiger"' & -- \\
        \hline
    \end{tabular}
    \caption{Schlüssel-Wert-Konfiguration des
    \texttt{\textbackslash{}coverpage}-Kommandos.}
    \label{tab:coverpage-config}
\end{center}
\end{table}




%codeabschnitt %morekeywords makiert
\begin{lstlisting}[language={[LaTeX]TeX}, morekeywords={coverpage}] 
\coverpage[
    student/id=1234567,
    student/mail=email@inf.fu-berlin.de,
    thesis/type=Masterarbeit,
    thesis/examiner={Prof. Dr. Mia Maus}
]
{
    Prokrastination ist ein gut verstandenes Verhalten,
    das auch vor Abschlussarbeitern mit Informatik-Hintergrund
    nicht halt macht.
    % ...
}
\end{lstlisting}

%Klassen Schrift
%Footnote
Quellcode der Klasse \texttt{agse-thesis}:\footnote{Es ist nicht üblich,
den gesamten produzierten Quellcode bei einer Abschlussarbeit in Textform
abzugeben.}

