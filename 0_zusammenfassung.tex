%%% hidden subsection for a better structure in latex editor: "texifier"
\myComment{\section*{Zusammenfassung}} 
%
%\input{stichpunkte0.tex} %todo: remove after this sections completion
%
%\myNewSection
%\textbf{WisschenschaftlichesSchreiben:}
%\\ Zusammenfassung des Inhalts
%\\ - wird am häufigsten gelesen
%\\ - max länge vorgegeben (150-300 Wörter)
%
%
%
%
%
%
\textit{Ziele:} In dieser Arbeit wurde versucht, eine passende App für CLI-Terminkalender zu erstellen. \glqq Passend\grqq{} bedeutet in diesem Sinne, dass die Stärken von PC und Handy berücksichtigt werden und somit in die zu erstellende Anwendung einfließen.
%
%
%
%
%
\newline%
\newline%
\textit{Methoden:} 
	Dazu wurde zunächst versucht herauszufinden, was die Stärken von Handys und PCs überhaupt sind, indem die wesentlichen Unterschiede zwischen ihnen verglichen wurden.
	%Anforderungen
	Anschließend wurden mithilfe von Erhebungstechniken die Anforderungen an die zu erstellende App ermittelt und festgehalten.
	%Technologische Überlegungen
	Auf Basis dieser Anforderungen wurden Gedanken zur Auswahl der passenden Technologien gemacht. 
	%Design
	Danach wurde das Design der Anwendung erstellt und überdacht, da dies ein wichtiger Punkt für die Anforderungen sowie eine mögliche Stärken von Handys ist.
	%Implementierung
	Da nun die Anforderungen und das Design der Anwendung bekannt sind sowie die Technologie, mit der sie erstellt werden soll, wurde sich anschließend mit der Implementierung befassen.
	%Evaluation
	Zum Schluss wurde die Anwendung mit einer Anforderungsverifizierung evaluiert, um festzustellen, ob das Erstellte auch das erfüllt, was sich zuvor vorgenommen wurde.
%
%
%
% 
%
\newline%
\newline%
\textit{Ergebnisse}:
%Was
Eine wichtige Erkenntnis, die während der Erarbeitung der Stärken gewonnen wurde, ist, dass für die Intuitivität und Benutzerfreundlichkeit einer Handyanwendung das Design und dessen Richtlinien ein entscheidender Faktor sind. 
		%Handy
		Weiterhin kam es zu dem Ergebnis, dass Handys für jene Aufgaben gut geeignet sind, die kurzweilig sind oder wenig Zeit benötigen, die unterwegs gelöst werden sollen und einfach sowie intuitiv sein sollen. 
		\newline%
		%Pc
		Während PCs gut für jene Aufgaben sind, die viel Leistung benötigen, schnelle, präzise oder vielfältige Eingaben erfordern, viele Informationen gleichzeitig darstellen oder benötigen sowie viele Optionen und Konfigurationen anbieten oder benötigen.
	%Anwendung Resultat
	Laut der Evaluation entstand während der Arbeit eine Anwendung, die in der Lage ist, genau diese Stärken umzusetzen.
%
%
%
% 
%
\newline%
\newline%
\textit{Schlussfolgerungen:} 
Es wird angenommen, dass das Ziel dadurch erfüllt werden konnte. Weiter wird sogar vermutet, dass es sich bei dieser App um eine nützliche Anwendung handeln könnte.
%
%
%
%
%
\newline%
\newline%
\textit{Ausblick}:
	%Generell
			%Generell (rest auf app bezogen)
		Falls sich die Anwendung als nützlich beweist, könnte dies bedeuten, dass es für Anwendungen durchaus lohnenswert sein kann, diese unter Berücksichtigung der Stärken sowohl von Handys als auch von PCs zu entwickeln. Möglicherweise könnte dies zeigen, dass Anwendungen nicht unbedingt als eigenständige Einheiten genutzt werden müssen, sondern voneinander abhängig sein und dennoch oder gerade deshalb nützlich sein können.
	\newline%
	%App
		%evaluation
		Um jedoch die Nützlichkeit abschließend bewerten zu können, benötigt es weiterer Evaluation. %Insbesondere ein Nutzertest könnte hierbei nützliche Erkenntnisse liefern.
		Auch konnte die Anwendung noch nicht vollständig implementiert werden. So fehlen noch einige wenige Funktionen, Design Anpassungen und ausprägungen von nicht funktionalen Anforderungen.