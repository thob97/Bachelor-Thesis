\subsection{Entwicklungsumgebung}
\myTextTodo{Maybe Zusammenfassung:Im folgendem Abschnitt werden benutzte Anwendungen und dessen Versionen genannt und begründet warum genau diese benutzen werden. }
%Was: Einleitung
In diesem Abschnitt werden die benutzten Software und dessen Versionen genannt.
%Warum: Einfluss
Das hat mehrere Gründe. Zu einem kann die Wahl von Software und dessen Versionen einen starken Einfluss auf das Endprodukt ausüben. So kann durch die neue Flutter Version zum Beispiel \glqq Menu bars and cascading menus\grqq benutzt werden, während ältere Versionen diese Funktionen nicht unterstützen\cite{tech_flutterUpdate}. Außerdem wäre es mit der Wahl der Software Windows nicht möglich Applikationen für iOS zu entwickelt. 
	%Warum: Nachbilden + Kompatibilität
	Ein weiterer Grund, warum die verwendeten Softwares genannt werden, ist, dass so das Nachbilden der Applikation garantiert sein sollte. Mit verschiedener Software oder anderen Versionen ist es aus Erfahrung durchaus Vorstellbar, dass es ansonsten zu Komplikationen kommen kann. So hat zum Beispiel während der Ausarbeitung Android Studio nach einem Update nicht mehr richtig funktioniert. Es kam also durch eine neuere Version zu Komplikationen, wodurch wichtige Arbeitszeit verlogen ging.
		%->Neuste Versionen
		Deswegen wurden zu diesem Zeitpunkt in der Arbeit alle Softwares auf die neuste Version aktualisiert und danach bis zum Ende der Arbeit auch auf diesen Versionen belassen. Einerseits sollten damit möglichst alle neuen Funktionen und Bugfixes [benutzbar werden]. Aber viel wichtiger noch sollten weitere Komplikationen in der Zukunft damit verhindert werden.
		\newline

\begin{enumerate}
	%Was+Warum: Betriebssystem -> iOS Apps
	\item Betriebsystem: Für das Betriebssystem standen Windows und MacOS zur verfügung. Es wurde MacOs entschieden, da auf diesem Betriebsystem iOS sowie Android Apps entwickelt werden können. Version: macOS Ventura 13.2.1
	%Was+Warum: IDE
	\item IDE: Flutter empfiehlt unteranderem Visual Studio Code, Android Studio oder Emacs als Editor zu nutzen\cite{tech_ideSuggestion}. Da Android Studio genau wie Flutter beide von Google entwickelt wurden, wird erwartet, dass der Editor besonders gut auf das Framework abgestimmt ist. Deshalb und weil Android Studio [genau passend] für die App-Entwicklung ausgelegt ist, wurde er schlussendlich als Editor ausgewählt. Version: 2022.1.1
	%Was: Flutter
	\item Flutter: Das benutzte Framework. Mehr dazu in \secref{subsection:auswahlDesFrameworks}. Version: 3.7.3
	%Was+Warum:Emulatoren
	\item Emulatoren:%
		%Warum:Testen
		Zum testen der App benötigt es Emulatoren. Um sicherzustellen, dass die Anwendung auf Android und iOS läuft, wurde jeweils ein iPhone und ein Pixel-Phone zum testen benutzt. 
		%Was+Warum: Reales Handy -> Performance
		Des Weiteren wurde auch noch ein echtes Handy zum Testen verwendet, da unter realen Handy Bedienungen das testen der Performance aussagekräftiger ist. Dabei wurde gezielt das relativ alte Handy iPhone SE 2 gewählt. 
			%Warum: Alter
			Dadurch lässt sich dann gut prüfen, ob die Anwendung auch von älterer und schwächerer Hardware unterstützt wird. 
			%Warum: größe
			Außerdem ist das Handy im Gegensatz zu neuen Handys relativ klein mit einer Bildschirmdiagonalen von vier Zoll. Wodurch sich prüfen lässt, ob das Design der App auch auf kleineren Bildschirmen funktioniert. 
		%Versionen
		Versionen: IPhone 14 iOS 16.2, Pixel 6 Android 13.0, iPhone SE1 iOS 15.7.3
	%Was+Warum: Weiteres
	\item Weiteres: Da die folgende Software standartmäßig für die Entwicklung benötigt benötigt wird, wird außer der Nennung der Version nicht näher auf sie eingegangen. Xcode 14.2, Android SDK Platform-Tools: 34.0.0
\end{enumerate}

%OS-Umgebung -> Iphone + Android entwickelbar
%Ide: Von Flutter empfohlen + vorherige erfahrung mit intellij umgebungen
%Versionen: neuste aber danach nicht weiter geupdatet... dart, flutter, ide, os, emulatoren
%Emulatoren: echte hardware + viele emulatoren