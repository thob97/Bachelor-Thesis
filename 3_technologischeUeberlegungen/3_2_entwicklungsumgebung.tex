\subsection{Entwicklungsumgebung}\label{subsection:entwicklungsumgebung}%
%Was: Einleitung
In diesem Abschnitt werden die verwendeten Software und deren Versionen aufgelistet. %
%Warum: Einfluss
Die Auswahl der Software und ihrer Versionen kann bereits einen Einfluss auf das Endprodukt haben. So wäre es beispielsweise durch die Wahl von Windows nicht möglich, Anwendungen für iOS zu entwickeln.\newline%
	%Warum: Nachbilden + Kompatibilität
	Darüber hinaus wird durch die Nennung der Software sichergestellt, dass die Anwendung reproduzierbar ist. Durch die Verwendung unterschiedlicher Software oder anderer Versionen können erfahrungsgemäß Komplikationen auftreten. Zum Beispiel funktionierte Android Studio nach einem Update während der Arbeit nicht mehr ordnungsgemäß. %
		%->Neuste Versionen
		Daher wurden alle verwendeten Software zu diesem Zeitpunkt auf die neueste Version aktualisiert und bis zum Ende der Arbeit auf dieser Version beibehalten. %Einerseits sollten damit möglichst alle neuen Funktionen und Bugfixes [benutzbar werden]. Aber viel wichtiger noch sollten weitere Komplikationen in der Zukunft damit verhindert werden. \newline%

\begin{enumerate}
	%Was+Warum: Betriebssystem -> iOS Apps
	\item Betriebsystem: Für das Betriebssystem standen Windows und MacOS zur Verfügung. Es wurde sich für MacOS entschieden, da auf diesem Betriebssystem die Entwicklung von iOS- und Android-Apps möglich ist. Version: macOS Ventura 13.2.1%

	%Was+Warum: IDE
	\item IDE: Flutter empfiehlt unter anderem Visual Studio Code, Android Studio oder Emacs als Editor\cite{tech_ideSuggestion}. Da sowohl Flutter als auch Android Studio von Google entwickelt wurden, wird erwartet, dass der Editor besonders gut auf das Framework abgestimmt ist. Aus diesem Grund und weil Android Studio speziell für die App-Entwicklung ausgelegt ist, wurde er schlussendlich ausgewählt. Version: 2022.1.1%

	%Was: Flutter
	\item Framework: Flutter. Mehr dazu in \secref{subsection:auswahlDesFrameworks}. Version: 3.7.3%

	%Was: Versionsverwaltung
	\item Versionsverwaltung: 
		Für die Versionsverwaltung wurde GitLab gewählt, %
		%Warum: Erfahrung und hosting
		da im Studium damit bereits Erfahrung gesammelt wurde.\newline%
		%Was+Warum: Einstellungen -> Verständlich + weiterentwickelndes
		Um ein verständliches und leicht weiterentwickelbares Projekt zu erstellen, wurden einige Einstellungen in GitLab getroffen. %
			%konventional commits
			So werden beispielsweise einheitliche und aussagekräftige Commit-Nachrichten verwendet, sodass sie auch für Dritte die Nachricht nachvollziehen können. Dabei wurde sich an das bereits existierende Regelwerk \glqq Conventional Commits\grqq{}\cite{tech_conventionalCommits} gehalten. %
			%ci/cd pipelines
			Darüber hinaus wurde eine CI/CD-Pipeline erstellt, um Tests automatisch durchzuführen. Auf diese Weise werden Entwickler automatisch auf mögliche Fehler ihrer Änderungen aufmerksam gemacht. %
			%Version
			Version: git 2.37.1 (Apple Git-137.1)

	%Was+Warum:Emulatoren
	\item Emulatoren: %
		%Warum:Testen
		Zum Testen der App werden Emulatoren verwendet. Um sicherzustellen, dass die Anwendung auf Android und iOS sowie auf den neuesten und älteren Betriebssystemversionen läuft, werden sowohl ein iPhone- als auch ein Android-Emulator mit der jeweils neuesten und ältesten verfügbaren Version verwendet. %
		%Was+Warum: Reales Handy -> Performance
		Zusätzlich wurde ein echtes Handy für Tests verwendet, da vermutet wird, dass unter realen Bedingungen die Performance-Tests aussagekräftiger sind. Hierbei wurde das iPhone SE1 gezielt ausgewählt, da es sich um ein relativ altes Handy handelt. %
			%Warum: Alter
			Dadurch sollte sich prüfen lassen, ob die Anwendung auch von älterer und schwächerer Hardware unterstützt wird. % 
			%Warum: größe
			Außerdem hat das Handy im Gegensatz zu neueren Geräten eine relativ kleine Bildschirmdiagonale von vier Zoll. Dadurch lässt sich prüfen, ob das App-Design auch auf kleineren Bildschirmen funktioniert. %
		%Versionen
		Versionen: IPhone 14 iOS 16.2 \& 13.7, Pixel 6 Android 13.0 \& 5.0, iPhone SE1 iOS 15.7.3%
		
	\item CLI-Terminalkalender: When. Mehr dazu im \secref{section:tech:sub:cli_terminkalender}. Version 1.1.45
		
	%Was+Warum: Weiteres
	\item Weiteres: Die folgende Software wird von einigen der zuvor genannten Technologien standardmäßig benötigt. Daher wird hier lediglich die Version aufgeführt, ohne näher darauf einzugehen: Xcode 14.2, Android SDK Platform-Tools: 34.0.0, DevTools: 2.20.1, Dart 2.19.2%
\end{enumerate}

%OS-Umgebung -> Iphone + Android entwickelbar
%Ide: Von Flutter empfohlen + vorherige erfahrung mit intellij umgebungen
%Versionen: neuste aber danach nicht weiter geupdatet... dart, flutter, ide, os, emulatoren
%Emulatoren: echte hardware + viele emulatoren