\subsection{Entwicklungsumgebung}\label{subsection:entwicklungsumgebung}\myCheckmark%
%Was: Einleitung
In diesem Abschnitt werden die benutzten Software und dessen Versionen genannt. %
%Warum: Einfluss
Denn die Wahl von Software und dessen Versionen kann bereits einen Einfluss auf das Endprodukt ausüben. So wäre es zum Beispiel durch die Wahl von Windows nicht möglich Applikationen für iOS zu entwickelt.\newline%
	%Warum: Nachbilden + Kompatibilität
	Außerdem wird durch die Nennung der Software das Nachbilden der Applikation garantiert. Mit verschiedener Software oder anderen Versionen ist es aus Erfahrung durchaus Vorstellbar, dass es zu Komplikationen kommen kann. So wurde zum Beispiel während der Ausarbeitung Android Studio nach einem Update nicht mehr funktionsfähig. %
		%->Neuste Versionen
		Deswegen wurden zu diesem Zeitpunkt in der Arbeit alle Softwares auf die neuste Version aktualisiert und danach bis zum Ende der Arbeit auch auf diesen Versionen belassen. %Einerseits sollten damit möglichst alle neuen Funktionen und Bugfixes [benutzbar werden]. Aber viel wichtiger noch sollten weitere Komplikationen in der Zukunft damit verhindert werden. \newline%

\begin{enumerate}
	%Was+Warum: Betriebssystem -> iOS Apps
	\item Betriebsystem: Für das Betriebssystem standen Windows und MacOS zur verfügung. Es wurde MacOs entschieden, da auf diesem Betriebsystem für iOS sowie Android Apps entwickelt werden können. Version: macOS Ventura 13.2.1%

	%Was+Warum: IDE
	\item IDE: Flutter empfiehlt unteranderem Visual Studio Code, Android Studio oder Emacs als Editor zu nutzen\cite{tech_ideSuggestion}. Da Android Studio genau wie Flutter beide von Google entwickelt wurden, wird erwartet, dass der Editor besonders gut auf das Framework abgestimmt ist. Deshalb und weil Android Studio für die App-Entwicklung ausgelegt ist, wurde er schlussendlich als Editor ausgewählt. Version: 2022.1.1%

	%Was: Flutter
	\item Framework: Flutter. Mehr dazu in \secref{subsection:auswahlDesFrameworks}. Version: 3.7.3%

	%Was: Versionsverwaltung
	\item Versionsverwaltung: 
		Für die Versionsverwaltung wurde sich für GitLab entschieden. %
		%Warum: Erfahrung und hosting
		Einerseits da im Studium damit bereits Erfahrung gesammelt wurde und andererseits weil die Universität ihre eigene Version dazu bereitstellt.\newline%
		%Was+Warum: Einstellungen -> Verständlich + weiterentwickelndes
		Um ein möglichst Verständliches und einfach zu weiterentwickelndes Projekt zu erstellen, wurden außerdem einige Einstellung in GitLab getroffen. %
			%konventional commits
			So werden einerseits einheitliche und ausschlaggebende Commit-Nachrichten benutzt, damit diese bei späterer Betrachtung verständlich sind. Dabei wurde sich an das bereits existierende Regelwerk \glqq Conventional Commits\grqq{}\cite{tech_conventionalCommits} gehalten. %
			%ci/cd pipelines
			Des Weiteren wurde eine CI/CD pipeline erstellt um Tests automatisch zu überprüfen. Dadurch werden Entwickler automatisch auf mögliche Fehler ihrer Änderungen aufmerksam gemacht. %
			%Version
			Version: git 2.37.1 (Apple Git-137.1)

	%Was+Warum:Emulatoren
	\item Emulatoren: %
		%Warum:Testen
		Zum testen der App werden Emulatoren benutzt. Um dabei sicherzustellen, dass die Anwendung auf Android und iOS läuft sowie auf den neusten und älteren Betriebssystemversionen, wurde jeweils ein iPhone und ein Android Emulator mit der neusten und älteste verfügbaren Version zum testen benutzt. %
		%Was+Warum: Reales Handy -> Performance
		Des Weiteren wurde auch noch ein echtes Handy zum Testen verwendet, da vermutet wird, dass unter realen Bedienungen das testen der Performance aussagekräftiger ist. Dabei wurde mit dem iPhone SE1 gezielt ein relativ altes Handy gewählt. %
			%Warum: Alter
			Dadurch lässt sich nämlich gut prüfen, ob die Anwendung auch von älterer und schwächerer Hardware unterstützt wird. % 
			%Warum: größe
			Außerdem ist das Handy im Gegensatz zu neuen Handys relativ klein mit einer Bildschirmdiagonalen von vier Zoll. Wodurch sich prüfen lässt, ob das Design der App auch auf kleineren Bildschirmen funktioniert. %
		%Versionen
		Versionen: IPhone 14 iOS 16.2 \& 13.7, Pixel 6 Android 13.0 \& 5.0, iPhone SE1 iOS 15.7.3%
		
	\item CLI-Terminalkalender: When. Mehr dazu im \secref{subsections:cli_termincalendar}. Version 1.1.45
		
	%Was+Warum: Weiteres
	\item Weiteres: Die folgende Software wird standardmäßig durch einige der zuvor genanten Technologien benötigt. Dementsprechend wird außer der Nennung der Version nicht näher auf sie eingegangen: Xcode 14.2, Android SDK Platform-Tools: 34.0.0, DevTools: 2.20.1, Dart 2.19.2%
\end{enumerate}

%OS-Umgebung -> Iphone + Android entwickelbar
%Ide: Von Flutter empfohlen + vorherige erfahrung mit intellij umgebungen
%Versionen: neuste aber danach nicht weiter geupdatet... dart, flutter, ide, os, emulatoren
%Emulatoren: echte hardware + viele emulatoren