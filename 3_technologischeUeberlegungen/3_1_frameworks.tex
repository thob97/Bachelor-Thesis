\subsection{Auswahl des Frameworks}\label{subsection:auswahlDesFrameworks}\myCheckmark%
%Einleitung
In diesem Abschnitt wird die Frage behandelt, wie die Anwendung erstellt werden soll. %
%Was: App vs Web.
Dabei bestehen die Optionen die Anwendung als eine Webseite oder als Applikation zu erstellen. %
	%Warum: Plattformunabhängig VS UserPreference, Gesten, Design Richtlinien
	Zwar bietet eine Webseiten einige Vorteile über Applikationen, zum Beispiel, dass sie Plattformunabhängig sind. Wie zuvor in \secref{section:pcVsPhone} erwähnt, können Apps durch Richtlinien und Gesten aber um einiges intuitiver sein. Außerdem, vielleicht genau deswegen, scheinen Nutzer auch generell eher Apps zu präferieren\cite{pcVsphone_mobileAppVsWebTimeSpent}. %
	%Auswirkung: -> App
	Daher soll die Anwendung als Applikation entwickelt werden. %

\myNewSection
%Was: Framework
Nun kann sich für ein Framework entschieden werden. %
%Was+Def: Native vs Crossplatform
Hier [bestehen] die Optionen ein Native-Frameworks oder Cross-Platform-Frameworks zu wählen.\newline% 
	%Todo: remove useless beispiel:
		%Dabei sind Native-Frameworks diejenigen, welche die für die Platform spezifischen tools benutzt. Während Cross-Platform-Frameworks ihre eigenen tools anbieten.
		%Warum: Native: Performance
	Generell bieten Native-Frameworks eine bessere Performance als Cross-Platform-Frameworks \cite{tech_performanceReactNativeVsFlutter1, tech_performanceReactNativeVsFlutter2}. %
	%Was: Native: Aussehen
	Außerdem sehen und fühlen sich die nativ erstellte Apps einheitlich mit der Platform an, da für diese Apps plattformspezifischen Funktionen und Komponenten benutze werden.
		%Todo: remove useless beispiel:
			%, wie zum Beispiel die Schieberegler von iOS [\ref{pic:schieberegler}]. 
		%Warum: intuitive
		Das würde wahrscheinlich zu einer intuitiveren Benutzung für den Nutzer führen, da dies für ihm bereits bekannte Muster und Funktionen wären. %
		%Auswirkung: 
		Da Performance und Benutzbarkeit zwei Anforderungen für diese Arbeit sind scheinen native-SDKs eine gute Wahl für diese Arbeit zu sein.\newline%
	%Was: Crossplatform: Relativierung
	Jedoch wurde sich für eine Cross-Platform-Framework entschieden. %
		%Performance
		Einerseits wird der Performance-Verlust als marginal eingeschätzt, da es sich bei der zu erstellenden App wahrscheinlich um eine simple Anwendung ohne schwierige Berechnungen oder aufwändigen Animationen handeln wird. 
		%Aussehen
		Andererseits kann die grafische Oberfläche auch versucht werden mit Cross-Platform-Frameworks passend für das System zu erstellen. Zwar wäre das ein größerer Aufwand als wie bei Native-Frameworks,
		%Warum: Zeit + Crossplatform
		aber dafür sind die Anwendung von Cross-Platform-Frameworks mit iOS und Android kompatible. Wie in \nameref{section:anforderungen} besprochen, ist dies der wichtigste Punkt, um eine große Reichweite zu ermöglichen. Zwar wäre dies auch möglich, indem für iOS und Android jeweils eine eigene Codebasis über native-SDKs erstellt werden, jedoch müssten dementsprechend dann auch zwei Codebasen gepflegt werden. Durch die relativ kurze Bearbeitungszeit dieser Arbeit scheint diese Idee also weniger sinnvoll. Stattdessen wird versucht mithilfe Cross-Platform-Frameworks möglichst schnell und Zeiteffizient eine evaluierbare Anwendung für beide Plattformen zu entwickeln. Falls sich die App später als Nützlich und beliebt herausstellen, kann immer noch eine Native-Entwicklung angefangen werden.
	
\myNewSection
%Was: Flutter vs React Native
Zu leztz muss sich für ein konkretes Framework entschieden werden. %
	%Warum: Weiterentwickelbarkeit
	Wie in \nameref{section:anforderungen} erwähnt, ist eine Anforderung die Weiterentwickelbarkeit. Daher ist es einerseits wichtig ein Framework zu wählen, welches möglichst lang unterstützt wird, damit eine Weiterentwicklung in der Zukunft überhaupt möglich ist.\newline%
	%%Warum: Beliebtheit
	Andererseits muss auch beachtet werden, dass das Framework beliebt ist und von vielen Personen benutzt wird, damit überhaupt die Möglichkeit besteht, dass sich ein Interessent findet.\newline%
	%Quelle:
	Laut der Update-History und der Sternbewertung von Github sind React-Native mit Flutter die beiden beliebtesten Frameworks für die Cross-Platform-Appentwicklung\cite{tech_flutterStars, tech_reactNativStars}.\newline%
%Warum: Beliebtheit
Wenn man die beiden Frameworks dementsprechend miteinander vergleicht, scheint Flutter beliebter zu sein. So hat Flutter mit 150k Sternen auf GitHub rund 38\% mehr als React-Native\cite{tech_flutterStars, tech_reactNativStars}. Ein ähnliches Anzeichen zeigt auch Google-Trends. So hat Futter fast doppelt soviel Suchanfragen auf Google wie React-Native\cite{tech_googleTrendsFlutterVsReactNative}.\newline%
%Warum: performance
Im Bereich Performance verhält es sich ähnlich. Laut zwei Analysen von inVerita scheint Flutter Ressourcen sparender und schneller als React Native zu sein\cite{tech_performanceReactNativeVsFlutter1, tech_performanceReactNativeVsFlutter2}.\newline%
%Warum: UI
Dafür bietet React Native den Vorteil Plattformspezifische Komponenten und Funktionen zu nutzen. Wie zuvor erwähnt könnte dass zu einer besseren Benutzbarkeit führen.\newline%
%Warum: vordefinierte features
Flutter bietet hingegen, und weshalb sich letztendlich auch für dieses Framework entschieden wurde, viele vordefinierte Komponenten und Features an. So hat React-Native lediglich 25 Core-Komponenten, während Flutter allein für Animationen bereits 22 Komponenten besitzt\cite{tech_componentsFlutter, tech_componentsReactNative}. Diese bereits existierenden Komponenten, können dabei helfen an Zeit zu sparen, da diese sonst [selbst und neu/eigenständig] definiert werden müssten.%