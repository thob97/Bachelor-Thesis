\subsection{Auswahl des Frameworks}\label{subsection:auswahlDesFrameworks}%
%Einleitung
In diesem Abschnitt wird die Frage behandelt, wie die Anwendung erstellt werden soll. %
%Was: App vs Web.
Dabei bestehen die Optionen die Anwendung als eine Webseite oder als Applikation zu erstellen. %
	%Warum: Plattformunabhängig VS UserPreference, Gesten, Design Richtlinien
	Zwar bieten Webseiten einige Vorteile im Vergleich zu Applikationen, zum Beispiel die Plattformunabhängigkeit. Wie zuvor in \secref{section:pcVsPhone} erwähnt, können Apps jedoch durch Richtlinien und Gesten aber um einiges intuitiver sein und Nutzer scheinen diese generell zu präferieren\cite{pcVsphone_mobileAppVsWebTimeSpent}. %
	%Auswirkung: -> App
	Daher soll die Anwendung als Applikation entwickelt werden. %
%
\newline
\myNewSection
%Was: Framework
Nun kann sich für ein Framework entschieden werden. %
%Was+Def: Native vs Crossplatform
Die erste Wahl liegt dabei zwischen einem Native-Framework oder Cross-Platform-Framework.\newline% 
	%Todo: remove useless beispiel:
		%Dabei sind Native-Frameworks diejenigen, welche die für die Platform spezifischen tools benutzt. Während Cross-Platform-Frameworks ihre eigenen tools anbieten.
		%Warum: Native: Performance
	Generell scheinen Native-Frameworks eine bessere Performance als Cross-Platform-Frameworks zu bieten \cite{tech_performanceReactNativeVsFlutter1, tech_performanceReactNativeVsFlutter2}. %
	%Was: Native: Aussehen
	Außerdem sehen und fühlen sich die nativ erstellte Apps einheitlich mit der Plattform an, da für diese Apps plattformspezifischen Funktionen und Komponenten benutze werden.
		%Todo: remove useless beispiel:
			%, wie zum Beispiel die Schieberegler von iOS [\ref{pic:schieberegler}]. 
		%Warum: intuitive
		Das würde wahrscheinlich zu einer intuitiveren Benutzung für den Nutzer führen, da dies für ihm bereits bekannte Muster und Funktionen wären. %
		%Auswirkung: 
		Da Performance und Benutzbarkeit zwei Anforderungen für diese Arbeit sind scheinen native-SDKs eine gute Wahl für diese Arbeit zu sein.\newline%
	%Was: Crossplatform: Relativierung
	Jedoch wurde sich für eine Cross-Platform-Framework entschieden. %
		%Performance
		Einerseits wird der Performance-Verlust als marginal eingeschätzt, da es sich bei der zu erstellenden App wahrscheinlich um eine simple Anwendung ohne schwierige Berechnungen oder aufwändigen Animationen handeln wird. 
		%Aussehen
		Andererseits kann die grafische Oberfläche auch versucht werden mit Cross-Platform-Frameworks passend für das System zu erstellen. Zwar wäre das ein größerer Aufwand als bei Native-Frameworks,
		%Warum: Zeit + Crossplatform
		aber dafür sind die Anwendungen von Cross-Platform-Frameworks mit iOS und Android kompatibel. Wie in \secref{section:anforderungen} besprochen, ist dies der wichtigste Punkt, um eine große Reichweite zu ermöglichen. Zwar wäre dies auch möglich, indem für iOS und Android jeweils eine eigene Codebasis über native SDKs erstellt würde, jedoch müssten dann auch zwei Codebasen gepflegt werden. Durch die relativ kurze Bearbeitungszeit dieser Arbeit erscheint diese Idee weniger sinnvoll. Stattdessen wird versucht, mithilfe von Cross-Platform-Frameworks möglichst schnell und zeiteffizient eine evaluierbare Anwendung für beide Plattformen zu entwickeln. Falls sich die App später als nützlich und beliebt herausstellt, kann immer noch eine Native-Entwicklung gestartet werden.
	
\myNewSection
%Was: Flutter vs React Native
Zuletzt muss eine Entscheidung für ein konkretes Framework getroffen werden. %
	%Warum: Weiterentwickelbarkeit
	Wie in \nameref{section:anforderungen} erwähnt, ist die Weiterentwickelbarkeit eine wichtige Anforderung. Daher ist es einerseits entscheidend, ein Framework auszuwählen, das möglichst lange unterstützt wird, um eine zukünftige Weiterentwicklung zu gewährleisten.\newline%
	%%Warum: Beliebtheit
	Andererseits muss auch berücksichtigt werden, dass das Framework beliebt ist und von vielen Personen genutzt wird, um die Chance zu erhöhen, dass Interessenten gefunden werden können.\newline%
	%Quelle:
	Basierend auf der Update-Historie und der Sternbewertung auf Github sind React-Native und Flutter die beiden beliebtesten Frameworks für die Cross-Platform-App-Entwicklung\cite{tech_flutterStars, tech_reactNativStars}.\newline%
%Warum: Beliebtheit
Wenn man die beiden Frameworks dementsprechend miteinander vergleicht, scheint Flutter beliebter zu sein. Es hat mit 150.000 Sternen auf GitHub etwa 38\% mehr als React Native\cite{tech_flutterStars, tech_reactNativStars}. Ein ähnliches Ergebnis zeigt sich auch bei Google Trends, da Flutter fast doppelt so viele Suchanfragen wie React Native hat\cite{tech_googleTrendsFlutterVsReactNative}.\newline%
%Warum: performance
Im Bereich der Performance verhält es sich ähnlich. Laut zwei Analysen von inVerita scheint Flutter ressourcensparender und schneller als React Native zu sein\cite{tech_performanceReactNativeVsFlutter1, tech_performanceReactNativeVsFlutter2}.\newline%
%Warum: UI
Dafür bietet React Native den Vorteil, plattformspezifische Komponenten und Funktionen nutzen zu können. Wie zuvor erwähnt wurde, könnte dies zu einer verbesserten Benutzbarkeit führen.
\newline%
%Warum: vordefinierte features
Flutter bietet im Vergleich zu React-Native viele vorgefertigte Komponenten und Features, was letztendlich zur Entscheidung für dieses Framework geführt hat. Während React-Native nur 25 Core-Komponenten hat, besitzt Flutter allein für Animationen bereits 22 Komponenten\cite{tech_componentsFlutter, tech_componentsReactNative}. Diese vordefinierten Komponenten können dabei helfen, zeiteffizient vorzugehen, da sie sonst möglicherweise selbst implementiert werden müssten.%
%
%
%
%