%%% hidden subsection for a better structure in latex editor: "texifier"
\myComment{\subsection*{Übersicht}}
%%%Einleitung: Was+Warum
In diesem Abschnitt werden sich Gedanken zur Auswahl der Technologien gemacht, da diese Auswahl bereits Auswirkungen auf die funktionalen und nicht-funktionalen Anforderungen haben kann.\newline%
%%%Übersicht: xWas
\textbf{Übersicht:} %
Zunächst wird behandelt, mithilfe welchen Frameworks die App erstellt werden soll. Anschließend wird sich für einen CLI-Terminkalender, für die Darstellungen der Erinnerungen und für das Dateiformat der Konfigurationsdatei entschieden. Danach wird die Wahl der verwendeten Software und Entwicklungsumgebung begründet und schließlich werden für die Arbeit relevante Pakete ausgewählt.\newline%
%%%%Ergebnis
\textbf{Ergebnisse:} %
	%Framework
	Im ersten \secref{subsection:auswahlDesFrameworks} wurde behandelt, ob die Anwendung als App oder Webseite und mithilfe von Native oder Cross Platform Frameworks erstellt werden soll. Die Wahl fiel dabei erstens auf eine App, da diese in der Regel intuitiver ausfallen können und beliebter sind, und zweitens auf Cross Platform Frameworks, da damit Apps für iOS und Android erstellt werden können. Zuletzt wurden noch die beiden Frameworks Flutter und React Native miteinander verglichen. Aufgrund von Beliebtheit, Performance und der Annahme, dass damit effizient gearbeitet werden kann, fiel die Wahl auf Flutter.\newline%
	%Terminkalender & Konfigurationsdatei
	Die \nameref{section:tech:sub:cli_terminkalender} viel auf When und das \nameref{section:tech:sub:konfigurationsdateiformat} auf JSON, da durch dieser Wahl erhofft wird, da durch diese Entscheidungen erhofft wird, während der Entwicklung ein evaluierbares Produkt zu erstellen.%
	%Erinnerungen
	Weiter wurde für die \nameref{section:tech:sub:darstellung_der_erinnerungen} Issues gewählt, da diese ein passendes Format besitzen und zudem die GitHub-Webseite genutzt werden kann, um die Erinnerungen einzusehen.\newline%
	%Entwicklungsumgebung
	Im darauf folgenden \secref{subsection:entwicklungsumgebung} wurde entschieden, MacOS als Betriebssystem, Android Studio als IDE, GitLab zur Versionsverwaltung und Emulatoren sowie Handys zum Testen zu nutzen. Es wurde auch erwähnt, dass die Versionen dieser Software während der Arbeit nicht verändert werden, um so mögliche Komplikationen zu vermeiden.\newline%
	%Pakete
	Im letzten \secref{subsection:auswahlDerPakete} wurden die für die Arbeit benötigten Pakete nach ihrem Alter, ihrer Beliebtheit und ihrem Update-Verlauf ausgewählt. Die Wahl fiel auf json\_serializable, tests und flutter\_lints, da sie die Lesbarkeit des Codes verbessern und nützliche Funktionen bieten. Weiter wurde das Paket github ausgewählt, da es das einzige verfügbare Paket ist, das eine Schnittstelle zum Backend bietet. Schließlich wurde das Paket syncfusion\_flutter\_calendar ausgewählt, da es eine Kalendardarstellung bereitstellt und somit Zeit gespart wird, da diese nicht selbst implementiert werden muss.%
%
%
%
%
%Todo - Remove
%\myComment{
%	%Old AllÜbersicht 	
%	\myNewSection
%	\myTextTodo{
%	\textbf{Abschnitte der Arbeit}\\
%	%Technologische Überlegungen -> wichtig da erfüllt Anforderungen + erst nach der Erhebung
%	Im darauf folgenden \secref{section:technologischeUeberlegungen} wird sich Gedanken über die Auswahl von Technologien gemacht. Das hat den Grund, da bereits die Auswahl von Technologien Auswirkungen auf Funktionale und nicht Funktionale Anforderungen haben können. Daher ist es auch wichtig, diesen Abschnitt erst nach der Anforderungserhebung zu behandeln.
%		%Beispiel
%		Man stelle sich vor es wird zuerst ein Framework welches bekannt für seine langsame Ausführung ist gewählt und erst zu einen späteren Zeitpunkt wird die Anforderung einer \dq schnelle Performance\dq erhoben. Durch diesem Konflikt müssten die Wahl des Framework neu überdacht werden, was wichtige Bearbeitungszeit verschwenden könnte.\newline%
%	}
%
%}