\subsection{Auswahl der Pakete}\label{subsection:auswahlDerPakete}
%Einleitung
In diesem Abschnitt wird die Auswahl der Pakete behandelt. Zunächst wird erläutert, warum und nach welchen Kriterien die Pakete generell ausgewählt wurden. Anschließend werden die wichtigsten Pakete aufgelistet und ihre Verwendung begründet.\newline%
%Generelle Kriterien
	%Was
	Die Pakete wurden nach einem ähnlichen Verfahren wie in \secref{subsection:auswahlDesFrameworks} ausgewählt. Es wurde dabei auf ihr Alter, ihre Beliebtheit und den Verlauf ihrer Aktualisierungen geachtet. % 
	%Warum -> long time support
	Dabei wird nämlich angenommen, dass bei einer starken Ausprägung dieser drei Kriterien es wahrscheinlicher ist, dass ein Paket noch lange Zeit weitere Unterstützung und Aktualisierungen erhält. Das ist vorteilhaft, wenn die erstellte Anwendung auch in Zukunft relevant bleiben soll. %
	%Warum -> 
	Zudem kann eine gute Dokumentation und Bekanntheit der Pakete die Verständlichkeit erleichtern. % 
	%Auswirkung
	Daher wurde, falls mehrere Pakete mit ähnlichen Funktionen zur Verfügung stehen, stets das ausgewählt, welches in diesen drei Punkten am besten abschneidet.%
%
%Packages
\myNewSection
Pakete:
\begin{itemize}
	
	\item json\_serializable\cite{tech_packageJson}: %
		%Was
		Dieses Paket erstellt automatisch Klassen für JSON-Serialisierung und -Deserialisierung, die für die Übersetzung von Daten zu JSON-Objekten und umgekehrt benötigt werden. % 
		%Warum: Zeit
		Einerseits wird durch die automatische Erstellung etwas Zeit gespart. % 
		%Warum: Einheitlich
		Einerseits spart die automatische Erstellung Zeit, andererseits haben die automatisch erstellten Klassen alle das gleiche Muster, wodurch der Code und das Projekt einheitlicher werden und eine bessere Struktur und Übersichtlichkeit entsteht. %
		%Warum: Popularität
		Das Paket ist mit 2.550 Likes und 99\% Popularität\footnote{Popularität ist eine eigene Bewertungskategorie auf dart.dev.} das beliebteste in dieser Liste. %
		%Auswirkung: Weiterentwickelbarkeit
		Aufgrund dieser beiden Punkte wird erwartet, dass andere Programmierer diese Klassen leichter erkennen und verstehen können, was letztendlich die Verständlichkeit erleichtern sollte.%
		 

	\item tests\cite{tech_packageTest}: %
		%Was
		Aus demselben Grund wurde das Testpaket für Flutter ausgewählt. Es stellt einheitliche Muster für das Schreiben von Tests bereit, was zu einer übersichtlichen Struktur führt. %
		%Warum: Popularität -> Weiterentwickelbarkeit
		%[Dementsprechend wird bei diesem Paket, genau wie beim vorherigen, aus der Beliebtheit und des einheitlichen Musters daraus geschlossen, dass die Nutzung davon die Verständlichkeit verbessert.]%
		
	\item lint\cite{tech_packageLints}: %
		%Was
		Linter legen eine Reihe von Regeln für Programmierpraktiken fest, zum Beispiel, dass eine Textzeile nicht länger als 80 Zeichen sein darf oder dass die Benennung von Variablen stets mit einem kleinen Buchstaben beginnen muss. %
		%Warum: Zeitsparen
		Durch die Nutzung eines solchen Pakets wird einerseits Zeit gespart, da sich der Entwickler weniger Gedanken um den Code-Style machen muss. Stattdessen können die vordefinierten Regeln des Paketes verwendet werden. %
		%Warum: Einheitlich
		Andererseits trägt es dazu bei, den Quellcode generell übersichtlicher zu gestalten, da durch das Paket in jeder Datei eine einheitliche Formatierung und Stil verwendet wird. %
		%Warum: Nützliche Regeln
		%Damit es zu diesen Vorteilen kommt, sollten die Regeln [gut durchdacht] und passend zur Programmiersprache und Framework sein. Da es sich beim Erstelle des Pakets um \glqq flutter.dev\grqq{} handelt, [wird genau das angenommen.] %
		Zunächst wurde das Paket flutter\_lints\cite{tech_packageFlutterLints} ausgewählt, da vermutet wurde, dass die Regeln gut durchdacht und der Programmiersprache und dem Framework angemessen sind, da es sich bei den Erstellern des Pakets um \glqq flutter.dev\grqq{} handelt. Beim Programmieren stellte sich jedoch heraus, dass zwar viele Regeln angeboten werden, aber die meisten standardmäßig deaktiviert sind. Das Pakets ist also darauf ausgelegt vom Entwickler angepasst zu werden. Das Ziel war jedoch, einen Linter mit vordefinierten Regeln zu verwenden. Daher wurde auf das Paket lint umgestellt. Dieses bietet eine Reihe von vordefinierten Regeln an und folgt dem Dart Style Guide\cite{tech_packageDartStyle}. Somit sollten die gleichen Vorteile wie bei flutter\_lints erzielt werden.
		%Warum: Beliebtheit
		%Des Weiteren erfreut sich auch dieses Paket erneut großer Beliebtheit. Daher wird auch hier erneut davon ausgegangen, dass durch die Beliebtheit und einheitlichen Muster, auch dieses Package die Leserlichkeit für andere Personen erleichtern.%
	

	\item github\cite{tech_packageGithub}: 
		%Warum: umfangreiche und zeitaufwändiges Umfangen
		Um mit dem Backend kommunizieren zu können, wird eine Schnittstelle benötigt, die die Programmiersprache mit dem Backend verbindet. Eine solche Schnittstelle zu erstellen ist eine umfangreiche und zeitaufwändige Aufgabe. Es würde also den Zeitrahmen dieser Arbeit sprengen, wenn versucht würde, diese Aufgabe eigenständig zu bewältigen. %
		%Was: github package
		Daher wird stattdessen das Paket \glqq github\grqq{} verwendet, das sich genau dieser Aufgabe widmet. Es stellt eine Verbindung zur GitHub-API bereit, damit über die Programmiersprache mit GitHub interagiert werden kann. %
		%Nachteil: Alternative
		Dieses Paket ist dabei das einzige, welches eine solche Schnittstelle für Flutter zur Verfügung stellt. Dies könnte sich als Nachteil herausstellen, da bei Komplikationen, Fehlern oder fehlender weiterer Unterstützung das Paket aufgrund fehlender Alternativen nicht ersetzt werden kann.%
		
	%\item shared\_preferences \& path\_provider um persistente und temporäre daten auf dem Handy speichern zu können.
	\item syncfusion\_flutter\_calendar\cite{tech_packageCalendar}: Um Zeit zu sparen und sicherzustellen, dass während der Bearbeitungszeit ein funktionsfähiges Produkt erstellt wird, wird vorerst ein vorhandenes Package für die Terminansichten verwendet. [Bei Bedarf kann dies jederzeit durch ein eigenes Design ersetzt werden.]  
	%\item flutter\_local\_notifications Um eine simple Schnittstelle zu Notificationen für Android und iOS zu haben.
\end{itemize}
