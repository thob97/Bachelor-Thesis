\subsection{Auswahl der Pakete}\myCheckmark
%Einleitung
In diesem Abschnitt wird sich mit der Auswahl der Pakete auseinandergesetzt. Dabei wird zuerst erklärt warum und nach welchen Kriterien generell die Pakete ausgesucht wurden. Anschließend werden die wichtigsten Pakete aufgelistet und ihre Daseinsberechtigung/Benutzung/Verwendung begründet.

%Generelle Kriterien
\myNewSection
	%Was
	Die Pakete wurden nach einem ähnlichen Verfahren wie auch in \secref{subsection:auswahlDesFrameworks} ausgesucht. So wurden bei Ihnen generell auf ihr Alter, ihre Beliebtheit und den Verlauf ihrer Aktualisierungen geachtet. 
	%Warum -> long time support
	Dabei wird nämlich angenommen, dass bei einer starken Ausprägung dieser drei Kriterien, es wahrscheinlicher ist, dass ein Paket noch für lange Zeit weitere Unterstützung und Aktualisierungen erhält. Und das ist wiederum nützlich wenn die zu erstellende Anwendung auch in der Zukunft noch bestehen soll. 
	%Warum -> 
	Außerdem kann es die Verständlichkeit, einer unserer nicht funktionalen Anforderungen, erleichtern, wenn die gewählten Pakete guten Support haben und etabliert/bereits bekannt sind. 
	%Auswirkung
	Aus diesem Grund wurde, falls mehrere Pakete mit ähnlichen Funktionen zur verfügung stehen, [stets] das ausgewählt, welches in diesem drei Punkten am besten abschneidet.

%Packages
\myNewSection
Pakete:
\begin{itemize}
	%Was
	\item json\_serializable\cite{tech_packageJson}: Dieses Package erstellt Json-Klassen automatisch, welche für die Übersetzung von Daten zu Json-Objekten und umgekehrt benötigt werden. 
		%Warum: Zeit
		Einerseits wird durch die automatische Erstellung etwas Zeit gespart. 
		%Warum: Einheitlich + Popularität -> Weiterentwickelbarkeit
		Andererseits haben die automatisch erstellten Klassen alle das selbe Muster. Dadurch wird der Code und das Projekt einheitlicher, wodurch eine bessere Struktur und Überblick geschaffen wird.
		Außerdem ist das Package mit 2550 Likes und 99\% Popularität\footnote{eine eigene Kategorie auf dart.dev} das beliebteste Paket dieser Liste. Durch diese beiden Punkte wird erwartet, das es andere Programmierer leichter fällt diese Klassen erkennen und verstehen. Im Endeffekt sollte das also die Verständlichkeit erleichtern.
		 
	%Was
	\item tests\cite{tech_packageTest}: Aus dem gleichen Grund wurde das Test Package für Flutter gewählt. Es legt ein einheitliches Muster für das Schreiben von Tests vor, was zu einer übersichtlichen Struktur führt. 
		%Warum: Popularität -> Weiterentwickelbarkeit
		So wird bei diesem Paket, genau wie beim vorherigen, aus der Beliebtheit und des einheitlichen Musters daraus geschlossen, dass die Nutzung davon die Verständlichkeit verbessert.
	%Was
	\item flutter\_lints\cite{tech_packageLints}: Das Package flutter\_lints legt eine Reihe von Regeln für nützliche  Programmierpraktiken vor. So zum Beispiel das eine Textzeile nicht länger als 80 Zeichen lang sein darf oder das die Benennung von Variablen stets mit einem kleinen Buchstaben anfangen.
		%Warum: Zeitsparen
		Durch die Nutzung dieses Paketes wird etwas Zeit gespart, da man sich weniger Gedanken um den Code-Style machen muss. Stattdessen kann man die vordefinierten Regeln vom Ersteller des Paketes benutzen. Beim Ersteller dieses Paketes handelt es sich dabei um \glqq flutter.dev\grqq. Daher wird auch angenommen, dass die ausgewählten Style Entscheidungen und Programmierpraktiken gut durchdacht und passend zum Framework und der Programmiersprache sind. 
		%Warum: Einheitlich
		Außerdem macht es den Quellcode generell übersichtlicher, da durch das Paket in jeder Datei eine einheitliche Formatierung und Style genutzt wird.
		%Warum: Beliebtheit
		Des Weiteren erfreut sich auch dieses Paket erneut großer Beliebtheit. Daher wird auch hier erneut davon ausgegangen, dass durch die Beliebtheit und einheitlichen Muster, auch dieses Package die Leserlichkeit für andere Personen erleichtern.
	
	%Warum: Zeit + Tests
	\item github\cite{tech_packageGithub}: Um mit dem Backend kommunizieren zu können benötigt es einer Schnittstelle von der Programmiersprache zum Backend. Eine solche Schnittstelle zu erstellen ist eine umfangreiche und zeitaufwändiges Umfangen. Es würde also nicht nur den Rahmen sondern auch die Zeit dieser Arbeit sprengen, wenn versucht würde diese Aufgabe eigenständig zu bewältigen. Daher wird stattdessen das Paket \glqq github\grqq benutzt, welche sich genau dieser Aufgabe widmet. 
		%Was
		So stellt das Package \dq github\dq eine API bereit, damit über dart mit GitHub interagiert werden kann. 
		%Alternative
		Dieses Paket ist dabei das einzige, welches eine solche Schnittstelle zur verfügung stellt. Das heißt also, dass es für dieses Package keine alternativen gibt. Das könnte sich als Nachteil herausstellen, da bei Komplikationen, Fehler oder keiner weiteren Unterstützung das Paket nicht gewechselt werden kann. 

\end{itemize}
