\subsection{Konfigurationsdateiformat}\label{section:tech:sub:konfigurationsdateiformat}%
%Was
Wie im \secref{subsection:anforderung:funktionaleAnforderungen} wurde erwähnt, dass es möglich sein soll, Einstellungen für die Anwendung über den PC zu konfigurieren. Dazu soll die Konfigurationsdatei mithilfe von GitHub zwischen dem Handy und dem PC übertragen werden.\newline%
%Frage
Nun wurde sich die Frage gestellt, in welchem Format die Datei vorliegen sollte.\newline%
Es wäre wahrscheinlich passend, eine Textdatei zu verwenden, da diese zur Unix-Philosophie\cite{tech_unix-philosophie} und damit zu CLI-Programmen passen würde. %
Jedoch wurde zunächst entschieden, eine JSON-Datei zu verwenden, da sie einfacher mit Flutter verwendet werden kann und dadurch Arbeitszeit eingespart wird. %
Zudem kann das Dateiformat auch nach der Bachelorarbeit immer noch in eine Textdatei umgewandelt werden.%