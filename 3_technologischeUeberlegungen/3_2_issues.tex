\subsection{Darstellung der Erinnerungen}\label{section:tech:sub:darstellung_der_erinnerungen}%
%Was
Wie im \secref{subsection:anforderung:funktionaleAnforderungen} erwähnt, soll es die Funktion geben Erinnerungen zu erstellen. Diese Erinnerungen sollen mithilfe von GitHub zwischen dem Handy und den Pc übertragen werden.\newline%
%Frage
%Die nun zu stellende Frage ist, wie genau die Erinnerungen über GitHub übertragen werden sollen.\newline%
%as Files
Eine Möglichkeit dafür wäre es, die Erinnerungen als Dateien auf das GitHub Repository zu speichern. Die jeweiligen Dateien müssten darauf nur von beiden Geräten herunterladen werden.\newline%
Eine andere Möglichkeit, für welches sich letztendlich auch entschieden wurde, ist es die Erinnerungen als GitHub Issues darzustellen. Dies hätte den Vorteil, dass die Erinnerung neben der App und dem Terminal auch über die Webseite [betrachtbar/ersichtlich] sind. Des Weiteren bietet sich das Format von Issues auch für Erinnerungen an. So bieten sie die Möglichkeit ein Titel, eine Beschreibung, Dateien sowie verschiedene Statuse hinzuzufügen.\newline%
Ein Nachteil davon ist es jedoch, das es nicht möglich ist über das Programm git Issues anzuzeigen. Deshalb benötigt es eine weitere CLI-Anwendung, welche diese Funktion unterstützt. Da GitHub aber bereits ein passendes Programm anbietet\cite{tech_github-cli}, wird dies als kein zu [verheerender] Nachteil eingeschätzt.