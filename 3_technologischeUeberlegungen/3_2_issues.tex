\subsection{Darstellung der Erinnerungen}\label{section:tech:sub:darstellung_der_erinnerungen}%
%Was
Wie im \secref{subsection:anforderung:funktionaleAnforderungen} erwähnt, soll es die Funktion geben Erinnerungen zu erstellen die mithilfe von GitHub zwischen dem Handy und dem PC übertragen werden.\newline%
%Frage
%Die nun zu stellende Frage ist, wie genau die Erinnerungen über GitHub übertragen werden sollen.\newline%
%as Files
Eine Möglichkeit hierfür wäre, die Erinnerungen als Dateien im GitHub-Repository zu speichern. Die jeweiligen Dateien können dann von beiden Geräten heruntergeladen werden.\newline%
Eine andere Möglichkeit, für welches sich letztendlich auch entschieden wurde, ist es die Erinnerungen als GitHub Issues darzustellen. Dies hätte den Vorteil, dass die Erinnerung neben der App und dem Terminal auch über die Webseite darstellbar sind. Darüber hinaus eignet sich das Format von Issues gut für Erinnerungen, da es die Möglichkeit bietet, einen Titel, eine Beschreibung, Dateien und verschiedene Status hinzuzufügen.\newline%
Ein Nachteil besteht jedoch darin, dass es nicht möglich ist, die GitHub Issues direkt über das Programm Git anzuzeigen. Daher wird eine weitere CLI-Anwendung benötigt, die diese Funktion unterstützt. Da GitHub jedoch bereits eine passende Anwendung anbietet\cite{tech_github-cli}, wird dies nicht als zu gravierender Nachteil betrachtet.
%
%
%