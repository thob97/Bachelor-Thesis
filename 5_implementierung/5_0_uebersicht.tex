%%% hidden subsection for a better structure in latex editor: "texifier"
\myComment{\subsection*{Übersicht}} 
%Einleitung
%Übersicht
%Zusammenfassung




\myTextTodo{
%Implementierung - Was + Warum wenig Implementierung wiedergegeben
\textbf{Abschnitte der Arbeit}\\
An dieser Stelle sollten die wichtigsten Erkenntnisse erhoben worden sein. Nun gehen wir über zum \secref{section:implementierung}. Hier wird betrachtet wie die Software aufgebaut sein soll. Also geht es auch hier unter anderem um das \dq innere Design\dq.\newline%
Die Implementierung, ein Großteil der eigentlichen Arbeit von diesem Kapitel, wird nicht wiedergegeben. Jede einzelne Entscheidung und Erkenntnis der Implementierung zu erwähnen und begründen würde nicht nur den Rahmen sprengen, sonder auch sehr ermüdend für den Leser werden. Daher werden nur die wichtigsten Entscheidungen, Schwierigkeiten und Erkenntnisse erwähnt.\newline%
}
%
%\myTextTodo{
%Aufzählung, Alternativen, Entscheidungen, \\
%LEITFADEN: 1.Was ist die Eigenschaft, 2.Warum ist es wichtig, 3.Wie wird es umgesetzt\\
%} %%%
%
%
%\myNewSection
%auto setup -> einfachheit für das handy -> anforderungen\newline
%
%\myNewSection
%Github config file:\newline
%json weil vs text. erst json aus einfachheit später auch noch txt files möglich da dies besser zu cli terminkalendar passen würde
%mögliche optionen...
%konfiguration welche auf dem handy nötig ist: token+repoName+configPath
%kommenatre in json nicht möglich zum erklären der variabeln. Lösung: erklärung in den variabel namen vs extra kommentarVariabeln = 'string'.
%\newline
%config file: besonders lange variabeln namen. normalerweise nicht sinnvoll und da kürzere beim programmieren genau so verständlich seien können. da aber der enduser nicht das gleiche wissen wie ein interner entwickler der app hat, wurden lange variabel namen genutzt, damit die bedeutung klar wird. + json akzeptiert keine commantare.