\subsection{Verbindung zur Datenbank} %
%Einleitung: 
Wie bereits erwähnt, wurde für die Datenbankverbindung eine vorhandene API verwendet. Im Folgenden werden Schwierigkeiten und Besonderheiten bei der Verwendung dieser aufgeführt.
%
%
%
%
%
%
%
\subsubsection{Authentikation \& Berechtigungen}%
%
Die Authentizierung über einen Benutzernamen und ein Passwort wird von der Api nicht unterstützt\cite{imp_github_userPasswordAuthentication}. Deshalb wird die Authentifizierung mit Tokens benutzt. %
Dies stellte aber kein Problem dar, da die Verwendung von Tokens ohnehin bevorzugt wird. Denn dadurch kann der Nutzer sicher sein, dass die Anwendung mit den gegebenen Zugriffsrechten nichts Schädliches ausführen kann.%
%
\newline%
Zwei der verwendeten Funktionen der API erfordern besondere Zugriffsrechte. Das Erstellen eines Repositorys erfordert die \glqq repo\grqq{} Berechtigung\cite{imp_github_createRepo}, während das Erstellen und Aktualisieren von Dateien die \glqq workflow\grqq{} Berechtigung erfordert\cite{imp_github_createFile}. Um den Benutzer über diese Informationen zu informieren, muss das Token auf diese Berechtigungen getestet werden. Jedoch unterstützt die API diese Funktion nicht mehr\cite{imp_github_tokensScopeDiscontinued}. Stattdessen bietet GitHub jedoch die Möglichkeit, sich die Berechtigungen über die Antwort einer URL anzeigen zu lassen\cite{imp_github_tokensScopeCurl}. Um dies in der Anwendung zu nutzen, wird die URL mit curl aufgerufen und die Antwort nach den passenden Berechtigungen geparst.%
\newline%
In der Dokumentation konnte zwar nichts dazu gefunden werden, jedoch hat sich beim Testen herausgestellt, dass diese Methode, um die Berechtigungen zu erlangen, nur von den \glqq classic Tokens\grqq{} unterstützt wird. Für die neueren \glqq beta Tokens\grqq{} sendet der Server keine Antwort zurück. Daher werden derzeit \glqq beta Tokens\grqq{} in der App nicht unterstützt.%
%%
%%
%%
%%
%%
\subsubsection{Dateien herunterladen}%
%Was: Endpoint
Die API bietet einen Endpunkt an, mit dem Dateien direkt aus einem Repository heruntergeladen werden können\cite{imp_github_1mb100mbDownloadFile}. %
	%Warum
	Allerdings fügt dieser Endpunkt in Flutter einige nicht konventionelle Zeichen hinzu, was zu ungültigen heruntergeladenen Dateien führt. Durch Tests wurde festgestellt, dass es sich dabei um Leerzeichen und ein \textbackslash n-Zeichen handelt. Wenn diese entfernt werden, kann die Datei normal verwendet werden.%
\newline%
%Was:Filesize
Beim weiteren Testen wurde festgestellt, dass dieser Endpunkt nur Dateien bis zu einer Größe von maximal 1 MB unterstützt. Diese Beschränkung ist aus Erfahrung zu klein, wenn Fotos, Videos und Audio heruntergeladen werden sollen. %
%Auswirkung
Dementsprechend wird der Endpunkt nicht mehr direkt genutzt. Stattdessen werden die Dateien über den GitHub-Link heruntergeladen. Dadurch können Dateien mit einer Größe von bis zu 100 MB heruntergeladen werden.%
%%
%%
%%
%%
%%
\subsubsection{GitHub Issues}%
%Was: Issues -> App, CLI, Webview
Wie zuvor erwähnt, sollen Erinnerungen über GitHub Issues ermöglicht werden. Dadurch ist es möglich, die Listen über die App, das CLI oder die Webview anzuzeigen und zu bearbeiten.%
%Was:Unsupported
Leider ist es nicht möglich, über die API Dateien zu Issues hinzuzufügen\cite{imp_github_issueFilesUnsupported}. %
	%Auswirkung: herrausfinden
	Um herauszufinden, wie Dateien dennoch zu Issues hinzugefügt werden können, wurde versucht herrauszufinden wie Daten über die Webansicht zu Issues hinzugefügt werden. Es stellte sich heraus, dass die Dateien nicht direkt in den Issues gespeichert werden, sondern lediglich durch einen Link referenziert werden. %
	%Auswirkung: Funktion erstellen
	Mit dieser Information konnte eine passende Funktion erstellt werden: Zunächst wird die GitHub API genutzt, um ein Issue zu erstellen. Anschließend wird die Datei, die dem Issue hinzugefügt werden soll, auf das entsprechende Repository hochgeladen. Schließlich wird ein Link zur Datei dem Issue hinzugefügt, wodurch die Datei im Webview als solche angezeigt wird. %
\newline%
%Was: Embedded
Die GitHub-Webansicht unterstützt Dateien eingebettet darzustellen, anstatt sie nur als Link anzuzeigen. Beispielsweise wird bei einer PNG-Datei direkt das Bild angezeigt und bei einer MP4-Datei wird das entsprechende Video wiedergegeben. %
%Warum: (Zeitersparnis)
Dies hat den Vorteil, dass es dem Nutzer Aufwand spart, da sie die Datei direkt in der Webansicht betrachten können, ohne sie zuvor herunterladen zu müssen.%
%Auswirkung:
Daher sollen auch die mit der App erstellten Issues von dieser Funktion profitieren. Um herauszufinden, wie die WebView Dateien einbetten kann, wurden zahlreiche Tests durchgeführt und die Dokumentation zurate gezogen\cite{imp_github_syntaxing}. Dabei stellte sich heraus, dass einige der in der Dokumentation erwähnten Funktionen für Issues nicht funktionieren, wie zum Beispiel \glqq Relative Link\grqq{}, die in Markdown-Dateien\footnote{Eine reine Textdatei, die auf GitHub Syntaxunterstützung bietet, um den Text zu formatieren.} verwendet werden können, aber nicht in Issues.
Bei den Tests konnte erfolgreich die Einbindung von Bildern in die WebView reproduziert werden, jedoch war es nicht möglich, Videos einzubetten. Es scheint, dass Videos nur dann in die WebView eingebettet werden können, wenn sie über eine interne GitHub-URL der Form \glqq https://user-images.githubusercontent.com/...\grqq{} verfügbar sind. Diese URLs können jedoch anscheinend nur generiert werden, wenn die Dateien über die WebView hinzugefügt wurden. Daher ist es derzeit nicht möglich, eingebettete Videos in der WebView mit der App zu erstellen.%
\newline
\myNewSection
%Was+Warum:CLI
Um die Issues auch im CLI ansprechend darzustellen, wurde versucht, bei deren Erstellung ein geeignetes Format zu verwenden.%
\newline%
Ein Issue besteht immer aus einem Titel und einer Beschreibung. Die Beschreibung wird so formatiert, dass die erste Zeile die Beschreibung der Erinnerung enthält und die folgenden Zeilen die Dateien.%
\newline%
Die Dateien müssen für die WebView im Format \glqq [ ]\{url\}\grqq{} gekennzeichnet werden. Innerhalb der \glqq[ ]\grqq{} wird der Dateipfad angegeben, um eine weitere Referenz auf die Datei für die CLI bereitzustellen. %
Um eine klare Struktur im Repository zu gewährleisten, werden die Dateien der Issues dabei in Ordnern der Form \glqq issue\{issueNummer\}\grqq{} gespeichert.%
\newline%
Für eine Beispieldarstellung aller drei Ansichten siehe Abbildung \ref{pic_webview}, \ref{pic_reminder_cli}, \ref{pic:reminder_handy}.
%%
%%
%%
%%
%%
\myComment{

%\subsubsection{Endpunkte}
%Von der Api wurden folgende Endpunkte benutzt. createRepository getRepository
%repo: für create repository: https://docs.github.com/en/rest/repos/repos?apiVersion=2022-11-28#create-a-repository-for-the-authenticated-user
%workflow: für create und update files in repo: https://docs.github.com/en/rest/repos/contents?apiVersion=2022-11-28#create-a-file
%
%Weitere benutzte funktionen der Api(benötigen keine extra scopes):
%get repository content: https://docs.github.com/en/rest/repos/contents?apiVersion=2022-11-28#get-repository-content
%List issues assigned to the authenticated user :https://docs.github.com/en/rest/issues/issues?apiVersion=2022-11-28#list-issues-assigned-to-the-authenticated-user
%create an issue: https://docs.github.com/en/rest/issues/issues?apiVersion=2022-11-28#create-an-issue
%update an issue: https://docs.github.com/en/rest/issues/issues?apiVersion=2022-11-28#update-an-issue
 

}